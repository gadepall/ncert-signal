\begin{enumerate}[label=\thechapter.\arabic*,ref=\thechapter.\theenumi]
\numberwithin{equation}{enumi}
\numberwithin{figure}{enumi}
\numberwithin{table}{enumi}
\item 
	The $Z$-transform of $p(n)$ is defined as
\begin{align}
P(z) = \sum_{n=-\infty}^{\infty}p(n)z^{-n}
\label{eq:ztrans}
\end{align}
\item If 
\begin{align}
	p(n) &= p_1(n)* p_2(n),
	\\
	P(z)&=P_1(z)P_2(z)
\end{align}
The above property follows from Fourier analysis and is fundamental to signal processing. 
\item For a Geometric progression defined as follows
\begin{align}
	 x\brak{n} =x\brak{0}r^nu\brak{n}
\end{align}  
\begin{align}
               X\brak{z} &= \sum_{n=-\infty}^{\infty}x\brak{n}z^{-n}\\
               &=\sum_{n=0}^{\infty}x\brak{0}r^nz^{-n}\\
                &=\sum_{n=0}^{\infty}x\brak{0}\brak{rz^{-1}}^n\\
               &= \frac{x\brak{0}}{1-rz^{-1}} & \abs{rz^{-1}}<1\\ 
               ROC &\implies \abs{z}>\abs{r} 
\end{align}
\end{enumerate}
