\iffalse
\let\negmedspace\undefined
\let\negthickspace\undefined
\documentclass[journal,12pt,twocolumn]{IEEEtran}
\usepackage{cite}
\usepackage{amsmath,amssymb,amsfonts,amsthm}
\usepackage{algorithmic}
\usepackage{graphicx}
\usepackage{textcomp}
\usepackage{xcolor}
\usepackage{txfonts}
\usepackage{listings}
\usepackage{enumitem}
\usepackage{mathtools}
\usepackage{gensymb}
\usepackage{comment}
\usepackage[breaklinks=true]{hyperref}
\usepackage{tkz-euclide} 
\usepackage{listings}
\usepackage{gvv}                                        
\def\inputGnumericTable{}                                 
\usepackage[latin1]{inputenc}                                
\usepackage{color}                                            
\usepackage{array}                                            
\usepackage{longtable}                                       
\usepackage{calc}                                             
\usepackage{multirow}                                         
\usepackage{hhline}                                           
\usepackage{ifthen}                                           
\usepackage{lscape}
\usepackage{amsmath}
\usepackage{caption}
\newtheorem{theorem}{Theorem}[section]
\newtheorem{problem}{Problem}
\newtheorem{proposition}{Proposition}[section]
\newtheorem{lemma}{Lemma}[section]
\newtheorem{corollary}[theorem]{Corollary}
\newtheorem{example}{Example}[section]
\newtheorem{definition}[problem]{Definition}
\newcommand{\BEQA}{\begin{eqnarray}}
\newcommand{\EEQA}{\end{eqnarray}}
\newcommand{\define}{\stackrel{\triangle}{=}}
\theoremstyle{remark}
\newtheorem{rem}{Remark}
\begin{document}

\bibliographystyle{IEEEtran}
\vspace{3cm}

\title{NCERT 10.5.3 10Q}
\author{EE22BTECH11010 - Venkatesh D Bandawar $^{*}$% <-this % stops a space
}
\maketitle
\newpage
\bigskip

\renewcommand{\thefigure}{\theenumi}
\renewcommand{\thetable}{\theenumi}

\textbf{Question:} Show that $a_0$ , $a_1$ , $a_2$
, . . ., $a_n$
, . . . form an AP where an is defined as below :
\begin{enumerate}
    \item $a_n$ = $(3 + 4n)$ 
    \item $a_n$ = $(9 - 5n)$ 
\end{enumerate}
Also find the sum of the first 15 terms in each case.

\textbf{Solution:} 
\fi
\begin{table}[!h] 
    \centering
    \begin{tabular}{|c|c|c|}
\hline
     \textbf{Parameter} & \textbf{Description} & \textbf{Value} \\
     \hline
     \multirow{3}{*}{$x_i(n)$} & \multirow{3}{*}{$i^{th}$Discrete signal} & $(3+4n)u(n)$\\
     \cline{3-3}
     & & $(9-5n)u(n)$\\
     \hline
     \multirow{3}{*}{$x_i(0)$} & \multirow{3}{*}{First term of $i^{th}$AP} & $3$ \\
     \cline{3-3}
     & & $9$\\
     \hline
     \multirow{3}{*}{$d_i$} & \multirow{3}{*}{common difference of $i^{th}$AP} & $4$ \\
     \cline{3-3}
     & & $-5$\\
     \hline
\end{tabular}

    \caption{Given parameters}
    \label{given parameters list}
    \end{table}
\begin{enumerate} 
    \item From equation \eqref{eq:APSum}
    \begin{align}
        X(z) &= \frac{3}{1-z^{-1}} + \frac{4.z^{-1}}{(1-z^{-1})^2} ; |z|>1\\
        \because y(n) &= x(n) * u(n)\\
        Y(z) &= X(z)U(z)\\
        &= \sbrak{\frac{3}{\brak{1-z^{-1}}^2} + \frac{4 z^{-1}}{\brak{1-z^{-1}}^3}} 
    \end{align}
    Using contour integration for inverse Z transformation,
    \begin{align}
        y(14) &= \frac{1}{2\pi j}\int Y(z) z^{13} dz \\
         &= \frac{1}{2\pi j}\int \frac{3.z^{15}}{(z-1)^{2}} dz + \frac{1}{2\pi j}\int \frac{4.z^{15}}{(z-1)^{3}} dz\\
        \because R&=\frac{1}{\brak {m-1}!}\lim\limits_{z\to a}\frac{d^{m-1}}{dz^{m-1}}\brak {{(z-a)}^{m}f\brak z}\\
        R_1 &=\frac{1}{1!}\lim\limits_{z\to 1}\frac{d}{dz}\brak {(z-1)^2.\frac{3.z^{15}}{(z-1)^{2}}}\\
        &= 45\\
        R_2 &=\frac{1}{2!}\lim\limits_{z\to 1}\frac{d^2}{dz^2}\brak {(z-1)^3.\frac{4.z^{15}}{(z-1)^{3}}}\\
        &= 420\\
        \implies y(14) &= R_1 + R_2 \\
        &= 465
    \end{align}
    
    \begin{figure}[!h] 
    \centering
    \includegraphics[width=\columnwidth]{ncert-maths/10/5/3/10/figs/signal_x1.png}
    \caption{$x_1(n)=(3+4n)u(n)$}
    \label{fig:Graph1_math.10.5.3.10}
    \end{figure}

    \begin{figure}[!h] 
    \centering
    \includegraphics[width=\columnwidth]{ncert-maths/10/5/3/10/figs/signal_y1.png}
    \caption{$x_1(n)=(2n^2+5n+3)u(n)$}
    \label{fig:Graph2_math.10.5.3.10}
    \end{figure}
    
    \item From equation  \eqref{eq:APSum}
    \begin{align}
         X(z) &= \frac{9}{1-z^{-1}} - \frac{5.z^{-1}}{(1-z^{-1})^2} ; |z|>1\\
        \because y(n) &= x(n) * u(n)\\
        Y(z) &= X(z)U(z)\\
         &= \sbrak{\frac{9}{\brak{1-z^{-1}}^2} - \frac{5 z^{-1}}{\brak{1-z^{-1}}^3}}
    \end{align}
    Using contour integration for inverse Z transformation,
    \begin{align}
       y(14) &= \frac{1}{2\pi j}\int Y(z) z^{13} dz \\
         &= \frac{1}{2\pi j}\int \frac{9.z^{15}}{(z-1)^{2}} dz - \frac{1}{2\pi j}\int \frac{5.z^{15}}{(z-1)^{3}} dz\\
        \because R&=\frac{1}{\brak {m-1}!}\lim\limits_{z\to a}\frac{d^{m-1}}{dz^{m-1}}\brak {{(z-a)}^{m}f\brak z}\\
        R_1 &=\frac{1}{1!}\lim\limits_{z\to 1}\frac{d}{dz}\brak {(z-1)^2.\frac{9.z^{15}}{(z-1)^{2}}}\\
         &= 135\\
        R_2 &=\frac{1}{2!}\lim\limits_{z\to 1}\frac{d^2}{dz^2}\brak {(z-1)^3.\frac{5.z^{15}}{(z-1)^{3}}}\\
         &= 525\\
        \implies y(14) &= R_1 - R_2 \\
        &= -390
    \end{align}
    
    \begin{figure}[!h] 
    \centering
    \includegraphics[width=\columnwidth]{ncert-maths/10/5/3/10/figs/signal_x2.png}
    \caption{$x_2(n)=(9-5n)u(n)$}
    \label{fig:Graph3_math.10.5.3.10}
    \end{figure}

    \begin{figure}[!h] 
    \centering
    \includegraphics[width=\columnwidth]{ncert-maths/10/5/3/10/figs/signal_y2.png}
    \caption{$x_2(n)=(-5n^2+13n+18)u(n)$}
    \label{fig:Graph4_math.10.5.3.10}
    \end{figure}

\end{enumerate}

