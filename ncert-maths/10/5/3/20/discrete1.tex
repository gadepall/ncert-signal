\let\negmedspace\undefined
\let\negthickspace\undefined
\documentclass[journal,12pt,twocolumn]{IEEEtran}
\usepackage{cite}
\usepackage{amsmath,amssymb,amsfonts,amsthm}
\usepackage{algorithmic}
\usepackage{graphicx}
\usepackage{textcomp}
\usepackage{xcolor}
\usepackage{txfonts}
\usepackage{listings}
\usepackage{enumitem}
\usepackage{mathtools}
\usepackage{gensymb}
\usepackage{comment}
\usepackage[breaklinks=true]{hyperref}
\usepackage{tkz-euclide}
\usepackage{listings}
\usepackage{gvv}
\def\inputGnumericTable{}
\usepackage[latin1]{inputenc}
\usepackage{color}
\usepackage{array}
\usepackage{longtable}
\usepackage{calc}
\usepackage{multirow}
\usepackage{hhline}
\usepackage{ifthen}
\usepackage{lscape}

\newtheorem{theorem}{Theorem}[section]
\newtheorem{problem}{Problem}
\newtheorem{proposition}{Proposition}[section]
\newtheorem{lemma}{Lemma}[section]
\newtheorem{corollary}[theorem]{Corollary}
\newtheorem{example}{Example}[section]
\newtheorem{definition}[problem]{Definition}
\newcommand{\BEQA}{\begin{eqnarray}}
\newcommand{\EEQA}{\end{eqnarray}}
\newcommand{\define}{\stackrel{\triangle}{=}}
\theoremstyle{remark}
\newtheorem{rem}{Remark}
\begin{document}

\bibliographystyle{IEEEtran}
\vspace{3cm}

\title{NCERT Discrete - 10.5.3.20}
\author{EE23BTECH1205 - Avani Chouhan$^{*}$% <-this % stops a space
}
\maketitle
\newpage
\bigskip

\renewcommand{\thefigure}{\theenumi}
\renewcommand{\thetable}{\theenumi}

\vspace{3cm}
\textbf{Question : 10.5.3.20} 
The sum of some terms of G.P. is 315 whose first term and the common ratio are $5$ and $2$ , respectively. Find the last term and the number of terms.\\
\solution

\begin{table}
  \centering
  
  \begin{tabular}{|c|c|c|c|}
\hline
\textbf{Parameter}&\textbf{Description} &\textbf{subquestion}& \textbf{Value}\\
\hline
     \multirow{4}{*}{$\Delta \theta$} & \multirow{4}{*}{$\theta_1 - \theta_2$} &\brak{a}& 6.4$\pi$ \, radians \\
     \cline{3-4}
     & & \brak{b}& 0.8$\pi$ \, radians \\
     \cline{3-4}
     & &\brak{c}& $\pi$ \, radians \\
     \cline{3-4}
     & & \brak{d} & $\dfrac{3\pi}{2\vphantom{\brak{0.1}}}$ \, radians \\
     \hline
\end{tabular}

  \caption{Input Parameters}
  \label{tab:10.5.3.20table1}
\end{table}
\begin{align}
x(n) = x(0)r^{n}u(n)
\label{eq:10.5.3.20eq}
\end{align}
From \eqref{eq:gpz}
\begin{align}
X(z) =\frac{5}{1-2z^{-1}} \quad \abs{z} > \abs{2}
\end{align}
By contour integration:
\begin{align}
y(n) &= x(0)\brak{\frac{r^{n+1}-1}{r-1}}u(n)\\
315 &= 5\brak{2^{n+1}- 1}  \\
\implies n &= 5
\end{align}
The number of terms is \(n + 1 = 6\)\\
From \eqref{eq:10.5.3.20eq}:
\begin{align}
x(5) &= 5\brak{2^{5}}\\
 &= 160 
\end{align}

\begin{figure}
    \centering
    \includegraphics[width=\columnwidth]{figs/plot1.png}
    \caption{Stem plot of x(n)}
    \label{fig:10.5.3.20fig1}
\end{figure}
\begin{figure}
    \centering
    \includegraphics[width=\columnwidth]{figs/plot2.png}
    \caption{Stem plot of y(n)}
    \label{fig:10.5.3.20fig2}
\end{figure}
\end{document}
