\iffalse
\let\negmedspace\undefined
\let\negthickspace\undefined
\documentclass[journal,12pt,twocolumn]{IEEEtran}
\usepackage{cite}
\usepackage{amsmath,amssymb,amsfonts,amsthm}
\usepackage{algorithmic}
\usepackage{graphicx}
\usepackage{textcomp}
\usepackage{xcolor}
\usepackage{txfonts}
\usepackage{listings}
\usepackage{enumitem}
\usepackage{mathtools}
\usepackage{gensymb}
\usepackage{comment}
\usepackage[breaklinks=true]{hyperref}
\usepackage{tkz-euclide} 
\usepackage{listings}
\usepackage{gvv}                                        
\def\inputGnumericTable{}                                 
\usepackage[latin1]{inputenc}                                
\usepackage{color}                                            
\usepackage{array}                                            
\usepackage{longtable}                                       
\usepackage{calc}                                             
\usepackage{multirow}                                         
\usepackage{hhline}                                           
\usepackage{ifthen}                                           
\usepackage{lscape}

\newtheorem{theorem}{Theorem}[section]
\newtheorem{problem}{Problem}
\newtheorem{proposition}{Proposition}[section]
\newtheorem{lemma}{Lemma}[section]
\newtheorem{corollary}[theorem]{Corollary}
\newtheorem{example}{Example}[section]
\newtheorem{definition}[problem]{Definition}
\newcommand{\BEQA}{\begin{eqnarray}}
\newcommand{\EEQA}{\end{eqnarray}}
\newcommand{\define}{\stackrel{\triangle}{=}}
\theoremstyle{remark}
\newtheorem{rem}{Remark}
\begin{document}

\bibliographystyle{IEEEtran}
\vspace{3cm}

\title{10.5.3.19}
\author{EE23BTECH11065 - prem sagar}
\maketitle
\newpage

\bigskip 

\renewcommand{\thefigure}{\arabic{figure}}
\renewcommand{\thetable}{\arabic{table}}
\textbf{Question}:\\  200 logs are stacked in the following manner: 20 logs in the bottom row, 19 in the next row,18 in the row next to it and so on (see Fig \ref{fig:10.5.3.19.q}). In how many rows are the 200 logs placed and how many logs are in the top row?
\begin{figure}[h]
    \centering
    \includegraphics[width=1\linewidth]{ncert-maths/10/5/3/19/figs/question.png}
    \caption{ }
    \label{fig:10.5.3.19.q}
\end{figure}
\fi
\\\\\textbf{Solution}:
\begin{table}[!ht]
  \def\arraystretch{1.5}
  \centering
  \renewcommand\thetable{1}
  \begin{tabular}{|c|c|c|}
   \hline
   \textbf{Symbol} & \textbf{Value}& \textbf{Description} \\
   \hline
        $ x\brak{0}$ & $20$ & first term of AP\\
        \hline
        d & ${-1}$ & common difference\\
        \hline
        $x\brak{n}$ &  & $\brak{x\brak{0}+nd}u\brak{n}$\\
        \hline
        $y\brak{n}$ & $200$ & \\
        \hline
\end{tabular}

  \caption{input parameters}
  \label{tab:10.5.3.19}
  \end{table}
\\\\by the differentiation property:
\\\begin{align}
 x\brak{n}\overset{z}{\longleftrightarrow} (-z)\frac{dX\brak{z}}{dz}
 \\\implies n u\brak{n}\overset{z}{\longleftrightarrow} \frac{z^{-1}}{\brak{1-z^{-1}}^2},\abs z>1
 \label{eq:10.5.3.19.2}
 \\\implies n^2 u\brak{n}\overset{z}{\longleftrightarrow} \frac{z^{-1}\brak{z^{-1}+1}}{\brak{1-z^{-1}}^3} ,\abs z>1
 \end{align}
 \begin{align}
 \implies Z^{-1}\biggl[\frac{1}{\brak{1-z^{-1}}^2}\biggr]&=\brak{n+1}u\brak{n}
 \label{eq:10.5.3.19.4}
\\\implies Z^{-1}\biggl[\frac{z^{-1}}{\brak{1-z^{-1}}^3}\biggr]&=\frac{n\brak{n+1}}{2}u\brak{n}
\label{eq:10.5.3.19.5}
 \end{align}
 from \eqref{eq:10.5.3.19.2}
\begin{align} 
\implies X\brak{Z}&=\frac{20}{1-z^{-1}}-\frac{z^{-1}}{\brak{1-z^{-1}}^2}\,,\abs{z}>1
\label{eq:10.5.3.19.9}
\end{align}
from \eqref{eq:10.5.3.19.9}
\begin{align}
y\brak{n}&=x\brak{n}*u\brak{n}
\\\implies Y\brak{Z}&=X\brak{Z}U\brak{Z}
\\&=\frac{20}{\brak{1-z^{-1}}^2}-\frac{z^{-1}}{\brak{1-z^{-1}}^3}\,,\abs{z}>1
\end{align}
substituting \eqref{eq:10.5.3.19.4}, \eqref{eq:10.5.3.19.5}:
\begin{align}
y\brak{n}&=\brak{20\brak{n+1}-\frac{n\brak{n+1}}{2}}u\brak{n}
\\&=\frac{\brak{n+1}\brak{40-n}}{2}
\end{align}
since number of logs=200
\begin{align}
 200&=\frac{\brak{n+1}\brak{40-n}}{2}   
\\n&=24,15
\end{align}
\\for n=24
\begin{align}
x\brak{24}&=20-24
\\&={-4}
\end{align}
but logs can't be negative
\\for n=15
\begin{align}
x\brak{15}&=20-15
\\&=5
\end{align}
\\so number of rows=15
\\number of logs=5
\\\begin{figure}[h]
  \renewcommand\thefigure{1}
    \centering
    \includegraphics[width=1\linewidth]{ncert-maths/10/5/3/19/figs/figure_plot.png}
    \caption{plot of y\brak{n} v/s n}
\end{figure}
