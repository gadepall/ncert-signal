\documentclass[12pt]{article}
\usepackage{amsmath}
\usepackage{graphicx}
\usepackage{float}
\usepackage{amssymb}
\usepackage{pgfplots}
\pgfplotsset{compat=1.18}
\newcommand{\tabref}[1]{Table~\ref{#1}}
\newcommand{\figref}[1]{Figure~\ref{#1}}
\providecommand{\abs}[1]{\left\vert#1\right\vert}

\begin{document}

\title{Discrete Assignment}
\author{Mohana Eppala\\ EE23BTECH11018}
\maketitle

\section*{Problem Statement}
Determine the AP whose third term is 16 and the 7th term exceeds the 5th term by 12. 
\section*{Solution}
\begin{table}[H]
\begin{table}[h!!]
\renewcommand\thetable{1}
    \centering
        \begin{tabular}{|c|c|c|}
	    \hline
	            Variable&             Description&value\\\hline
		                 $r$&            common ratio&2    \\\hline
				      $x\brak{7}$&              eighth term&192  \\\hline
				           $x\brak{n}$&General term of sequence&None \\\hline
					        $X\brak{z}$&    Z-Transform Equation&None \\\hline
						    \end{tabular}
						        \caption{Variables Used and their Descriptions}
							    \label{tab 11.9.3.2.1}
							    \end{table}

\end{table}


From \tabref{tab:1}
\begin{align}
    x(0) +6d - x(0) - 4d &= 12 \\ \implies
    2d &= 12\\ \implies
    d &= 6
\end{align}
Also,
\begin{align}
     x(0) + 2d &= 16 \\ \implies
	x(0) + 2(6) &= 16 \\ \implies
	x(0) &= 4 \\
	\therefore x(n) &= 6n + 4 
\end{align}
From \tabref{tab:1}
\begin{align}
	X(z) &= x(0)  \frac{1}{1-z^{-1}} + d \frac{z^{-1}}{(1 - z^{-1})^2} \\
	&= 4 \frac{1}{1-z^{-1}} + 6 \frac{z^{-1}}{(1 - z^{-1})^2} \\
	&= \frac{4+2z^{-1}}{(1-z^{-1})^2} \quad \abs{z} > 1
\end{align}



\begin{figure}[H]
    \centering
    \includegraphics{figs/fig1.png}
    \caption{Given AP}
    \label{fig}
\end{figure}

\end{document}

