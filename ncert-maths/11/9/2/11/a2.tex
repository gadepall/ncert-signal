% \iffalse
\let\negmedspace\undefined
\let\negthickspace\undefined
\documentclass[journal,12pt,twocolumn]{IEEEtran}
\usepackage{cite}
\usepackage{amsmath,amssymb,amsfonts,amsthm}
\usepackage{algorithmic}
\usepackage{graphicx}
\usepackage{textcomp}
\usepackage{xcolor}
\usepackage{txfonts}
\usepackage{listings}
\usepackage{enumitem}
\usepackage{mathtools}
\usepackage{gensymb}
\usepackage{comment}
\usepackage[breaklinks=true]{hyperref}
\usepackage{tkz-euclide} 
\usepackage{listings}
\usepackage{gvv}                                        
\def\inputGnumericTable{}                                 
\usepackage[latin1]{inputenc}                                
\usepackage{color}                                            
\usepackage{array}                                            
\usepackage{longtable}                                       
\usepackage{calc}                                             
\usepackage{multirow}                                         
\usepackage{hhline}                                           
\usepackage{ifthen}                                           
\usepackage{lscape}

\newtheorem{theorem}{Theorem}[section]
\newtheorem{problem}{Problem}
\newtheorem{proposition}{Proposition}[section]
\newtheorem{lemma}{Lemma}[section]
\newtheorem{corollary}[theorem]{Corollary}
\newtheorem{example}{Example}[section]
\newtheorem{definition}[problem]{Definition}
\newcommand{\BEQA}{\begin{eqnarray}}
\newcommand{\EEQA}{\end{eqnarray}}
\newcommand{\define}{\stackrel{\triangle}{=}}
\theoremstyle{remark}
\newtheorem{rem}{Remark}
\begin{document}
\parindent 0px

\bibliographystyle{IEEEtran}
\vspace{3cm}

\title{Assignment\\[1ex]11.9.2 - 11}
\author{EE23BTECH11034 - Prabhat Kukunuri$^{}$% <-this % stops a space
}
\maketitle
\newpage
\bigskip

\renewcommand{\thefigure}{\theenumi}
\renewcommand{\thetable}{\theenumi}
\section*{Question}
Sum of the first p, q and r terms of an A.P. are a, b and c, respectively.

Prove that $\dfrac{a}{p}\brak{q-r}+\dfrac{b}{q}\brak{r-p}+\dfrac{c}{r}\brak{p-q}=0$
\section*{Solution}
\begin{table}[h]
    \centering
    \begin{tabular}{|c|c|c|}
\hline 
   \textbf{Parameter}  &\textbf{Description} &\textbf{Value} \\
\hline
&&\\
$I_r$&Net Intensity of light at $\Delta x =\dfrac{\lambda}{3}$ &$\dfrac{K}{4}$ \\&&\\
\hline
\end{tabular}

    \caption{Variable description}
    \label{tab:11.9.2.11.1}
\end{table}
\begin{align}
    y\brak{n}&=\dfrac{n+1}{2}\brak{2x\brak{0}+nd}u\brak{n}
\end{align}
Using y\brak{n},
\begin{align}
    a&=\dfrac{p}{2}\brak{2x\brak{0}+\brak{p-1}d}\label{eq:2}\\
    b&=\dfrac{q}{2}\brak{2x\brak{0}+\brak{q-1}d}\label{eq:3}\\
    c&=\dfrac{r}{2}\brak{2x\brak{0}+\brak{r-1}d}\label{eq:4}
\end{align}
which can be represented as,
\begin{align}
    &p.x\brak{0}+\dfrac{p\brak{p-1}}{2}.d+a.\brak{-1}=0\\
    &q.x\brak{0}+\dfrac{q\brak{q-1}}{2}.d+b.\brak{-1}=0\\
    &r.x\brak{0}+\dfrac{r\brak{r-1}}{2}.d+c.\brak{-1}=0
\end{align}
resulting in the matrix equation,
\begin{align}
    \myvec{p&\frac{p\brak{p-1}}{2}&a\\q&\frac{q\brak{q-1}}{2}&b\\r&\frac{r\brak{r-1}}{2}&c\\}\vec{x}=0\label{eq:8}
\end{align}
where,
\begin{align}
    \vec{x}=\myvec{x\brak{0}\\d\\-1}
\end{align}
solving the equations \eqref{eq:2},\eqref{eq:3} and \eqref{eq:4} by row reducing the matrix in $\eqref{eq:8}$,
    \begin{align}
    \myvec{
        p&\frac{p\brak{p-1}}{2}&a\\
        q&\frac{q\brak{q-1}}{2}&b\\
        r&\frac{r\brak{r-1}}{2}&c\\
    }
    \xleftrightarrow[R_{1}\leftarrow\frac{R_{1}}{p}, R_{2}\leftarrow\frac{R_{2}}{q}]{R_{3}\leftarrow\frac{R_{3}}{r}} 
    \myvec{
        1&\frac{p-1}{2}&\frac{a}{p}\\
        1&\frac{q-1}{2}&\frac{b}{q}\\
        1&\frac{r-1}{2}&\frac{c}{r}\\
    }\\
   \xleftrightarrow[R_{2}\leftarrow R_{2}-R_{1}]{R_{3}\leftarrow R_{3}-R_{1}} 
    \myvec{
        1&\frac{p-1}{2}&\frac{a}{p}\\
        0&\frac{q-p}{2}&\frac{b}{q}-\frac{a}{p}\\
        0&\frac{r-p}{2}&\frac{c}{r}-\frac{a}{p}\\
    }\\
    \xleftrightarrow{R_2\leftarrow\frac{R_{2}}{\frac{q-p}{2}}}
    \myvec{
        1&\frac{p-1}{2}&\frac{a}{p}\\
        0&1&\brak{\frac{b}{q}-\frac{a}{p}}\frac{2}{q-p}\\
        0&\frac{r-p}{2}&\frac{c}{r}-\frac{a}{p}\\
    }\\
    \xleftrightarrow[R_{1}\leftarrow R_{1}-\frac{p-1}{2}R_{2}]{R_{3}\leftarrow R_{3}-\frac{r-p}{2}R_{2}}
    \myvec{
        1&0&\frac{a}{p}-\frac{\brak{\frac{b}{q}-\frac{a}{p}}\brak{p-1}}{q-p}\\
        0&1&\brak{\frac{b}{q}-\frac{a}{p}}\frac{2}{q-p}\\
        0&0&\brak{\frac{c}{r}-\frac{a}{p}}-\frac{\brak{\frac{b}{q}-\frac{a}{p}}\brak{r-p}}{q-p}\\
    }\\
    \implies
    \myvec{
        1&0&\frac{aq\brak{q-1}-bp\brak{p-1}}{pq\brak{q-p}}\\
        0&1&\brak{\frac{b}{q}-\frac{a}{p}}\frac{2}{q-p}\\
        0&0&\frac{\frac{a}{p}\brak{r-q}+\frac{b}{q}\brak{p-r}+\frac{c}{r}\brak{q-p}}{q-p}\\
    }
\end{align}
After row reduction of matrix we get,
\begin{align}
    x\brak{0}=\brak{\frac{aq\brak{q-1}-bp\brak{p-1}}{pq\brak{q-p}}}\\
    d=\brak{\frac{b}{q}-\frac{a}{p}}\frac{2}{q-p}\\
    \frac{\frac{a}{p}\brak{r-q}+\frac{b}{q}\brak{p-r}+\frac{c}{r}\brak{q-p}}{q-p}=0\\
    \therefore{\frac{a}{p}\brak{q-r}+\frac{b}{q}\brak{r-p}+\frac{c}{r}\brak{p-q}}=0
\end{align}
\begin{align}
    &x \brak{n} \system{Z} X \brak{z}\\
    &X\brak{z}=\frac{aq\brak{q-1}-bp\brak{p-1}}{pq\brak{q-p}\brak{1-z^{-1}}}+\frac{2\brak{\frac{b}{q}-\frac{a}{p}}z^{-1}}{\brak{q-p}\brak{1-z^{-1}}^{2}}\\
    &R.O.C\brak{|z|>1}
\end{align}
\begin{figure}[ht]
    \centering
    \includegraphics[width=\columnwidth]{figs/Figure_1.png}
    \caption{Plot of x(n) $vs$ n}
    \label{fig:11.9.2.11.2}
\end{figure}
\begin{table}[ht]
    \centering
    \def\arraystretch{1.5}
    \begin{tabular}{|c|c|c|c|}
\hline
\textbf{Parameter}&\textbf{Description} &\textbf{subquestion}& \textbf{Value}\\
\hline
     \multirow{4}{*}{$\Delta \theta$} & \multirow{4}{*}{$\theta_1 - \theta_2$} &\brak{a}& 6.4$\pi$ \, radians \\
     \cline{3-4}
     & & \brak{b}& 0.8$\pi$ \, radians \\
     \cline{3-4}
     & &\brak{c}& $\pi$ \, radians \\
     \cline{3-4}
     & & \brak{d} & $\dfrac{3\pi}{2\vphantom{\brak{0.1}}}$ \, radians \\
     \hline
\end{tabular}

    \caption{Verified Values}
    \label{tab:11.9.2.11.3}
\end{table}
\end{document}