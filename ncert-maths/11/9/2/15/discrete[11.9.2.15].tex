\iffalse
\let\negmedspace\undefined
\let\negthickspace\undefined
\documentclass[journal,12pt,twocolumn]{IEEEtran}
\usepackage{xparse}
\usepackage{cite}
\usepackage{amsmath,amssymb,amsfonts,amsthm}
\usepackage{algorithmic}
\usepackage{graphicx}
\usepackage{textcomp}
\usepackage{xcolor}
\usepackage{txfonts}
\usepackage{listings}
\usepackage{enumitem}
\usepackage{mathtools}
\usepackage{gensymb}
\usepackage{comment}
\usepackage[breaklinks=true]{hyperref}
\usepackage{tkz-euclide} 
\usepackage{listings}
\usepackage{gvv}
\def\inputGnumericTable{}                                 
\usepackage[latin1]{inputenc}                                
\usepackage{color}                                            
\usepackage{array}                                            
\usepackage{longtable}                                       
\usepackage{calc}                                             
\usepackage{multirow}                                         
\usepackage{hhline}                                           
\usepackage{ifthen}                                           
\usepackage{lscape}

\newtheorem{theorem}{Theorem}[section]
\newtheorem{problem}{Problem}
\newtheorem{proposition}{Proposition}[section]
\newtheorem{lemma}{Lemma}[section]
\newtheorem{corollary}[theorem]{Corollary}
\newtheorem{example}{Example}[section]
\newtheorem{definition}[problem]{Definition}
\newcommand{\BEQA}{\begin{eqnarray}}
\newcommand{\EEQA}{\end{eqnarray}}
\newcommand{\define}{\stackrel{\triangle}{=}}
\theoremstyle{remark}
\newtheorem{rem}{Remark}
\begin{document}

\bibliographystyle{IEEEtran}
\vspace{3cm}


\title{NCERT DISCRETE 11.9.2.15}
\author{EE23BTECH11046 - Poluri Hemanth$^{*}$}
\maketitle
\textbf{Question:}
If \( \frac{a^n +b^n}{a^{n-1} + b^{n-1}}\)is A.M between $a$ and $b$, then find value of $n$.
\break
\textbf{Solution:}
\fi
\begin{table}[h!]
        \setlength{\arrayrulewidth}{0.2mm}
\setlength{\tabcolsep}{15pt}
\renewcommand{\arraystretch}{1.15}


\begin{table}[ht]
  \centering
  \begin{tabular}{|c|c|c|}
    \hline
    	Symbol & Parameters & value\\
    \hline
	  $u\brak{n}$ & unit step function & 1, if n$\geq$ 0; \\& &0 otherwise \\
    \hline
	  $x\brak{n}$ & general term of the series & $\brak{n+1}\brak{n+3}u\brak{n}$ \\
    \hline 
	 $X\brak{z}$ & Z-transform of $x\brak{n}$ & ? \\
    \hline
  \end{tabular}
  \vspace{0.3cm}
  \caption{Input Parameters}
  \label{tab:24.11.9.1.1}
\end{table}


        \caption{parameters}
\end{table}
\\A.M of two numbers $a$,$b$ is $\frac{a+b}{2}$.
\begin{align}
	x(n)&=x(0)+n\cdot d\cdot u(n)\label{1}
\end{align}
Where,
\begin{align}
	x(1)&=\frac{x(0)^n +x(2)^n}{x(0)^{n-1} + x(2)^{n-1}}\label{r}\\ 
	&=\frac{a+b}{2}\\
	\Rightarrow\frac{x(0)^n +x(2)^n}{x(0)^{n-1} + x(2)^{n-1}}&= \frac{x(0)+x(2)}{2}  \\
    \Rightarrow x(0)^{n-1}(x(0)-x(2))&=x(2)^{n-1}(x(0)-x(2))\label{2}
\end{align}
From \eqref{2}
\begin{align}
        \Rightarrow n
        \begin{cases}
                =1  &\text{if } a\neq b\\
                \in R &\text{if } a=b
        \end{cases}
\end{align}
From \eqref{1}\\
\begin{align}
	d&=x(1)-x(0)\\
	&=\frac{a+b}{2}-a\\
	&=\frac{b-a}{2}\label{d}
\end{align}
Using $Z$ transform.
\begin{align}
	x(n)&\Large\xleftrightarrow{\mathcal{Z}}X(z)\\
	X(z)&=\frac{a}{1-z^{-1}}+\frac{dz^{-1}}{(1-z^{-1})^2}\label{8}\\
	\notag\text{From \eqref{d}}\\
	X(z)&=\frac{a}{1-z^{-1}}+\frac{(b-a)z^{-1}}{2(1-z^{-1})^2}
\end{align}

% \end{document}
