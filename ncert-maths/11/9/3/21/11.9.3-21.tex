\iffalse
\let\negmedspace\undefined
\let\negthickspace\undefined
\documentclass[journal,12pt,onecolumn]{IEEEtran}
\usepackage{cite}
\usepackage{amsmath,amssymb,amsfonts,amsthm}
\usepackage{algorithmic}
\usepackage{graphicx}
\usepackage{textcomp}
\usepackage{xcolor}
\usepackage{txfonts}
\usepackage{listings}
\usepackage{enumitem}
\usepackage{mathtools}
\usepackage{gensymb}
\usepackage{comment}
\usepackage[breaklinks=true]{hyperref}
\usepackage{tkz-euclide} 
\usepackage{listings}
\usepackage{gvv}                                        
\def\inputGnumericTable{}                                 
\usepackage[latin1]{inputenc}                                
\usepackage{color}                                            
\usepackage{array}                                            
\usepackage{longtable}                                       
\usepackage{calc}                                             
\usepackage{multirow}                                         
\usepackage{hhline}                                           
\usepackage{ifthen}                                           
\usepackage{lscape}
\newtheorem{theorem}{Theorem}[section]
\newtheorem{problem}{Problem}
\newtheorem{proposition}{Proposition}[section]
\newtheorem{lemma}{Lemma}[section]
\newtheorem{corollary}[theorem]{Corollary}
\newtheorem{example}{Example}[section]
\newtheorem{definition}[problem]{Definition}
\newcommand{\BEQA}{\begin{eqnarray}}
\newcommand{\EEQA}{\end{eqnarray}}
\newcommand{\define}{\stackrel{\triangle}{=}}
\theoremstyle{remark}
\newtheorem{rem}{Remark}
\begin{document}
\bibliographystyle{IEEEtran}
\vspace{3cm}

\title{NCERT-discrete : 11.9.3 - 21}
\author{EE23BTECH11025 - Anantha Krishnan $^{}$% <-this % stops a space
}
\maketitle
\bigskip

\renewcommand{\thefigure}{\theenumi}
\renewcommand{\thetable}{\theenumi}
\section{question}
Find four numbers forming a geometric progression in which the third term is greater than the first term by 9, and the second term is greater than the $4^{th}$ by 18.\\

\textbf{Solutions :}
\fi
\begin{table}[ht!]
\centering
\begin{tabular}{ |c|c|c| } 
 \hline
Symbols & Description & Values  \\
\hline
 $r$ & Common ratio of the GP & -2\\
 \hline
 $x(n)$ & $(n+1)^{th}$ term of the Sequence & $x(0)r^{n}u(n)$\\
 \hline
 $x(0)$ & First term of the GP & 3\\
\hline
 $x(2)-x(0)$ & First constraint & 9\\
 \hline
 $x(1)-x(3)$& Second constraint & 18\\
 \hline
\end{tabular}
\caption{Parameters, Descriptions, and Values}
\label{table:ee25-tab2}
\end{table}





From the constraints given in \ref{table:ee25-tab2}:
   \begin{align}
x\brak{0}r^2 - 9 &= x\brak{0} \label{eq:ee25-a1}\\
x\brak{0}r + 18 &= x\brak{0}r^3 \label{eq:ee25-a2}\\
\implies x(0)(r^2-1) &= 9 \label{eq:ee25-a3}\\
\implies x (0)r(r^2-1) &=18
\label{eq:ee25-a4}
\end{align}
By dividing $\eqref{eq:ee25-a3}$ and $\eqref{eq:ee25-a4}$ and solving ,we get: 
\begin{align}
    \implies
    x\brak{0} &= 3\\
    \implies
    r &= -2
\end{align}

 Z-Transform for x\brak{n} :
    Using $\eqref{eq:ztrans}$ :
    \begin{align}
    X\brak{z} &= \frac{1}{1+2z^{-1}},\quad \abs {z}>\abs{2} 
    \end{align}
    
    \begin{figure}[!ht]    
    \centering
    \graphicspath{ {ncert-maths/11/9/3/21/figs/} }
\includegraphics[width=\columnwidth]{graph_1}
\caption{ $x\brak{n}$ vs n }
\label{graph:ee25-ag2}
\end{figure}

