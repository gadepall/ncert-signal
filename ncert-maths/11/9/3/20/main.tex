\iffalse
\let\negmedspace\undefined
\let\negthickspace\undefined
\documentclass[journal,12pt,twocolumn]{IEEEtran}
\usepackage{cite}
\usepackage{amsmath,amssymb,amsfonts,amsthm}
\usepackage{algorithmic}
\usepackage{graphicx}
\usepackage{textcomp}
\usepackage{xcolor}
\usepackage{txfonts}
\usepackage{listings}
\usepackage{enumitem}
\usepackage{mathtools}
\usepackage{gensymb}
\usepackage{comment}
\usepackage[breaklinks=true]{hyperref}
\usepackage{tkz-euclide} 
\usepackage{listings}
\usepackage{gvv}                                        
\def\inputGnumericTable{}                                 
\usepackage[latin1]{inputenc}                                
\usepackage{color}                                            
\usepackage{array}                                            
\usepackage{longtable}                                       
\usepackage{calc}                                             
\usepackage{multirow}                                         
\usepackage{hhline}                                           
\usepackage{ifthen}                                           
\usepackage{lscape}
\newtheorem{theorem}{Theorem}[section]
\newtheorem{problem}{Problem}
\newtheorem{proposition}{Proposition}[section]
\newtheorem{lemma}{Lemma}[section]
\newtheorem{corollary}[theorem]{Corollary}
\newtheorem{example}{Example}[section]
\newtheorem{definition}[problem]{Definition}
\newcommand{\BEQA}{\begin{eqnarray}}
\newcommand{\EEQA}{\end{eqnarray}}
\newcommand{\define}{\stackrel{\triangle}{=}}
\theoremstyle{remark}
\newtheorem{rem}{Remark}
\begin{document}

\bibliographystyle{IEEEtran}
\vspace{3cm}

\title{DISCRETE}
\author{EE23BTECH11006 - Ameen Aazam$^{*}$% <-this % stops a space
}
\maketitle
\newpage
\bigskip

\renewcommand{\thefigure}{\theenumi}
\renewcommand{\thetable}{\theenumi}

\vspace{3cm}
\textbf{Question :}
Show that the products of the corresponding terms of the sequences $a, ar, ar^2, \ldots ar^{n-1}$ and $A, AR, AR^2, \ldots AR^{n-1}$ form a G.P., and find the common ratio.

\solution
\fi
\begin{table}[htbp]
    \centering
    \begin{tabular}{|c|c|c|} \hline
      \textbf{Input Parameters} & \textbf{Values} & \textbf{Description} \\ \hline
      $a$ & & First term of $1^{st}$ G.P. \\ \hline
      $r$ & & Common ratio of $1^{st}$ G.P. \\ \hline
      $x_1\brak{n}$ & $x_1\brak{n}=ar^nu\brak{n}$& General term of $1^{st}$ G.P. \\ \hline
      $X_1\brak{z}$ & & z-Transform of $1^{st}$ G.P. \\ \hline
      $A$ & & First term of $2^{nd}$ G.P. \\ \hline
      $R$ & & Common ratio of $2^{nd}$ G.P. \\ \hline
      $x_2\brak{n}$ & $x_1\brak{n}=AR^nu\brak{n}$& General term of $2^{nd}$ G.P. \\ \hline
      $X_2\brak{z}$ & & z-Transform of $2^{nd}$ G.P. \\ \hline
    \end{tabular}
    \vspace{3pt}
    \caption{Parameters}
\end{table}

General term of the $n^{th}$ term of the $1^{st}$ G.P.,
\begin{align}
    x_1\brak{n}=ar^nu\brak{n}
\end{align}
Now the sequence in the z domain would be,
\begin{align}
    X_1\brak{z}&=\sum\limits_{n=-\infty}^{\infty}ar^nu\brak{n}z^{-n} \\
    &=\frac{a}{1-rz^{-1}},\hspace{0.5cm}\abs{z}>\abs{r}
\end{align}
General for the $2^{nd}$ G.P. is given as,
\begin{align}
    x_2\brak{n}=AR^nu\brak{n}
\end{align}
And the z-Transform,
\begin{align}
    X_2\brak{z}&=\sum\limits_{n=-\infty}^{\infty}AR^nu\brak{n}z^{-n} \\
    &=\frac{A}{1-Rz^{-1}},\hspace{0.5cm}\abs{z}>\abs{R}
\end{align}
Now taking the product will result in a sequence as,
\begin{align}
    y\brak{n}&=x_1\brak{n}x_2\brak{n} \\
    &=aA\brak{rR}^nu\brak{n} \label{eq:11.9.3.20.1}
\end{align}
z-Transform of the resulting sequence,
\begin{align}
    Y\brak{z}&=\sum\limits_{n=-\infty}^{\infty}aA\brak{rR}^nu\brak{n}z^{-n} \\
    &=\frac{aA}{1-rRz^{-1}},\hspace{0.5cm}\abs{z}>\abs{rR}
\end{align}
So, form \ref{eq:11.9.3.20.1}, taking the ratio of two consecutive terms,
\begin{align}
    \frac{y\brak{n}}{y\brak{n-1}}&=\frac{aA\brak{rR}^nu\brak{n}}{aA\brak{rR}^{n-1}u\brak{n-1}} \\
    &=rR
\end{align}
As we can see the ratio of any two consecutive terms, $rR$, is a constant. Which means the product of the corresponding terms of the two G.P.s results in another G.P.
And the common ratio is $rR$.
%\end{document}

