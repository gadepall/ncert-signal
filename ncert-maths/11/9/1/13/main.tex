\iffalse
\let\negmedspace\undefined
\let\negthickspace\undefined
\documentclass[journal,12pt,twocolumn]{IEEEtran}
\usepackage{cite}
\usepackage{amsmath,amssymb,amsfonts,amsthm}
\usepackage{algorithmic}
\usepackage{graphicx}
\usepackage{textcomp}
\usepackage{xcolor}
\usepackage{txfonts}
\usepackage{listings}
\usepackage{enumitem}
\usepackage{mathtools}
\usepackage{float}
\usepackage{gensymb}
\usepackage{comment}
\usepackage[breaklinks=true]{hyperref}
\usepackage{tkz-euclide} 
\usepackage{listings}
\usepackage{gvv}                                        
\def\inputGnumericTable{}                                 
\usepackage[latin1]{inputenc}                                
\usepackage{color}                                            
\usepackage{array}                                            
\usepackage{longtable}                                       
\usepackage{calc}                                             
\usepackage{multirow}                                         
\usepackage{hhline}                                           
\usepackage{ifthen}                                           
\usepackage{lscape}
\usepackage{amsmath}
\newtheorem{theorem}{Theorem}[section]
\newtheorem{problem}{Problem}
\newtheorem{proposition}{Proposition}[section]
\newtheorem{lemma}{Lemma}[section]
\newtheorem{corollary}[theorem]{Corollary}
\newtheorem{example}{Example}[section]
\newtheorem{definition}[problem]{Definition}
\newcommand{\BEQA}{\begin{eqnarray}}
\newcommand{\EEQA}{\end{eqnarray}}
\newcommand{\define}{\stackrel{\triangle}{=}}
\theoremstyle{remark}
\newtheorem{rem}{Remark}
\begin{document}
\bibliographystyle{IEEEtran}
\title{NCERT 11.9.1.13Q}
\author{EE23BTECH11015 - DHANUSH V NAYAK$^{*}$% <-this % stops a space
}
\maketitle
\newpage
\bigskip
\renewcommand{\thefigure}{\arabic{figure}}
\renewcommand{\thetable}{\theenumi}
\textbf{Question:} Write the first five terms of each of the sequences in Exercises 11 to 13 and obtain the corresponding series:\\
$a_1=a_2=2,$\hspace{5pt} $a_n=a_{n-1} -1,$\hspace{5pt} $n>2$\\
\solution
\fi
\begin{table}[H]
    \centering
    \renewcommand\thetable{1}
    \setlength{\extrarowheight}{9pt}
    \resizebox{0.5\textwidth}{!}{
    \begin{tabular}{|c|c|c|}
    \hline
    \textbf{Parameter} & \textbf{Description} & \textbf{Value} \\ \hline
    $x\brak{0}$ & First term &2 \\ \hline
    $x\brak{1}$ & Second term &2 \\ \hline
    ROC & Region of convergence & $\left\{ z : \left|\sum_{n=-\infty}^{\infty} x(n)z^{-n}\right| < \infty \vphantom {\brak{{0.3pt}}}\right\}$ \\ \hline 
    $x(n)$ & General term & $x(n) = 
    \begin{cases}
        ? & ; n \geq 0 \\
        0 & ; n < 0 \\
    \end{cases}$ \\ \hline
    \end{tabular}}
    \caption{Parameter Table}
    \label{tab:11.9.1.13}
    \end{table}
\begin{align}
    x\brak{n} - x\brak{n-1} &= 2u\brak{n}-2u\brak{n-1}-u\brak{n-2}\label{eq:11.9.1.13.1}\\
X\brak{z}- z^{-1}X\brak{z} &= \frac{2}{\brak{1-z^{-1}}} - \frac{z^{-2}}{\brak{1-z^{-1}}}- \frac{2z^{-1}}{\brak{1-z^{-1}}}\\
    X\brak{z} &= \frac{2-2z^{-1}-z^{-2}}{\brak{1-z^{-1}}^2}  ,   \abs{z} >1
\end{align}
Using partial fractions
\begin{align}
    X\brak{z} &= \frac{2z^{-1}}{\brak{1-z^{-1}}} - \frac{z^{-2}}{\brak{1-z^{-1}}^2} + 2\label{eq:11.9.1.13.5}
\end{align}
Taking inverse $Z$-transform by result of equation \eqref{eq:11.9.5.26.9} in equation \eqref{eq:11.9.1.13.5}:
\begin{align}
    x\brak{n} &= 2u\brak{n}+\brak{1-n}u\brak{n-1}\label{eq:11.9.1.13.6}
\end{align}
\begin{figure}[H]
    \centering
    \includegraphics[width=1\columnwidth]{ncert-maths/11/9/1/13/figs/Theory_vs_Simulation.png}
    \caption{Comparison of Theory and Simulated Values}
    \label{fig:11.9.1.13.1}
\end{figure}
From the figure\figref{fig:11.9.1.13.1} we can see that the theoretical and simulated values overlap. 
%\end{document}
