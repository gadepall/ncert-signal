% \iffalse
\let\negmedspace\undefined
\let\negthickspace\undefined
\documentclass[journal,12pt,twocolumn]{IEEEtran}
\usepackage{cite}
\usepackage{amsmath,amssymb,amsfonts,amsthm}
\usepackage{algorithmic}
\usepackage{graphicx}
\usepackage{textcomp}
\usepackage{xcolor}
\usepackage{txfonts}
\usepackage{listings}
\usepackage{enumitem}
\usepackage{mathtools}
\usepackage{gensymb}
\usepackage{comment}
\usepackage[breaklinks=true]{hyperref}
\usepackage{tkz-euclide} 
\usepackage{listings}
\usepackage{gvv}                                        
\def\inputGnumericTable{}                                 
\usepackage[latin1]{inputenc}                                
\usepackage{color}                                            
\usepackage{array}                                            
\usepackage{longtable}                                       
\usepackage{calc}                                             
\usepackage{multirow}                                         
\usepackage{hhline}                                           
\usepackage{ifthen}                                           
\usepackage{lscape}
\usepackage{siunitx}
\usepackage{flushend}
\usepackage[siunitx]{circuitikz}
\usepackage{caption}

\newtheorem{theorem}{Theorem}[section]
\newtheorem{problem}{Problem}
\newtheorem{proposition}{Proposition}[section]
\newtheorem{lemma}{Lemma}[section]
\newtheorem{corollary}[theorem]{Corollary}
\newtheorem{example}{Example}[section]
\newtheorem{definition}[problem]{Definition}
\newcommand{\BEQA}{\begin{eqnarray}}
	\newcommand{\EEQA}{\end{eqnarray}}
\newcommand{\define}{\stackrel{\triangle}{=}}
\theoremstyle{remark}
\newtheorem{rem}{Remark}
\begin{document}
	
	\bibliographystyle{IEEEtran}
	\vspace{3cm}
	
	\title{NCERT 9.1 Q.14}
	\author{EE23BTECH11203 - Adarsh A$^{*}$% <-this % stops a space
	}
	\maketitle
	%\newpage
	\bigskip
	
	\renewcommand{\thefigure}{\theenumi}
	\renewcommand{\thetable}{\theenumi}
	
	
	\vspace{0.2cm}
	\linespread{1.1}
	
	
	\textbf{\fontsize{13.5}{20}\selectfont Question : }
	\fontsize{13.5}{20}\selectfont The Fibonacci sequence is defined by $1 = a_1 = a_2$ and \par $a_n = a_{n-1} + a_{n-2}$ , $n$ $\textgreater$ 2
	
	\vspace{0.2cm}
	Find $\dfrac{a_{n + 1}}{a_n}$, for $n$ = 1, 2, 3, 4, 5
	
	\vspace{0.3cm}
	\textbf{\fontsize{13.5}{20}\selectfont Answer : }
	\vspace{0.3cm}
	\fontsize{13.5}{20}\selectfont 
	
	\begin{tabular}{|m{2cm}|m{2cm}|m{2cm}|}
    \hline
    \textbf{Symbol} & \textbf{Value} & \textbf{Description}\\ [1ex]
    \hline
        $x$ & $x\brak{0}r^4$ & $x\brak{4}$ \\ [1ex]
    \hline
        $y$ & $x\brak{0}r^{10}$ & $x\brak{10}$\\ [1ex]
    \hline
        $z$ & $x\brak{0}r^{16}$ & $x\brak{16}$\\ [1ex]
    \hline
        $r$ & ? & $\frac{x\brak{n}}{x\brak{n-1}}$\\[1ex]
    \hline \vspace{0.1cm}
        $x\brak{0}$ & ? & First term \\[1ex]
    \hline
        $x\brak{n}$ & $x\brak{0}r^nu\brak{n}$ & General Term \\ [1ex]
    \hline
    \end{tabular}

	
	\vspace{-0.2cm}
	
	\fontsize{13.5}{20}\selectfont
	Here, $a_1 = 1, a_2 = 1$
	
	$a_n = a_{n-1} + a_{n-2}$ , $n$ $\textgreater$ 2 \hfill (1)
	
	\vspace{0.2cm}
	
	%Re-writing this equation,
	
	%\vspace{0.1cm}
	
	%$x(n)$ = $x(n-1)$ + $x(n-2)$ + $u(-n)$, $n$ $\geq$ 0 \hfill (2)
	
	%\vspace{0.2cm}
	
	Applying $z$ transform,
	
	\vspace{-0.8cm}
	
	\begin{align}
		X\brak z	&= z^{-1} X(z) + z^{-2} X(z) + z^{-0} \tag{2} \\[6pt]
		&= \dfrac{1}{1 - z^{-1} - z^{-2}} \tag{3} \\[10pt]
		&= \dfrac{1}{(1 - \alpha z^{-1})(1 - \beta z^{-1})} \hspace{0.2cm}, \hspace{0.3cm}\lvert \hspace{0.1cm} z \hspace{0.1cm}\rvert \hspace{0.1cm} \textgreater \hspace{0.1cm} \lvert \hspace{0.1cm} \alpha \hspace{0.1cm} \rvert  \tag{4}				
	\end{align}
	
	\vspace{0.3cm}
	
	Where, $\alpha$ = $\dfrac{1 +\sqrt{5}}{2}$ and $\beta$ = $\dfrac{1 -\sqrt{5}}{2}$ 
	
	\vspace{0.4cm}
	
	Using partial fractions,
	
	\begin{align}
		X(z) &= \dfrac{\alpha}{(\alpha - \beta)} \dfrac{1}{(1 - \alpha z^{-1})} - \dfrac{\beta}{(\alpha - \beta)} \dfrac{1}{(1 - \beta z^{-1})} \tag{5}
	\end{align}
	
	$a^n u(n)$
	$\xleftarrow[]{\hspace{0.4cm}{\mathcal{Z}}\hspace{0.1cm}}\xrightarrow[]{}$
	$\dfrac{1}{1 - a z^{-1}}$ \hspace{0.2cm} $\lvert \hspace{0.1cm} z \hspace{0.1cm}\rvert \hspace{0.1cm} \textgreater \hspace{0.1cm} \lvert \hspace{0.1cm} a \hspace{0.1cm} \rvert$
	
	\vspace{0.4cm}
	
	Substituting this result,
	
	\vspace{-0.5cm}
	
	\begin{align}
		x(n) &= \dfrac{\alpha}{(\alpha - \beta)} (\alpha^n u(n)) - \dfrac{\beta}{(\alpha - \beta)} (\beta^n u(n)) \tag{6}
	\end{align}
	
	$x(n)$ = $\dfrac{\alpha^{n+1} - \beta^{n+1} }{\alpha - \beta}$ $u(n)$ \hfill (7)
	
	\vspace{0.6cm}
	
	$x(n)$ = $\dfrac{(1 + \sqrt{5})^{n+1} - (1 - \sqrt{5})^{n+1} }{2^{n+1} \sqrt{5}}$ $u(n)$\hfill (8)
	
	\vspace{0.6cm}
	
	$y(n)$ = $\dfrac{x(n+1)}{x(n)}$ \hfill (9)
	
	\vspace{0.6cm}
	
	$y(n)$ = $\dfrac{1}{2}$ $\Bigg[$ $\dfrac{(1 + \sqrt{5})^{n+2} - (1 - \sqrt{5})^{n+2}}{(1 + \sqrt{5})^{n+1} - (1 - \sqrt{5})^{n+1}}$ $\Bigg]$ \hfill (10)
	
	\newpage
	
	\begin{figure}[htbp]
		\centering
		\includegraphics[width=0.6\textwidth]{figures/fig1.png}
		\caption*{\hspace{2cm} (a) Plot of $x(n)$ $vs$ $n$}
	\end{figure}
	
	\begin{figure}[htbp]
		\centering
		\includegraphics[width=0.6\textwidth]{figures/fig2.png}
		\caption*{\hspace{2cm} (b) Plot of $y(n)$ $vs$ $n$}
		
	\end{figure}
	
	
\end{document}