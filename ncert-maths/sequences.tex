\begin{enumerate}[label=\thesection.\arabic*,ref=\thesection.\theenumi]
\item Find the number of terms in each of the following APs. 
\begin{enumerate}
    \item 7, 13, 19, ... 205

    \item 18, 15$\frac{1}{2}$, 13, ... -47
\end{enumerate}
\solution
\iffalse
\let\negmedspace\undefined
\let\negthickspace\undefined
\documentclass[journal,12pt,twocolumn]{IEEEtran}
\usepackage{cite}
\usepackage{amsmath,amssymb,amsfonts,amsthm}
\usepackage{algorithmic}
\usepackage{graphicx}
\usepackage{textcomp}
\usepackage{xcolor}
\usepackage{txfonts}
\usepackage{listings}
\usepackage{enumitem}
\usepackage{mathtools}
\usepackage{gensymb}
\usepackage{comment}
\usepackage[breaklinks=true]{hyperref}
\usepackage{tkz-euclide} 
\usepackage{listings}
\usepackage{gvv}                                        
\def\inputGnumericTable{}                                 
\usepackage[latin1]{inputenc}                                
\usepackage{color}                                            
\usepackage{array}                                            
\usepackage{longtable}                                       
\usepackage{calc}                                             
\usepackage{multirow}                                         
\usepackage{hhline}                                           
\usepackage{ifthen}                                           
\usepackage{lscape}
\usepackage{placeins}
\usepackage{xparse}


\newtheorem{theorem}{Theorem}[section]
\newtheorem{problem}{Problem}
\newtheorem{proposition}{Proposition}[section]
\newtheorem{lemma}{Lemma}[section]
\newtheorem{corollary}[theorem]{Corollary}
\newtheorem{example}{Example}[section]
\newtheorem{definition}[problem]{Definition}
\newcommand{\BEQA}{\begin{eqnarray}}
\newcommand{\EEQA}{\end{eqnarray}}
\newcommand{\define}{\stackrel{\triangle}{=}}
\theoremstyle{remark}
\newtheorem{rem}{Remark}



\begin{document}

\bibliographystyle{IEEEtran}
\vspace{3cm}

\Large\title{NCERT Question 10.5.2.5}
\large\author{EE23BTECH11032 - Kaustubh Parag Khachane $^{*}$% <-this % stops a space
}
\maketitle
\newpage
\bigskip

\renewcommand{\thefigure}{\theenumi}
\renewcommand{\thetable}{\theenumi}
\large\textbf{Question 10.5.2.5} : \normalsize Find the number of terms in each of the following APs. Then express each term as x\brak{n} and find the z transform, ROC and plot the graph for x\brak{n}: 
\begin{enumerate}
    \item 7, 13, 19, ... 205

    \item 18, 15$\frac{1}{2}$, 13, ... -47
\end{enumerate}


\solution
\fi
\begin{table}[ht] 
\centering
\setlength{\extrarowheight}{8pt}
\begin{tabular}{|c|l|l|} 
 \hline
  \textbf{Parameter} & \textbf{Used to denote } & \textbf{Values} \\ 
 \hline
 $x_{i}$\brak{n} & $n^{th}$ term of $i^{th}$ series $\brak{i =\brak{1,2}}$  & $\brak{x_{i}\brak{0} + nd_{i}}u\brak{n}$ \\
 \hline
$x_{i}$\brak{0} & First term of $i^{th} $ AP &\multicolumn{1}{|p{1.5cm}|}{\centering $x_{1}\brak{0} = 7$ \\ $x_{2}\brak{0} = 18$ }\\
 \hline
  $d_{i}$ & Commmon difference of $i^{th}$ AP&\multicolumn{1}{|p{1.5cm}|}{\centering $d_{1} = 6 $ \\ $d_{2} = -2.5$}\\
 \hline

\end{tabular}
 \vspace{4mm}
 \caption{Parameter Table}
 \label{tab:table0}
\end{table}

The number of terms in the AP x\brak{n} is given by: 
\begin{align}  \label{eq:eq12}
    \frac{x\brak{n} - x\brak{0}}{d} + 1
\end{align}
\begin{align}
    &X_i(z) = \frac{x_i\brak{0}}{1 - z^{-1}} + d_i\frac{z^{-1}}{\brak{1-z^{-1}}^2} \text{ , for i=1,2} \label{eq:eq3}\\
    &\text{ROC : $\abs{z} > 1$ as it is an AP}   
\end{align}
\begin{enumerate}
    \item 
\begin{align}
x_{1}\brak{n} &= \brak{7 + \brak{n}6}u\brak{n}
\end{align}
Using the values in \tabref{tab:table0} and equation \eqref{eq:eq12},
\begin{align}
    k_1 = \frac{205 - 7}{6} + 1 = 34
\end{align}

Using the values in \tabref{tab:table0} and equation \eqref{eq:eq3} :
\begin{align}
 X_1\brak{z} = \frac{7 - z^{-1}}{\brak{1-z^{-1}}^2}
\end{align}

ROC is $\abs{z} > 1$
 
   \item
   
\begin{align}
    x_{2}\brak{n} &= \brak{18 + n\brak{-2.5}u\brak{n}}
\end{align}

Using the values in \tabref{tab:table0} and equation \eqref{eq:eq12},
\begin{align}
    k_2 = \frac{-47 - 18}{-2.5} + 1 = 27
\end{align}

Using the values in \tabref{tab:table0} and equation \eqref{eq:eq3} :
\begin{align} 
 X_2\brak{z} = \frac{18 - \brak{20.5}z^{-1}}{\brak{1 - z^{-1}}^2}
\end{align}

ROC is $\abs{z} > 1$.

\begin{figure}[!ht]
\centering
\begin{center}
\includegraphics[width=\columnwidth]{ncert-maths/10/5/2/5/figs/Figure_1}
\caption{Plot of $x_1\brak{n}$}
\end{center}
\end{figure}

\begin{figure}[!ht]
\centering
\begin{center}
\includegraphics[width=\columnwidth]{ncert-maths/10/5/2/5/figs/Figure_2}
\caption{Plot of $x_2\brak{n}$}
\end{center}
\end{figure}

\end{enumerate}
%\end{document}



\item For what value of $ n$, are the $ nth$ terms of two A.Ps: 63, 65, 67,\dots and 3, 10, 17,\dots equal?
\solution
\iffalse
\let\negmedspace\undefined
\let\negthickspace\undefined
\documentclass[journal,12pt,twocolumn]{IEEEtran}
\usepackage{cite}
\usepackage{amsmath,amssymb,amsfonts,amsthm}
\usepackage{algorithmic}
\usepackage{graphicx}
\usepackage{textcomp}
\usepackage{xcolor}
\usepackage{txfonts}
\usepackage{listings}
\usepackage{enumitem}
\usepackage{mathtools}
\usepackage{gensymb}
\usepackage{comment}
\usepackage[breaklinks=true]{hyperref}
\usepackage{tkz-euclide}
\usepackage{listings}
\usepackage{gvv}
\def\inputGnumericTable{}
\usepackage[latin1]{inputenc}
\usepackage{color}
\usepackage{array}
\usepackage{longtable}
\usepackage{calc}
\usepackage{multirow}
\usepackage{hhline}
\usepackage{ifthen}
\usepackage{lscape}

\newtheorem{theorem}{Theorem}[section]
\newtheorem{problem}{Problem}
\newtheorem{proposition}{Proposition}[section]
\newtheorem{lemma}{Lemma}[section]
\newtheorem{corollary}[theorem]{Corollary}
\newtheorem{example}{Example}[section]
\newtheorem{definition}[problem]{Definition}
\newcommand{\BEQA}{\begin{eqnarray}}
\newcommand{\EEQA}{\end{eqnarray}}
\newcommand{\define}{\stackrel{\triangle}{=}}
\theoremstyle{remark}
\newtheorem{rem}{Remark}
\begin{document}

\bibliographystyle{IEEEtran}
\vspace{3cm}

\title{NCERT Discrete 10.5.2 -15}
\author{EE23BTECH11057 - Shakunaveti Sai Sri Ram Varun$^{}$% &lt;-this % stops a space
}
\maketitle
\newpage
\bigskip

\vspace{2cm}
\textbf{Question: }
For what value of $ n$, are the $ nth$ terms of two A.Ps: 63, 65, 67,\dots and 3, 10, 17,\dots equal?\\
\vspace{0.5cm}
\textbf{Solution}:
\fi

\begin{table}[htbp] 
\centering
\begin{tabular}{|c|c|c|c|}
    \hline
    \textbf{Parameter} & \textbf{Sub-question} & \textbf{Description} & \textbf{Value} \\
    \hline
    \multirow{2}{*}{$x_i\brak{0}$} & $x_1\brak{0}$ & $1^{st}$ term of $1^{st}$ A.P. & 63 \\
    \cline{2-4}
    & $x_2\brak{0}$ & $1^{st}$ term of $2^{nd}$ A.P. & \phantom{0}3 \\
    \hline
    \multirow{2}{*}{$d_i$} & $d_1$ & Common difference of $1^{st}$ A.P. & \phantom{0}2 \\
    \cline{2-4}
    & $d_2$ & Common difference of $2^{nd}$ A.P. & \phantom{0}7 \\
    \hline
\end{tabular}

\caption{input values}
\label{tab: table10.5.2.15}
\end{table}
\begin{align}
x_i\brak{n} &= x\brak{0}u\brak{n} + dnu\brak{n}\\
X\brak{z} &= \frac{x\brak{0}}{1-z^{-1}} + \frac{dz^{-1}}{\brak{1-z^{-1}}^{2}} \quad |z|>1
\end{align}
\begin{enumerate}
\item
\begin{align}
x_1\brak{n} &= 63u\brak{n} + 2nu\brak{n} \\
%To find $ X_1\brak{z}$:
X_1\brak{z} &= \frac{63}{1-z^{-1}} + \frac{2z^{-1}}{\brak{1-z^{-1}}^{2}}  \quad |z|>1
\end{align}
\item
\begin{align}
x_2\brak{n} &= 3u\brak{n} + 7nu\brak{n}\\ 
%To find $ X_2\brak{z}$ :\\
X_2\brak{z} &= \frac{3}{1-z^{-1}} + \frac{7z^{-1}}{\brak{1-z^{-1}}^{2}} \quad |z|>1
\end{align}
\item

given,
\begin{align}
 x_1\brak{n} &= x_2\brak{n}\\
\therefore 63 + 2n &= 7n+3\\
\implies n &=12
\end{align}
\begin{figure}[h!]
    \includegraphics[width = \columnwidth]{ncert-maths/10/5/2/15/figs/Figure_1.png}
    \caption{Graphs of $ x_1\brak{n}$ and $ x_2\brak{n}$ and both are equal at $ n=12$}
    \label{fig: fig10.5.2.15}
\end{figure}
\end{enumerate}



\item Two APs have the same common difference.The difference between their $100${th} terms is 100,what is the difference between their $1000${th} terms?

\solution
\iffalse
\let\negmedspace\undefined
\let\negthickspace\undefined
\documentclass[journal,12pt,onecolumn]{IEEEtran}
\usepackage{cite}
\usepackage{amsmath,amssymb,amsfonts,amsthm}
\usepackage{algorithmic}
\usepackage{graphicx}
\usepackage{textcomp}
\usepackage{xcolor}
\usepackage{txfonts}
\usepackage{listings}
\usepackage{enumitem}
\usepackage{mathtools}
\usepackage{gensymb}
\usepackage{comment}
\usepackage[breaklinks=true]{hyperref}
\usepackage{tkz-euclide} 
\usepackage{listings}
\usepackage{gvv}                                        
\def\inputGnumericTable{}                                 
\usepackage[latin1]{inputenc}                                
\usepackage{color}                                            
\usepackage{array}                                            
\usepackage{longtable}                                       
\usepackage{calc}                                             
\usepackage{multirow}                                         
\usepackage{hhline}                                           
\usepackage{ifthen}                                           
\usepackage{lscape}
\newtheorem{theorem}{Theorem}[section]
\newtheorem{problem}{Problem}
\newtheorem{proposition}{Proposition}[section]
\newtheorem{lemma}{Lemma}[section]
\newtheorem{corollary}[theorem]{Corollary}
\newtheorem{example}{Example}[section]
\newtheorem{definition}[problem]{Definition}
\newcommand{\BEQA}{\begin{eqnarray}}
\newcommand{\EEQA}{\end{eqnarray}}
\newcommand{\define}{\stackrel{\triangle}{=}}
\theoremstyle{remark}
\newtheorem{rem}{Remark}
\begin{document}
\bibliographystyle{IEEEtran}
\vspace{3cm}
\title{NCERT 11.9.2 16Q}
\author{EE23BTECH11021 - GANNE GOPI CHANDU$^{*}$% <-this % stops a space
}
\maketitle
\bigskip
\renewcommand{\thefigure}{\theenumi}
\renewcommand{\thetable}{\theenumi}
\bibliographystyle{IEEEtran}
\textbf{Question}\\
Between 1 and 31, m numbers have been inserted in such a way that the resulting sequence is an A.P. and 
the ratio of 7 th and (m - 1) th numbers is 5:9. Find the value of m.\\
\textbf{Solution}\\
\fi
\begin{table}[!h]
\begin{center}
\renewcommand\thetable{1}
\begin{tabular}{ |c|c|c| } 
  \hline
    Symbol & Value & description \\ 
  \hline
  $x(0)$ & $1$ & First term of A.P  \\ 
  \hline
  $x(n)$ & $31$ & $\brak{n+1}\text{th}$ term \\
  \hline
  $\frac{x\brak{7}}{x\brak{m-1}}$ & $\frac{5}{9}$ & ratio of $7$ th  and $(m-1)$ th numbers\\ 
  \hline
  $n$ & $m+2$ & number of terms \\
  \hline
\end{tabular}
\end{center}
\caption{}
\end{table}\\
The last term is
\begin{align}
x(n)&=x(0)+\brak{n}d\\
\implies31 &= 1 + \brak{m + 1}d \\
\implies30 &= \brak{m + 1}d \\
\implies\frac{30}{m + 1} &= d \label{eq11.9.2.4}
\end{align}
Now $7$th and $\brak{m-1}$th terms
\begin{align}
x\brak{7} &= x(0) + 7d\label{eq11.9.2.5}\\
x\brak{m-1} &= x(0) + \brak{m-1}d\label{eq11.9.2.6}
\end{align}
From  equations \eqref{eq11.9.2.5} and \eqref{eq11.9.2.6}\\
\begin{align}
   \frac{x(0) + 7d}{x(0) + \brak{m-1}d} &= \frac{5}{9} \label{eq11.9.2.7}
\end{align}
Substituting  \eqref{eq11.9.2.4} in \eqref{eq11.9.2.7}\\
\begin{align}
\implies \frac{1+7\brak{{\frac{30}{m+1}}}}{1+\brak{{m-1}}\brak{\frac{30}{m+1}}} &= \frac{5}{9} \\
\implies \frac{m+1+210}{m+1+30m-30} &= \frac{5}{9}\\
\implies \frac{m+181}{31m-29} &= \frac{5}{9}\\
\implies 9m+1899 &=155m-145\\
\implies 155m-9m &=1899+145\\
\implies 146m &=2044\\
\implies m &=14
\end{align}
Therefore, $m = 14$ .\\
 \text{General term of AP is} \\
\begin{align}
    x\brak{n}&=\brak{2n+1}u(n)\\
    x\brak{n}&=\brak{2n}u\brak{n}+u\brak{n}
\end{align}
\begin{figure}
    \centering
    \includegraphics[width=1.0\linewidth]{ncert-maths/11/9/2/16/figs/test.png}
    \caption{Plot of x(n) vs n}
    \label{fig:11.9.2.1}
\end{figure}\\
The Z-Transform is\\
\begin{align}
    X\brak{z}&=2\brak{\dfrac{z}{\brak{z-1}^{2}}}+U\brak{z}\\
    &=\dfrac{2z}{\brak{z-1}^{2}}+\dfrac{1}{1-z^{-1}}\\
    X\brak{z}&=\dfrac{z^2+z}{\brak{z-1}^{2}} \quad{|z|>1}
\end{align}


\item Check whether -150 is a term of the AP: 11,8,5,2,....

 \solution
 \let\negmedspace\undefined
\let\negthickspace\undefined
\documentclass[journal,12pt,onecolumn]{IEEEtran}
\usepackage{cite}
\usepackage{amsmath,amssymb,amsfonts,amsthm}
\usepackage{algorithmic}
\usepackage{graphicx}
\usepackage{textcomp}
\usepackage{xcolor}
\usepackage{txfonts}
\usepackage{listings}
\usepackage{enumitem}
\usepackage{mathtools}
\usepackage{gensymb}
\usepackage{comment}
\usepackage[breaklinks=true]{hyperref}
\usepackage{tkz-euclide} % loads  TikZ and tkz-base
\usepackage{listings}
\usepackage[latin1]{inputenc}                                
\usepackage{color}                                            
\usepackage{array}                                            
\usepackage{longtable}                                       
\usepackage{calc}                                             
\usepackage{multirow}                                         
\usepackage{hhline}                                           
\usepackage{ifthen}                                           
\usepackage{lscape}
\usepackage{caption}


\newtheorem{theorem}{Theorem}[section]
\newtheorem{problem}{Problem}
\newtheorem{proposition}{Proposition}[section]
\newtheorem{lemma}{Lemma}[section]
\newtheorem{corollary}[theorem]{Corollary}
\newtheorem{example}{Example}[section]
\newtheorem{definition}[problem]{Definition}
%\newtheorem{thm}{Theorem}[section] 
%\newtheorem{defn}[thm]{Definition}
%\newtheorem{algorithm}{Algorithm}[section]
%\newtheorem{cor}{Corollary}
\newcommand{\BEQA}{\begin{eqnarray}}
\newcommand{\EEQA}{\end{eqnarray}}
\newcommand{\define}{\stackrel{\triangle}{=}}
\theoremstyle{remark}
\newtheorem{rem}{Remark}
%\bibliographystyle{ieeetr}

\begin{document}

%
\providecommand{\pr}[1]{\ensuremath{\Pr\left(#1\right)}}
\providecommand{\prt}[2]{\ensuremath{p_{#1}^{\left(#2\right)} }}        % own macro for this question
\providecommand{\qfunc}[1]{\ensuremath{Q\left(#1\right)}}
\providecommand{\sbrak}[1]{\ensuremath{{}\left[#1\right]}}
\providecommand{\lsbrak}[1]{\ensuremath{{}\left[#1\right.}}
\providecommand{\rsbrak}[1]{\ensuremath{{}\left.#1\right]}}
\providecommand{\brak}[1]{\ensuremath{\left(#1\right)}}
\providecommand{\lbrak}[1]{\ensuremath{\left(#1\right.}}
\providecommand{\rbrak}[1]{\ensuremath{\left.#1\right)}}
\providecommand{\cbrak}[1]{\ensuremath{\left\{#1\right\}}}
\providecommand{\lcbrak}[1]{\ensuremath{\left\{#1\right.}}
\providecommand{\rcbrak}[1]{\ensuremath{\left.#1\right\}}}
\newcommand{\sgn}{\mathop{\mathrm{sgn}}}
\providecommand{\abs}[1]{\left\vert#1\right\vert}
\providecommand{\res}[1]{\Res\displaylimits_{#1}} 
\providecommand{\norm}[1]{\left\lVert#1\right\rVert}
%\providecommand{\norm}[1]{\lVert#1\rVert}
\providecommand{\mtx}[1]{\mathbf{#1}}
\providecommand{\mean}[1]{E\left[ #1 \right]}
\providecommand{\cond}[2]{#1\middle|#2}
\providecommand{\fourier}{\overset{\mathcal{F}}{ \rightleftharpoons}}
\newenvironment{amatrix}[1]{%
  \left(\begin{array}{@{}*{#1}{c}|c@{}}
}{%
  \end{array}\right)
}
%\providecommand{\hilbert}{\overset{\mathcal{H}}{ \rightleftharpoons}}
%\providecommand{\system}{\overset{\mathcal{H}}{ \longleftrightarrow}}
        %\newcommand{\solution}[2]{\textbf{Solution:}{#1}}
\newcommand{\solution}{\noindent \textbf{Solution: }}
\newcommand{\cosec}{\,\text{cosec}\,}
\providecommand{\dec}[2]{\ensuremath{\overset{#1}{\underset{#2}{\gtrless}}}}
\newcommand{\myvec}[1]{\ensuremath{\begin{pmatrix}#1\end{pmatrix}}}
\newcommand{\mydet}[1]{\ensuremath{\begin{vmatrix}#1\end{vmatrix}}}
\newcommand{\myaugvec}[2]{\ensuremath{\begin{amatrix}{#1}#2\end{amatrix}}}
\providecommand{\rank}{\text{rank}}
\providecommand{\pr}[1]{\ensuremath{\Pr\left(#1\right)}}
\providecommand{\qfunc}[1]{\ensuremath{Q\left(#1\right)}}
        \newcommand*{\permcomb}[4][0mu]{{{}^{#3}\mkern#1#2_{#4}}}
\newcommand*{\perm}[1][-3mu]{\permcomb[#1]{P}}
\newcommand*{\comb}[1][-1mu]{\permcomb[#1]{C}}
\providecommand{\qfunc}[1]{\ensuremath{Q\left(#1\right)}}
\providecommand{\gauss}[2]{\mathcal{N}\ensuremath{\left(#1,#2\right)}}
\providecommand{\diff}[2]{\ensuremath{\frac{d{#1}}{d{#2}}}}
\providecommand{\myceil}[1]{\left \lceil #1 \right \rceil }
\newcommand\figref{Fig.~\ref}
\newcommand\tabref{Table~\ref}
\newcommand{\sinc}{\,\text{sinc}\,}
\newcommand{\rect}{\,\text{rect}\,}
%%
%       %\newcommand{\solution}[2]{\textbf{Solution:}{#1}}
%\newcommand{\solution}{\noindent \textbf{Solution: }}
%\newcommand{\cosec}{\,\text{cosec}\,}
%\numberwithin{equation}{section}
%\numberwithin{equation}{subsection}
%\numberwithin{problem}{section}
%\numberwithin{definition}{section}
%\makeatletter
%\@addtoreset{figure}{problem}
%\makeatother

%\let\StandardTheFigure\thefigure
\let\vec\mathbf

\bibliographystyle{IEEEtran}

\vspace{3cm}
\title{Assignment}
\author{EE23BTECH11001 - Aashna Sahu}
\maketitle
\bigskip

\renewcommand{\thefigure}{\theenumi}
\renewcommand{\thetable}{\theenumi}
%\renewcommand{\theequation}{\theenumi}
Q:Check whether -150 is a term of the AP: 11,8,5,2,....

 \solution

\begin{align}
x(n)&=x(0)+nd\\
n&=\frac{x(n)-x(0)}{d}
\end{align}
\begin{align}
x(n)-x(0) &\equiv 0 \pmod{d}
\end{align}
On substitutings values\\
\begin{align}
-161 &\equiv 2 \pmod{-3}
\end{align}
Thus -150 is not a term of the given AP.
\begin{align}
 \boxed{x(n)=(11-3n)\times u(n)}   
\end{align}

\begin{align}
   X(z)&=\frac{11}{1-z^{-1}}-\frac{3z^{-1}}{(1-z^{-1})^2}\quad
    |z|>1
\end{align}

    \begin{table}[h]
    \centering
    
        \begin{tabular}{|c|c|c|}
\hline 
   \textbf{Parameter}  &\textbf{Description} &\textbf{Value} \\
\hline
&&\\
$I_r$&Net Intensity of light at $\Delta x =\dfrac{\lambda}{3}$ &$\dfrac{K}{4}$ \\&&\\
\hline
\end{tabular}

        
    \caption{Input parameters}
    \label{tab:Table1}
\end{table}
\newpage
\begin{figure}[h]
  \centering
  \includegraphics[width=1.2\columnwidth]{figs/Figure_1.png}
  \captionsetup {justification=centering}
  \caption{Representation of x(n)}
  \label{fig:fig1}
\end{figure}
\end{document}

 

 \item Write the first five terms of the sequence \(a_n = \frac{n(n^2+5)}{4}\).

\solution
\input{ncert-maths/11/9/1/6/file1.tex}


\item
\begin{enumerate}
\item 30th term of the AP: 10, 7, 4, $\ldots$ is 
\item 11th term of the AP: $-3, -\frac{1}{2}, 2, \ldots$ is
\end{enumerate}
\solution
\let\negmedspace\undefined
\let\negthickspace\undefined
\documentclass[journal,12pt,twocolumn]{IEEEtran}
\usepackage{cite}
\usepackage{amsmath,amssymb,amsfonts,amsthm}
\usepackage{algorithmic}
\usepackage{graphicx}
\usepackage{textcomp}
\usepackage{xcolor}
\usepackage{txfonts}
\usepackage{listings}
\usepackage{enumitem}
\usepackage{mathtools}
\usepackage{gensymb}
\usepackage{comment}
\usepackage[breaklinks=true]{hyperref}
\usepackage{tkz-euclide}
\usepackage{listings}
\usepackage{gvv}
\def\inputGnumericTable{}
\usepackage[latin1]{inputenc}
\usepackage{color}
\usepackage{array}
\usepackage{longtable}
\usepackage{calc}
\usepackage{multirow}
\usepackage{hhline}
\usepackage{ifthen}
\usepackage{lscape}

\newtheorem{theorem}{Theorem}[section]
\newtheorem{problem}{Problem}
\newtheorem{proposition}{Proposition}[section]
\newtheorem{lemma}{Lemma}[section]
\newtheorem{corollary}[theorem]{Corollary}
\newtheorem{example}{Example}[section]
\newtheorem{definition}[problem]{Definition}
\newcommand{\BEQA}{\begin{eqnarray}}
\newcommand{\EEQA}{\end{eqnarray}}
\newcommand{\define}{\stackrel{\triangle}{=}}
\theoremstyle{remark}
\newtheorem{rem}{Remark}
\begin{document}

\bibliographystyle{IEEEtran}
\vspace{3cm}

\title{NCERT Discrete - 10.5.2.2}
\author{EE23BTECH11058 - Sindam Ananya$^{*}$% <-this % stops a space
}
\maketitle
\newpage
\bigskip

\renewcommand{\thefigure}{\theenumi}
\renewcommand{\thetable}{\theenumi}

\vspace{3cm}
\textbf{Question 10.5.2.2:} 
\begin{enumerate}
\item 30th term of the AP: 10, 7, 4, $\ldots$ is 
\item 11th term of the AP: $-3, -\frac{1}{2}, 2, \ldots$ is
\end{enumerate}
\solution
\begin{table}[h!]
    \centering
    \begin{tabular}{|c|c|c|}
\hline 
   \textbf{Parameter}  &\textbf{Description} &\textbf{Value} \\
\hline
&&\\
$I_r$&Net Intensity of light at $\Delta x =\dfrac{\lambda}{3}$ &$\dfrac{K}{4}$ \\&&\\
\hline
\end{tabular}

    \caption{Input Parameters}
    \label{tab:table1}
    \end{table}
\begin{equation}
    x_i(n) = \sbrak{x_i(0) + nd_i} u(n)
    \label{eq:eq1}
\end{equation}
\begin{enumerate}
\item From \eqref{eq:eq1} \tabref{tab:table1} :
\begin{align}
x_1(n) &= \sbrak{10 -3n}u(n)\\
x_1(29) &= -77\\
X_1(z) &= \frac{10 - 13z^{-1}}{(1-z^{-1})^2} \quad \abs{z} > 1
\end{align}
\item From \eqref{eq:eq1} and \tabref{tab:table1} :
\begin{align}
x_2(n) &= \sbrak{-3 + \frac{5}{2}n}u(n)\\
x_2(10) &= 22\\
X_2(z) &= \frac{0.5z^{-1}-3}{(1-z^{-1})^2} \quad \abs{z}> 1
\end{align}
\end{enumerate}
\begin{figure}[h!]
    \centering
    \includegraphics[width=\columnwidth]{figs/plot.png}
    \caption{stem plots of $x_1(n)$ and $x_2(n)$}
    \label{fig:1}
\end{figure}
\end{document}



\item Write the first five terms of the sequence whose nth term is $\frac{2n-3}{6}$ and obtain the Z transform of the series
\solution
\let\negmedspace\undefined
\let\negthickspace\undefined
\documentclass[journal,12pt,twocolumn]{IEEEtran}
\usepackage{cite}
\usepackage{amsmath,amssymb,amsfonts,amsthm}
\usepackage{algorithmic}
\usepackage{graphicx}
\usepackage{textcomp}
\usepackage{xcolor}
\usepackage{txfonts}
\usepackage{listings}
\usepackage{enumitem}
\usepackage{mathtools}
\usepackage{gensymb}
\usepackage{comment}
\usepackage[breaklinks=true]{hyperref}
\usepackage{tkz-euclide} 
\usepackage{listings}
\usepackage{gvv}                                        
\def\inputGnumericTable{}                                 
\usepackage[latin1]{inputenc}                                
\usepackage{color}                                            
\usepackage{array}                                            
\usepackage{longtable}                                       
\usepackage{calc}                                             
\usepackage{multirow}                                         
\usepackage{hhline}                                           
\usepackage{ifthen}                                           
\usepackage{lscape}

\newtheorem{theorem}{Theorem}[section]
\newtheorem{problem}{Problem}
\newtheorem{proposition}{Proposition}[section]
\newtheorem{lemma}{Lemma}[section]
\newtheorem{corollary}[theorem]{Corollary}
\newtheorem{example}{Example}[section]
\newtheorem{definition}[problem]{Definition}
\newcommand{\BEQA}{\begin{eqnarray}}
\newcommand{\EEQA}{\end{eqnarray}}
\newcommand{\define}{\stackrel{\triangle}{=}}
\theoremstyle{remark}
\newtheorem{rem}{Remark}

\begin{document}
\bibliographystyle{IEEEtran}

\vspace{3cm}

\title{}
\author{EE23BTECH11047 - Deepakreddy P
}
\maketitle
\newpage
\bigskip

\section*{Exercise 9.1}

\noindent \textbf{4} \quad Write the first five terms of the sequence whose nth term is $\frac{2n-3}{6}$ and obtain the Z transform of the series\\
\solution
\begin{align}
x \brak{n} &= \frac{2n-1}{6} \brak{u\brak{n}}
\label{x(n)}
\end{align}

\begin{figure}[h]
   \centering
   \includegraphics[width=1\columnwidth]{figs/plot.png}
   \caption{Plot of x(n) vs n}
   \label{fig: 9.1.4.1}
\end{figure}

\begin{align}
X(z) &= {\frac{3z^{-1}-1}{6(1-z^{-1})^{2}}\quad|z|>1}
\end{align}


\end{document}


 \item For what values of x, the numbers $-\frac{2}{7}\,,x,-\frac{7}{2}\,$ are in G.P ?

\solution
\iffalse
\let\negmedspace\undefined
\let\negthickspace\undefined
\documentclass[journal,12pt,twocolumn]{IEEEtran}
\usepackage{cite}
\usepackage{amsmath,amssymb,amsfonts,amsthm}
\usepackage{algorithmic}
\usepackage{graphicx}
\usepackage{textcomp}
\usepackage{xcolor}
\usepackage{txfonts}
\usepackage{listings}
\usepackage{enumitem}
\usepackage{mathtools}
\usepackage{gensymb}
\usepackage{comment}
\usepackage[breaklinks=true]{hyperref}
\usepackage{tkz-euclide} 
\usepackage{listings}
\usepackage{gvv}                                        
\def\inputGnumericTable{}                                 
\usepackage[latin1]{inputenc}                                
\usepackage{color}                                            
\usepackage{array}                                            
\usepackage{longtable}                                       
\usepackage{calc}                                             
\usepackage{multirow}                                         
\usepackage{hhline}                                           
\usepackage{ifthen}                                           
\usepackage{lscape}

\newtheorem{theorem}{Theorem}[section]
\newtheorem{problem}{Problem}
\newtheorem{proposition}{Proposition}[section]
\newtheorem{lemma}{Lemma}[section]
\newtheorem{corollary}[theorem]{Corollary}
\newtheorem{example}{Example}[section]
\newtheorem{definition}[problem]{Definition}
\newcommand{\BEQA}{\begin{eqnarray}}
\newcommand{\EEQA}{\end{eqnarray}}
\newcommand{\define}{\stackrel{\triangle}{=}}
\theoremstyle{remark}
\newtheorem{rem}{Remark}
\begin{document}

\bibliographystyle{IEEEtran}
\vspace{3cm}

\title{11.9.3.6}
\author{EE23BTECH11022 - G DILIP REDDY}
\maketitle
\newpage

\bigskip

\renewcommand{\thefigure}{\theenumi}
\renewcommand{\thetable}{\theenumi}
\textbf{Question}:\\
For what values of x, the numbers $-\frac{2}{7}\,,x,-\frac{7}{2}\,$ are in G.P ?
\\\\
\textbf{Solution: }\\
\fi
\begin{table}[h]
    \centering
    \begin{tabular}[12.1pt]{ |c| c| c|}
    \hline
    \textbf{Variable} & \textbf{Description} &\textbf{Value}\\ 
    \hline
    $x(0)$ & First term of the GP &$-\brak{\frac{2}{7}}$ \\
    \hline 
    $x(1)$ & Second term of the GP &$x$ \\
    \hline 
    $x(2)$ & Third term of the GP &$-\brak{\frac{7}{2}}$ \\
    \hline 
    $r$ & Common ratio of the GP & \\
    \hline
    $x(n)$ & General term & $x(0)\,r^n\,u(n)$\\
    \hline    
\end{tabular}

    \caption{Variables Used}
    \label{tab:table_11.9.3.6}
\end{table}
Let $r$ be the common ratio\\
From \tabref{tab:table_11.9.3.6}:
\begin{align}
\implies \frac{x}{\brak{-\frac{2}{7}\,}}\,&= \frac{\brak{-\frac{7}{2}\,}}{x}\,=r \\
x^2&=\brak{-\frac{2}{7}\,}\cdot\brak{-\frac{7}{2}\,}\\
x&=\pm 1\\
\implies r&=\pm \frac{7}{2}\,\\\notag
\end{align}
The signal corresponding to this is 
\begin{align}
x(n)=\brak{-\frac{2}{7}}\brak{\pm \frac{7}{2}}^n\,u(n)
\end{align}
Applying z-Transform :
\begin{align}
\implies X_1(z)&=\brak{\frac{1}{7}}\brak{\frac{4}{7z^{-1}+2}\,}
\quad \abs{z}>\frac{7}{2}\\
\implies X_2(z)&=\brak{\frac{1}{7}}\brak{\frac{4}{7z^{-1}-2}\,}
\quad \abs{z}>\frac{7}{2}
\end{align}
\begin{figure}[h]
    \centering
    \includegraphics[width=1.1\linewidth]{ncert-maths/11/9/3/6/figs/graph1.png}
    \caption{Stem Plot of $x_1$(n)}
    \label{stemplot1}
\end{figure}
\begin{figure}[h]
    \centering
    \includegraphics[width=1.1\linewidth]{ncert-maths/11/9/3/6/figs/graph2.png}
    \caption{Stem Plot of $x_2(n)$}
    \label{stemplot2}
\end{figure}
%\end{document}



\item Find the $20^{th}$ and $n^{th}$ terms of the G.P $\frac{5}{2}$, $\frac{5}{4}$, $\frac{5}{8}$,.....

\solution
% \iffalse
\let\negmedspace\undefined
\let\negthickspace\undefined
\documentclass[journal,12pt,twocolumn]{IEEEtran}
\usepackage{cite}
\usepackage{amsmath,amssymb,amsfonts,amsthm}
\usepackage{algorithmic}
\usepackage{graphicx}
\usepackage{textcomp}
\usepackage{xcolor}
\usepackage{txfonts}
\usepackage{listings}
\usepackage{enumitem}
\usepackage{mathtools}
\usepackage{gensymb}
\usepackage{comment}
\usepackage[breaklinks=true]{hyperref}
\usepackage{tkz-euclide} 
\usepackage{listings}
\usepackage{gvv}                                        
\def\inputGnumericTable{}                                 
\usepackage[latin1]{inputenc}                                
\usepackage{color}                                            
\usepackage{array}                                            
\usepackage{longtable}                                       
\usepackage{calc}                                             
\usepackage{multirow}                                         
\usepackage{hhline}                                           
\usepackage{ifthen}                                           
\usepackage{lscape}
\usepackage{caption}
\newtheorem{theorem}{Theorem}[section]
\newtheorem{problem}{Problem}
\newtheorem{proposition}{Proposition}[section]
\newtheorem{lemma}{Lemma}[section]
\newtheorem{corollary}[theorem]{Corollary}
\newtheorem{example}{Example}[section]
\newtheorem{definition}[problem]{Definition}
\newcommand{\BEQA}{\begin{eqnarray}}
\newcommand{\EEQA}{\end{eqnarray}}
\newcommand{\define}{\stackrel{\triangle}{=}}
\theoremstyle{remark}
\newtheorem{rem}{Remark}
\begin{document}
\parindent 0px
\bibliographystyle{IEEEtran}
\vspace{3cm}

\title{NCERT 11.9.3 1Q}
\author{EE23BTECH11013 - Avyaaz$^{*}$% <-this % stops a space
}
\maketitle
\newpage
\bigskip

\renewcommand{\thefigure}{\arabic{figure}}
\renewcommand{\thetable}{\arabic{table}}
\large\textbf{\textsl{Question:}}
Find the $20^{th}$ and $n^{th}$ terms of the G.P $\frac{5}{2}$, $\frac{5}{4}$, $\frac{5}{8}$,.....

\solution
 \begin{table}[htbp]
     \centering
     \setlength{\extrarowheight}{8pt}
    \begin{tabular}{|c|c|c|}
\hline 
   \textbf{Parameter}  &\textbf{Description} &\textbf{Value} \\
\hline
&&\\
$I_r$&Net Intensity of light at $\Delta x =\dfrac{\lambda}{3}$ &$\dfrac{K}{4}$ \\&&\\
\hline
\end{tabular}

     \caption{Parameters}
     \label{tab:table1.11.9.3.1}
 \end{table} 

% \begin{align}
%    x(n) = \dfrac{5}{2}\left(\dfrac{1}{2}\right)^n 
% \end{align}

% \begin{align}
% 	x \brak{n} & \system{Z} X \brak{z} \\
%    % x(n) &=\dfrac{5}{2}\left(\dfrac{1}{2}\right)^n u(n) \\
%     \therefore X(z) &= \sum_{n=-\infty}^{\infty}x(n)z^{-n}\label{eq:z-transform}  
% \end{align}
% Here, 
%          $    u(n) = \begin{cases}
%                 0 &\text{for } n < 0 \\
%                 1 & \text{for } n \geq 0
%             \end{cases}$       
 
%  \vspace{1cm}
From \tabref{tab:table1.11.9.3.1}:
\(Z\)-Transform of \(x(n)\):
\begin{align}
% \implies X(z) &= \sum_{n=-\infty}^{\infty}\left(\dfrac{5}{2}\left(\dfrac{1}{2}\right)^n u(n)\right) z^{-n} \\
 % \implies X(z) &= \dfrac{5}{2}\sum_{n=0}^{\infty}\left(\dfrac{z
 % ^{-1}}{2}\right)^n \\
\implies X(z) &=\dfrac{5}{2}\left(\dfrac{1}{1-\frac{z^{-1}}{2}}\right) ;\cbrak{z\in\mathbb{C} : |z|>\dfrac{1}{2}}
\end{align}

\begin{figure}[ht]
    \centering
    \includegraphics[width = \columnwidth]{figs/stem_plot.png}
    \caption{}
	\label{fig:graph1.11.9.3.1}
\end{figure} 

\bibliographystyle{IEEEtran}
\end{document}



\item 
Which term of the following sequences:\\
(a) 2,$2\sqrt{2}$,4\dots is 128
\quad(b) $\sqrt{3}$,3,$3\sqrt{3}$\dots is 729\\
(c) $\frac{1}{3}$,$\frac{1}{9}$,$\frac{1}{27}$\dots is $\frac{1}{19683}$ \\
\solution
\iffalse
\let\negmedspace\undefined
\let\negthickspace\undefined
\documentclass[journal,12pt,twocolumn]{IEEEtran}
\usepackage{cite}
\usepackage{amsmath,amssymb,amsfonts,amsthm}
\usepackage{algorithmic}
\usepackage{graphicx}
\usepackage{textcomp}
\usepackage{xcolor}
\usepackage{txfonts}
\usepackage{listings}
\usepackage{enumitem}
\usepackage{mathtools}
\usepackage{gensymb}
\usepackage{comment}
\usepackage[breaklinks=true]{hyperref}
\usepackage{tkz-euclide} 
\usepackage{listings}
\usepackage{gvv}                                        
\def\inputGnumericTable{}                                 
\usepackage[latin1]{inputenc}                                
\usepackage{color}                                            
\usepackage{array}                                            
\usepackage{longtable}                                       
\usepackage{calc}                                             
\usepackage{multirow}                                         
\usepackage{hhline}                                           
\usepackage{ifthen}                                           
\usepackage{lscape}
\usepackage[center]{caption} % center the captions to figure

\newtheorem{theorem}{Theorem}[section]
\newtheorem{problem}{Problem}
\newtheorem{proposition}{Proposition}[section]
\newtheorem{lemma}{Lemma}[section]
\newtheorem{corollary}[theorem]{Corollary}
\newtheorem{example}{Example}[section]
\newtheorem{definition}[problem]{Definition}
\newcommand{\BEQA}{\begin{eqnarray}}
\newcommand{\EEQA}{\end{eqnarray}}
\newcommand{\define}{\stackrel{\triangle}{=}}
\theoremstyle{remark}
\newtheorem{rem}{Remark}
\begin{document}

\newcolumntype{M}[1]{>{\centering\arraybackslash}m{#1}}
\newcolumntype{N}{@{}m{0pt}@{}}

\bibliographystyle{IEEEtran}
\vspace{3cm}

\title{NCERT 11.9.3 5Q} 
\author{ee23btech11223 - Soham Prabhakar More% <-this % stops a space
}
\maketitle
\newpage
\bigskip

\renewcommand{\thefigure}{\theenumi}
\renewcommand{\thetable}{\theenumi}

\bibliographystyle{IEEEtran}

\textbf{Question:}\\
Which term of the following sequences:\\
(a) 2,$2\sqrt{2}$,4\dots is 128
\quad(b) $\sqrt{3}$,3,$3\sqrt{3}$\dots is 729\\
(c) $\frac{1}{3}$,$\frac{1}{9}$,$\frac{1}{27}$\dots is $\frac{1}{19683}$
\fi 
For a general GP series and $k > 0$,
\begin{align}
    x\brak{k} &= x\brak{0}r^k \\
    \therefore k &= \log_r{\frac{x\brak{k}}{x\brak{0}}} \label{eq:gsoln}
\end{align}
And the Z-transform $X\brak{z}$:
\begin{align}
    X\brak{z} &= \frac{x\brak{0}}{1 - rz^{-1}} \quad {\abs{z} > \abs{r}} \label{eq:zresult}
\end{align}

\begin{enumerate}[label=(\alph*)]
\item By \tabref{Table:1}, \eqref{eq:gsoln} and \tabref{Table:1}: % prob:a
\begin{align}
    x_1\brak{n} &= x_1\brak{0} r_1^nu\brak{n} \\
    k_1 &= \log_{r_1}{\frac{128}{x_1\brak{0}}} \\
    \therefore k_1 &= 12 \\
	X_1\brak{z} &= \frac{2}{1 - \sqrt{2}z^{-1}} \quad \abs{z} > \sqrt{2}
\end{align}

\begin{figure}[h!]
    \renewcommand\thefigure{1}
    \centering
    \includegraphics[width=\columnwidth]{ncert-maths/11/9/3/5/figs/a.png}
    \caption[short]{Plot of $x_1$\brak{n} vs n. See \tabref{Table:1}}
    \label{fig:img1}
\end{figure}



\item By \eqref{eq:gsoln}, \eqref{eq:zresult} and \tabref{Table:1}: % prob:b
\begin{align}
    x_2\brak{n} &= x_2\brak{0} r_2^nu\brak{n} \\
    k_2 &= \log_{r_2}{\frac{729}{x_2\brak{0}}} \\
    \therefore k_2 &= 11 \\
    X_2\brak{z} &= \frac{\sqrt{3}}{1 - \sqrt{3}z^{-1}} \quad \abs{z} > \sqrt{3} 
\end{align}

\begin{figure}[h!]
    \renewcommand\thefigure{2}
    \centering
    \includegraphics[width=\columnwidth]{ncert-maths/11/9/3/5/figs/b.png}
    \caption[short]{Plot of $x_2$\brak{n} vs n. See \tabref{Table:1}}
    \label{fig:img2}
\end{figure}

\item By \eqref{eq:gsoln}, \eqref{eq:zresult} and \tabref{Table:1}: % prob:c
\begin{align}
    x_3\brak{n} &= x_3\brak{0} r_3^nu\brak{n} \\
    k_3 &= \log_{r_3}{\frac{1}{19683 x_3\brak{0}}} \\
    \therefore k_3 &= 8 \\
    X_3\brak{z} &= \frac{1}{3 - z^{-1}} \quad \abs{z} > \frac{1}{3}
\end{align}

\begin{figure}[h!]
    \renewcommand\thefigure{3}
    \centering
    \includegraphics[width=0.9\columnwidth]{ncert-maths/11/9/3/5/figs/c.png}
    \caption[short]{Plot of $x_3$\brak{n} vs n. See \tabref{Table:1}}
    \label{fig:img3}
\end{figure}

\begin{table}[ht]
\begin{tabular}{|c|c|c|}
    \hline 
    \textbf{Parameter}&\textbf{Description} &\textbf{Value}\\
    \hline 
    $r_i$ & Common ratio of G.P (a),(b),(c) & $\sqrt{2}, \sqrt{3}, \frac{1}{3}$ \\
    \hline
    $x_i(0)$ & Initial Values & $2, \sqrt{3}, \frac{1}{3}$ \\
    \hline
    $x_i(k_i)$ & Given Values & $128, 729, \frac{1}{19683}$ \\
    \hline 
    $k_i$ & Desired index & $12, 11, 8$ \\
    \hline 
    $x_i\brak{n}$ & Series & $x_i\brak{0}r_i^nu\brak{n}$ \\
    \hline
	$X_i\brak{z}$ & Z-Transform of $x_i\brak{n}$ & $\frac{x\brak{0}}{1-rz^{-1}}$ \\
    \hline
\end{tabular}

\caption{Table of parameters}
\label{Table:1}


\end{table}

\end{enumerate}

Find the $20^{th}$ and $n^{th}$ terms of the G.P $\frac{5}{2}$, $\frac{5}{4}$, $\frac{5}{8}$,.....

% \item 
% Which term of the following sequences:\\
% (a) 2,$2\sqrt{2}$,4\dots is 128
% \quad(b) $\sqrt{3}$,3,$3\sqrt{3}$\dots is 729\\
% (c) $\frac{1}{3}$,$\frac{1}{9}$,$\frac{1}{27}$\dots is $\frac{1}{19683}$ \\
% \solution
% \iffalse
\let\negmedspace\undefined
\let\negthickspace\undefined
\documentclass[journal,12pt,twocolumn]{IEEEtran}
\usepackage{cite}
\usepackage{amsmath,amssymb,amsfonts,amsthm}
\usepackage{algorithmic}
\usepackage{graphicx}
\usepackage{textcomp}
\usepackage{xcolor}
\usepackage{txfonts}
\usepackage{listings}
\usepackage{enumitem}
\usepackage{mathtools}
\usepackage{gensymb}
\usepackage{comment}
\usepackage[breaklinks=true]{hyperref}
\usepackage{tkz-euclide} 
\usepackage{listings}
\usepackage{gvv}                                        
\def\inputGnumericTable{}                                 
\usepackage[latin1]{inputenc}                                
\usepackage{color}                                            
\usepackage{array}                                            
\usepackage{longtable}                                       
\usepackage{calc}                                             
\usepackage{multirow}                                         
\usepackage{hhline}                                           
\usepackage{ifthen}                                           
\usepackage{lscape}
\usepackage[center]{caption} % center the captions to figure

\newtheorem{theorem}{Theorem}[section]
\newtheorem{problem}{Problem}
\newtheorem{proposition}{Proposition}[section]
\newtheorem{lemma}{Lemma}[section]
\newtheorem{corollary}[theorem]{Corollary}
\newtheorem{example}{Example}[section]
\newtheorem{definition}[problem]{Definition}
\newcommand{\BEQA}{\begin{eqnarray}}
\newcommand{\EEQA}{\end{eqnarray}}
\newcommand{\define}{\stackrel{\triangle}{=}}
\theoremstyle{remark}
\newtheorem{rem}{Remark}
\begin{document}

\newcolumntype{M}[1]{>{\centering\arraybackslash}m{#1}}
\newcolumntype{N}{@{}m{0pt}@{}}

\bibliographystyle{IEEEtran}
\vspace{3cm}

\title{NCERT 11.9.3 5Q} 
\author{ee23btech11223 - Soham Prabhakar More% <-this % stops a space
}
\maketitle
\newpage
\bigskip

\renewcommand{\thefigure}{\theenumi}
\renewcommand{\thetable}{\theenumi}

\bibliographystyle{IEEEtran}

\textbf{Question:}\\
Which term of the following sequences:\\
(a) 2,$2\sqrt{2}$,4\dots is 128
\quad(b) $\sqrt{3}$,3,$3\sqrt{3}$\dots is 729\\
(c) $\frac{1}{3}$,$\frac{1}{9}$,$\frac{1}{27}$\dots is $\frac{1}{19683}$
\fi 
For a general GP series and $k > 0$,
\begin{align}
    x\brak{k} &= x\brak{0}r^k \\
    \therefore k &= \log_r{\frac{x\brak{k}}{x\brak{0}}} \label{eq:gsoln}
\end{align}
And the Z-transform $X\brak{z}$:
\begin{align}
    X\brak{z} &= \frac{x\brak{0}}{1 - rz^{-1}} \quad {\abs{z} > \abs{r}} \label{eq:zresult}
\end{align}

\begin{enumerate}[label=(\alph*)]
\item By \tabref{Table:1}, \eqref{eq:gsoln} and \tabref{Table:1}: % prob:a
\begin{align}
    x_1\brak{n} &= x_1\brak{0} r_1^nu\brak{n} \\
    k_1 &= \log_{r_1}{\frac{128}{x_1\brak{0}}} \\
    \therefore k_1 &= 12 \\
	X_1\brak{z} &= \frac{2}{1 - \sqrt{2}z^{-1}} \quad \abs{z} > \sqrt{2}
\end{align}

\begin{figure}[h!]
    \renewcommand\thefigure{1}
    \centering
    \includegraphics[width=\columnwidth]{ncert-maths/11/9/3/5/figs/a.png}
    \caption[short]{Plot of $x_1$\brak{n} vs n. See \tabref{Table:1}}
    \label{fig:img1}
\end{figure}



\item By \eqref{eq:gsoln}, \eqref{eq:zresult} and \tabref{Table:1}: % prob:b
\begin{align}
    x_2\brak{n} &= x_2\brak{0} r_2^nu\brak{n} \\
    k_2 &= \log_{r_2}{\frac{729}{x_2\brak{0}}} \\
    \therefore k_2 &= 11 \\
    X_2\brak{z} &= \frac{\sqrt{3}}{1 - \sqrt{3}z^{-1}} \quad \abs{z} > \sqrt{3} 
\end{align}

\begin{figure}[h!]
    \renewcommand\thefigure{2}
    \centering
    \includegraphics[width=\columnwidth]{ncert-maths/11/9/3/5/figs/b.png}
    \caption[short]{Plot of $x_2$\brak{n} vs n. See \tabref{Table:1}}
    \label{fig:img2}
\end{figure}

\item By \eqref{eq:gsoln}, \eqref{eq:zresult} and \tabref{Table:1}: % prob:c
\begin{align}
    x_3\brak{n} &= x_3\brak{0} r_3^nu\brak{n} \\
    k_3 &= \log_{r_3}{\frac{1}{19683 x_3\brak{0}}} \\
    \therefore k_3 &= 8 \\
    X_3\brak{z} &= \frac{1}{3 - z^{-1}} \quad \abs{z} > \frac{1}{3}
\end{align}

\begin{figure}[h!]
    \renewcommand\thefigure{3}
    \centering
    \includegraphics[width=0.9\columnwidth]{ncert-maths/11/9/3/5/figs/c.png}
    \caption[short]{Plot of $x_3$\brak{n} vs n. See \tabref{Table:1}}
    \label{fig:img3}
\end{figure}

\begin{table}[ht]
\begin{tabular}{|c|c|c|}
    \hline 
    \textbf{Parameter}&\textbf{Description} &\textbf{Value}\\
    \hline 
    $r_i$ & Common ratio of G.P (a),(b),(c) & $\sqrt{2}, \sqrt{3}, \frac{1}{3}$ \\
    \hline
    $x_i(0)$ & Initial Values & $2, \sqrt{3}, \frac{1}{3}$ \\
    \hline
    $x_i(k_i)$ & Given Values & $128, 729, \frac{1}{19683}$ \\
    \hline 
    $k_i$ & Desired index & $12, 11, 8$ \\
    \hline 
    $x_i\brak{n}$ & Series & $x_i\brak{0}r_i^nu\brak{n}$ \\
    \hline
	$X_i\brak{z}$ & Z-Transform of $x_i\brak{n}$ & $\frac{x\brak{0}}{1-rz^{-1}}$ \\
    \hline
\end{tabular}

\caption{Table of parameters}
\label{Table:1}


\end{table}

\end{enumerate}

Find the $20^{th}$ and $n^{th}$ terms of the G.P $\frac{5}{2}$, $\frac{5}{4}$, $\frac{5}{8}$,.....

% \item 
% Which term of the following sequences:\\
% (a) 2,$2\sqrt{2}$,4\dots is 128
% \quad(b) $\sqrt{3}$,3,$3\sqrt{3}$\dots is 729\\
% (c) $\frac{1}{3}$,$\frac{1}{9}$,$\frac{1}{27}$\dots is $\frac{1}{19683}$ \\
% \solution
% \iffalse
\let\negmedspace\undefined
\let\negthickspace\undefined
\documentclass[journal,12pt,twocolumn]{IEEEtran}
\usepackage{cite}
\usepackage{amsmath,amssymb,amsfonts,amsthm}
\usepackage{algorithmic}
\usepackage{graphicx}
\usepackage{textcomp}
\usepackage{xcolor}
\usepackage{txfonts}
\usepackage{listings}
\usepackage{enumitem}
\usepackage{mathtools}
\usepackage{gensymb}
\usepackage{comment}
\usepackage[breaklinks=true]{hyperref}
\usepackage{tkz-euclide} 
\usepackage{listings}
\usepackage{gvv}                                        
\def\inputGnumericTable{}                                 
\usepackage[latin1]{inputenc}                                
\usepackage{color}                                            
\usepackage{array}                                            
\usepackage{longtable}                                       
\usepackage{calc}                                             
\usepackage{multirow}                                         
\usepackage{hhline}                                           
\usepackage{ifthen}                                           
\usepackage{lscape}
\usepackage[center]{caption} % center the captions to figure

\newtheorem{theorem}{Theorem}[section]
\newtheorem{problem}{Problem}
\newtheorem{proposition}{Proposition}[section]
\newtheorem{lemma}{Lemma}[section]
\newtheorem{corollary}[theorem]{Corollary}
\newtheorem{example}{Example}[section]
\newtheorem{definition}[problem]{Definition}
\newcommand{\BEQA}{\begin{eqnarray}}
\newcommand{\EEQA}{\end{eqnarray}}
\newcommand{\define}{\stackrel{\triangle}{=}}
\theoremstyle{remark}
\newtheorem{rem}{Remark}
\begin{document}

\newcolumntype{M}[1]{>{\centering\arraybackslash}m{#1}}
\newcolumntype{N}{@{}m{0pt}@{}}

\bibliographystyle{IEEEtran}
\vspace{3cm}

\title{NCERT 11.9.3 5Q} 
\author{ee23btech11223 - Soham Prabhakar More% <-this % stops a space
}
\maketitle
\newpage
\bigskip

\renewcommand{\thefigure}{\theenumi}
\renewcommand{\thetable}{\theenumi}

\bibliographystyle{IEEEtran}

\textbf{Question:}\\
Which term of the following sequences:\\
(a) 2,$2\sqrt{2}$,4\dots is 128
\quad(b) $\sqrt{3}$,3,$3\sqrt{3}$\dots is 729\\
(c) $\frac{1}{3}$,$\frac{1}{9}$,$\frac{1}{27}$\dots is $\frac{1}{19683}$
\fi 
For a general GP series and $k > 0$,
\begin{align}
    x\brak{k} &= x\brak{0}r^k \\
    \therefore k &= \log_r{\frac{x\brak{k}}{x\brak{0}}} \label{eq:gsoln}
\end{align}
And the Z-transform $X\brak{z}$:
\begin{align}
    X\brak{z} &= \frac{x\brak{0}}{1 - rz^{-1}} \quad {\abs{z} > \abs{r}} \label{eq:zresult}
\end{align}

\begin{enumerate}[label=(\alph*)]
\item By \tabref{Table:1}, \eqref{eq:gsoln} and \tabref{Table:1}: % prob:a
\begin{align}
    x_1\brak{n} &= x_1\brak{0} r_1^nu\brak{n} \\
    k_1 &= \log_{r_1}{\frac{128}{x_1\brak{0}}} \\
    \therefore k_1 &= 12 \\
	X_1\brak{z} &= \frac{2}{1 - \sqrt{2}z^{-1}} \quad \abs{z} > \sqrt{2}
\end{align}

\begin{figure}[h!]
    \renewcommand\thefigure{1}
    \centering
    \includegraphics[width=\columnwidth]{ncert-maths/11/9/3/5/figs/a.png}
    \caption[short]{Plot of $x_1$\brak{n} vs n. See \tabref{Table:1}}
    \label{fig:img1}
\end{figure}



\item By \eqref{eq:gsoln}, \eqref{eq:zresult} and \tabref{Table:1}: % prob:b
\begin{align}
    x_2\brak{n} &= x_2\brak{0} r_2^nu\brak{n} \\
    k_2 &= \log_{r_2}{\frac{729}{x_2\brak{0}}} \\
    \therefore k_2 &= 11 \\
    X_2\brak{z} &= \frac{\sqrt{3}}{1 - \sqrt{3}z^{-1}} \quad \abs{z} > \sqrt{3} 
\end{align}

\begin{figure}[h!]
    \renewcommand\thefigure{2}
    \centering
    \includegraphics[width=\columnwidth]{ncert-maths/11/9/3/5/figs/b.png}
    \caption[short]{Plot of $x_2$\brak{n} vs n. See \tabref{Table:1}}
    \label{fig:img2}
\end{figure}

\item By \eqref{eq:gsoln}, \eqref{eq:zresult} and \tabref{Table:1}: % prob:c
\begin{align}
    x_3\brak{n} &= x_3\brak{0} r_3^nu\brak{n} \\
    k_3 &= \log_{r_3}{\frac{1}{19683 x_3\brak{0}}} \\
    \therefore k_3 &= 8 \\
    X_3\brak{z} &= \frac{1}{3 - z^{-1}} \quad \abs{z} > \frac{1}{3}
\end{align}

\begin{figure}[h!]
    \renewcommand\thefigure{3}
    \centering
    \includegraphics[width=0.9\columnwidth]{ncert-maths/11/9/3/5/figs/c.png}
    \caption[short]{Plot of $x_3$\brak{n} vs n. See \tabref{Table:1}}
    \label{fig:img3}
\end{figure}

\begin{table}[ht]
\input{ncert-maths/11/9/3/5/tables/table.tex}
\end{table}

\end{enumerate}

Find the $20^{th}$ and $n^{th}$ terms of the G.P $\frac{5}{2}$, $\frac{5}{4}$, $\frac{5}{8}$,.....

% \item 
% Which term of the following sequences:\\
% (a) 2,$2\sqrt{2}$,4\dots is 128
% \quad(b) $\sqrt{3}$,3,$3\sqrt{3}$\dots is 729\\
% (c) $\frac{1}{3}$,$\frac{1}{9}$,$\frac{1}{27}$\dots is $\frac{1}{19683}$ \\
% \solution
% \input{ncert-maths/11/9/3/5/main.tex}
% \pagebreak

%\end{document}


% \pagebreak

%\end{document}


% \pagebreak

%\end{document}


\clearpage

\item The number of bacteria in a certain culture doubles every hour. If there were 30 bacteria present in the culture originally, how many bacteria will be present at the end of $2^{nd}$ hour, $4^{th}$ hour and $n^{th}$ hour?

\solution
\iffalse
\let\negmedspace\undefined
\let\negthickspace\undefined
\documentclass[journal,12pt,twocolumn]{IEEEtran}
\usepackage{cite}
\usepackage{amsmath,amssymb,amsfonts,amsthm}
\usepackage{algorithmic}
\usepackage{graphicx}
\usepackage{textcomp}
\usepackage{xcolor}
\usepackage{txfonts}
\usepackage{listings}
\usepackage{enumitem}
\usepackage{mathtools}
\usepackage{gensymb}
\usepackage{comment}
\usepackage[breaklinks=true]{hyperref}
\usepackage{tkz-euclide}
\usepackage{listings}
\usepackage{gvv}
\def\inputGnumericTable{}
\usepackage[latin1]{inputenc}
\usepackage{color}
\usepackage{array}
\usepackage{longtable}
\usepackage{calc}
\usepackage{multirow}
\usepackage{hhline}
\usepackage{ifthen}
\usepackage{lscape}

\newtheorem{theorem}{Theorem}[section]
\newtheorem{problem}{Problem}
\newtheorem{proposition}{Proposition}[section]
\newtheorem{lemma}{Lemma}[section]
\newtheorem{corollary}[theorem]{Corollary}
\newtheorem{example}{Example}[section]
\newtheorem{definition}[problem]{Definition}
\newcommand{\BEQA}{\begin{eqnarray}}
\newcommand{\EEQA}{\end{eqnarray}}
\newcommand{\define}{\stackrel{\triangle}{=}}
\theoremstyle{remark}
\newtheorem{rem}{Remark}
\begin{document}

\bibliographystyle{IEEEtran}
\vspace{3cm}

\title{NCERT Discrete - 11.9.3.30}
\author{EE23BTECH11007 - Aneesh Kadiyala$^{*}$% <-this % stops a space
}
\maketitle
\newpage
\bigskip

\renewcommand{\thefigure}{\theenumi}
\renewcommand{\thetable}{\theenumi}

\vspace{3cm}
\textbf{Question 11.9.3.30:} The number of bacteria in a certain culture doubles every hour. If there were 30 bacteria present in the culture originally, how many bacteria will be present at the end of $2^{nd}$ hour, $4^{th}$ hour and $n^{th}$ hour?
\\
\solution
\fi
\begin{table}[h!]
    \begin{tabular}{ | c | c | c | }
    \hline
    Parameter & Value & Description \\
    \hline
    $x(0)$ & 30 & Initial no. of bacteria\\
    \hline
    $r$ & 2 & Ratio of no. of bacteria at end of \\
    & & hour to start of hour (Common Ratio) \\
    \hline
    $x(n)$ & $r^nx(0)u(n)$ & $n^{th}$ term of the GP \\
    \hline
\end{tabular}
    \caption{Input Parameters}
    \label{tab:ncert_maths_11_9_3_30}
\end{table}
From \tabref{tab:ncert_maths_11_9_3_30}:
\begin{align}
x(2) &= 120 \\
x(4) &= 480 \\
x(n) &= 30(2^n)u(n)
\end{align}
\begin{figure}[h!]
    \centering
    \includegraphics[width=\columnwidth]{ncert-maths/11/9/3/30/figs/11_9_3_30.png}
    \caption{Plot of $x(n)$ vs $n$. See \tabref{tab:ncert_maths_11_9_3_30} for details.}
    \label{fig:ncert_maths_11_9_3_30}
\end{figure}
\begin{align}
X(z) = \frac{30z^{-1}}{1 - 2z^{-1}} \quad \abs{z} > 2
\end{align}
%\end{document}


\item Ramkali saved Rs 5 in the first week of a year and then increased her weekly savings by Rs 1.75. If in the $n$th week, her weekly savings become Rs 20.75, find $n$.

\solution
\let\negmedspace\undefined
\let\negthickspace\undefined
\documentclass[journal,12pt,twocolumn]{IEEEtran}
\usepackage{cite}
\usepackage{amsmath,amssymb,amsfonts,amsthm}
\usepackage{algorithmic}
\usepackage{graphicx}
\usepackage{textcomp}
\usepackage{xcolor}
\usepackage{txfonts}
\usepackage{listings}
\usepackage{enumitem}
\usepackage{mathtools}
\usepackage{gensymb}
\usepackage[breaklinks=true]{hyperref}
\usepackage{tkz-euclide} % loads  TikZ and tkz-base
\usepackage{listings}
\usepackage{gvv}


\newtheorem{theorem}{Theorem}[section]
\newtheorem{problem}{Problem}
\newtheorem{proposition}{Proposition}[section]
\newtheorem{lemma}{Lemma}[section]
\newtheorem{corollary}[theorem]{Corollary}
\newtheorem{example}{Example}[section]
\newtheorem{definition}[problem]{Definition}

\newcommand{\BEQA}{\begin{eqnarray}}
\newcommand{\EEQA}{\end{eqnarray}}
\newcommand{\define}{\stackrel{\triangle}{=}}
\theoremstyle{remark}
\newtheorem{rem}{Remark}

\graphicspath{./figs/}

%\bibliographystyle{ieeetr}
\begin{document}
%

\bibliographystyle{IEEEtran}


\vspace{3cm}

\title{
	%	\logo{
	Assignment-1 

	\large{EE:1205 Signals and Systems}

	Indian Institute of Technology, Hyderabad
	%	}
}
\author{Kunal Thorawade

EE23BTECH11035
}	

\maketitle


\newpage

%\tableofcontents

\bigskip
 
 \renewcommand{\thefigure}{\theenumi}
 \renewcommand{\thetable}{\theenumi}
 %\renewcommand{\theequation}{\theenumi}

 \section{\Large Question:}  Ramkali saved Rs 5 in the first week of a year and then increased her weekly savings by Rs 1.75. If in the $n$th week, her weekly savings become Rs 20.75, find $n$.

 \section{\Large Solution:} 
 \begin{tabular}{|c|c|c|}
\hline 
   \textbf{Parameter}  &\textbf{Description} &\textbf{Value} \\
\hline
&&\\
$I_r$&Net Intensity of light at $\Delta x =\dfrac{\lambda}{3}$ &$\dfrac{K}{4}$ \\&&\\
\hline
\end{tabular}


 \begin{align} 
	 x(n) &= x(0) + (n)(d)
	 \\ 20.75 &= 5 + (n)(1.75)  
	 \\ \implies 15.75 &= (n)(1.75)
	 \\ \implies n &= \frac{15.75}{1.75}
	 \\ \implies n &= 9
	 \\x(n) &= 5u(n) + 1.75nu(n)
 \end{align}
 The Z-transform of a sequence $x(n)$ is given by:
 \begin{align}
	  X(z) &= \frac{5z^{-1}}{1-z^{-1}}+\frac{1.75z^{-1}}{(1-z^{-1})^{2}} ; |z| > 1
 \end{align}

 \begin{figure}
	     \centering
	         \includegraphics[width = 8cm]{figs/fig1.png}
		     \caption{Plot of $x(n) = 5 + 1.75n$}
		         \label{fig:enter-label}
 \end{figure}
\end{document}



\item Show that the sum of $\brak {m+n}^{th}$ and $\brak {m-n}^{th}$ terms of an $A.P.,$ is equal to twice the $m^{th}$ terms.    \\
\solution
% \iffalse
\let\negmedspace\undefined
\let\negthickspace\undefined
\documentclass[journal,12pt,twocolumn]{IEEEtran}
\usepackage{cite}
\usepackage{amsmath,amssymb,amsfonts,amsthm}
\usepackage{algorithmic}
\usepackage{graphicx}
\usepackage{textcomp}
\usepackage{xcolor}
\usepackage{txfonts}
\usepackage{listings}
\usepackage{enumitem}
\usepackage{mathtools}
\usepackage{gensymb}
\usepackage{comment}
\usepackage[breaklinks=true]{hyperref}
\usepackage{tkz-euclide} 
\usepackage{listings}
\usepackage{gvv}                                        
\def\inputGnumericTable{}                                 
\usepackage[latin1]{inputenc}                                
\usepackage{color}                                            
\usepackage{array}                                            
\usepackage{longtable}                                       
\usepackage{calc}                                             
\usepackage{multirow}                                         
\usepackage{hhline}                                           
\usepackage{ifthen}                                           
\usepackage{lscape}

\newtheorem{theorem}{Theorem}[section]
\newtheorem{problem}{Problem}
\newtheorem{proposition}{Proposition}[section]
\newtheorem{lemma}{Lemma}[section]
\newtheorem{corollary}[theorem]{Corollary}
\newtheorem{example}{Example}[section]
\newtheorem{definition}[problem]{Definition}
\newcommand{\BEQA}{\begin{eqnarray}}
\newcommand{\EEQA}{\end{eqnarray}}
\newcommand{\define}{\stackrel{\triangle}{=}}
\theoremstyle{remark}
\newtheorem{rem}{Remark}
\begin{document}
\parindent 0px
\bibliographystyle{IEEEtran}
\title{Assignment 11.9.5\_1Q}
\author{EE22BTECH11219 - Rada Sai Sujan$^{}$% <-this % stops a space
}
\maketitle
\newpage
\bigskip
\section*{Question}
Show that the sum of $\brak {m+n}^{th}$ and $\brak {m-n}^{th}$ terms of an $A.P.,$ is equal to twice the $m^{th}$ terms.    \\
\solution

\begin{table}[ht]
    \centering
    \def\arraystretch{1.5}
    \begin{tabular}{|c|c|c|}
    \hline
    PARAMETER & VALUE & DESCRIPTION  \\ \hline
    $$x\brak0$$ & $$x\brak{0}$$ & First term \\ \hline
    $$d$$ & $$d$$ & common difference \\ \hline
    $$x(n)$$ & $$[x\brak{0}+nd]u\brak n$$ & General term of the series  \\ \hline
  \end{tabular}

    \caption{Parameter Table1}
    \label{tab:10.9.5.1.1}
\end{table}
For an $AP$,
\begin{align}
    x\brak{n}&=[x\brak{0}+nd]u\brak{n}   \\
    \implies x\brak{m+n}+x\brak{m-n}&=[x\brak{0}+\brak{m+n}d]+[x\brak{0}+\brak{m-n}d] \\
    &=2[x\brak{0}+md]   \\
    \therefore x\brak{m+n}+x\brak{m-n}&=2x\brak{m}
\end{align}
\begin{table}[ht]
    \centering
    \def\arraystretch{1.5}
    \begin{tabular}{|p{4.5cm}|p{4.5cm}|}
    \hline
      $$x\brak{0}$$ & $$3$$  \\ \hline
      $$d$$ & $$2$$  \\ \hline
      $$m$$ & $$6$$  \\ \hline
      $$n$$ & $$2$$  \\ \hline
      $$x\brak{m+n}$$ & $$19$$  \\ \hline
      $$x\brak{m-n}$$ & $$11$$  \\ \hline
      $$x\brak{m}$$ & $$15$$  \\ \hline
    \end{tabular}

    \caption{Verified Values}
    \label{tab:10.9.5.1.2}
\end{table}
\end{document}



\item The sum of the first three terms of a G.P is $39/10$ and their product is $1$. Find the common ratio and the terms.\\
\solution
\let\negmedspace\undefined
\let\negthickspace\undefined
\documentclass[journal,12pt,twocolumn]{IEEEtran}
\usepackage{cite}
\usepackage{amsmath,amssymb,amsfonts,amsthm}
\usepackage{algorithmic}
\usepackage{graphicx}
\usepackage{textcomp}
\usepackage{xcolor}
\usepackage{txfonts}
\usepackage{listings}
\usepackage{enumitem}
\usepackage{mathtools}
\usepackage{gensymb}
\usepackage{comment}
\usepackage[breaklinks=true]{hyperref}
\usepackage{tkz-euclide}
\usepackage{listings}
\usepackage{gvv}
\def\inputGnumericTable{}
\usepackage[latin1]{inputenc}
\usepackage{color}
\usepackage{array}
\usepackage{longtable}
\usepackage{calc}
\usepackage{multirow}
\usepackage{hhline}
\usepackage{ifthen}
\usepackage{lscape}

\newtheorem{theorem}{Theorem}[section]
\newtheorem{problem}{Problem}
\newtheorem{proposition}{Proposition}[section]
\newtheorem{lemma}{Lemma}[section]
\newtheorem{corollary}[theorem]{Corollary}
\newtheorem{example}{Example}[section]
\newtheorem{definition}[problem]{Definition}
\newcommand{\BEQA}{\begin{eqnarray}}
\newcommand{\EEQA}{\end{eqnarray}}
\newcommand{\define}{\stackrel{\triangle}{=}}
\theoremstyle{remark}
\newtheorem{rem}{Remark}
\begin{document}

\bibliographystyle{IEEEtran}
\vspace{3cm}

\title{NCERT Discrete - 11.9.3.12}
\author{EE23BTECH11058 - Sindam Ananya$^{*}$% <-this % stops a space
}
\maketitle
\newpage
\bigskip

\renewcommand{\thefigure}{\theenumi}
\renewcommand{\thetable}{\theenumi}

\vspace{3cm}
\textbf{Question : 11.9.3.12} 
The sum of the first three terms of a G.P is $39/10$ and their product is $1$. Find the common ratio and the terms.\\
\solution
\begin{table}[h!]
    \centering
    \begin{tabular}{|c|c|c|}
        \hline
        \textbf{Parameter} & \textbf{Value} & \textbf{Description} \\
        \hline
        $x(0)$ & & First term \\
        \hline
        $r$ & & Common ratio \\
        \hline
        $x(0)^3r^3$ & 1 & Product of terms \\
        \hline
        $x(0)$ + $x(0)r$ + $x(0)r^2$ & $\frac{39}{10}$ & Sum of terms \\
        \hline
    \end{tabular}

    \caption{Input Parameters}
    \label{tab:11.9.3.12table1}
\end{table}
\begin{equation}
y(n) = x(0)\brak{\frac{r^{n+1}-1}{r-1}}u(n)
\label{eq:11.9.3.12eq1}
\end{equation}
From \tabref{tab:11.9.3.12table1} and \eqref{eq:11.9.3.12eq1} :
\begin{align}
y(2) &= x(0)\brak{\frac{r^3-1}{r-1}}\\
\frac{39}{10} &= x(0)\brak{r^2+r+1}\\
\implies \frac{39r}{10} &= r^2+r+1 \quad \brak{\because x(0)r = 1}\\
\implies (2r-5)(5r-2) &=0\\
\implies r &= \frac{2}{5} \quad or \quad \frac{5}{2}
\end{align}
\begin{enumerate}
      \item If $r = \frac{2}{5}$, then terms are $\frac{5}{2}$, $1$, $\frac{2}{5}$.
      \item If $r = \frac{5}{2}$, then terms are $\frac{2}{5}$, $1$, $\frac{5}{2}$.
\end{enumerate}
\begin{figure}[h!]
    \centering
    \includegraphics[width=\columnwidth]{figs/graph1.png}
    \caption{stem plots of GP if $r=\frac{2}{5}$}
    \label{fig:11.9.3.12_1}
\end{figure}
\begin{figure}[h!]
    \centering
    \includegraphics[width=\columnwidth]{figs/graph2.png}
    \caption{stem plots of GP if $r=\frac{5}{2}$}
    \label{fig:11.9.3.12_2}
\end{figure}
\end{document}




\item The ratio of the A.M and G.M of two positive numbers $a$ and $b$ is $m:n$. Show that $a:b = \brak{ m + \sqrt{m^2 - n^2}} : \brak{ m - \sqrt{m^2 - n^2}}$.\\
\solution
\input{ncert-maths/11/9/5/19/file2.tex}

\item The sum of three numbers in an arithmetic progression (AP) is $24$ and the product of those three numbers is $440$, find the values of the three numbers.\\
\solution
\iffalse
\let\negmedspace\undefined
\let\negthickspace\undefined
\documentclass[journal,12pt,twocolumn]{IEEEtran}
\usepackage{cite}
\usepackage{amsmath,amssymb,amsfonts,amsthm}
\usepackage{algorithmic}
\usepackage{graphicx}
\usepackage{textcomp}
\usepackage{xcolor}
\usepackage{txfonts}
\usepackage{listings}
\usepackage{enumitem}
\usepackage{mathtools}
\usepackage{gensymb}
\usepackage{comment}
\usepackage[breaklinks=true]{hyperref}
\usepackage{tkz-euclide} 
\usepackage{listings}
\usepackage{gvv}

\def\inputGnumericTable{}                                
\usepackage[latin1]{inputenc}                 
\usepackage{color}                            
\usepackage{array}                            
\usepackage{longtable}                        
\usepackage{calc}                            
\usepackage{multirow}                      
\usepackage{hhline}                           
\usepackage{ifthen}                          
\usepackage{lscape}
\usepackage{amsmath}
\newtheorem{theorem}{Theorem}[section]
\newtheorem{problem}{Problem}
\newtheorem{proposition}{Proposition}[section]
\newtheorem{lemma}{Lemma}[section]
\newtheorem{corollary}[theorem]{Corollary}
\newtheorem{example}{Example}[section]
\newtheorem{definition}[problem]{Definition}
\newcommand{\BEQA}{\begin{eqnarray}}
\newcommand{\EEQA}{\end{eqnarray}}
\newcommand{\define}{\stackrel{\triangle}{=}}
\theoremstyle{remark}
\newtheorem{rem}{Remark}


\begin{document}
%

\bibliographystyle{IEEEtran}


\vspace{3cm}

\title{
%	\logo{
Discrete 11.9.2 

\large{EE:1205 Signals and System}

Indian Institute of Technology, Hyderabad
%	}
}
\author{Prashant Maurya

EE23BTECH11218
\maketitle

\newpage

%\tableofcontents

\bigskip

\renewcommand{\thefigure}{\arabic{figure}}
\renewcommand{\thetable}{\arabic{table}}
\flushleft{\textbf{Question-2 :} Find the sum of all natural numbers lying between 100 and 1000, which are
multiples of 5.}\\
\bigskip
\textbf{Solution:}
\fi
\begin{table}[!h]
	\centering
	\begin{tabular}{|c|c|c|}
    \hline
     Parameter & Description & Value \\
    \hline
     $x(0)$ & First Term & 105\\
     \hline
     $d$ & Common Difference & 5\\
    \hline
    $n$ & Total terms & 179 \\ 
    \hline
    $x(178)$ & Last Term & 995\\
    \hline
    $m$ & No of poles & 3\\
    \hline
\end{tabular}

	\vspace{6 pt}
	\caption{Given Parameters}
\end{table}
\begin{align}
	x\brak{n}= & \brak{105+5n}\brak{u(n)}
\end{align}
On taking Z transform
\begin{align}
X\brak{z}= &\frac {x\brak{0}} {\brak{1-z^{-1}}} + \frac {dz^{-1}} {\brak{1-z^{-1}}^2} \\
= &\dfrac{105}{1-z^{-1}} + \frac{5z^{-1}}{\brak{1-z^{-1}}^{2}}\\
\implies X\brak{z}=& \dfrac{105-100z^{-1}}{{\brak{1-z^{-1}}}^2} \quad |z|>1\\
y\brak{n}=& x\brak{n}* u\brak{n}\\
\implies Y\brak{z}=& X\brak{z}U\brak{z}\\
=& \dfrac{105-100z^{-1}}{{(1-z^{-1})}^2}\frac1 {\brak{1-z^{-1}}} \\
=& \dfrac{105-100z^{-1}}{\brak{1-z^{-1}}^{3}} \quad |z|>1
\end{align}
Using contour integration to find the inverse Z-transform:\\
\begin{align}
    \implies y\brak{178}=&\dfrac{1}{2\pi j}\oint_{C}Y\brak{z} \;z^{177} \;dz\\
    =&\dfrac{1}{2\pi j}\oint_{C}\dfrac{\brak{105-100z^{-1}}{z^{177}}}{\brak{{1-z^{-1}}}^{3}} \;dz 
\end{align}
We can observe that there is only a 3 times repeated pole at $z=1$,
\begin{align}
    \implies R&=\dfrac{1}{\brak {m-1}!}\lim\limits_{z\to a}\dfrac{d^{m-1}}{dz^{m-1}}\brak {{\brak{z-a}}^{m}f\brak z}  
\end{align}
\begin{align}
    &=\dfrac{1}{\brak {2}!}\lim\limits_{z\to 1}\dfrac{d^{2}}{dz^{2}}\brak {{\brak{z-1}}^{3}\dfrac{\brak{105-100z^{-1}}z^{180}}{{\brak{z-1}}^3}}\\
    &=\dfrac{1}{2}{\lim\limits_{z\to 1}\dfrac{d^2}{dz^2}\brak{105z^{180}-100z^{179}}}\\
     &=98450
\end{align}
\begin{align}
    \therefore y\brak{178}=98450
\end{align}
\begin{figure}[ht]
    \centering
    \includegraphics[width=\columnwidth]{ncert-maths/11/9/2/2/figures/fig1.png}
    \caption{Plot of $x(n)$ $vs$ $n$}
\end{figure}
%\end{document}

\pagebreak

\item The sum of some terms of G.P. is $315$ whose first term and the common ratio are $5$ and $2$ , respectively. Find the last term and the number of terms.\\
\solution
 \iffalse
\let\negmedspace\undefined
\let\negthickspace\undefined
\documentclass[journal,12pt,twocolumn]{IEEEtran}
\usepackage{cite}
\usepackage{amsmath,amssymb,amsfonts,amsthm}
\usepackage{algorithmic}
\usepackage{graphicx}
\usepackage{textcomp}
\usepackage{xcolor}
\usepackage{txfonts}
\usepackage{listings}
\usepackage{enumitem}
\usepackage{mathtools}
\usepackage{gensymb}
\usepackage{comment}
\usepackage[breaklinks=true]{hyperref}
\usepackage{tkz-euclide} 
\usepackage{listings}
\usepackage{gvv}                                        
\def\inputGnumericTable{}                                 
\usepackage[latin1]{inputenc}                                
\usepackage{color}                                            
\usepackage{array}                                            
\usepackage{longtable}                                       
\usepackage{calc}                                             
\usepackage{multirow}                                         
\usepackage{hhline}                                           
\usepackage{ifthen}                                           
\usepackage{lscape}
\newtheorem{theorem}{Theorem}[section]
\newtheorem{problem}{Problem}
\newtheorem{proposition}{Proposition}[section]
\newtheorem{lemma}{Lemma}[section]
\newtheorem{corollary}[theorem]{Corollary}
\newtheorem{example}{Example}[section]
\newtheorem{definition}[problem]{Definition}
\newcommand{\BEQA}{\begin{eqnarray}}
\newcommand{\EEQA}{\end{eqnarray}}
\newcommand{\define}{\stackrel{\triangle}{=}}
\theoremstyle{remark}

\newtheorem{rem}{Remark}
\begin{document}
\parindent 0px
\bibliographystyle{IEEEtran}
\title{Assignment 11.9.5\_6Q}
\author{EE23BTECH11028 - Kamale Goutham$^{}$% <-this % stops a space
}
\maketitle
\newpage
\bigskip
\textbf{Question}\\
Find the sum of all two digit numbers which when divided by $4$,yields $1$ as reminder?\\
\solution 
\fi
Input parameters are:\\
\begin{table}[ht]
    \centering
    \def\arraystretch{1.5}
    \footnotesize
\begin{tabular}{|p{2cm}|p{2.5cm}|p{2.3cm}|}
    \hline
    PARAMETER & VALUE & DESCRIPTION  \\ \hline
    $$x\brak0$$ & $$13$$ & First term \\ \hline
    $$d$$ & $$4$$ & common difference \\ \hline
    $$x(n)$$ & $$[13+4n]u\brak n$$ & General term of the series  \\ \hline
  \end{tabular}

    \caption{INPUT PARAMETER TABLE}
    \label{tab:11.9.5.6}
\end{table}
  \begin{align}
    \text{x(n)}=&x(0)+n{\text{d}}\\
    n=&\frac{97-13}{4}=21\\
\end{align}
From \ref{eq:11.9.5.26.2}
\begin{align}
X(z)=&\frac{13-9z^{-1}}{(1-z^{-1})^2},|z|>1\\
y(n)=&x(n)*u(n)\\ Y(z)=&X(z)U(z)\\\implies Y(z)=&{\frac{13-9z^{-1}}{(1-z^{-1})^{3}}},|z|>1
\end{align}
Using contour integration to find the inverse z-transform,
\begin{align}
    y(n)=&\frac{1}{2\pi j}\oint_{C}Y(z) z^{n-1} dz  \\y(21)=&\frac{1}{2\pi j}\oint_{C}{\frac{(13-9z^{-1})z^{20}}{(1-z^{-1})^{3}}}
\end{align}
We can observe that the pole is repeated $3$ times and thus $m=3$,
\begin{align}
    R&=\frac{1}{\brak {m-1}!}\lim\limits_{z\to a}\frac{d^{m-1}}{dz^{m-1}}\brak {{(z-a)}^{m}f\brak z}  \\&=\frac{1}{\brak {2}!}\lim\limits_{z\to 1}\frac{d^{2}}{dz^{2}}\brak{13z^{23}-9z^{22}}\\R&=1210\\
        \therefore &y\brak{21}=1210
\end{align}
Therefore, the sum of all two-digit numbers that, when divided by 4, yield a remainder of 1 is 1210.\\
\begin{figure}[h]
  \centering
  \includegraphics[width=\columnwidth]{ncert-maths/11/9/5/6/figs/fig1.png}
  \caption{y(n) = $13 + 15n+2n^2$}
\end{figure}
%\end{document}

\pagebreak

\item  Find the sum of n terms of the A.P. whose kth term is \(5k + 1\).\\
\solution
\documentclass[journal,12pt,twocolumn]{IEEEtran}
\usepackage{cite}
\usepackage{amsmath,amssymb,amsfonts,amsthm}
\usepackage{algorithmic}
\usepackage{graphicx}
\usepackage{textcomp}
\usepackage{xcolor}
\usepackage{txfonts}
\usepackage{listings}
\usepackage{enumitem}
\usepackage{mathtools}
\usepackage{gensymb}
\usepackage{comment}
\usepackage[breaklinks=true]{hyperref}
\usepackage{tkz-euclide} 
\usepackage{textgreek}                       
\usepackage{circuitikz}
\usepackage{pgfplots}                            
\usepackage[latin1]{inputenc}                                
\usepackage{color}                                            
\usepackage{array}                                            
\usepackage{longtable}                                       
\usepackage{calc}                                             
\usepackage{multirow}                                         
\usepackage{hhline}                                           
\usepackage{ifthen}                                           
\usepackage{lscape}


\newtheorem{theorem}{Theorem}[section]
\newtheorem{problem}{Problem}
\newtheorem{proposition}{Proposition}[section]
\newtheorem{lemma}{Lemma}[section]
\newtheorem{corollary}[theorem]{Corollary}
\newtheorem{example}{Example}[section]
\newtheorem{definition}[problem]{Definition}
\newcommand{\BEQA}{\begin{eqnarray}}
\newcommand{\EEQA}{\end{eqnarray}}
\newcommand{\define}{\stackrel{\triangle}{=}}
\theoremstyle{remark}
\newtheorem{rem}{Remark}

\begin{document}
\providecommand{\brak}[1]{\ensuremath{\left(#1\right)}}
\bibliographystyle{IEEEtran}
\vspace{3cm}

\title{NCERT 11.9.2 Q7}
\author{EE23BTECH11204 - Ashley Ann Benoy$^{*}$}% <-this % stops a space
\maketitle
\newpage
\bigskip
\bibliographystyle{IEEEtran}
\textbf{Question: Find the sum of n terms of the A.P. whose kth term is \(5k + 1\).}\\
\solution
\begin{table}[h!]
    \centering
    \resizebox{6cm}{!}{
        
\begin{tabular}{|c|c|c|}
\hline
\textbf{Symbol} & \textbf{Value} & \textbf{Parameter} \\
\hline
\(x(0)\) & \(1 \) & First Term \\
\hline
\(x(n)\) & \((5n+1)u(n)\) & kth Term \\
\hline
\(d\) & \(5 \) & Common Difference \\
\hline
\end{tabular}


    }
    \\
    \caption{Given Parameters}
    \label{tab:given_params}  
\end{table}

Apply the Z-transform to \( x\brak{n} \):
\begin{align}
X\brak{z} = \frac{5z^{-1}}{\brak{1 - z^{-1}}^2} + \frac{1}{\brak{1 - z^{-1}}}
\quad |z|>1
\end{align}

Sum of First \( n \) Terms:

\begin{align}
y\brak{n} = x\brak{n} * u\brak{n}
\end{align}

Applying Z transform on both sides:
\begin{align}
    Y\brak{z} &= X\brak{z}U\brak{z}
\end{align}

\begin{align}
&=\frac{1}{\brak{1 - z^{-1}}^2} + \frac{5}{2} \cdot \frac{2z^{-1}}{\brak{1 - z^{-1}}^3} 
\end{align}
\\
Now we can compare the  above pairs as;
\begin{align}
nu\brak{n} \xleftrightarrow{\text{Z}} \frac{z^{-1}}{(1 - z^{-1})^2}
\end{align}
\begin{align}
u\brak{n} \xleftrightarrow{\text{Z}} \frac{1}{(1 - z^{-1})}
\end{align}
\begin{align}
n\brak{n-1}u\brak{n} \xleftrightarrow{\text{Z}} \frac{2z^{-1}}{(1 - z^{-1})^3}
\end{align}
On referring the above equations and comparing, we can obtain the  Z transform inverse as follows:

\begin{align}
y\brak{n} = \brak{n+1 }u\brak{n} + \frac{5}{2} n\brak{n-1} u\brak{n}
\end{align}
\begin{align}
&= \brak{n+1 + \frac{5}{2} n\brak{n-1}}u\brak{n}
\end{align}
Since we are taking n starting from 0 we replace n with n+1 to make our simulation match with the theory\\Therefore, we have got the sum of n terms as:\\
\begin{align}
y\brak{n}= \brak{n+2 + \frac{5}{2} n\brak{n+1}}u\brak{n+1}
\end{align}
The stem plot is given as
\begin{figure}[h!]
  \centering
  \includegraphics[width=\columnwidth]{figs/Figure_1.png}
  \label{fig:Stem_Plot}
\end{figure}
\end{document}


\pagebreak

\item How many 3 digit numbers are divisible by 7? \\
\solution
\iffalse
\let\negmedspace\undefined
\let\negthickspace\undefined
\documentclass[journal,12pt,twocolumn]{IEEEtran}
\usepackage{cite}
\usepackage{amsmath,amssymb,amsfonts,amsthm}
\usepackage{graphicx}
\usepackage{textcomp}
\usepackage{xcolor}
\usepackage{txfonts}
\usepackage{listings}
\usepackage{enumitem}
\usepackage{mathtools}
\usepackage{gensymb}
\usepackage[breaklinks=true]{hyperref}
\usepackage{tkz-euclide} % loads  TikZ and tkz-base
\usepackage{listings}
\usepackage{gvv}
\usepackage{booktabs}

%
%\usepackage{setspace}
%\usepackage{gensymb}
%\doublespacing
%\singlespacing

%\usepackage{graphicx}
%\usepackage{amssymb}
%\usepackage{relsize}
%\usepackage[cmex10]{amsmath}
%\usepackage{amsthm}
%\interdisplaylinepenalty=2500
%\savesymbol{iint}
%\usepackage{txfonts}
%\restoresymbol{TXF}{iint}
%\usepackage{wasysym}
%\usepackage{amsthm}
%\usepackage{iithtlc}
%\usepackage{mathrsfs}
%\usepackage{txfonts}
%\usepackage{stfloats}
%\usepackage{bm}
%\usepackage{cite}
%\usepackage{cases}
%\usepackage{subfig}
%\usepackage{xtab}
%\usepackage{longtable}
%\usepackage{multirow}

%\usepackage{algpseudocode}
%\usepackage{enumitem}
%\usepackage{mathtools}
%\usepackage{tikz}
%\usepackage{circuitikz}
%\usepackage{verbatim}
%\usepackage{tfrupee}
%\usepackage{stmaryrd}
%\usetkzobj{all}
%    \usepackage{color}                                            %%
%    \usepackage{array}                                            %%
%    \usepackage{longtable}                                        %%
%    \usepackage{calc}                                             %%
%    \usepackage{multirow}                                         %%
%    \usepackage{hhline}                                           %%
%    \usepackage{ifthen}                                           %%
  %optionally (for landscape tables embedded in another document): %%
%    \usepackage{lscape}     
%\usepackage{multicol}
%\usepackage{chngcntr}
%\usepackage{enumerate}

%\usepackage{wasysym}
%\documentclass[conference]{IEEEtran}
%\IEEEoverridecommandlockouts
% The preceding line is only needed to identify funding in the first footnote. If that is unneeded, please comment it out.

\newtheorem{theorem}{Theorem}[section]
\newtheorem{problem}{Problem}
\newtheorem{proposition}{Proposition}[section]
\newtheorem{lemma}{Lemma}[section]
\newtheorem{corollary}[theorem]{Corollary}
\newtheorem{example}{Example}[section]
\newtheorem{definition}[problem]{Definition}
%\newtheorem{thm}{Theorem}[section] 
%\newtheorem{defn}[thm]{Definition}
%\newtheorem{algorithm}{Algorithm}[section]
%\newtheorem{cor}{Corollary}
\newcommand{\BEQA}{\begin{eqnarray}}
\newcommand{\EEQA}{\end{eqnarray}}
\newcommand{\define}{\stackrel{\triangle}{=}}
\theoremstyle{remark}
\newtheorem{rem}{Remark}

%\bibliographystyle{ieeetr}
\begin{document}
%

\bibliographystyle{IEEEtran}


\vspace{3cm}

\title{
%	\logo{
Discrete Assignment 

\large{EE:1205 Signals and Systems}

Indian Institute of Technology, Hyderabad
%	}
}
\author{Abhey Garg

EE23BTECH11202
}	


% make the title area
\maketitle

\newpage

%\tableofcontents

\bigskip

\renewcommand{\thefigure}{\arabic{figure}}
\renewcommand{\thetable}{\arabic{table}}
\renewcommand{\theequation}{\arabic{equation}}

\section{Question 10.5.2.13}
How many 3 digit numbers are divisible by 7?
\section{Solution}
\fi

\begin{table}[ht]
\centering
\setlength{\extrarowheight}{8pt}
\caption{Input Parameters}
\begin{tabular}{|c|l|l|} 
\hline
\textbf{Parameter} & \textbf{Used to denote} & \textbf{Values} \\
\hline
$x$\brak{0}  & First three digit number divisible by 7 & \multicolumn{1}{|p{1.3cm}|}{\centering $105$ }\\
\hline
$x$\brak{k-1} & Last three digit number divisible by 7 & \multicolumn{1}{|p{1.3cm}|}{\centering ? } \\
\hline
$d$ & Common difference of A.P & \multicolumn{1}{|p{1.3cm}|}{\centering $7$ } \\
\hline
$k$ & Number of 3 digit terms divisible by 7 & \multicolumn{1}{|p{1.3cm}|}{\centering ? }\\
\hline
\end{tabular}
 \vspace{4mm}
 \label{tab:table0}
\end{table}

We can use modular arithmetic to determine last three digit number divisible by 7 . 
\begin{align}
x(k-1) \equiv 0  \; \text{mod}\; 7
\end{align}
So we need to find the largest multiple of 7 less than 1000. We can find this by subtracting the remainder when 1000 is divided by 7 from 1000.
\begin{align}
1000 - (1000 \; \text{mod}\;7) &= 1000-6
\end{align} 
\begin{align}
x(k-1) = 994
\end{align}
From \tabref{tab:table012}, the number of terms in the AP, $k$ is:
\begin{align}
k = \frac{x(k-1)-x(0)}{d} + 1
\end{align}
\begin{align}
\frac{994 - 105}{7} + 1 = 128
\end{align}
Taking z transform  using  \ref{eq:apz} :
\begin{align}
 X\brak{z} = \frac{105 - 98z^{-1}}{\brak{1-z^{-1}}^2} \quad |z| > 1 
\end{align}


\begin{figure}[!ht]
\centering
\begin{center}
\includegraphics[width=\columnwidth]{ncert-maths/10/5/2/13/figs/Fig1.png}
\caption{Plot of $x\brak{n}$}
\end{center}
\end{figure}
	
%\end{document}


\item A person writes a letter to four of his friends. He asks each one of them to copy the letter and mail to four different persons with instruction that they move the chain similarly. Assuming that the chain is not broken and that it costs 50 paise to mail one letter. Find the amount spent on the postage when 8th set of letter is mailed.\\
\solution
%\iffalse
\let\negmedspace\undefined
\let\negthickspace\undefined
\documentclass[journal,12pt,twocolumn]{IEEEtran}
\usepackage{cite}
\usepackage{amsmath,amssymb,amsfonts,amsthm}
\usepackage{algorithmic}
\usepackage{graphicx}
\usepackage{textcomp}
\usepackage{xcolor}
\usepackage{txfonts}
\usepackage{listings}
\usepackage{enumitem}
\usepackage{mathtools}
\usepackage{gensymb}
\usepackage{comment}
\usepackage[breaklinks=true]{hyperref}
\usepackage{tkz-euclide} 
\usepackage{listings}
\usepackage{gvv}  
\usepackage{tikz}
\usepackage{circuitikz} 
\usepackage{caption}

\def\inputGnumericTable{}                                
\usepackage[latin1]{inputenc}                 
\usepackage{color}                            
\usepackage{array}                            
\usepackage{longtable}                        
\usepackage{calc}                            
\usepackage{multirow}                      
\usepackage{hhline}                           
\usepackage{ifthen}                          
\usepackage{lscape}
\usepackage{amsmath}
\newtheorem{theorem}{Theorem}[section]
\newtheorem{problem}{Problem}
\newtheorem{proposition}{Proposition}[section]
\newtheorem{lemma}{Lemma}[section]
\newtheorem{corollary}[theorem]{Corollary}
\newtheorem{example}{Example}[section]
\newtheorem{definition}[problem]{Definition}
\newcommand{\BEQA}{\begin{eqnarray}}
\newcommand{\EEQA}{\end{eqnarray}}
\newcommand{\define}{\stackrel{\triangle}{=}}
\theoremstyle{remark}
\newtheorem{rem}{Remark}


\begin{document}
%

\bibliographystyle{IEEEtran}


\vspace{3cm}

\title{
%	\logo{
Discrete 11.9.5

\large{EE:1205 Signals and System}

Indian Institute of Technology, Hyderabad
%	}
}
\author{Prashant Maurya

EE23BTECH11218
}	
%\title{
%	\logo{Matrix Analysis through Octave}{\begin{center}\includegraphics[scale=.24]{tlc}\end{center}}{}{HAMDSP}
%}


% paper title
% can use linebreaks \\ within to get better formatting as desired
%\title{Matrix Analysis through Octave}
%
%
% author names and IEEE memberships
% note positions of commas and nonbreaking spaces ( ~ ) LaTeX will not break
% a structure at a ~ so this keeps an author's name from being broken across
% two lines.
% use \thanks{} to gain access to the first footnote area
% a separate \thanks must be used for each paragraph as LaTeX2e's \thanks
% was not built to handle multiple paragraphs
%

%\author{<-this % stops a space
%\thanks{}}
%}
% note the % following the last \IEEEmembership and also \thanks - 
% these prevent an unwanted space from occurring between the last author name
% and the end of the author line. i.e., if you had this:
% 
% \author{....lastname \thanks{...} \thanks{...} }
%                     ^------------^------------^----Do not want these spaces!
%
% a space would be appended to the last name and could cause every name on that
% line to be shifted left slightly. This is one of those "LaTeX things". For
% instance, "\textbf{A} \textbf{B}" will typeset as "A B" not "AB". To get
% "AB" then you have to do: "\textbf{A}\textbf{B}"
% \thanks is no different in this regard, so shield the last } of each \thanks
% that ends a line with a % and do not let a space in before the next \thanks.
% Spaces after \IEEEmembership other than the last one are OK (and needed) as
% you are supposed to have spaces between the names. For what it is worth,
% this is a minor point as most people would not even notice if the said evil
% space somehow managed to creep in.



% The paper headers
%\markboth{Journal of \LaTeX\ Class Files,~Vol.~6, No.~1, January~2007}%
%{Shell \MakeLowercase{\textit{et al.}}: Bare Demo of IEEEtran.cls for Journals}
% The only time the second header will appear is for the odd numbered pages
% after the title page when using the twoside option.
% 
% *** Note that you probably will NOT want to include the author's ***
% *** name in the headers of peer review papers.                   ***
% You can use \ifCLASSOPTIONpeerreview for conditional compilation here if
% you desire.




% If you want to put a publisher's ID mark on the page you can do it like
% this:
%\IEEEpubid{0000--0000/00\$00.00~\copyright~2007 IEEE}
% Remember, if you use this you must call \IEEEpubidadjcol in the second
% column for its text to clear the IEEEpubid mark.



% make the title area
\maketitle

\newpage

%\tableofcontents

\bigskip

\renewcommand{\thefigure}{\arabic{figure}}
\renewcommand{\thetable}{\arabic{table}}
%\renewcommand{\theequation}{\theenumi}

%\begin{abstract}
%%\boldmath
%In this letter, an algorithm for evaluating the exact analytical bit error rate  (BER)  for the piecewise linear (PL) combiner for  multiple relays is presented. Previous results were available only for upto three relays. The algorithm is unique in the sense that  the actual mathematical expressions, that are prohibitively large, need not be explicitly obtained. The diversity gain due to multiple relays is shown through plots of the analytical BER, well supported by simulations. 
%
%\end{abstract}
% IEEEtran.cls defaults to using nonbold math in the Abstract.
% This preserves the distinction between vectors and scalars. However,
% if the journal you are submitting to favors bold math in the abstract,
% then you can use LaTeX's standard command \boldmath at the very start
% of the abstract to achieve this. Many IEEE journals frown on math
% in the abstract anyway.

% Note that keywords are not normally used for peerreview papers.
%\begin{IEEEkeywords}
%Cooperative diversity, decode and forward, piecewise linear
%\end{IEEEkeywords}



% For peer review papers, you can put extra information on the cover
% page as needed:
% \ifCLASSOPTIONpeerreview
% \begin{center} \bfseries EDICS Category: 3-BBND \end{center}
% \fi
%
% For peerreview papers, this IEEEtran command inserts a page break and
% creates the second title. It will be ignored for other modes.
%\IEEEpeerreviewmaketitle	

\textbf{Question 29: }A person writes a letter to four of his friends. He asks each one of them to copy the letter and mail to four different persons with instruction that they move the chain similarly. Assuming that the chain is not broken and that it costs 50 paise to mail one letter. Find the amount spent on the postage when 8th set of letter is mailed.\\

\textbf{Solution}
\begin{table}[!h]
	\centering
	\begin{tabular}{|c|c|c|}
\hline 
   \textbf{Parameter}  &\textbf{Description} &\textbf{Value} \\
\hline
&&\\
$I_r$&Net Intensity of light at $\Delta x =\dfrac{\lambda}{3}$ &$\dfrac{K}{4}$ \\&&\\
\hline
\end{tabular}

	\vspace{6 pt}
	\caption{Given Parameters}
	\label{tab:enter-label}
\end{table}
\begin{align}
x\brak{n}=x\brak{0}r^nu\brak{n}\
\end{align}
On taking Z transform
\begin{align}
X\brak{z}=&\dfrac{x\brak{0}}{1-rz^{-1}}  \quad |z|>|r|\\
 		 =&\dfrac{4}{1-4z^{-1}}
\end{align}
\begin{align}
y\brak{n} =& x\brak{n}* u\brak{n}\\
\implies Y\brak{z} =& X\brak{z} U\brak{z}\\
                   =& \dfrac{4}{\brak{1-4z^{-1}}}\dfrac{1}{\brak{1-z^{-1}}} \quad |z|>|r| \cap |z|>|1|    
\end{align}
Using contour integration to find the inverse Z-transform:\\
\begin{align}
    \implies y\brak{7}=&\dfrac{1}{2\pi j}\oint_{C}Y\brak{z} \;z^{6} \;dz\\
    =&\dfrac{1}{2\pi j}\oint_{C}\dfrac{4z^6}{\brak{1-4z^{-1}}\brak{1-z^{-1}}}\;dz \\
    =&\dfrac{4}{3} \brak{\frac{1}{2\pi j} {\oint_C \dfrac{z^{9}}{z - 4} \:dz -  \frac{1}{2\pi j}{\oint_C \frac{z^{9}}{z - 1}}}} dz
\end{align}
We know that 
\begin{align}
    \implies R&=\dfrac{1}{\brak {m-1}!}\lim\limits_{z\to a}\dfrac{d^{m-1}}{dz^{m-1}}\brak {{\brak{z-a}}^{m}f\brak z}  
\end{align}
For first contour integral,
\begin{align}
R_1 =& \frac{1}{\brak{1 - 1}!} \lim_{z \to a} \brak{\brak{z - a}f\brak{z}}\\
	=& r^{n+1}
\end{align}
For second contour integral,
\begin{align}
R_2 =& \frac{1}{\brak{1 - 1}!} \lim_{z \to a} \brak{\brak{z - a}f\brak{z}}\\
	=& 1
\end{align}
The sum of $n$ terms of a GP is given by :
\begin{align}
s\brak{n}=&\dfrac{x\brak{0}}{r-1} \brak{R_1 -R_2}\\
		 =& 87380\\
\end{align}
\begin{align}
\therefore \text{Total amount spent on postage}=& 87380 \times 0.5 \\ 
             =& Rs.\; 43690
\end{align}

\begin{figure}[ht]
	\centering
    \includegraphics[width=\columnwidth]{figs/figure1.png}
    \caption{Plot of $x(n)$ $vs$ $n$}
    \label{fig: 1}
\end{figure}
\end{document}
\pagebreak

\item If $a$, $b$, $c$ are in A.P.; $b$, $c$, $d$ are in G.P and $\frac{1}{c}$, $\frac{1}{d}$, $\frac{1}{e}$ are in A.P. prove that $a$, $c$, $e$ are in G.P.\\
\solution
\let\negmedspace\undefined
\let\negthickspace\undefined
\documentclass[journal,12pt,twocolumn]{IEEEtran}

\usepackage{cite}
\usepackage{amsmath,amssymb,amsfonts,amsthm}
\usepackage{algorithmic}
\usepackage{graphicx}
\usepackage{textcomp}
\usepackage{xcolor}
\usepackage{txfonts}
\usepackage{listings}
\usepackage{enumitem}
\usepackage{mathtools}
\usepackage{gensymb}
\usepackage[breaklinks=true]{hyperref}
\usepackage{tkz-euclide} % loads  TikZ and tkz-base
\usepackage{listings}
\usepackage{circuitikz}
\usepackage{graphicx}

%\newcounter{MYtempeqncnt}
\DeclareMathOperator*{\Res}{Res}
%\renewcommand{\baselinestretch}{2}
\renewcommand\thesection{\arabic{section}}
\renewcommand\thesubsection{\thesection.\arabic{subsection}}
\renewcommand\thesubsubsection{\thesubsection.\arabic{subsubsection}}

\renewcommand\thesectiondis{\arabic{section}}
\renewcommand\thesubsectiondis{\thesectiondis.\arabic{subsection}}
\renewcommand\thesubsubsectiondis{\thesubsectiondis.\arabic{subsubsection}}

% correct bad hyphenation here
\hyphenation{op-tical net-works semi-conduc-tor}
\def\inputGnumericTable{}                                 %%

\lstset{
	frame=single,
	breaklines=true,
	columns=fullflexible
}



\newtheorem{theorem}{Theorem}[section]
\newtheorem{problem}{Problem}
\newtheorem{proposition}{Proposition}[section]
\newtheorem{lemma}{Lemma}[section]
\newtheorem{corollary}[theorem]{Corollary}
\newtheorem{example}{Example}[section]
\newtheorem{definition}[problem]{Definition}
\newcommand{\BEQA}{\begin{eqnarray}}
	\newcommand{\EEQA}{\end{eqnarray}}
\newcommand{\define}{\stackrel{\triangle}{=}}
\newcommand\figref{Fig.~\ref}
\newcommand\tabref{Table~\ref}
\bibliographystyle{IEEEtran}
%\bibliographystyle{ieeetr}


\providecommand{\mbf}{\mathbf}
\providecommand{\pr}[1]{\ensuremath{\Pr\left(#1\right)}}
\providecommand{\qfunc}[1]{\ensuremath{Q\left(#1\right)}}
\providecommand{\sbrak}[1]{\ensuremath{{}\left[#1\right]}}
\providecommand{\lsbrak}[1]{\ensuremath{{}\left[#1\right.}}
\providecommand{\rsbrak}[1]{\ensuremath{{}\left.#1\right]}}
\providecommand{\brak}[1]{\ensuremath{\left(#1\right)}}
\providecommand{\lbrak}[1]{\ensuremath{\left(#1\right.}}
\providecommand{\rbrak}[1]{\ensuremath{\left.#1\right)}}
\providecommand{\cbrak}[1]{\ensuremath{\left\{#1\right\}}}
\providecommand{\lcbrak}[1]{\ensuremath{\left\{#1\right.}}
\providecommand{\rcbrak}[1]{\ensuremath{\left.#1\right\}}}
\theoremstyle{remark}
\newtheorem{rem}{Remark}
\newcommand{\sgn}{\mathop{\mathrm{sgn}}}
\providecommand{\abs}[1]{\left\vert#1\right\vert}
\providecommand{\res}[1]{\Res\displaylimits_{#1}}
\providecommand{\norm}[1]{\left\lVert#1\right\rVert}
%\providecommand{\norm}[1]{\lVert#1\rVert}
\providecommand{\mtx}[1]{\mathbf{#1}}
\providecommand{\mean}[1]{E\left[ #1 \right]}
\providecommand{\fourier}{\overset{\mathcal{F}}{ \rightleftharpoons}}
%\providecommand{\hilbert}{\overset{\mathcal{H}}{ \rightleftharpoons}}
\providecommand{\system}{\overset{\mathcal{H}}{ \longleftrightarrow}}
%\newcommand{\solution}[2]{\textbf{Solution:}{#1}}
\newcommand{\solution}{\noindent \textbf{Solution: }}
\newcommand{\cosec}{\,\text{cosec}\,}
\providecommand{\dec}[2]{\ensuremath{\overset{#1}{\underset{#2}{\gtrless}}}}
\newcommand{\myvec}[1]{\ensuremath{\begin{pmatrix}#1\end{pmatrix}}}
\newcommand{\mydet}[1]{\ensuremath{\begin{vmatrix}#1\end{vmatrix}}}
\renewcommand{\abstractname}{Question}

\let\vec\mathbf

	
	\vspace{3cm}
	
	


\newcommand{\permcomb}[4][0mu]{{{}^{#3}\mkern#1#2_{#4}}}
\newcommand{\comb}[1][-1mu]{\permcomb[#1]{C}}

%\IEEEpeerreviewmaketitle

\newcommand \tab [1][1cm]{\hspace*{#1}}
%\newcommand{\Var}{$\sigma ^2$}
\usepackage{amssymb}
\usepackage{amsmath}
\title{
	
\title{NCERT Discrete 11.5.9 Q20}
\author{EE23BTECH11061 - SWATHI DEEPIKA$^{*}$% <-this % stops a space
}


}
\begin{document}

\maketitle

\textbf{Question:} 
If $a$,$b$,$c$ are in A.P.;$b$,$c$,$d$ are in G.P and $\frac{1}{c}$, $\frac{1}{d}$, $\frac{1}{e}$ are in A.P. prove that $a$,$c$,$e$ are in G.P.
\solution
 \begin{table}[h]
 	\centering
 	\resizebox{6 cm}{!}{
 		\begin{tabular}{|c|c|c|}
\hline 
   \textbf{Parameter}  &\textbf{Description} &\textbf{Value} \\
\hline
&&\\
$I_r$&Net Intensity of light at $\Delta x =\dfrac{\lambda}{3}$ &$\dfrac{K}{4}$ \\&&\\
\hline
\end{tabular}

 	}
 	\vspace{6 pt}
 	\caption{Parameters}
 	\label{tab:swa_tabel} 
 \end{table}

 
\begin{align}
b-a = c-b\\
2b=a+c \label{eq: sw1}
\end{align}
\begin{align}
c^2 = b\times d
\end{align}
\begin{align}
d= \frac{c^2}{b} \label{eq: sw2}
\end{align}
\begin{align}
\frac{1}{d} - \frac{1}{c} = \frac{1}{e} - \frac{1}{d}\\
\frac{2}{d} = \frac{1}{c} + \frac{1}{e}
\end{align}
From \eqref{eq: sw2},
\begin{align}
\frac{2b}{c^2} = \frac{1}{c} + \frac{1}{e}
\end{align}

From \eqref{eq: sw1},
\begin{align}\frac{a + c}{c^2} = \frac{1}{c} + \frac{1}{e}\\
\frac{a}{c^2} + \frac{1}{c} = \frac{1}{c} + \frac{1}{e}\\
a \times e = {c}^2
\end{align}

So, $a$,$c$,$e$ are in G.P\\

\begin{enumerate}

\item For $y(n)$:
    \begin{align}
        y(n) &= a\left(\frac{c}{a}\right)^n u(n)
    \end{align}
    \begin{equation*}
     y(n)  \xleftrightarrow{\mathcal{Z}} Y(z)
     \end{equation*}
    \begin{align}
        Y(z) &= \frac{c}{1-\frac{c}{a}z^{-1}}, \quad \left|z\right|>\left|\frac{c}{a}\right|
    \end{align}
    
    \item For $x_1(n)$:
    \begin{align}
        x_1(n) &= (b + n(b-a))u(n)
    \end{align}
    \begin{equation*}
   x_1(n)  \xleftrightarrow{\mathcal{Z}} X_1(z)
    \end{equation*}
    \begin{align}
        X_1(z) &= \frac{a}{1-z^{-1}} + \frac{(b-a)z^{-1}}{(1-z^{-1})^2} , \quad |z| > 1
    \end{align}

    \item For $x_2(n)$:
    \begin{align}
        x_2(n) &= b\left(\frac{c}{b}\right)^n u(n)
    \end{align}
    \begin{equation*}
     x_2(n)  \xleftrightarrow{\mathcal{Z}} X_2(z)
     \end{equation*}
    \begin{align}
        X_2(z) &= \frac{c}{1-\frac{c}{b}z^{-1}},  \quad \left|z\right|>\left|\frac{c}{b}\right|
    \end{align}

    \item For $x_3(n)$:
    \begin{align}
        x_3(n) &= \left(\frac{1}{c} + n\left(\frac{1}{c} - \frac{1}{d}\right)\right)u(n)
    \end{align}
    \begin{equation*}
  x_3(n)  \xleftrightarrow{\mathcal{Z}} X_3(z)
   \end{equation*}
    \begin{align}
        X_3(z) &= \frac{\frac{1}{c}}{1-z^{-1}} + \left(\frac{1}{d} - \frac{1}{c}\right)\frac{z^{-1}}{(1-z^{-1})^2} , \quad |z| > 1
    \end{align}
\end{enumerate}


\end{document}






\pagebreak

\item Find the 31st term of an AP whose $11$th term is 38 and the $16$th term is 73.\\ 
\solution
% \iffalse
\let\negmedspace\undefined
\let\negthickspace\undefined
\documentclass[journal,12pt,twocolumn]{IEEEtran}
\usepackage{cite}
\usepackage{amsmath,amssymb,amsfonts,amsthm}
\usepackage{algorithmic}
\usepackage{graphicx}
\usepackage{textcomp}
\usepackage{xcolor}
\usepackage{txfonts}
\usepackage{listings}
\usepackage{enumitem}
\usepackage{mathtools}
\usepackage{gensymb}
\usepackage{comment}
\usepackage[breaklinks=true]{hyperref}
\usepackage{tkz-euclide} 
\usepackage{listings}
\usepackage{gvv}                                        
\def\inputGnumericTable{}                                 
\usepackage[latin1]{inputenc}                                
\usepackage{color}                                            
\usepackage{array}                                            
\usepackage{longtable}                                       
\usepackage{calc}                                             
\usepackage{multirow}                                         
\usepackage{hhline}                                           
\usepackage{ifthen}                                           
\usepackage{lscape}
\usepackage{caption}
\newtheorem{theorem}{Theorem}[section]
\newtheorem{problem}{Problem}
\newtheorem{proposition}{Proposition}[section]
\newtheorem{lemma}{Lemma}[section]
\newtheorem{corollary}[theorem]{Corollary}
\newtheorem{example}{Example}[section]
\newtheorem{definition}[problem]{Definition}
\newcommand{\BEQA}{\begin{eqnarray}}
\newcommand{\EEQA}{\end{eqnarray}}
\newcommand{\define}{\stackrel{\triangle}{=}}
\theoremstyle{remark}
\newtheorem{rem}{Remark}
\begin{document}
\parindent 0px
\bibliographystyle{IEEEtran}
\vspace{3cm}

\title{NCERT 11.9.3 1Q}
\author{EE23BTECH11013 - Avyaaz$^{*}$% <-this % stops a space
}
\maketitle
\newpage
\bigskip

\renewcommand{\thefigure}{\arabic{figure}}
\renewcommand{\thetable}{\arabic{table}}
\large\textbf{\textsl{Question:}}
Find the $20^{th}$ and $n^{th}$ terms of the G.P $\frac{5}{2}$, $\frac{5}{4}$, $\frac{5}{8}$,.....

\solution
 \begin{table}[htbp]
     \centering
     \setlength{\extrarowheight}{8pt}
    \begin{tabular}{|c|c|c|}
\hline 
   \textbf{Parameter}  &\textbf{Description} &\textbf{Value} \\
\hline
&&\\
$I_r$&Net Intensity of light at $\Delta x =\dfrac{\lambda}{3}$ &$\dfrac{K}{4}$ \\&&\\
\hline
\end{tabular}

     \caption{Parameters}
     \label{tab:table1.11.9.3.1}
 \end{table} 

% \begin{align}
%    x(n) = \dfrac{5}{2}\left(\dfrac{1}{2}\right)^n 
% \end{align}

% \begin{align}
% 	x \brak{n} & \system{Z} X \brak{z} \\
%    % x(n) &=\dfrac{5}{2}\left(\dfrac{1}{2}\right)^n u(n) \\
%     \therefore X(z) &= \sum_{n=-\infty}^{\infty}x(n)z^{-n}\label{eq:z-transform}  
% \end{align}
% Here, 
%          $    u(n) = \begin{cases}
%                 0 &\text{for } n < 0 \\
%                 1 & \text{for } n \geq 0
%             \end{cases}$       
 
%  \vspace{1cm}
From \tabref{tab:table1.11.9.3.1}:
\(Z\)-Transform of \(x(n)\):
\begin{align}
% \implies X(z) &= \sum_{n=-\infty}^{\infty}\left(\dfrac{5}{2}\left(\dfrac{1}{2}\right)^n u(n)\right) z^{-n} \\
 % \implies X(z) &= \dfrac{5}{2}\sum_{n=0}^{\infty}\left(\dfrac{z
 % ^{-1}}{2}\right)^n \\
\implies X(z) &=\dfrac{5}{2}\left(\dfrac{1}{1-\frac{z^{-1}}{2}}\right) ;\cbrak{z\in\mathbb{C} : |z|>\dfrac{1}{2}}
\end{align}

\begin{figure}[ht]
    \centering
    \includegraphics[width = \columnwidth]{figs/stem_plot.png}
    \caption{}
	\label{fig:graph1.11.9.3.1}
\end{figure} 

\bibliographystyle{IEEEtran}
\end{document}


\pagebreak
\item If $a\left(\frac{1}{b} + \frac{1}{c}\right)$, $b\left(\frac{1}{c} + \frac{1}{a}\right)$, $c\left(\frac{1}{a} + \frac{1}{b}\right)$ are in arithmetic progression (AP), prove that $a$, $b$, $c$ are also in AP. \\
\solution
% \iffalse
\let\negmedspace\undefined
\let\negthickspace\undefined
\documentclass[journal,12pt,twocolumn]{IEEEtran}
\usepackage{cite}
\usepackage{amsmath,amssymb,amsfonts,amsthm}
\usepackage{algorithmic}
\usepackage{graphicx}
\usepackage{textcomp}
\usepackage{xcolor}
\usepackage{txfonts}
\usepackage{listings}
\usepackage{enumitem}
\usepackage{mathtools}
\usepackage{gensymb}
\usepackage{comment}
\usepackage[breaklinks=true]{hyperref}
\usepackage{tkz-euclide} 
\usepackage{listings}
\usepackage{gvv}                                        
\def\inputGnumericTable{}                                 
\usepackage[latin1]{inputenc}                                
\usepackage{color}                                            
\usepackage{array}                                            
\usepackage{longtable}                                       
\usepackage{calc}                                             
\usepackage{multirow}                                         
\usepackage{hhline}                                           
\usepackage{ifthen}                                           
\usepackage{lscape}
\usepackage{caption}
\newtheorem{theorem}{Theorem}[section]
\newtheorem{problem}{Problem}
\newtheorem{proposition}{Proposition}[section]
\newtheorem{lemma}{Lemma}[section]
\newtheorem{corollary}[theorem]{Corollary}
\newtheorem{example}{Example}[section]
\newtheorem{definition}[problem]{Definition}
\newcommand{\BEQA}{\begin{eqnarray}}
\newcommand{\EEQA}{\end{eqnarray}}
\newcommand{\define}{\stackrel{\triangle}{=}}
\theoremstyle{remark}
\newtheorem{rem}{Remark}
\begin{document}
\parindent 0px
\bibliographystyle{IEEEtran}
\vspace{3cm}

\title{NCERT 11.9.3 1Q}
\author{EE23BTECH11013 - Avyaaz$^{*}$% <-this % stops a space
}
\maketitle
\newpage
\bigskip

\renewcommand{\thefigure}{\arabic{figure}}
\renewcommand{\thetable}{\arabic{table}}
\large\textbf{\textsl{Question:}}
Find the $20^{th}$ and $n^{th}$ terms of the G.P $\frac{5}{2}$, $\frac{5}{4}$, $\frac{5}{8}$,.....

\solution
 \begin{table}[htbp]
     \centering
     \setlength{\extrarowheight}{8pt}
    \begin{tabular}{|c|c|c|}
\hline 
   \textbf{Parameter}  &\textbf{Description} &\textbf{Value} \\
\hline
&&\\
$I_r$&Net Intensity of light at $\Delta x =\dfrac{\lambda}{3}$ &$\dfrac{K}{4}$ \\&&\\
\hline
\end{tabular}

     \caption{Parameters}
     \label{tab:table1.11.9.3.1}
 \end{table} 

% \begin{align}
%    x(n) = \dfrac{5}{2}\left(\dfrac{1}{2}\right)^n 
% \end{align}

% \begin{align}
% 	x \brak{n} & \system{Z} X \brak{z} \\
%    % x(n) &=\dfrac{5}{2}\left(\dfrac{1}{2}\right)^n u(n) \\
%     \therefore X(z) &= \sum_{n=-\infty}^{\infty}x(n)z^{-n}\label{eq:z-transform}  
% \end{align}
% Here, 
%          $    u(n) = \begin{cases}
%                 0 &\text{for } n < 0 \\
%                 1 & \text{for } n \geq 0
%             \end{cases}$       
 
%  \vspace{1cm}
From \tabref{tab:table1.11.9.3.1}:
\(Z\)-Transform of \(x(n)\):
\begin{align}
% \implies X(z) &= \sum_{n=-\infty}^{\infty}\left(\dfrac{5}{2}\left(\dfrac{1}{2}\right)^n u(n)\right) z^{-n} \\
 % \implies X(z) &= \dfrac{5}{2}\sum_{n=0}^{\infty}\left(\dfrac{z
 % ^{-1}}{2}\right)^n \\
\implies X(z) &=\dfrac{5}{2}\left(\dfrac{1}{1-\frac{z^{-1}}{2}}\right) ;\cbrak{z\in\mathbb{C} : |z|>\dfrac{1}{2}}
\end{align}

\begin{figure}[ht]
    \centering
    \includegraphics[width = \columnwidth]{figs/stem_plot.png}
    \caption{}
	\label{fig:graph1.11.9.3.1}
\end{figure} 

\bibliographystyle{IEEEtran}
\end{document}

\newpage
\item If \(\frac{a^n +b^n}{a^{n-1} + b^{n-1}}\) is A.M between $a$ and $b$, then find value of $n$.\\
\solution
\iffalse
\let\negmedspace\undefined
\let\negthickspace\undefined
\documentclass[journal,12pt,twocolumn]{IEEEtran}
\usepackage{xparse}
\usepackage{cite}
\usepackage{amsmath,amssymb,amsfonts,amsthm}
\usepackage{algorithmic}
\usepackage{graphicx}
\usepackage{textcomp}
\usepackage{xcolor}
\usepackage{txfonts}
\usepackage{listings}
\usepackage{enumitem}
\usepackage{mathtools}
\usepackage{gensymb}
\usepackage{comment}
\usepackage[breaklinks=true]{hyperref}
\usepackage{tkz-euclide} 
\usepackage{listings}
\usepackage{gvv}
\def\inputGnumericTable{}                                 
\usepackage[latin1]{inputenc}                                
\usepackage{color}                                            
\usepackage{array}                                            
\usepackage{longtable}                                       
\usepackage{calc}                                             
\usepackage{multirow}                                         
\usepackage{hhline}                                           
\usepackage{ifthen}                                           
\usepackage{lscape}

\newtheorem{theorem}{Theorem}[section]
\newtheorem{problem}{Problem}
\newtheorem{proposition}{Proposition}[section]
\newtheorem{lemma}{Lemma}[section]
\newtheorem{corollary}[theorem]{Corollary}
\newtheorem{example}{Example}[section]
\newtheorem{definition}[problem]{Definition}
\newcommand{\BEQA}{\begin{eqnarray}}
\newcommand{\EEQA}{\end{eqnarray}}
\newcommand{\define}{\stackrel{\triangle}{=}}
\theoremstyle{remark}
\newtheorem{rem}{Remark}
\begin{document}

\bibliographystyle{IEEEtran}
\vspace{3cm}


\title{NCERT DISCRETE 11.9.2.15}
\author{EE23BTECH11046 - Poluri Hemanth$^{*}$}
\maketitle
\textbf{Question:}
If \( \frac{a^n +b^n}{a^{n-1} + b^{n-1}}\)is A.M between $a$ and $b$, then find value of $n$.
\break
\textbf{Solution:}
\fi
\begin{table}[h!]
        \setlength{\arrayrulewidth}{0.2mm}
\setlength{\tabcolsep}{15pt}
\renewcommand{\arraystretch}{1.15}


\begin{table}[ht]
  \centering
  \begin{tabular}{|c|c|c|}
    \hline
    	Symbol & Parameters & value\\
    \hline
	  $u\brak{n}$ & unit step function & 1, if n$\geq$ 0; \\& &0 otherwise \\
    \hline
	  $x\brak{n}$ & general term of the series & $\brak{n+1}\brak{n+3}u\brak{n}$ \\
    \hline 
	 $X\brak{z}$ & Z-transform of $x\brak{n}$ & ? \\
    \hline
  \end{tabular}
  \vspace{0.3cm}
  \caption{Input Parameters}
  \label{tab:24.11.9.1.1}
\end{table}


        \caption{parameters}
\end{table}
\\A.M of two numbers $a$,$b$ is $\frac{a+b}{2}$.
\begin{align}
	x(n)&=x(0)+n\cdot d\cdot u(n)\label{1}
\end{align}
Where,
\begin{align}
	x(1)&=\frac{x(0)^n +x(2)^n}{x(0)^{n-1} + x(2)^{n-1}}\label{r}\\ 
	&=\frac{a+b}{2}\\
	\Rightarrow\frac{x(0)^n +x(2)^n}{x(0)^{n-1} + x(2)^{n-1}}&= \frac{x(0)+x(2)}{2}  \\
    \Rightarrow x(0)^{n-1}(x(0)-x(2))&=x(2)^{n-1}(x(0)-x(2))\label{2}
\end{align}
From \eqref{2}
\begin{align}
        \Rightarrow n
        \begin{cases}
                =1  &\text{if } a\neq b\\
                \in R &\text{if } a=b
        \end{cases}
\end{align}
From \eqref{1}\\
\begin{align}
	d&=x(1)-x(0)\\
	&=\frac{a+b}{2}-a\\
	&=\frac{b-a}{2}\label{d}
\end{align}
Using $Z$ transform.
\begin{align}
	x(n)&\Large\xleftrightarrow{\mathcal{Z}}X(z)\\
	X(z)&=\frac{a}{1-z^{-1}}+\frac{dz^{-1}}{(1-z^{-1})^2}\label{8}\\
	\notag\text{From \eqref{d}}\\
	X(z)&=\frac{a}{1-z^{-1}}+\frac{(b-a)z^{-1}}{2(1-z^{-1})^2}
\end{align}

% \end{document}

\pagebreak

\item The 17th term of ap exceeds its 10th term by 7. FInd its common difference?\\
 \solution
\iffalse
\let\negmedspace\undefined
\let\negthickspace\undefined
\documentclass[journal,12pt,onecolumn]{IEEEtran}
\usepackage{cite}
\usepackage{amsmath,amssymb,amsfonts,amsthm}
\usepackage{algorithmic}
\usepackage{graphicx}
\usepackage{textcomp}
\usepackage{xcolor}
\usepackage{txfonts}
\usepackage{listings}
\usepackage{enumitem}
\usepackage{mathtools}
\usepackage{gensymb}
\usepackage{comment}
\usepackage[breaklinks=true]{hyperref}
\usepackage{tkz-euclide} 
\usepackage{listings}
\usepackage{gvv}                                        
\def\inputGnumericTable{}                                 
\usepackage[latin1]{inputenc}                                
\usepackage{color}                                            
\usepackage{array}                                            
\usepackage{longtable}                                       
\usepackage{calc}                                             
\usepackage{multirow}                                         
\usepackage{hhline}                                           
\usepackage{ifthen}                                           
\usepackage{lscape}

\newtheorem{theorem}{Theorem}[section]
\newtheorem{problem}{Problem}
\newtheorem{proposition}{Proposition}[section]
\newtheorem{lemma}{Lemma}[section]
\newtheorem{corollary}[theorem]{Corollary}
\newtheorem{example}{Example}[section]
\newtheorem{definition}[problem]{Definition}
\newcommand{\BEQA}{\begin{eqnarray}}
 \newcommand{\EEQA}{\end{eqnarray}}
\newcommand{\define}{\stackrel{\triangle}{=}}
\theoremstyle{remark}
\newtheorem{rem}{Remark}
\begin{document}
 \bibliographystyle{IEEEtran}
 \vspace{3cm}
 \title{\textbf{10.5.2.11}}
 \author{EE23BTECH11048-Ponugumati Venkata Chanakya$^{*}$% <-this % stops a space
 }
 \maketitle

 \bigskip
 \renewcommand{\thefigure}{\theenumi}
 \renewcommand{\thetable}{\theenumi}
 \textbf{QUESTION:}
 The 17th term of ap exceeds its 10th term by 7. FInd its common difference?\\
 \solution
\fi
 \begin{align}
     x(n) &= \{x(0)+nd\}u(n) \label{eq 10.5.2.11_1}\\
     x(17)-x(10) &= 7\\
    \implies {x(0)+17d}-{x(0)+10d} &= 7\\
    \implies 17d-10d &= 7\\
    \implies 7d &= 7\\
    \implies d &= 1
 \end{align}

 
 \begin{table}[!ht]
    \centering
        
      \begin{tabular}{|c|c|c|} 
      \hline
\textbf{Variable}& \textbf{Description}& \textbf{Value}\\\hline
         $x(n)$& $n^{th}$ term of AP&none\\\hline
          $d$&common difference between the terms of AP&none\\\hline
          $x(17)-x(10)$& difference of $17^{th}$  and $10^{th}$ term of X &$7$ \\ \hline
         
    \end{tabular}

    \caption{input parameters}
    \label{tab:10_5_2_11}
\end{table}
Taking Z-Transform:
\begin{enumerate}
    \item $\mathcal{Z}\{u(n)\}$
\begin{align}
    u(n) \system{Z} \frac{1}{1-z^{-1}} \{\abs{z} > 1\} \label{eq 10.5.2.11_7}
\end{align}
    \item $\mathcal{Z}\{nu(n)\}$ 
\begin{align}
    nu(n) \system{Z} \frac{z^{-1}}{(1-z^{-1})^2}\, \{\abs{z} > 1\} \label{eq 10.5.2.11_8} 
\end{align}0
Taking Z-Transform of \eqref{eq 10.5.2.11_1} using \eqref{eq 10.5.2.11_7}and \eqref{eq 10.5.2.11_8}
\begin{align}
    X(n)=100\frac{1}{1-z^{-1}} +\frac{z^{-1}}{(1-z^{-1})^2}\
\end{align}
\end{enumerate}
Let \\
\begin{align}
x(n)&= \lbrace 101,102,103,...\rbrace 
\end{align}
\begin{figure}
    \centering
    \includegraphics{ncert-maths/10/5/2/11/figs/fig1.png}
    \caption{ }
\end{figure}
 
 %\end{document}

 \pagebreak
 \item If $p^{th},q^{th},r^{th} $ term of a GP are $a,b$ and $c$  respectively Prove that \\
\begin{align*}
    a^{q-r}b^{r-p}c^{p-q}=1
\end{align*}
\solution
\iffalse
\let\negmedspace\undefined
\let\negthickspace\undefined
\documentclass[journal,12pt,twocolumn]{IEEEtran}
\usepackage{cite}
\usepackage{amsmath,amssymb,amsfonts,amsthm}
\usepackage{algorithmic}
\usepackage{graphicx}
\usepackage{textcomp}
\usepackage{xcolor}
\usepackage{txfonts}
\usepackage{listings}
\usepackage{enumitem}
\usepackage{mathtools}
\usepackage{gensymb}
\usepackage{comment}
\usepackage[breaklinks=true]{hyperref}
\usepackage{tkz-euclide} 
\usepackage{listings}
\usepackage{gvv}                                        
\def\inputGnumericTable{}                                 
\usepackage[latin1]{inputenc}                                
\usepackage{color}                                            
\usepackage{array}                                            
\usepackage{longtable}                                       
\usepackage{calc}                                             
\usepackage{multirow}                                         
\usepackage{hhline}                                           
\usepackage{ifthen}                                           
\usepackage{lscape}

\newtheorem{theorem}{Theorem}[section]
\newtheorem{problem}{Problem}
\newtheorem{proposition}{Proposition}[section]
\newtheorem{lemma}{Lemma}[section]
\newtheorem{corollary}[theorem]{Corollary}
\newtheorem{example}{Example}[section]
\newtheorem{definition}[problem]{Definition}
\newcommand{\BEQA}{\begin{eqnarray}}
 \newcommand{\EEQA}{\end{eqnarray}}
\newcommand{\define}{\stackrel{\triangle}{=}}
\theoremstyle{remark}
\newtheorem{rem}{Remark}
\begin{document}
 \bibliographystyle{IEEEtran}
 \vspace{3cm}
 \title{\textbf{11.9.3.22}}
 \author{EE23BTECH11048-Ponugumati Venkata Chanakya$^{*}$% <-this % stops a space
 }
 \maketitle
 \newpage
 \bigskip
 \renewcommand{\thefigure}{\theenumi}
 \renewcommand{\thetable}{\theenumi}
 \textbf{QUESTION:}
If $p^{th},q^{th},r^{th} $ term of a GP are $a,b$ and $c$  respectively Prove that \\
\begin{align*}
    a^{q-r}b^{r-p}c^{p-q}=1
\end{align*}
\solution
\fi
\begin{align}
x(n)&=(x(0)d^n) u (n) \label{eq 11.9.3.22_1}\\
a&=x(p) = (x(0)d^p)\\
b&=x(q) = (x(0)d^q)\\
c&=x(r) = (x(0)d^r)\\
a^{q-r}b^{r-p}c^{p-q}&=x(0)^{q-r} d^{p(q-r)} x(0)^{r-p} d^{q(r-p)} x(0)^{p-q} d^{r(p-q)} \\
&= x(0)^{q-r+r-p+p-q} d^{p(q-r)+q(r-p)+r(p-q)}\\
&=x(0)^0 d^0\\
a^{q-r}b^{r-p}c^{p-q} &=1
\end{align}\

 \begin{table}[!ht]
    \centering
        \begin{tabular}{|c|c|c|} 
      \hline
\textbf{Variable}& \textbf{Description}& \textbf{Value}\\\hline
         $x(n)$& $n^{th}$ term of GP&none\\\hline
         $x(0)$& First term of GP&none\\\hline
          $d$&common ratio between the terms of GP&none\\\hline
          $x(p)$& a &$x(0)d^p$ \\ \hline
          $x(q)$& b &$x(0)d^q$ \\ \hline
          $x(r)$& c &$x(0)d^r$ \\ \hline
    \end{tabular}

    \caption{input parameters}
    \label{tab:11_9_3_22}
\end{table}

Taking Z-Transform:
\begin{enumerate}
    \item $\mathcal{Z}\{u(n)\}$
\begin{align}
    u(n) \system{Z} \frac{1}{1-z^{-1}} \{\abs{z} > 1\}\label{eq 11.9.3.22_9} 
\end{align}
    \item $\mathcal{Z}\{d^{n}u(n)\}$ 
\begin{align}
    nu(n) \system{Z} \frac{z^{-1}}{(1-dz^{-1})}\, \{\abs{z} > \abs{d}\} \label{eq 11.9.3.22_10}
    \end{align}
    Taking Z-Transform of \eqref{eq 11.9.3.22_1} using \eqref{eq 11.9.3.22_9}and \eqref{eq 11.9.3.22_10}
    \begin{align}
    X(z) &= \frac{x(0)}{1-dz^{-1}} \qquad |z| > |d|
     \end{align}
\end{enumerate}
%\end{document}

\pagebreak

\item Write the first five terms of the sequence whose $n^{th}$ \text{term is} : $x(n) = (-1)^{n-1}5^{n+1}$.\\
\solution
% \iffalse
\let\negmedspace\undefined
\let\negthickspace\undefined
\documentclass[journal,12pt,twocolumn]{IEEEtran}
\usepackage{cite}
\usepackage{amsmath,amssymb,amsfonts,amsthm}
\usepackage{algorithmic}
\usepackage{graphicx}
\usepackage{textcomp}
\usepackage{xcolor}
\usepackage{txfonts}
\usepackage{listings}
\usepackage{enumitem}
\usepackage{mathtools}
\usepackage{gensymb}
\usepackage{comment}
\usepackage[breaklinks=true]{hyperref}
\usepackage{tkz-euclide} 
\usepackage{listings}
\usepackage{gvv}                                        
\def\inputGnumericTable{}                                 
\usepackage[latin1]{inputenc}                                
\usepackage{color}                                            
\usepackage{array}                                            
\usepackage{longtable}                                       
\usepackage{calc}                                             
\usepackage{multirow}                                         
\usepackage{hhline}                                           
\usepackage{ifthen}                                           
\usepackage{lscape}
\usepackage{caption}
\newtheorem{theorem}{Theorem}[section]
\newtheorem{problem}{Problem}
\newtheorem{proposition}{Proposition}[section]
\newtheorem{lemma}{Lemma}[section]
\newtheorem{corollary}[theorem]{Corollary}
\newtheorem{example}{Example}[section]
\newtheorem{definition}[problem]{Definition}
\newcommand{\BEQA}{\begin{eqnarray}}
\newcommand{\EEQA}{\end{eqnarray}}
\newcommand{\define}{\stackrel{\triangle}{=}}
\theoremstyle{remark}
\newtheorem{rem}{Remark}
\begin{document}
\parindent 0px
\bibliographystyle{IEEEtran}
\vspace{3cm}

\title{NCERT 11.9.3 1Q}
\author{EE23BTECH11013 - Avyaaz$^{*}$% <-this % stops a space
}
\maketitle
\newpage
\bigskip

\renewcommand{\thefigure}{\arabic{figure}}
\renewcommand{\thetable}{\arabic{table}}
\large\textbf{\textsl{Question:}}
Find the $20^{th}$ and $n^{th}$ terms of the G.P $\frac{5}{2}$, $\frac{5}{4}$, $\frac{5}{8}$,.....

\solution
 \begin{table}[htbp]
     \centering
     \setlength{\extrarowheight}{8pt}
    \begin{tabular}{|c|c|c|}
\hline 
   \textbf{Parameter}  &\textbf{Description} &\textbf{Value} \\
\hline
&&\\
$I_r$&Net Intensity of light at $\Delta x =\dfrac{\lambda}{3}$ &$\dfrac{K}{4}$ \\&&\\
\hline
\end{tabular}

     \caption{Parameters}
     \label{tab:table1.11.9.3.1}
 \end{table} 

% \begin{align}
%    x(n) = \dfrac{5}{2}\left(\dfrac{1}{2}\right)^n 
% \end{align}

% \begin{align}
% 	x \brak{n} & \system{Z} X \brak{z} \\
%    % x(n) &=\dfrac{5}{2}\left(\dfrac{1}{2}\right)^n u(n) \\
%     \therefore X(z) &= \sum_{n=-\infty}^{\infty}x(n)z^{-n}\label{eq:z-transform}  
% \end{align}
% Here, 
%          $    u(n) = \begin{cases}
%                 0 &\text{for } n < 0 \\
%                 1 & \text{for } n \geq 0
%             \end{cases}$       
 
%  \vspace{1cm}
From \tabref{tab:table1.11.9.3.1}:
\(Z\)-Transform of \(x(n)\):
\begin{align}
% \implies X(z) &= \sum_{n=-\infty}^{\infty}\left(\dfrac{5}{2}\left(\dfrac{1}{2}\right)^n u(n)\right) z^{-n} \\
 % \implies X(z) &= \dfrac{5}{2}\sum_{n=0}^{\infty}\left(\dfrac{z
 % ^{-1}}{2}\right)^n \\
\implies X(z) &=\dfrac{5}{2}\left(\dfrac{1}{1-\frac{z^{-1}}{2}}\right) ;\cbrak{z\in\mathbb{C} : |z|>\dfrac{1}{2}}
\end{align}

\begin{figure}[ht]
    \centering
    \includegraphics[width = \columnwidth]{figs/stem_plot.png}
    \caption{}
	\label{fig:graph1.11.9.3.1}
\end{figure} 

\bibliographystyle{IEEEtran}
\end{document}

\pagebreak
\item The ratio of sums of m and n terms of an A.P. is $m^2:n^2$.Show
that the ratio of $m^{th}$ and $n^{th}$ term is (2m-1):(2n-1).\\
\solution
\pagebreak

\item If $a$ and $b$ are the roots of $x^{2} -3x + p = 0$ and $c$ , $d$ are roots of $x^{2} - 12x + q = 0$ where $a,b,c,d$ form a G.P. Prove that $(q+p) : (q-p)$ = 17:15 .\\
\solution
\pagebreak


\item Write the first five terms in the sequence defined recursively as follows:
\[ a_{0} = 3 \]
\[ a_{n} = 3a_{n-1} + 2 \quad \text{for } n > 0 \]
\solution 
\pagebreak


\item \begin{align}
\frac{a+bx}{a-bx}=\frac{b+cx}{b-cx}=\frac{c+dx}{c-dx}
\end{align}
then show that a,b,c,d are in G.P\\
\solution
\iffalse
\let\negmedspace\undefined
\let\negthickspace\undefined
\documentclass[journal,12pt,twocolumn]{IEEEtran}
\usepackage{cite}
\usepackage{amsmath,amssymb,amsfonts,amsthm}
\usepackage{algorithmic}
\usepackage{graphicx}
\usepackage{textcomp}
\usepackage{xcolor}
\usepackage[justification=centering]{caption}
\usepackage{txfonts}
\usepackage{listings}
\usepackage{enumitem}
\usepackage{mathtools}
\usepackage{gensymb}
\usepackage{comment}
\usepackage[breaklinks=true]{hyperref}
\usepackage{tkz-euclide} 
\usepackage{listings}
\usepackage{gvv}                                        
\def\inputGnumericTable{}                                 
\usepackage[latin1]{inputenc}                                
\usepackage{color}                                            
\usepackage{array}                                            
\usepackage{longtable}                                       
\usepackage{calc}                                             
\usepackage{multirow}                                         
\usepackage{hhline}                                           
\usepackage{ifthen}                                           
\usepackage{lscape}

\newtheorem{theorem}{Theorem}[section]
\newtheorem{problem}{Problem}
\newtheorem{proposition}{Proposition}[section]
\newtheorem{lemma}{Lemma}[section]
\newtheorem{corollary}[theorem]{Corollary}
\newtheorem{example}{Example}[section]
\newtheorem{definition}[problem]{Definition}
\newcommand{\BEQA}{\begin{eqnarray}}
\newcommand{\EEQA}{\end{eqnarray}}
\newcommand{\define}{\stackrel{\triangle}{=}}
\theoremstyle{remark}
\newtheorem{rem}{Remark}
\begin{document}

\bibliographystyle{IEEEtran}
\vspace{3cm}

\title{11.9.5-13}
\author{EE23BTECH11033-killana jaswanth}
\maketitle
\newpage

\bigskip

\renewcommand{\thefigure}{\theenumi}
\renewcommand{\thetable}{\theenumi}
question:\begin{align}
\frac{a+bx}{a-bx}=\frac{b+cx}{b-cx}=\frac{c+dx}{c-dx}
\end{align}
then show that a,b,c,d are in G.P\\\\
solution:\\
\fi
      let,
\begin{align}
\frac{b}{a}=\frac{c}{b}=\frac{d}{c}=r
\end{align}
\\\begin{table}[!ht]
 \centering
  \begin{tabular}{|c|c|c|}
\hline
\textbf{parameter}& \textbf{description}& \textbf{value}
\\\hline
\multirow{3}{1em}\\$x\brak{0}$&first term&$a$
\\\hline
$x\brak{1}$&second term&$b$
\\\hline
$x\brak{2}$&third term&$c$
\\\hline
$x\brak{3}$&fourth term&$d$
\\\hline
$r$&common ratio&$\frac{b}{a}$
\\\hline
$n$&no of terms&$4$
\\\hline
$x\brak{n}$&$n/^{th}$ term&$x\brak{0}r^{n}$
\\\hline
\end{tabular}



   \caption{input parameters}
   \label{tab:11.9.5.13}
   \end{table}
\begin{align}
\frac{a+bx}{a-bx}&=\frac{a+arx}{a-arx}\\
&=\frac{1+rx}{1-rx}\\
\frac{b+cx}{b-cx}&=\frac{ar+ar^2x}{ar-ar^2x}\\
&=\frac{1+rx}{1-rx}\\
\frac{c+dx}{c-dx}&=\frac{ar^2+ar^3x}{ar^2-ar^3x}\\
&=\frac{1+rx}{1-rx}
\end{align}
As, equations\begin{align} \brak{4}=\brak{6}=\brak{8}
\end{align}
so, a,b,c,d are in G.P\\\\
Applying z-transform\\
\begin{align}
X\brak{z}&=\frac{a^2}{a-bz^{-1}} \quad \abs{z}>\abs{\frac{b}{a}}
\end{align}
%\end{document}

\pagebreak


\item Sum of the first p, q and r terms of an A.P. are a, b and c, respectively.\\
Prove that $\dfrac{a}{p}\brak{q-r}+\dfrac{b}{q}\brak{r-p}+\dfrac{c}{r}\brak{p-q}=0$\hfill{NCERT-discrete 11.9.2.11}\\
\solution
% \iffalse
\let\negmedspace\undefined
\let\negthickspace\undefined
\documentclass[journal,12pt,twocolumn]{IEEEtran}
\usepackage{cite}
\usepackage{amsmath,amssymb,amsfonts,amsthm}
\usepackage{algorithmic}
\usepackage{graphicx}
\usepackage{textcomp}
\usepackage{xcolor}
\usepackage{txfonts}
\usepackage{listings}
\usepackage{enumitem}
\usepackage{mathtools}
\usepackage{gensymb}
\usepackage{comment}
\usepackage[breaklinks=true]{hyperref}
\usepackage{tkz-euclide} 
\usepackage{listings}
\usepackage{gvv}                                        
\def\inputGnumericTable{}                                 
\usepackage[latin1]{inputenc}                                
\usepackage{color}                                            
\usepackage{array}                                            
\usepackage{longtable}                                       
\usepackage{calc}                                             
\usepackage{multirow}                                         
\usepackage{hhline}                                           
\usepackage{ifthen}                                           
\usepackage{lscape}

\newtheorem{theorem}{Theorem}[section]
\newtheorem{problem}{Problem}
\newtheorem{proposition}{Proposition}[section]
\newtheorem{lemma}{Lemma}[section]
\newtheorem{corollary}[theorem]{Corollary}
\newtheorem{example}{Example}[section]
\newtheorem{definition}[problem]{Definition}
\newcommand{\BEQA}{\begin{eqnarray}}
\newcommand{\EEQA}{\end{eqnarray}}
\newcommand{\define}{\stackrel{\triangle}{=}}
\theoremstyle{remark}
\newtheorem{rem}{Remark}
\begin{document}
\parindent 0px

\bibliographystyle{IEEEtran}
\vspace{3cm}

\title{Assignment\\[1ex]11.9.2 - 11}
\author{EE23BTECH11034 - Prabhat Kukunuri$^{}$% <-this % stops a space
}
\maketitle
\newpage
\bigskip

\renewcommand{\thefigure}{\theenumi}
\renewcommand{\thetable}{\theenumi}
\section*{Question}
Sum of the first p, q and r terms of an A.P. are a, b and c, respectively.

Prove that $\dfrac{a}{p}\brak{q-r}+\dfrac{b}{q}\brak{r-p}+\dfrac{c}{r}\brak{p-q}=0$
\section*{Solution}
\begin{table}[h]
    \centering
    \begin{tabular}{|c|c|c|}
\hline 
   \textbf{Parameter}  &\textbf{Description} &\textbf{Value} \\
\hline
&&\\
$I_r$&Net Intensity of light at $\Delta x =\dfrac{\lambda}{3}$ &$\dfrac{K}{4}$ \\&&\\
\hline
\end{tabular}

    \caption{Variable description}
    \label{tab:11.9.2.11.1}
\end{table}
\begin{align}
    y\brak{n}&=\dfrac{n+1}{2}\brak{2x\brak{0}+nd}u\brak{n}
\end{align}
Using y\brak{n},
\begin{align}
    a&=\dfrac{p}{2}\brak{2x\brak{0}+\brak{p-1}d}\label{eq:2}\\
    b&=\dfrac{q}{2}\brak{2x\brak{0}+\brak{q-1}d}\label{eq:3}\\
    c&=\dfrac{r}{2}\brak{2x\brak{0}+\brak{r-1}d}\label{eq:4}
\end{align}
which can be represented as,
\begin{align}
    &p.x\brak{0}+\dfrac{p\brak{p-1}}{2}.d+a.\brak{-1}=0\\
    &q.x\brak{0}+\dfrac{q\brak{q-1}}{2}.d+b.\brak{-1}=0\\
    &r.x\brak{0}+\dfrac{r\brak{r-1}}{2}.d+c.\brak{-1}=0
\end{align}
resulting in the matrix equation,
\begin{align}
    \myvec{p&\frac{p\brak{p-1}}{2}&a\\q&\frac{q\brak{q-1}}{2}&b\\r&\frac{r\brak{r-1}}{2}&c\\}\vec{x}=0\label{eq:8}
\end{align}
where,
\begin{align}
    \vec{x}=\myvec{x\brak{0}\\d\\-1}
\end{align}
solving the equations \eqref{eq:2},\eqref{eq:3} and \eqref{eq:4} by row reducing the matrix in $\eqref{eq:8}$,
    \begin{align}
    \myvec{
        p&\frac{p\brak{p-1}}{2}&a\\
        q&\frac{q\brak{q-1}}{2}&b\\
        r&\frac{r\brak{r-1}}{2}&c\\
    }
    \xleftrightarrow[R_{1}\leftarrow\frac{R_{1}}{p}, R_{2}\leftarrow\frac{R_{2}}{q}]{R_{3}\leftarrow\frac{R_{3}}{r}} 
    \myvec{
        1&\frac{p-1}{2}&\frac{a}{p}\\
        1&\frac{q-1}{2}&\frac{b}{q}\\
        1&\frac{r-1}{2}&\frac{c}{r}\\
    }\\
   \xleftrightarrow[R_{2}\leftarrow R_{2}-R_{1}]{R_{3}\leftarrow R_{3}-R_{1}} 
    \myvec{
        1&\frac{p-1}{2}&\frac{a}{p}\\
        0&\frac{q-p}{2}&\frac{b}{q}-\frac{a}{p}\\
        0&\frac{r-p}{2}&\frac{c}{r}-\frac{a}{p}\\
    }\\
    \xleftrightarrow{R_2\leftarrow\frac{R_{2}}{\frac{q-p}{2}}}
    \myvec{
        1&\frac{p-1}{2}&\frac{a}{p}\\
        0&1&\brak{\frac{b}{q}-\frac{a}{p}}\frac{2}{q-p}\\
        0&\frac{r-p}{2}&\frac{c}{r}-\frac{a}{p}\\
    }\\
    \xleftrightarrow[R_{1}\leftarrow R_{1}-\frac{p-1}{2}R_{2}]{R_{3}\leftarrow R_{3}-\frac{r-p}{2}R_{2}}
    \myvec{
        1&0&\frac{a}{p}-\frac{\brak{\frac{b}{q}-\frac{a}{p}}\brak{p-1}}{q-p}\\
        0&1&\brak{\frac{b}{q}-\frac{a}{p}}\frac{2}{q-p}\\
        0&0&\brak{\frac{c}{r}-\frac{a}{p}}-\frac{\brak{\frac{b}{q}-\frac{a}{p}}\brak{r-p}}{q-p}\\
    }\\
    \implies
    \myvec{
        1&0&\frac{aq\brak{q-1}-bp\brak{p-1}}{pq\brak{q-p}}\\
        0&1&\brak{\frac{b}{q}-\frac{a}{p}}\frac{2}{q-p}\\
        0&0&\frac{\frac{a}{p}\brak{r-q}+\frac{b}{q}\brak{p-r}+\frac{c}{r}\brak{q-p}}{q-p}\\
    }
\end{align}
After row reduction of matrix we get,
\begin{align}
    x\brak{0}=\brak{\frac{aq\brak{q-1}-bp\brak{p-1}}{pq\brak{q-p}}}\\
    d=\brak{\frac{b}{q}-\frac{a}{p}}\frac{2}{q-p}\\
    \frac{\frac{a}{p}\brak{r-q}+\frac{b}{q}\brak{p-r}+\frac{c}{r}\brak{q-p}}{q-p}=0\\
    \therefore{\frac{a}{p}\brak{q-r}+\frac{b}{q}\brak{r-p}+\frac{c}{r}\brak{p-q}}=0
\end{align}
\begin{align}
    &x \brak{n} \system{Z} X \brak{z}\\
    &X\brak{z}=\frac{aq\brak{q-1}-bp\brak{p-1}}{pq\brak{q-p}\brak{1-z^{-1}}}+\frac{2\brak{\frac{b}{q}-\frac{a}{p}}z^{-1}}{\brak{q-p}\brak{1-z^{-1}}^{2}}\\
    &R.O.C\brak{|z|>1}
\end{align}
\begin{figure}[ht]
    \centering
    \includegraphics[width=\columnwidth]{figs/Figure_1.png}
    \caption{Plot of x(n) $vs$ n}
    \label{fig:11.9.2.11.2}
\end{figure}
\begin{table}[ht]
    \centering
    \def\arraystretch{1.5}
    \begin{tabular}{|c|c|c|c|}
\hline
\textbf{Parameter}&\textbf{Description} &\textbf{subquestion}& \textbf{Value}\\
\hline
     \multirow{4}{*}{$\Delta \theta$} & \multirow{4}{*}{$\theta_1 - \theta_2$} &\brak{a}& 6.4$\pi$ \, radians \\
     \cline{3-4}
     & & \brak{b}& 0.8$\pi$ \, radians \\
     \cline{3-4}
     & &\brak{c}& $\pi$ \, radians \\
     \cline{3-4}
     & & \brak{d} & $\dfrac{3\pi}{2\vphantom{\brak{0.1}}}$ \, radians \\
     \hline
\end{tabular}

    \caption{Verified Values}
    \label{tab:11.9.2.11.3}
\end{table}
\end{document}
\pagebreak

\item The pth, qth and rth terms of an AP are a,b,c respectively. Show that
\begin{align*} (q-r)a + (r-p)b +(p-q)c =0 \end{align*}
\solution
\iffalse
\let\negmedspace\undefined
\let\negthickspace\undefined
\documentclass[journal,12pt,onecolumn]{IEEEtran}
\usepackage{cite}
\usepackage{amsmath,amssymb,amsfonts,amsthm}
\usepackage{algorithmic}
\usepackage{graphicx}
\usepackage{textcomp}
\usepackage{xcolor}
\usepackage{txfonts}
\usepackage{listings}
\usepackage{enumitem}
\usepackage{mathtools}
\usepackage{gensymb}

\usepackage{tkz-euclide} % loads  TikZ and tkz-base
\usepackage{listings}



\newtheorem{theorem}{Theorem}[section]
\newtheorem{problem}{Problem}
\newtheorem{proposition}{Proposition}[section]
\newtheorem{lemma}{Lemma}[section]
\newtheorem{corollary}[theorem]{Corollary}
\newtheorem{example}{Example}[section]
\newtheorem{definition}[problem]{Definition}
%\newtheorem{thm}{Theorem}[section] 
%\newtheorem{defn}[thm]{Definition}
%\newtheorem{algorithm}{Algorithm}[section]
%\newtheorem{cor}{Corollary}
\newcommand{\BEQA}{\begin{eqnarray}}
\newcommand{\EEQA}{\end{eqnarray}}
\newcommand{\system}[1]{\stackrel{#1}{\rightarrow}}

\newcommand{\define}{\stackrel{\triangle}{=}}
\theoremstyle{remark}
\newtheorem{rem}{Remark}
%\bibliographystyle{ieeetr}
\begin{document}
%
\providecommand{\pr}[1]{\ensuremath{\Pr\left(#1\right)}}
\providecommand{\prt}[2]{\ensuremath{p_{#1}^{\left(#2\right)} }}        % own macro for this question
\providecommand{\qfunc}[1]{\ensuremath{Q\left(#1\right)}}
\providecommand{\sbrak}[1]{\ensuremath{{}\left[#1\right]}}
\providecommand{\lsbrak}[1]{\ensuremath{{}\left[#1\right.}}
\providecommand{\rsbrak}[1]{\ensuremath{{}\left.#1\right]}}
\providecommand{\brak}[1]{\ensuremath{\left(#1\right)}}
\providecommand{\lbrak}[1]{\ensuremath{\left(#1\right.}}
\providecommand{\rbrak}[1]{\ensuremath{\left.#1\right)}}
\providecommand{\cbrak}[1]{\ensuremath{\left\{#1\right\}}}
\providecommand{\lcbrak}[1]{\ensuremath{\left\{#1\right.}}
\providecommand{\rcbrak}[1]{\ensuremath{\left.#1\right\}}}
\newcommand{\sgn}{\mathop{\mathrm{sgn}}}
\providecommand{\abs}[1]{\left\vert#1\right\vert}
\providecommand{\res}[1]{\Res\displaylimits_{#1}} 
\providecommand{\norm}[1]{\left\lVert#1\right\rVert}
%\providecommand{\norm}[1]{\lVert#1\rVert}
\providecommand{\mtx}[1]{\mathbf{#1}}
\providecommand{\mean}[1]{E\left[ #1 \right]}
\providecommand{\cond}[2]{#1\middle|#2}
\providecommand{\fourier}{\overset{\mathcal{F}}{ \rightleftharpoons}}
\newenvironment{amatrix}[1]{%
  \left(\begin{array}{@{}*{#1}{c}|c@{}}
}{%
  \end{array}\right)
}
%\providecommand{\hilbert}{\overset{\mathcal{H}}{ \rightleftharpoons}}
%\providecommand{\system}{\overset{\mathcal{H}}{ \longleftrightarrow}}
	%\newcommand{\solution}[2]{\textbf{Solution:}{#1}}
\newcommand{\solution}{\noindent \textbf{Solution: }}
\newcommand{\cosec}{\,\text{cosec}\,}
\providecommand{\dec}[2]{\ensuremath{\overset{#1}{\underset{#2}{\gtrless}}}}
\newcommand{\myvec}[1]{\ensuremath{\begin{pmatrix}#1\end{pmatrix}}}
\newcommand{\mydet}[1]{\ensuremath{\begin{vmatrix}#1\end{vmatrix}}}
\newcommand{\myaugvec}[2]{\ensuremath{\begin{amatrix}{#1}#2\end{amatrix}}}
\providecommand{\rank}{\text{rank}}
\providecommand{\pr}[1]{\ensuremath{\Pr\left(#1\right)}}
\providecommand{\qfunc}[1]{\ensuremath{Q\left(#1\right)}}
	\newcommand*{\permcomb}[4][0mu]{{{}^{#3}\mkern#1#2_{#4}}}
\newcommand*{\perm}[1][-3mu]{\permcomb[#1]{P}}
\newcommand*{\comb}[1][-1mu]{\permcomb[#1]{C}}
\providecommand{\qfunc}[1]{\ensuremath{Q\left(#1\right)}}
\providecommand{\gauss}[2]{\mathcal{N}\ensuremath{\left(#1,#2\right)}}
\providecommand{\diff}[2]{\ensuremath{\frac{d{#1}}{d{#2}}}}
\providecommand{\myceil}[1]{\left \lceil #1 \right \rceil }
\newcommand\figref{Fig.~\ref}
\newcommand\tabref{Table~\ref}
\newcommand{\sinc}{\,\text{sinc}\,}
\newcommand{\rect}{\,\text{rect}\,}
%%
%	%\newcommand{\solution}[2]{\textbf{Solution:}{#1}}
%\newcommand{\solution}{\noindent \textbf{Solution: }}
%\newcommand{\cosec}{\,\text{cosec}\,}
%\numberwithin{equation}{section}
%\numberwithin{equation}{subsection}
%\numberwithin{problem}{section}
%\numberwithin{definition}{section}
%\makeatletter
%\@addtoreset{figure}{problem}
%\makeatother

%\let\StandardTheFigure\thefigure
\let\vec\mathbf

\bibliographystyle{IEEEtran}





\bigskip

\renewcommand{\thefigure}{\theenumi}
\renewcommand{\thetable}{\theenumi}
%\renewcommand{\theequation}{\theenumi}


\title{Discrete Assignment}
\author{Praful Kesavadas\\ EE23BTECH11049}
\maketitle
\textbf{Question 11.9.5.15:}
The pth, qth and rth terms of an AP are a,b,c respectively. Show that
\begin{align*} \brak{q-r}a + \brak{r-p}b +\brak{p-q}c =0 \end{align*}
\solution\\
\fi
The AP has the following parameters
\begin{table}[ht!]
    \centering
    \begin{tabular}{|c|c|c|}
    \hline
    \textbf{Term} & \textbf{Value} & \textbf{Description}\\
    \hline
    $x\brak{0}$ & - & First term\\
    \hline
    $d$ & - & Common Difference\\
    \hline
    $x\brak{n}$ & $\brak{x\brak{0}+nd}u\brak{n}$ & General term\\
    \hline
    $x\brak{p}$ & $a$ & pth term\\
    \hline
    $x\brak{q}$ & $b$ & qth term\\
    \hline
    $x\brak{r}$ & $c$ & rth term\\
    \hline
  \end{tabular}
  

    \caption{Input Parameters}
\end{table}

Now,
\begin{align}
    x\brak{0}+pd &= a \label{eq:11.9.5.15.1} \\
    x\brak{0}+qd &= b \label{eq:11.9.5.15.2} \\
    x\brak{0}+rd &= c \label{eq:11.9.5.15.3} 
\end{align}
which can be represented as,
\begin{align}
    &x\brak{0}+p.d+a.\brak{-1}=0\\
    &x\brak{0}+q.d+b.\brak{-1}=0\\
    &x\brak{0}+r.d+c.\brak{-1}=0
\end{align}
resulting in the matrix equation,
\begin{align}
    \myvec{1&p&a\\1&q&b\\1&r&c\\}\vec{x}=\vec{0}\label{eq:11.9.5.15.4}
\end{align}
where,
\begin{align}
    \vec{x}=\myvec{x\brak{0}\\d\\-1}
\end{align}
solving the equations \eqref{eq:11.9.5.15.1},\eqref{eq:11.9.5.15.2} and \eqref{eq:11.9.5.15.3} by row reducing the matrix in $\eqref{eq:11.9.5.15.4}$,
    \begin{align}
    \myvec{
        1&p&a\\
        1&q&b\\
        1&r&c\\
    }
    &\xleftrightarrow[R_{2}\leftarrow R_2 - R_1]{ R_{3}\leftarrow R_3 - R_1}
    \myvec{
        1&p&a\\
        0&q-p&b-a\\
        0&r-p&c-a\\
    }\\
   &\xleftrightarrow{R_{2}\leftarrow \frac{R_2}{q-p}} 
    \myvec{
        1&p&a\\
        0&1&\frac{b-a}{q-p}\\
        0&r-p&c-a\\
    }\\
    &\xleftrightarrow{R_1\leftarrow R_1 - p.R_2}
    \myvec{
       1&0&a- p. \frac{b-a}{q-p}\\
       0&1&\frac{b-a}{q-p}\\
       0&r-p&c-a\\
    }\\
    &\xleftrightarrow{R_{3}\leftarrow R_{3}-\brak{r-p}.R_2}
    \myvec{
       1&0&a- p. \frac{b-a}{q-p}\\
       0&1&\frac{b-a}{q-p}\\
       0&0&\brak{c-a} - \frac{\brak{r-p}\brak{b-a}}{q-p}\\
    }\\
    &\implies
    \myvec{
          1&0&\frac{aq-pb}{q-p}\\
       0&1&\frac{b-a}{q-p}\\
       0&0&\frac{a\brak{r-q} + b\brak{p-r} + c\brak{q-p}}{q-p}\\
    }
\end{align}
After row reduction of matrix we get,
\begin{align}
    x\brak{0}=\frac{aq-pb}{q-p}\\
    d=\frac{b-a}{q-p}\\
    \frac{a\brak{r-q} + b\brak{p-r} + c\brak{q-p}}{q-p}=0\\
    \therefore{\brak{q-r}a + \brak{r-p}b +\brak{p-q}c =0}
\end{align}
\begin{align}
    &x \brak{n} \system{Z} X \brak{z}\\
    &X\brak{z}=\frac{aq-pb}{\brak{q-p}\brak{1-z^{-1}}}+\frac{\brak{b-a}z^{-1}}{\brak{q-p}\brak{1-z^{-1}}^{2}}\\
    &R.O.C\brak{|z|>1}
\end{align}
Hence proved
%\end{document}

\pagebreak

\item Find the sum to indicated number of term in each of the geometric progressions in $\sqrt{7} ,\sqrt{21} , 3\sqrt{7}, ....n$ terms\\
\solution
\let\negmedspace\undefined
\let\negthickspace\undefined
\documentclass[a4,12pt,onecolumn]{IEEEtran}
\usepackage{amsmath,amssymb,amsfonts,amsthm}
\usepackage{algorithmic}
\usepackage{graphicx}
\usepackage{textcomp}
\usepackage{xcolor}
\usepackage{txfonts}
\usepackage{listings}
\usepackage{enumitem}
\usepackage{mathtools}
\usepackage{gensymb}
\usepackage[breaklinks=true]{hyperref}
\usepackage{tkz-euclide}
\usepackage{listings}
\usepackage{gvv}
\begin{document}
\title{
\Huge\textbf{Discrete Assignment}\\
\Huge\textbf{EE1205} Signals and Systems\\
}
\large\author{Kurre Vinay\\EE23BTECH11036}
\maketitle
\textbf{Question 11.9.3.8:}
Find the sum to indicated number of term in each of the geometric progressions in $\sqrt{7} ,\sqrt{21} , 3\sqrt{7}, ....n$ terms\\
\solution
\begin{table}[h!]
 \begin{center}
\begin{tabular}{|c|c|c|}
   \hline
   variable&value&description  \\
   \hline
   $x(0)$ & $ \sqrt{7} $& first term of the geometric progession\\
   \hline
   $r$ & $\sqrt{3}$ & common ratio of the geometeric progression\\
   \hline
   $x(n)$ & $\sqrt{7(3^{n})}u\brak{n}$& $n^{th}$ term of the geometric progession\\
   \hline
   $y(n)$ &$\frac{x(0)(r^{n+1}-1)}{r-1}u\brak{n}$ &Sum of the n term of the geometric progression\\
   \hline 
\end{tabular}
\caption{Input parameters}
\end{center}
\end{table}

\begin{align}
X\brak{z} &= x\brak{0}\brak{\frac{1}{1-rz^{-1}}}, \quad{|rz^{-1}|<1}\\
y\brak{n} &= x\brak{n}*u\brak{n}\\
Y\brak{z} &= X\brak{z}U\brak{z}\\
&=\sqrt{7}\brak{\frac{1}{1-\sqrt{3}z^{-1}}}\brak{\frac{1}{1-z^{-1}}} ,\quad{|z|>\sqrt{3}}\\
&=\brak{\frac{\sqrt{7}}{\sqrt{3}-1}}\brak{\brak{\frac{\sqrt{3}}{1-\sqrt{3}z^{-1}}}-\brak{\frac{1}{1-z^{-1}}}}\\
\frac{1}{1-rz^{-1}} &\xleftrightarrow{\mathcal{Z}^{-1}}  r^nu(n), \quad{|z|>r}\\
y\brak{n} &= \sqrt{7}\brak{\frac{\sqrt{3}^{n+1}-1}{\sqrt{3}-1}}u(n) , \quad{|z|>\sqrt{3}}
\end{align}

\begin{figure}[ht!]
\includegraphics[width=\columnwidth]{figs/fig2.png}
\caption{\large{STEM PLOT OF $y\brak{n}$}}
\end{figure}
\end{document}


\item How many multiples of 4 lie between 10 and 250?\\
\solution
\pagebreak

\item if $a,b,c$ and $d$ are in GP then show that $(a^{2}+b^{2}+c^{2})(b^{2}+c^{2}+d^{2})=(ab+bc+cd)^{2}$\\
\solution
\pagebreak

\item In an A.P. the first term is 2 and the sum of the first five terms is one-fourth of the next five terms. Show that 20\textsuperscript{th} term is $-112$. \hfill(NCERT MATHS 11.9.2.3)\\
\solution
% \iffalse
\let\negmedspace\undefined
\let\negthickspace\undefined
\documentclass[journal,12pt,twocolumn]{IEEEtran}
\usepackage{cite}
\usepackage{amsmath,amssymb,amsfonts,amsthm}
\usepackage{algorithmic}
\usepackage{graphicx}
\usepackage{textcomp}
\usepackage{xcolor}
\usepackage{txfonts}
\usepackage{listings}
\usepackage{enumitem}
\usepackage{mathtools}
\usepackage{gensymb}
\usepackage{comment}
\usepackage[breaklinks=true]{hyperref}
\usepackage{tkz-euclide} 
\usepackage{listings}
\usepackage{gvv}                                        
\def\inputGnumericTable{}                                 
\usepackage[latin1]{inputenc}                                
\usepackage{color}                                            
\usepackage{array}                                            
\usepackage{longtable}                                       
\usepackage{calc}                                             
\usepackage{multirow}                                         
\usepackage{hhline}                                           
\usepackage{ifthen}                                           
\usepackage{lscape}

\newtheorem{theorem}{Theorem}[section]
\newtheorem{problem}{Problem}
\newtheorem{proposition}{Proposition}[section]
\newtheorem{lemma}{Lemma}[section]
\newtheorem{corollary}[theorem]{Corollary}
\newtheorem{example}{Example}[section]
\newtheorem{definition}[problem]{Definition}
\newcommand{\BEQA}{\begin{eqnarray}}
\newcommand{\EEQA}{\end{eqnarray}}
\newcommand{\define}{\stackrel{\triangle}{=}}
\theoremstyle{remark}
\newtheorem{rem}{Remark}

\begin{document}

\bibliographystyle{IEEEtran}
\vspace{3cm}

\title{NCERT 11.9.2.3}
\author{EE23BTECH11043 - BHUVANESH SUNIL NEHETE$^{*}$% <-this % stops a space
}
\maketitle
\newpage
\bigskip

\renewcommand{\thefigure}{\theenumi}
\renewcommand{\thetable}{\theenumi}

\bibliographystyle{IEEEtran}

\textbf{Question:}

In an A.P. the first term is 2 and the sum of the first five terms is one-fourth of the next five terms. Show  that 20\textsuperscript{th} term is $-112$.

\solution
\fi
\begin{table}[h]
\renewcommand\thetable{1}
    \centering
    \begin{tabular}{|c|c|c|}
        \hline
        \textbf{Parameter} & \textbf{Description} & \textbf{Value}\\
        \hline
        $x(0)$ & First term & $2$\\
        \hline
        $x(19)$ & $20\textsuperscript{th}$ term & $-112$\\
        \hline
        $y(n)$ & sum upto $n\textsuperscript{th}$ term & \\
        \hline
    \end{tabular}
    \caption{Input data}
  \label{input data}
\end{table}


General term can be written as
\begin{align}
    x\brak{n} = \brak{x\brak{0} + nd}u\brak{n}
\end{align}
By referreing \eqref{eq:apz}
\begin{align}
    X(z) &= \frac{x(0)}{1-z^{-1}} + \frac{dz^{-1}}{(1-z^{-1})^{2}}\label{eq2}
\end{align}
Taking the inverse Z-transform by contour integration by refering \eqref{eq:APSum},
\begin{align}
    y(n) &= x(0)\sbrak{(n + 1)u(n)} + \frac{d}{2}\sbrak{n(n + 1)u(n)}\\
    &= \frac{n+1}{2}\cbrak{2x(0) + nd}u(n)
\end{align}
Therefore, 
\begin{align}
    y\brak{4}=5x\brak{0}+10d\\
    y\brak{9}=10x\brak{0}+45d
\end{align}
Given, 
   \begin{align}
       \sum_{n=0}^{4}x\brak{n} = \frac{1}{4}\sum_{n=5}^{9}x\brak{n}
   \end{align}
Simplifying:
    \begin{align}
        y\brak{4} &= \frac{1}{4}\brak{y\brak{9}-y\brak{4}}\\
        \implies 5x\brak{0} + 10d &= \frac{1}{4}\brak{5x\brak{0} + 35d}\\
        x\brak{0} &= \frac{-d}{3}\\
        \implies d &= -6 \label{eq1}
    \end{align}
From \eqref{eq1} and \tabref{input data}:
    \begin{align}
        x\brak{n}&=\brak{2-6n}u\brak{n} \label{eq3}
   \end{align} 
From \eqref{eq3}:
    \begin{align}
        x\brak{19}&=x\brak{0}+19d\\ 
        &= -112
    \end{align}    
From \eqref{eq3} and \eqref{eq2}:
    \begin{align}
        X\brak{z}=\frac{2}{1-z^{-1}} - \frac{6z^{-1}}{\brak{1-z^{-1}}^{2}} \quad |z|>1
    \end{align}

    \begin{figure}[ht]
    \renewcommand\thefigure{1}
        \centering
        \includegraphics[width=1\linewidth]{ncert-maths/11/9/2/3/figs/Figure_1.png}
        \caption{graph of $x\brak{n} = 2 - 6n$}
    \end{figure}

%\end{document}

\pagebreak
\item If the 3rd and the 9th terms of an AP are 4 and -8, respectively, which term of this AP is zero? \\
\solution
\pagebreak
\item Find the sum of the products of the corresponding terms of the sequences $2, 4, 8, 16, 32$ and $128, 32, 8, 2, \frac{1}{2}$.
\solution
\pagebreak

\item Let the sum of $n,2n,3n$ terms of an AP be $S_1,S_2$ and $S_3$, respectively, show that $S_3=3(S_2-S_1)$\\
\solution
\iffalse
\let\negthickspace\undefined
\documentclass[journal,12pt,twocolumn]{IEEEtran}
\usepackage{cite}
\usepackage{amsmath,amssymb,amsfonts,amsthm}
\usepackage{algorithmic}
\usepackage{graphicx}
\usepackage{textcomp}
\usepackage{xcolor}
\usepackage{txfonts}
\usepackage{listings}
\usepackage{enumitem}
\usepackage{mathtools}
\usepackage{gensymb}
\usepackage{comment}
\usepackage[breaklinks=true]{hyperref}
\usepackage{tkz-euclide} 
\usepackage{listings}
\usepackage{gvv}                                        
\def\inputGnumericTable{}                                 
\usepackage[latin1]{inputenc}                                
\usepackage{color}                                            
\usepackage{array}                                            
\usepackage{longtable}                                       
\usepackage{calc}                                             
\usepackage{multirow}                                         
\usepackage{hhline}                                           
\usepackage{ifthen}                                           
\usepackage{lscape}
\usepackage{tfrupee}

\newtheorem{theorem}{Theorem}[section]
\newtheorem{problem}{Problem}
\newtheorem{proposition}{Proposition}[section]
\newtheorem{lemma}{Lemma}[section]
\newtheorem{corollary}[theorem]{Corollary}
\newtheorem{example}{Example}[section]
\newtheorem{definition}[problem]{Definition}
\newcommand{\BEQA}{\begin{eqnarray}}
\newcommand{\EEQA}{\end{eqnarray}}
\newcommand{\define}{\stackrel{\triangle}{=}}
\theoremstyle{remark}
\newtheorem{rem}{Remark}
\begin{document}

\bibliographystyle{IEEEtran}
\vspace{3cm}

\title{11.9.5.3}
\author{EE23BTECH11062 - V MANAS}
\maketitle
\newpage

\bigskip
\textbf{Question:}\\Let the sum of $n,2n,3n$ terms of an AP be $S_1,S_2$ and $S_3$, respectively, show that $S_3=3(S_2-S_1)$\\
\textbf{Solution:}
\fi
\begin{table}[h]
    \centering
    \begin{tabular}{|c|c|c|}
    \hline
    \textbf{Variable} & \textbf{Description}\\
    \hline
    x(0) & First term of AP\\
    \hline
    d & common difference in the AP\\
    \hline
    n & number of terms in AP\\
    \hline
    y(n) & sum of n terms of the AP\\
    \hline
\end{tabular}

    \caption{Variables Used}
    \label{tab:table_11.9.5.3}
\end{table}
\begin{figure}[h]
    \centering
    \includegraphics[width=\linewidth]{ncert-maths/11/9/5/3/figs/graph.png}
    \caption{Verification plot for the AP[y(n)=$\frac{n+1}{2}(2(5)+n(3))u(n)$]}
\end{figure}\\
By equation(\ref{eq:3/ap/contour})
\begin{align}
    y(n)&=\frac{n+1}{2}(2x(0)+nd)u(n)\\
    y(2n)&=\frac{2n+1}{2}(2x(0)+2nd)u(n)\\
    y(3n)&=\frac{3n+1}{2}(2x(0)+3nd)u(n)\\
    3(y(2n)-y(n))&=\frac{3n+1}{2}(2x(0)+3nd)u(n)
\end{align}

%\end{document}

\pagebreak

\item Show that the products of the corresponding terms of the sequences $a, ar, ar2, \ldots ar^{n-1}$ and $A, AR, AR2, \ldots AR^{n-1}$ form a G.P, and find the common ratio.
\solution
\iffalse
\let\negmedspace\undefined
\let\negthickspace\undefined
\documentclass[journal,12pt,twocolumn]{IEEEtran}
\usepackage{cite}
\usepackage{amsmath,amssymb,amsfonts,amsthm}
\usepackage{algorithmic}
\usepackage{graphicx}
\usepackage{textcomp}
\usepackage{xcolor}
\usepackage{txfonts}
\usepackage{listings}
\usepackage{enumitem}
\usepackage{mathtools}
\usepackage{gensymb}
\usepackage{comment}
\usepackage[breaklinks=true]{hyperref}
\usepackage{tkz-euclide} 
\usepackage{listings}
\usepackage{gvv}                                        
\def\inputGnumericTable{}                                 
\usepackage[latin1]{inputenc}                                
\usepackage{color}                                            
\usepackage{array}                                            
\usepackage{longtable}                                       
\usepackage{calc}                                             
\usepackage{multirow}                                         
\usepackage{hhline}                                           
\usepackage{ifthen}                                           
\usepackage{lscape}
\newtheorem{theorem}{Theorem}[section]
\newtheorem{problem}{Problem}
\newtheorem{proposition}{Proposition}[section]
\newtheorem{lemma}{Lemma}[section]
\newtheorem{corollary}[theorem]{Corollary}
\newtheorem{example}{Example}[section]
\newtheorem{definition}[problem]{Definition}
\newcommand{\BEQA}{\begin{eqnarray}}
\newcommand{\EEQA}{\end{eqnarray}}
\newcommand{\define}{\stackrel{\triangle}{=}}
\theoremstyle{remark}
\newtheorem{rem}{Remark}
\begin{document}

\bibliographystyle{IEEEtran}
\vspace{3cm}

\title{DISCRETE}
\author{EE23BTECH11006 - Ameen Aazam$^{*}$% <-this % stops a space
}
\maketitle
\newpage
\bigskip

\renewcommand{\thefigure}{\theenumi}
\renewcommand{\thetable}{\theenumi}

\vspace{3cm}
\textbf{Question :}
Show that the products of the corresponding terms of the sequences $a, ar, ar^2, \ldots ar^{n-1}$ and $A, AR, AR^2, \ldots AR^{n-1}$ form a G.P., and find the common ratio.

\solution
\fi
\begin{table}[htbp]
    \centering
    \begin{tabular}{|c|c|c|} \hline
      \textbf{Input Parameters} & \textbf{Values} & \textbf{Description} \\ \hline
      $a$ & & First term of $1^{st}$ G.P. \\ \hline
      $r$ & & Common ratio of $1^{st}$ G.P. \\ \hline
      $x_1\brak{n}$ & $x_1\brak{n}=ar^nu\brak{n}$& General term of $1^{st}$ G.P. \\ \hline
      $X_1\brak{z}$ & & z-Transform of $1^{st}$ G.P. \\ \hline
      $A$ & & First term of $2^{nd}$ G.P. \\ \hline
      $R$ & & Common ratio of $2^{nd}$ G.P. \\ \hline
      $x_2\brak{n}$ & $x_1\brak{n}=AR^nu\brak{n}$& General term of $2^{nd}$ G.P. \\ \hline
      $X_2\brak{z}$ & & z-Transform of $2^{nd}$ G.P. \\ \hline
    \end{tabular}
    \vspace{3pt}
    \caption{Parameters}
\end{table}

General term of the $n^{th}$ term of the $1^{st}$ G.P.,
\begin{align}
    x_1\brak{n}=ar^nu\brak{n}
\end{align}
Now the sequence in the z domain would be,
\begin{align}
    X_1\brak{z}&=\sum\limits_{n=-\infty}^{\infty}ar^nu\brak{n}z^{-n} \\
    &=\frac{a}{1-rz^{-1}},\hspace{0.5cm}\abs{z}>\abs{r}
\end{align}
General for the $2^{nd}$ G.P. is given as,
\begin{align}
    x_2\brak{n}=AR^nu\brak{n}
\end{align}
And the z-Transform,
\begin{align}
    X_2\brak{z}&=\sum\limits_{n=-\infty}^{\infty}AR^nu\brak{n}z^{-n} \\
    &=\frac{A}{1-Rz^{-1}},\hspace{0.5cm}\abs{z}>\abs{R}
\end{align}
Now taking the product will result in a sequence as,
\begin{align}
    y\brak{n}&=x_1\brak{n}x_2\brak{n} \\
    &=aA\brak{rR}^nu\brak{n} \label{eq:11.9.3.20.1}
\end{align}
z-Transform of the resulting sequence,
\begin{align}
    Y\brak{z}&=\sum\limits_{n=-\infty}^{\infty}aA\brak{rR}^nu\brak{n}z^{-n} \\
    &=\frac{aA}{1-rRz^{-1}},\hspace{0.5cm}\abs{z}>\abs{rR}
\end{align}
So, form \ref{eq:11.9.3.20.1}, taking the ratio of two consecutive terms,
\begin{align}
    \frac{y\brak{n}}{y\brak{n-1}}&=\frac{aA\brak{rR}^nu\brak{n}}{aA\brak{rR}^{n-1}u\brak{n-1}} \\
    &=rR
\end{align}
As we can see the ratio of any two consecutive terms, $rR$, is a constant. Which means the product of the corresponding terms of the two G.P.s results in another G.P.
And the common ratio is $rR$.
%\end{document}


\pagebreak

\item
The sum of the $4$th and $8$th terms of an AP is $24$ and the sum of the $6$th and $10$th terms is $44$. Find the first three terms of the AP.\\
\solution
\iffalse
\let\negmedspace\undefined
\let\negthickspace\undefined
\documentclass[journal,12pt,twocolumn]{IEEEtran}
\usepackage{cite}
\usepackage{amsmath,amssymb,amsfonts}
\usepackage{graphicx}
\usepackage{textcomp}
\usepackage{xcolor}
\usepackage{txfonts}
\usepackage{listings}
\usepackage{enumitem}
\usepackage{mathtools}
\usepackage{gensymb}
\usepackage{comment}
\usepackage[breaklinks=true]{hyperref}
\usepackage{tkz-euclide} 
\usepackage{listings}
\usepackage{gvv}                                        
\def\inputGnumericTable{}                                 
\usepackage[latin1]{inputenc}                                
\usepackage{color}                                            
\usepackage{array}                                            
\usepackage{longtable}                                       
\usepackage{calc}                                             
\usepackage{multirow}                                         
\usepackage{hhline}                                           
\usepackage{ifthen}                                           
\usepackage{lscape}
\usepackage[export]{adjustbox}

\newtheorem{theorem}{Theorem}[section]
\newtheorem{problem}{Problem}
\newtheorem{proposition}{Proposition}[section]
\newtheorem{lemma}{Lemma}[section]
\newtheorem{corollary}[theorem]{Corollary}
\newtheorem{example}{Example}[section]
\newtheorem{definition}[problem]{Definition}
\newcommand{\BEQA}{\begin{eqnarray}}
\newcommand{\EEQA}{\end{eqnarray}}
\newcommand{\define}{\stackrel{\triangle}{=}}
\newtheorem{rem}{Remark}

\begin{document}
\parindent 0px
\bibliographystyle{IEEEtran}

\vspace{3cm}

\title{}
\author{EE23BTECH11042 -  Khusinadha Naik$^{*}$
}
\maketitle
\newpage
\bigskip

% \renewcommand{\thefigure}{\theenumi}
% \renewcommand{\thetable}{\theenumi}


\section*{Exercise 5.2}

\noindent \textbf{18} \hspace{2pt}The sum of the 4th and 8th terms of an AP is 24 and the sum of the 6th and 10th terms is 44. Find the first three terms of the AP.\\
\noindent \textbf{Ans.}\\
\fi

\begin{table}[h]
\centering
\begin{tabular}{|c|c|c|}
        \hline
        \textbf{Parameter} & \textbf{Value} & \textbf{Description} \\
        \hline
        $x\brak{0}$ & ? & First term of AP \\
	\hline
	$d$ & ? & Common difference \\
        \hline
        $x\brak{3} + x\brak{7}$ & 24 & Sum of 4th , 8th term \\
        \hline
	$x\brak{5} + x\brak{9}$ & 44 & Sum of 6th , 10th term \\
	\hline
        $x(n)$ & $\brak{x\brak{0} + nd}u\brak{n}$ & General term \\
        \hline
\end{tabular}
\caption{Input parameters table}
\label{tab:10.5.2.18.1}



\end{table}

\noindent From \tabref{tab:10.5.2.18.1}

\begin{align}
x\brak{0}+3d + x\brak{0}+7d &= 24 \label{eq:10.5.2.18.1}\\
x\brak{0}+5d + x\brak{0}+9d &= 44 \label{eq:10.5.2.18.2}
\end{align}

\noindent Subtracting \eqref{eq:10.5.2.18.1} from \eqref{eq:10.5.2.18.2}

\begin{align}
4d &= 20 \\
\implies d &= 5  \label{eq:10.5.2.18.4}
\end{align}

\noindent Putting \eqref{eq:10.5.2.18.4} in \eqref{eq:10.5.2.18.1}

\begin{align}
2x\brak{0} + 10d &= 24 \\
2x\brak{0} + 10\brak{5} &= 24 \\
\implies x\brak{0} &= -13
\end{align}

Now , general term becomes
\begin{align}
x\brak{n} &= \brak{-13 + 5n}u\brak{n}
\end{align}

Taking Z-transform of $x\brak{n}$
\begin{align}
X\brak{z} = & \frac{-13}{1 - z^{-1}} + \frac{ 5z^{-1}}{\brak{1 - z^{-1}}^2}\\
X\brak{z}\implies & \frac{18z^{-1} - 13}{z^{-2} - 2z^{-1} + 1} \quad \text{, ROC: } |z| > 1 
\end{align}

\pagebreak

Plotting $x\brak{n}$ v $n$ :
\begin{figure}[h]
    \includegraphics[width=0.5\textwidth]{ncert-maths/10/5/2/18/figs/fig1.png}
    \caption{Given AP}
    \label{fig:10.5.2.18.1}
\end{figure}

From \figref{fig:10.5.2.18.1} first three terms will be
\begin{align}
\{x\brak{0},x\brak{1},x\brak{2}\} &= \{-13 , -8 , -3\}
\end{align}















%\end{document}

\newpage

\item
If A and G be A.M. and G.M., respectively between two positive numbers, prove that the numbers are $A \pm \sqrt{(A+G)(A-G)}$\\
\solution
\newpage

 \item
A man starts repaying a loan as first instalment of Rs.$100$. If he increases the
instalment by Rs $5$ every month, what amount he will pay in the $30^{th}$ instalment? \\
\solution
\iffalse
\let\negmedspace\undefined
\let\negthickspace\undefined
\documentclass[journal,12pt,twocolumn]{IEEEtran}
\usepackage{cite}
\usepackage{amsmath,amssymb,amsfonts,amsthm}
\usepackage{algorithmic}
\usepackage{graphicx}
\usepackage{textcomp}
\usepackage{xcolor}
\usepackage{txfonts}
\usepackage{listings}
\usepackage{enumitem}
\usepackage{mathtools}
\usepackage{gensymb}
\usepackage{comment}
\usepackage[breaklinks=true]{hyperref}
\usepackage{tkz-euclide} 
\usepackage{listings}
\usepackage{gvv}                                        
\def\inputGnumericTable{}                                 
\usepackage[latin1]{inputenc}                                
\usepackage{color}                                            
\usepackage{array}                                            
\usepackage{longtable}                                       
\usepackage{calc}                                             
\usepackage{multirow}                                         
\usepackage{hhline}                                           
\usepackage{ifthen}                                           
\usepackage{lscape}

\newtheorem{theorem}{Theorem}[section]
\newtheorem{problem}{Problem}
\newtheorem{proposition}{Proposition}[section]
\newtheorem{lemma}{Lemma}[section]
\newtheorem{corollary}[theorem]{Corollary}
\newtheorem{example}{Example}[section]
\newtheorem{definition}[problem]{Definition}
\newcommand{\BEQA}{\begin{eqnarray}}
\newcommand{\EEQA}{\end{eqnarray}}
\newcommand{\define}{\stackrel{\triangle}{=}}
\theoremstyle{remark}
\newtheorem{rem}{Remark}

\usepackage{graphicx}
\graphicspath{ {./Downloads/} }
\begin{document}

\bibliographystyle{IEEEtran}
\vspace{3cm}

\title{ASSIGNMENT 1}
\author{EE22BTECH11060 - TEJAVATH KUSHAL$^{*}$% <-this % stops a space
}
\maketitle
\newpage
\bigskip

\renewcommand{\thefigure}{\theenumi}
\renewcommand{\thetable}{\theenumi}


\maketitle
QUESTION 17:\\
A man starts repaying a loan as first instalment of Rs.$100$. If he increases the
instalment by Rs $5$ every month, what amount he will pay in the $30^{th}$ instalment?\\

SOLUTION:\\
\fi
\begin{table}[ht]
\setlength{\arrayrulewidth}{0.3mm}
\setlength{\tabcolsep}{15pt}
\renewcommand{\arraystretch}{1.5}



\begin{tabular}{ |p{1cm}|p{2cm}|p{2cm}| }
\hline
Parameter & Value & Description\\
\hline
$x(n)$ & $(x(0)+nd)u(n)$ & general term \\ \hline
$x(0)$ & $100$ & first term\\ \hline
$d$ & $5$ & Common difference\\ \hline
$x\brak{29}$ & $\brak{x\brak{0} + 29d}u\brak{n}$ & $30^{th}$term\\ \hline 

%$x(l)$ & Last($l^{th}$) term of series & 350\\
%$x(0)$ & Starting ($0^{th}$) term of series & 17 %\\
%\hline
%d & Common difference of AP & 9\\
%\hline
\end{tabular}
\caption{Parameters}



\end{table}

\begin{align}
x(n)&= 100 + 5n \\
x(29)&= x(0)+29d \\
x(29)&= 100+ 145 \\
x(29)&= 245
\end{align}
Z transform of $x(n)= 100 + 5n$,
\begin{align}
X \brak{z} & = \sum_{n=-\infty}^{\infty} x \brak{n}   z^{-n} \\
& = \sum_{n=-\infty}^{\infty}  \brak{100+5n} u \brak{n}   z^{-n} \\
\notag & = \sum_{n=-\infty}^{\infty} \brak{100} u \brak{n}   z^{-n} \\  &~~~~~~~+\sum_{n=-\infty}^{\infty} \brak{5n} u \brak{n}  z^{-n} \\
\implies X \brak{z} & = \frac{100}{1-z^{-1}} + \frac{5z^{-2}}{\brak{1-z^{-1}}^2} \\
ROC &: \abs{z} > 1 \notag
\end{align}


\pagebreak

\begin{figure}[h]
    %\caption{Stem Plot of $x\brak{n}$ v/s n}
    \includegraphics[width=0.5\textwidth]{ncert-maths/11/9/2/17/figs/x(n)_vs_n.png}
    \caption{Stem Plot of $x\brak{n}$ v/s n}
\end{figure}
%\end{document}

\newpage

\item 
Write the first five terms of the sequence $a_n = n(n+2)$. \\
\solution
\iffalse
\let\negmedspace\undefined
\let\negthickspace\undefined
\documentclass[journal,12pt,twocolumn]{IEEEtran}

\usepackage{cite}
\usepackage{amsmath,amssymb,amsfonts,amsthm}
\usepackage{algorithmic}
\usepackage{graphicx}
\usepackage{textcomp}
\usepackage{xcolor}
\usepackage{txfonts}
\usepackage{listings}
\usepackage{enumitem}
\usepackage{mathtools}
\usepackage{gensymb}
\usepackage[breaklinks=true]{hyperref}
\usepackage{tkz-euclide} % loads  TikZ and tkz-base
\usepackage{listings}
\usepackage{circuitikz}
\usepackage{graphicx}

%\newcounter{MYtempeqncnt}
\DeclareMathOperator*{\Res}{Res}
%\renewcommand{\baselinestretch}{2}
\renewcommand\thesection{\arabic{section}}
\renewcommand\thesubsection{\thesection.\arabic{subsection}}
\renewcommand\thesubsubsection{\thesubsection.\arabic{subsubsection}}

\renewcommand\thesectiondis{\arabic{section}}
\renewcommand\thesubsectiondis{\thesectiondis.\arabic{subsection}}
\renewcommand\thesubsubsectiondis{\thesubsectiondis.\arabic{subsubsection}}

% correct bad hyphenation here
\hyphenation{op-tical net-works semi-conduc-tor}
\def\inputGnumericTable{}                                 %%

\lstset{
	frame=single,
	breaklines=true,
	columns=fullflexible
}



\newtheorem{theorem}{Theorem}[section]
\newtheorem{problem}{Problem}
\newtheorem{proposition}{Proposition}[section]
\newtheorem{lemma}{Lemma}[section]
\newtheorem{corollary}[theorem]{Corollary}
\newtheorem{example}{Example}[section]
\newtheorem{definition}[problem]{Definition}
\newcommand{\BEQA}{\begin{eqnarray}}
	\newcommand{\EEQA}{\end{eqnarray}}
\newcommand{\define}{\stackrel{\triangle}{=}}
\newcommand\figref{Fig.~\ref}
\newcommand\tabref{Table~\ref}
\bibliographystyle{IEEEtran}
%\bibliographystyle{ieeetr}


\providecommand{\mbf}{\mathbf}
\providecommand{\pr}[1]{\ensuremath{\Pr\left(#1\right)}}
\providecommand{\qfunc}[1]{\ensuremath{Q\left(#1\right)}}
\providecommand{\sbrak}[1]{\ensuremath{{}\left[#1\right]}}
\providecommand{\lsbrak}[1]{\ensuremath{{}\left[#1\right.}}
\providecommand{\rsbrak}[1]{\ensuremath{{}\left.#1\right]}}
\providecommand{\brak}[1]{\ensuremath{\left(#1\right)}}
\providecommand{\lbrak}[1]{\ensuremath{\left(#1\right.}}
\providecommand{\rbrak}[1]{\ensuremath{\left.#1\right)}}
\providecommand{\cbrak}[1]{\ensuremath{\left\{#1\right\}}}
\providecommand{\lcbrak}[1]{\ensuremath{\left\{#1\right.}}
\providecommand{\rcbrak}[1]{\ensuremath{\left.#1\right\}}}
\theoremstyle{remark}
\newtheorem{rem}{Remark}
\newcommand{\sgn}{\mathop{\mathrm{sgn}}}
\providecommand{\abs}[1]{\left\vert#1\right\vert}
\providecommand{\res}[1]{\Res\displaylimits_{#1}}
\providecommand{\norm}[1]{\left\lVert#1\right\rVert}
%\providecommand{\norm}[1]{\lVert#1\rVert}
\providecommand{\mtx}[1]{\mathbf{#1}}
\providecommand{\mean}[1]{E\left[ #1 \right]}
\providecommand{\fourier}{\overset{\mathcal{F}}{ \rightleftharpoons}}
%\providecommand{\hilbert}{\overset{\mathcal{H}}{ \rightleftharpoons}}
\providecommand{\system}{\overset{\mathcal{H}}{ \longleftrightarrow}}
%\newcommand{\solution}[2]{\textbf{Solution:}{#1}}
\newcommand{\solution}{\noindent \textbf{Solution: }}
\newcommand{\cosec}{\,\text{cosec}\,}
\providecommand{\dec}[2]{\ensuremath{\overset{#1}{\underset{#2}{\gtrless}}}}
\newcommand{\myvec}[1]{\ensuremath{\begin{pmatrix}#1\end{pmatrix}}}
\newcommand{\mydet}[1]{\ensuremath{\begin{vmatrix}#1\end{vmatrix}}}
\renewcommand{\abstractname}{Question}

\let\vec\mathbf

	
	\vspace{3cm}
	
	


\newcommand{\permcomb}[4][0mu]{{{}^{#3}\mkern#1#2_{#4}}}
\newcommand{\comb}[1][-1mu]{\permcomb[#1]{C}}

%\IEEEpeerreviewmaketitle

\newcommand \tab [1][1cm]{\hspace*{#1}}
%\newcommand{\Var}{$\sigma ^2$}
\usepackage{amssymb}
\usepackage{amsmath}

\begin{document}
\bibliographystyle{IEEEtran}

\vspace{3cm}

\title{}
\author{EE23BTECH11024 - G.Karthik Yadav$^{*}$
}
\maketitle
\newpage
\bigskip




\section*{Exercise 9.1}
\noindent 1. \hspace{2pt}Write the first five terms of the sequence\\
$a_n = n \brak{n+2}$\\

\solution
\fi


\setlength{\arrayrulewidth}{0.2mm}
\setlength{\tabcolsep}{15pt}
\renewcommand{\arraystretch}{1.15}


\begin{table}[ht]
  \centering
  \begin{tabular}{|c|c|c|}
    \hline
    	Symbol & Parameters & value\\
    \hline
	  $u\brak{n}$ & unit step function & 1, if n$\geq$ 0; \\& &0 otherwise \\
    \hline
	  $x\brak{n}$ & general term of the series & $\brak{n+1}\brak{n+3}u\brak{n}$ \\
    \hline 
	 $X\brak{z}$ & Z-transform of $x\brak{n}$ & ? \\
    \hline
  \end{tabular}
  \vspace{0.3cm}
  \caption{Input Parameters}
  \label{tab:24.11.9.1.1}
\end{table}




from table \ref{tab:24.11.9.1.1}
\begin{align}
    X \brak{z} & = \sum_{n=-\infty}^{\infty}  \brak{n+1}\brak{n+3} u \brak{n}   z^{-n}  \\
    & = \sum_{n=-\infty}^{\infty}  \brak{n^{2}u\brak{n} + 4 n\, u\brak{n}  + 3u\brak{n} } z^{-n}
\end{align}    

Using  eq \eqref{eq:11.9.5.26.2} and eq \eqref{eq:11.9.5.26.3}
\begin{align}
     X \brak{z} & = \frac{ 3-z^{-1}}{\brak{1-z^{-1}}^3} \text{ ,}\qquad \abs{z}>1
\end{align}

\begin{figure}[ht]
   \centering
   \includegraphics[width=1\columnwidth]{ncert-maths/11/9/1/1/figs/plot1.png}
   \caption{Plot of x(n) vs n}
   \label{fig: 1.11.9.1.1}
\end{figure}





\newpage
\item If A.M. and G.M. of roots of a quadratic equation are 8 and 5,respectively,then obtain the quadratic equation.
\solution
\let\negmedspace\undefined
\let\negthickspace\undefined
\documentclass[journal,12pt,twocolumn]{IEEEtran}

\usepackage{cite}
\usepackage{amsmath,amssymb,amsfonts,amsthm}
\usepackage{graphicx}
\usepackage{textcomp}
\usepackage{xcolor}
\usepackage{txfonts}
\usepackage{listings}
\usepackage{enumitem}
\usepackage{mathtools}
\usepackage{gensymb}
\usepackage[breaklinks=true]{hyperref}
\usepackage{tkz-euclide} % loads  TikZ and tkz-base
\usepackage{listings}
\usepackage{circuitikz}
\usepackage{graphicx}

%\newcounter{MYtempeqncnt}
\DeclareMathOperator*{\Res}{Res}
%\renewcommand{\baselinestretch}{2}
\renewcommand\thesection{\arabic{section}}
\renewcommand\thesubsection{\thesection.\arabic{subsection}}
\renewcommand\thesubsubsection{\thesubsection.\arabic{subsubsection}}

\renewcommand\thesectiondis{\arabic{section}}
\renewcommand\thesubsectiondis{\thesectiondis.\arabic{subsection}}
\renewcommand\thesubsubsectiondis{\thesubsectiondis.\arabic{subsubsection}}

% correct bad hyphenation here
\hyphenation{op-tical net-works semi-conduc-tor}
\def\inputGnumericTable{}                                 %%

\lstset{
	frame=single,
	breaklines=true,
	columns=fullflexible
}



\newtheorem{theorem}{Theorem}[section]
\newtheorem{problem}{Problem}
\newtheorem{proposition}{Proposition}[section]
\newtheorem{lemma}{Lemma}[section]
\newtheorem{corollary}[theorem]{Corollary}
\newtheorem{example}{Example}[section]
\newtheorem{definition}[problem]{Definition}
\newcommand{\BEQA}{\begin{eqnarray}}
	\newcommand{\EEQA}{\end{eqnarray}}
\newcommand{\define}{\stackrel{\triangle}{=}}
\newcommand\figref{Fig.~\ref}
\newcommand\tabref{Table~\ref}
\bibliographystyle{IEEEtran}
%\bibliographystyle{ieeetr}


\providecommand{\mbf}{\mathbf}
\providecommand{\pr}[1]{\ensuremath{\Pr\left(#1\right)}}
\providecommand{\qfunc}[1]{\ensuremath{Q\left(#1\right)}}
\providecommand{\sbrak}[1]{\ensuremath{{}\left[#1\right]}}
\providecommand{\lsbrak}[1]{\ensuremath{{}\left[#1\right.}}
\providecommand{\rsbrak}[1]{\ensuremath{{}\left.#1\right]}}
\providecommand{\brak}[1]{\ensuremath{\left(#1\right)}}
\providecommand{\lbrak}[1]{\ensuremath{\left(#1\right.}}
\providecommand{\rbrak}[1]{\ensuremath{\left.#1\right)}}
\providecommand{\cbrak}[1]{\ensuremath{\left\{#1\right\}}}
\providecommand{\lcbrak}[1]{\ensuremath{\left\{#1\right.}}
\providecommand{\rcbrak}[1]{\ensuremath{\left.#1\right\}}}
\theoremstyle{remark}
\newtheorem{rem}{Remark}
\newcommand{\sgn}{\mathop{\mathrm{sgn}}}
\providecommand{\abs}[1]{\left\vert#1\right\vert}
\providecommand{\res}[1]{\Res\displaylimits_{#1}}
\providecommand{\norm}[1]{\left\lVert#1\right\rVert}
%\providecommand{\norm}[1]{\lVert#1\rVert}
\providecommand{\mtx}[1]{\mathbf{#1}}
\providecommand{\mean}[1]{E\left[ #1 \right]}
\providecommand{\fourier}{\overset{\mathcal{F}}{ \rightleftharpoons}}
%\providecommand{\hilbert}{\overset{\mathcal{H}}{ \rightleftharpoons}}
\providecommand{\system}{\overset{\mathcal{H}}{ \longleftrightarrow}}
%\newcommand{\solution}[2]{\textbf{Solution:}{#1}}
\newcommand{\solution}{\noindent \textbf{Solution: }}
\newcommand{\cosec}{\,\text{cosec}\,}
\providecommand{\dec}[2]{\ensuremath{\overset{#1}{\underset{#2}{\gtrless}}}}
\newcommand{\myvec}[1]{\ensuremath{\begin{pmatrix}#1\end{pmatrix}}}
\newcommand{\mydet}[1]{\ensuremath{\begin{vmatrix}#1\end{vmatrix}}}
\renewcommand{\abstractname}{Question}

\let\vec\mathbf

	
	\vspace{3cm}
	
	


\newcommand{\permcomb}[4][0mu]{{{}^{#3}\mkern#1#2_{#4}}}
\newcommand{\comb}[1][-1mu]{\permcomb[#1]{C}}

%\IEEEpeerreviewmaketitle

\newcommand \tab [1][1cm]{\hspace*{#1}}
%\newcommand{\Var}{$\sigma ^2$}
\usepackage{amssymb}
\usepackage{amsmath}
\title{
	
\title{NCERT Mathematics 11.9.3 Q32}
\author{EE23BTECH11213 - MUTHYALA NIKHITHA SRI
}


}
\begin{document}

\maketitle

\textbf{Question:} 
If A.M. and G.M. of roots of a quadratic equation are 8 and 5,respectively,then obtain the quadratic equation.

\textbf{Solution: }

\begin{table}[h]
 	\centering
 	\resizebox{6 cm}{!}{
 		\begin{tabular}{|c|c|c|}
\hline 
   \textbf{Parameter}  &\textbf{Description} &\textbf{Value} \\
\hline
&&\\
$I_r$&Net Intensity of light at $\Delta x =\dfrac{\lambda}{3}$ &$\dfrac{K}{4}$ \\&&\\
\hline
\end{tabular}

 	}
 	\caption{Input Parameters}
    \label{tab:table_9.3.32}
 \end{table}


\begin{align}
x_1 \cdot x_2 &= 25 \\
x_1 + x_2 &= 16 \\
\implies  x^2 - 16x + 25 &= 0 \\
\implies x_1 &= 8+\sqrt{39} \\
\implies x_2 &= 8-\sqrt{39} 
\end{align}

For AP, 
\begin{align}
x\brak{0} &= 8+\sqrt{39} \\
d &= -2\sqrt{39} \\
x\brak{n} &= \brak{8+\sqrt{39} + n\brak{-2\sqrt{39}}}u\brak{n} \\
X\brak{z} &= \frac{8+\sqrt{39}}{1 - z^{-1}} + \frac{\brak{-2\sqrt{39}}\cdot z^{-1}}{\brak{1 - z^{-1}}^2}  \quad \abs{z} > \abs{1} \\
\implies X\brak{z} &= \frac{8+\sqrt{39}-\brak{8+3\sqrt{39}}\cdot{z^{-1}}}{\brak{1 - z^{-1}}^2} \quad \abs{z} > \abs{1}
\end{align}

For GP,
\begin{align}
x\brak{0} &= 8+\sqrt{39} \\
r &= \frac{8-\sqrt{39}}{8+\sqrt{39}} \\
x\brak{n} &= \brak{\brak{8+\sqrt{39}}\cdot {\brak{\frac{8-\sqrt{39}}{8+\sqrt{39}}}}^{n}}u\brak{n} \\
X\brak{z} &= \frac{8+\sqrt{39}}{1-\frac{\brak{8-\sqrt{39}}z^{-1}}{8+\sqrt{39}}} \quad \abs{z} > \frac{103-16\sqrt{39}}{25}
\end{align}

\begin{figure}[h!]
    \centering
    \includegraphics[width=\columnwidth]{figs/f1.png}
    \caption{Plot of $f\brak{x} = x^2 - 16x + 25 = 0$}
    \label{fig:1}
\end{figure}

\begin{figure}[h!]
    \centering
    \includegraphics[width=\columnwidth]{figs/ap.png}
    \caption{Plot of $x\brak{n} = \brak{8+\sqrt{39} + n\brak{-2\sqrt{39}}}u\brak{n}$}
    \label{fig:2}
\end{figure}

\begin{figure}[h!]
    \centering
    \includegraphics[width=\columnwidth]{figs/gp.png}
    \caption{Plot of $x\brak{n} = \brak{\brak{8+\sqrt{39}}\cdot {\brak{\frac{8-\sqrt{39}}{8+\sqrt{39}}}}^{n}}u\brak{n}$}
    \label{fig:3}
\end{figure}

\end{document}

\pagebreak

\item An AP consists of $50$ terms of which $3^{rd}$ term is $12$ and the last term is $106$. Find the $29^{th}$ term.\\
\solution 
\iffalse
\documentclass[journal,12pt,twocolumn]{IEEEtran}
\usepackage{amsmath,amsfonts,amssymb,float,amsthm,gvv,listings,enumitem,mathtools,setspace}
\usepackage{graphicx}
\bibliographystyle{IEEEtran}
\vspace{3cm}
\title{NCERT Discrete}
\author{Pragnidhved Reddy\\EE23BTECH11050}
\date{}
\parindent 0px
\begin{document}
\maketitle
\newpage
\bigskip
\textbf{Question 10.5.2.8:}\\
An AP consists of $50$ terms of which $3^{rd}$ term is $12$ and the last term is $106$. Find the $29^{th}$ term.\\
\solution 
\fi
\begin{table}[H]
\centering
\begin{tabular}{|c|c|c|}\hline
\textbf{Parameter} & \textbf{Value} & \textbf{description}\\ \hline
$x(2)$ & $12$ & Third term\\ \hline
$x(49)$ & $106$ & Last term\\ \hline
$x(0)$ & $$ & First term \\ \hline
$d$ & $$ & Common difference\\ \hline
$x(n)$ & $(x(0)+nd)u(n)$ & general term \\ \hline
\end{tabular}
\caption{Input parameters}
\label{tab:table1_pragnidhved_8}
\end{table}
\begin{align}
\myvec{x(2) \\ x(49)}
&=
\myvec{1 & 2 \\ 1 & 49}
\myvec{x(0) \\ d}
\\[5pt]
\myvec{12 \\ 106}
&=
\myvec{x(0) + 2d \\ x(0) + 49d}
\\
\quad &\text{converting to augmented matrix}\\
&= \myvec{x(0)+2d &| 12 \\ x(0)+49d &| 106} \\
\quad R_2&\rightarrow R_2-R_1\\[5pt]
\label{eq:eq1_pragnidhved_8}
&\xrightarrow{R_2\rightarrow R_2 - R_1} \myvec{x(0)+2d &| 12 \\ 47d &| 94}
\end{align}
 From \eqref{eq:eq1_pragnidhved_8}, we get
\begin{align}
\implies &x(0)=8\\
\implies &d=2
\end{align}
From the \tabref{tab:table1_pragnidhved_8} :
\begin{align}
\implies x(n)&=(8+2n)u(n)
\end{align}
 Finding $x(28)$ :
\begin{align}
x(28)&=x(0)+28(2)\\
\implies x(28)&=64
\end{align}
 Z-transform :
\begin{align}
\implies &X(z)=\frac{8-6z^{-1}}{(1-z^{-1})^2} \quad \abs{z}>1
\end{align}\\[130pt]
\begin{figure}[h!]
    \centering
    \includegraphics[width=\columnwidth]{ncert-maths/10/5/2/8/figs/plot.png}
    \caption{graph of the given AP}
    \label{fig:fig1_10_5_8_050}
\end{figure}
%\end{document}

\pagebreak

\item The first term of an AP is $5$, the last term is $45$ and the sum is $400$. Find the number of terms and the common difference.\\
\solution
\iffalse
\let\negmedspace\undefined
\let\negthickspace\undefined
\documentclass[journal,12pt,twocolumn]{IEEEtran}
\usepackage{cite}
\usepackage{amsmath,amssymb,amsfonts,amsthm}
\usepackage{algorithmic}
\usepackage{graphicx}
\usepackage{textcomp}
\usepackage{xcolor}
\usepackage{txfonts}
\usepackage{listings}
\usepackage{enumitem}
\usepackage{mathtools}
\usepackage{gensymb}
\usepackage{comment}
\usepackage[breaklinks=true]{hyperref}
\usepackage{tkz-euclide} 
\usepackage{listings}
\usepackage{gvv}                                        
\def\inputGnumericTable{}                                 
\usepackage[latin1]{inputenc}                                
\usepackage{color}                                            
\usepackage{array}                                            
\usepackage{longtable}                              
\usepackage{calc}                                             
\usepackage{multirow}                                         
\usepackage{hhline}                                           
\usepackage{ifthen}                                           
\usepackage{lscape}

\newtheorem{theorem}{Theorem}[section]
\newtheorem{problem}{Problem}
\newtheorem{proposition}{Proposition}[section]
\newtheorem{lemma}{Lemma}[section]
\newtheorem{corollary}[theorem]{Corollary}
\newtheorem{example}{Example}[section]
\newtheorem{definition}[problem]{Definition}
\newcommand{\BEQA}{\begin{eqnarray}}
\newcommand{\EEQA}{\end{eqnarray}}
\newcommand{\define}{\stackrel{\triangle}{=}}
\theoremstyle{remark}
\newtheorem{rem}{Remark}
\begin{document}

\bibliographystyle{IEEEtran}
\vspace{3cm}

\title{NCERT DISCRETE}
\author{EE23BTECH11020 - Raghava Ganji$^{*}$% <-this % stops a space 
}
\maketitle
\newpage
\bigskip

\renewcommand{\thefigure}{\theenumi}
\renewcommand{\thetable}{\theenumi}

\textbf{Question 10.5.3.5:}
The first term of an AP is $5$, the last term is $45$ and the sum is $400$. Find the number of terms and the common difference.\\
\textbf{solution:}\\
\fi
Given AP is 5, \ldots, 45.\\\\
\begin{table}[h]
\centering
\begin{tabular}{|c|c|c|}\hline
$x\brak 0$ & $5$ & 1st\hspace{1mm}term\\ \hline
$x\brak n$ & $45$ & \brak {n+1}th\hspace{1mm}term\\ \hline
$y\brak n$ & $400$ & sum\hspace{1mm}of\hspace{1mm}\brak {n+1}\hspace{1mm}terms\\ \hline
n+1 & ? & no.of\hspace{1mm}terms\\ \hline
d & ? & common\hspace{1mm}difference\\ \hline
\end{tabular}
\caption{parameters}
\end{table}

\begin{align}
x\brak n&=x\brak 0+nd\\
40&=nd\label{10.5.3.5.2}\\
y\brak n&=\frac{n+1}{2} \sbrak {2x\brak 0+nd}\\
\implies n&=15\label{10.5.3.5.4}\\
\implies d&=\frac{8}{3}\label{10.5.3.5.5}
\end{align}
by substituting equation \eqref{10.5.3.5.4} in the  equation \eqref{10.5.3.5.2}, we get the equation \eqref{10.5.3.5.5}.\\
z transform of $x\brak n,y\brak n$ are $X\brak z,Y\brak z$
\begin{align}
X\brak z&=\frac{7}{3\brak{1-z^{-1}}}+\frac{8}{3\brak{1-z^{-1}}^2}\\
Y\brak z&=\frac{7}{3\brak{1-z^{-1}}^2}+\frac{8}{3\brak{1-z^{-1}}^3}
\end{align}
\begin{figure}
    \centering
    \includegraphics[width=1\columnwidth]{ncert-maths/10/5/3/5/figs/10.5.3.5.1.png}
    \caption{analysis of x\brak n}
\end{figure}
\begin{figure}
    \centering
    \includegraphics[width=1\columnwidth]{ncert-maths/10/5/3/5/figs/10.5.3.5.2.jpg}
    \caption{simulation vs analysis of y\brak n}
\end{figure}
%\end{document}

\pagebreak

\item Which term of the arithmetic progression (AP): \(3, 8, 13, 18, \ldots\) is \(78\)?\\
\solution
\input{ncert-maths/10/5/2/4/ass1.tex}
\pagebreak

\item The sum of two numbers is $6$ times their geometric mean, show that numbers are in the ratio $\dfrac{(3+2\sqrt{2})}{(3-2\sqrt{2})}$. 
\solution
\iffalse
\let\negmedspace\undefined
\let\negthickspace\undefined
\documentclass[journal,12pt,twocolumn]{IEEEtran}
\usepackage{cite}
\usepackage{amsmath,amssymb,amsfonts,amsthm}
\usepackage{algorithmic}
\usepackage{graphicx}
\usepackage{textcomp}
\usepackage{xcolor}
\usepackage{txfonts}
\usepackage{listings}
\usepackage{enumitem}
\usepackage{mathtools}
\usepackage{gensymb}
\usepackage{comment}
\usepackage[breaklinks=true]{hyperref}
\usepackage{tkz-euclide} 
\usepackage{listings}
\usepackage{gvv}                                        
\def\inputGnumericTable{}                                 
\usepackage[latin1]{inputenc}                                
\usepackage{color}                                            
\usepackage{array}                                            
\usepackage{longtable}                                       
\usepackage{calc}                                             
\usepackage{multirow}                                         
\usepackage{hhline}                                           
\usepackage{ifthen}                                           
\usepackage{lscape}
\usepackage{caption}
\newtheorem{theorem}{Theorem}[section]
\newtheorem{problem}{Problem}
\newtheorem{proposition}{Proposition}[section]
\newtheorem{lemma}{Lemma}[section]
\newtheorem{corollary}[theorem]{Corollary}
\newtheorem{example}{Example}[section]
\newtheorem{definition}[problem]{Definition}
\newcommand{\BEQA}{\begin{eqnarray}}
\newcommand{\EEQA}{\end{eqnarray}}
\newcommand{\define}{\stackrel{\triangle}{=}}
\theoremstyle{remark}
\newtheorem{rem}{Remark}
\begin{document}
\parindent 0px
\bibliographystyle{IEEEtran}
\vspace{3cm}

\title{NCERT 11.9.3 28Q}
\author{EE23BTECH11012 - Chavan Dinesh$^{*}$% <-this % stops a space
}
\maketitle
\newpage
\bigskip

\renewcommand{\thefigure}{\arabic{figure}}
\renewcommand{\thetable}{\arabic{table}}
\large\textbf{\textsl{Question:}}
The sum of two numbers is $6$ times their geometric mean, show that numbers are in the ratio $\dfrac{(3+2\sqrt{2})}{(3-2\sqrt{2})}$.

\solution
\fi
Let the two numbers be $x(0)$ and $x(2)$ such that $x(2)\geq x(0)$ 
% and $x(1)$ is G.M of $x(0)$ and $x(2)$.
\begin{table}[htbp]
    \centering
     \begin{tabular}{|c|c|c|}
        \hline
        \textbf{Parameter} & \textbf{Description} &\textbf{Value}\\
        \hline
        $x(0)$ & first number& \\
         \hline
        $r$ & common ratio &\\
        \hline
        $x(2)$ & second number & $x(0)r^2$\\
        \hline
        $x(1)$ & G.M & $x(0)r$\\
        \hline
        $x(n)$ & $(n + 1)^{th}$term & $(x(0)r^n)u(n)$\\
        \hline
    \end{tabular}
 

    \caption{Input table}
    \label{tab:parameter_table.11.9.3.28}
\end{table}

From \tabref{tab:parameter_table.11.9.3.28}:
\begin{align}
x(0) + x(2) &= 6x(1) \\
\implies x(0) + x(0)r^2 &= 6x(0)r \\
\implies r^2 - 6r +1 &= 0 \\
\implies r &= 3\pm 2\sqrt{2}\\
% \end{align}
% \begin{align}
    \therefore \frac{x(2)}{x(0)} &= (3 + 2\sqrt{2})^2 \\
    &=  \frac{(3+2\sqrt{2})}{(3-2\sqrt{2})}
\end{align}
\begin{align}
x(n) = (x(0)(3 + 2\sqrt{2})^n)u(n)
\end{align}
Taking z - Transform of $x(n)$:
\begin{align}
    X(z) = \frac{x(0)}{1 - (3 + 2\sqrt{2})z^{-1}} ; |z| > (3 + 2\sqrt{2})
\end{align}
\begin{figure}[ht]
    \centering
    \includegraphics[width = \columnwidth]{ncert-maths/11/9/3/28/figs/x_n_stem_plot.png}
    \caption{}
    \label{fig:graph1.11.9.3.28}
\end{figure}


\pagebreak

\item Find the value of $n$ so that $\frac{a^{n+1} + b^{n+1}}{a^{n}+b^{n}}$ may be the geometric mean between $a$ and $b$. \\
\solution
\documentclass[12pt]{article}
\usepackage{amsmath , amssymb}
\usepackage{graphicx}
\usepackage{float}
\usepackage{pgfplots}
\pgfplotsset{compat=1.18}
\newcommand{\tabref}[1]{Table~\ref{#1}}
\newcommand{\figref}[1]{Figure~\ref{#1}}
\providecommand{\abs}[1]{\left\vert#1\right\vert}

\begin{document}

\title{Discrete Assignment}
\author{Mohana Eppala\\ EE23BTECH11018}
\maketitle

\section*{Problem Statement}
Find the value of $n$ so that $\frac{a^{n+1} + b^{n+1}}{a^{n}+b^{n}}$ may be the geometric mean between $a$ and $b$.
\section*{Solution}

\begin{table}[H]
\centering
\begin{tabular}{|c|c|c|}
        \hline
        \textbf{Parameter} & \textbf{Value} & \textbf{Description} \\
        \hline
	$x(0)$ & $a$ & First term \\
        \hline
	$x(2)$ & $b$ & Third term \\
	\hline
	$x(1)$ & $\sqrt{ab}=\frac{a^{n+1}+b^{n+1}}{a^{n}+b^{n}}$ & Second term\\
	\hline
	$r$ & $\sqrt{\frac{b}{a}}$ & Common ratio \\
	\hline
        $n$ & - & Given variable \\
        \hline
	$x(k)$ & $ar^{k}u(k)$ & General term \\
	\hline
\end{tabular}
\caption{Input parameters table}
\label{tab:1}

\end{table}

Consider a GP as in \tabref{tab:1},
\begin{align}
	\therefore \frac{a^{n+1} + b^{n+1}}{a^{n}+b^{n}} &= x(1) \\
	\implies a^{n+1} + b^{n+1} &= a^{n+\frac{1}{2}}b^{\frac{1}{2}} + a^{\frac{1}{2}}b^{n+\frac{1}{2}} \\
\implies a^{n+\frac{1}{2}}(a^{\frac{1}{2}} - b^{\frac{1}{2}}) &= b^{n+\frac{1}{2}}(a^{\frac{1}{2}} - b^{\frac{1}{2}}) \\
\implies (\frac{a}{b})^{n+\frac{1}{2}} &= (\frac{a}{b})^{0} \\
\implies n &= -\frac{1}{2}
\end{align}
From \tabref{tab:1},
\begin{align}
	X(z) &= \frac{a}{1-(\sqrt{\frac{b}{a}})z^{-1}} \quad \abs{z}>\abs{\sqrt{\frac{b}{a}}}
\end{align}
\end{document}

\pagebreak

\item Which term of the AP : 121, 117, 113, \ldots, is its first negative term?\\
\solution
\pagebreak

\item The first term of a G.P. is $1$. The sum of the third term and fifth term is $90$. Find the common ratio of G.P.\\
\solution
\pagebreak

\item If the sum of first $p$ terms of an A.P. is equal to the sum of the first $q$ terms, then find the sum of the first $(p + q)$ terms.\\
\solution
\pagebreak
\item How many terms of G.P.$3$,$3^2$,$3^3$, \ldots are needed to give the sum $120$ ?\\
\solution
\iffalse
\let\negmedspace\undefined
\let\negthickspace\undefined
\documentclass[journal,12pt,twocolumn]{IEEEtran}
\usepackage{cite}
\usepackage{amsmath,amssymb,amsfonts,amsthm}
\usepackage{algorithmic}
\usepackage{graphicx}
\usepackage{textcomp}
\usepackage{xcolor}
\usepackage{txfonts}
\usepackage{listings}
\usepackage{enumitem}
\usepackage{mathtools}
\usepackage{gensymb}
\usepackage{comment}
\usepackage[breaklinks=true]{hyperref}
\usepackage{tkz-euclide} 
\usepackage{listings}
\usepackage{gvv}                                        
\def\inputGnumericTable{}                                 
\usepackage[latin1]{inputenc}                                
\usepackage{color}                                            
\usepackage{array}                                            
\usepackage{longtable}                                       
\usepackage{calc}                                             
\usepackage{multirow}                                         
\usepackage{hhline}                                           
\usepackage{ifthen}                                           
\usepackage{lscape}

\newtheorem{theorem}{Theorem}[section]
\newtheorem{problem}{Problem}
\newtheorem{proposition}{Proposition}[section]
\newtheorem{lemma}{Lemma}[section]
\newtheorem{corollary}[theorem]{Corollary}
\newtheorem{example}{Example}[section]
\newtheorem{definition}[problem]{Definition}
\newcommand{\BEQA}{\begin{eqnarray}}
\newcommand{\EEQA}{\end{eqnarray}}
\newcommand{\define}{\stackrel{\triangle}{=}}
\theoremstyle{remark}
\newtheorem{rem}{Remark}
\begin{document}
\bibliographystyle{IEEEtran}
\vspace{3cm}
\title{\textbf{11.9.3}}
\author{EE23BTECH11053-R.Rahul$^{*}$% <-this % stops a space
}
\maketitle
\newpage
\bigskip

\textbf{QUESTION:}\\
1.How many terms of G.P.$3$,$3^2$,$3^3$, \ldots are needed to give the sum $120$ ?\\

\solution
\fi
\vspace{-0.25cm}
\begin{table}[h]
  \centering
  \renewcommand{\arraystretch}{1.5}
\begin{tabular}{|c|c|c|}
\hline
Parameter & Description & Value \\\hline
\( n \) & No. of terms in the G.P &4 \\\hline
\(x(0) \) & first term in the G.P&3 \\\hline
\( r \) & common ratio in the G.P& 3 \\\hline
\(x(n)\) & $n^{th}$ term in G.P& $x(0)r^{n}u(n)$\\ \hline
\end{tabular}
\caption{variables}
  \label{tab:xn}
\end{table}
\begin{center}
    
\begin{align}
      X(z)&=\frac{x(0)}{1-rz^{-1}} \qquad |z|>|r|\\
      &=\frac{3}{1-3z^{-1}}\\
      U(z)&=\frac{1}{1-z^{-1}} \qquad |z|>1\\
      s(n)&= x(n)*u(n)\\
      S(z)&=X(z)U(z)\\
&=\brak{ \frac{3}{1-3z^{-1}}}\brak{\frac{1}{1-z^{-1}}}\quad|z| > 3 \end{align}
\end{center}
by using sum to n terms in G.P
\begin{align}
    s(n)&=a(\frac{r^{n}-1}{r-1})\\
    120&=\frac{3^{n+1}-3}{2}\\
    n&=4
\end{align}

\begin{figure}
  
  \includegraphics[width=\columnwidth]{ncert-maths/11/9/3/13/figs/download.png}
  \caption{Stem plot of x(n)}
\end{figure}

\begin{figure}
  
  \includegraphics[width=\columnwidth]{ncert-maths/11/9/3/13/figs/graph.png}
  \caption{Stem plot of s(n)}
\end{figure}
%\end{document}

\pagebreak

\item Find the sum of first $51$ terms of an $AP$ whose second and third terms are $14$ and $18$ respectively. \\
\solution
\iffalse
\let\negmedspace\undefined
\let\negthickspace\undefined
\documentclass[journal,12pt,onecolumn]{IEEEtran}
\usepackage{cite}
\usepackage{amsmath,amssymb,amsfonts,amsthm}
\usepackage{algorithmic}
\usepackage{graphicx}
\usepackage{textcomp}
\usepackage{xcolor}
\usepackage{txfonts}
\usepackage{listings}
\usepackage{enumitem}
\usepackage{mathtools}
\usepackage{gensymb}
\usepackage[breaklinks=true]{hyperref}
\usepackage{tkz-euclide} % loads  TikZ and tkz-base
\usepackage{listings}



\newtheorem{theorem}{Theorem}[section]
\newtheorem{problem}{Problem}
\newtheorem{proposition}{Proposition}[section]
\newtheorem{lemma}{Lemma}[section]
\newtheorem{corollary}[theorem]{Corollary}
\newtheorem{example}{Example}[section]
\newtheorem{definition}[problem]{Definition}
%\newtheorem{thm}{Theorem}[section] 
%\newtheorem{defn}[thm]{Definition}
%\newtheorem{algorithm}{Algorithm}[section]
%\newtheorem{cor}{Corollary}
\newcommand{\BEQA}{\begin{eqnarray}}
\newcommand{\EEQA}{\end{eqnarray}}
\newcommand{\system}[1]{\stackrel{#1}{\rightarrow}}
\newcommand{\define}{\stackrel{\triangle}{=}}
\theoremstyle{remark}
\newtheorem{rem}{Remark}
%\bibliographystyle{ieeetr}
\begin{document}
%
\providecommand{\pr}[1]{\ensuremath{\Pr\left(#1\right)}}
\providecommand{\prt}[2]{\ensuremath{p_{#1}^{\left(#2\right)} }}        % own macro for this question
\providecommand{\qfunc}[1]{\ensuremath{Q\left(#1\right)}}
\providecommand{\sbrak}[1]{\ensuremath{{}\left[#1\right]}}
\providecommand{\lsbrak}[1]{\ensuremath{{}\left[#1\right.}}
\providecommand{\rsbrak}[1]{\ensuremath{{}\left.#1\right]}}
\providecommand{\brak}[1]{\ensuremath{\left(#1\right)}}
\providecommand{\lbrak}[1]{\ensuremath{\left(#1\right.}}
\providecommand{\rbrak}[1]{\ensuremath{\left.#1\right)}}
\providecommand{\cbrak}[1]{\ensuremath{\left\{#1\right\}}}
\providecommand{\lcbrak}[1]{\ensuremath{\left\{#1\right.}}
\providecommand{\rcbrak}[1]{\ensuremath{\left.#1\right\}}}
\newcommand{\sgn}{\mathop{\mathrm{sgn}}}
\providecommand{\abs}[1]{\left\vert#1\right\vert}
\providecommand{\res}[1]{\Res\displaylimits_{#1}} 
\providecommand{\norm}[1]{\left\lVert#1\right\rVert}
%\providecommand{\norm}[1]{\lVert#1\rVert}
\providecommand{\mtx}[1]{\mathbf{#1}}
\providecommand{\mean}[1]{E\left[ #1 \right]}
\providecommand{\cond}[2]{#1\middle|#2}
\providecommand{\fourier}{\overset{\mathcal{F}}{ \rightleftharpoons}}
\newenvironment{amatrix}[1]{%
  \left(\begin{array}{@{}*{#1}{c}|c@{}}
}{%
  \end{array}\right)
}
%\providecommand{\hilbert}{\overset{\mathcal{H}}{ \rightleftharpoons}}
%\providecommand{\system}{\overset{\mathcal{H}}{ \longleftrightarrow}}
	%\newcommand{\solution}[2]{\textbf{Solution:}{#1}}
\newcommand{\solution}{\noindent \textbf{Solution: }}
\newcommand{\cosec}{\,\text{cosec}\,}
\providecommand{\dec}[2]{\ensuremath{\overset{#1}{\underset{#2}{\gtrless}}}}
\newcommand{\myvec}[1]{\ensuremath{\begin{pmatrix}#1\end{pmatrix}}}
\newcommand{\mydet}[1]{\ensuremath{\begin{vmatrix}#1\end{vmatrix}}}
\newcommand{\myaugvec}[2]{\ensuremath{\begin{amatrix}{#1}#2\end{amatrix}}}
\providecommand{\rank}{\text{rank}}
\providecommand{\pr}[1]{\ensuremath{\Pr\left(#1\right)}}
\providecommand{\qfunc}[1]{\ensuremath{Q\left(#1\right)}}
	\newcommand*{\permcomb}[4][0mu]{{{}^{#3}\mkern#1#2_{#4}}}
\newcommand*{\perm}[1][-3mu]{\permcomb[#1]{P}}
\newcommand*{\comb}[1][-1mu]{\permcomb[#1]{C}}
\providecommand{\qfunc}[1]{\ensuremath{Q\left(#1\right)}}
\providecommand{\gauss}[2]{\mathcal{N}\ensuremath{\left(#1,#2\right)}}
\providecommand{\diff}[2]{\ensuremath{\frac{d{#1}}{d{#2}}}}
\providecommand{\myceil}[1]{\left \lceil #1 \right \rceil }
\newcommand\figref{Fig.~\ref}
\newcommand\tabref{Table~\ref}
\newcommand{\sinc}{\,\text{sinc}\,}
\newcommand{\rect}{\,\text{rect}\,}
%%
%	%\newcommand{\solution}[2]{\textbf{Solution:}{#1}}
%\newcommand{\solution}{\noindent \textbf{Solution: }}
%\newcommand{\cosec}{\,\text{cosec}\,}
%\numberwithin{equation}{section}
%\numberwithin{equation}{subsection}
%\numberwithin{problem}{section}
%\numberwithin{definition}{section}
%\makeatletter
%\@addtoreset{figure}{problem}
%\makeatother

%\let\StandardTheFigure\thefigure
\let\vec\mathbf


\bibliographystyle{IEEEtran}
\title{SEQUENCE AND SERIES}
\author{EE23BTECH11059- Tejas Mehtre$^{*}$% <-this % stops a space
}
\maketitle




\bigskip

\renewcommand{\thefigure}{\theenumi}
\renewcommand{\thetable}{\theenumi}
%\renewcommand{\theequation}{\theenumi}
Find the sum of first 51 terms of an AP whose second and third terms are 14 and 18 respectively. \\
    \solution
    \fi
    \begin{table}[!ht]
    \centering
        \begin{tabular}{|c|c|c|} 
      \hline
\textbf{Variable}& \textbf{Description}& \textbf{Value}\\\hline
        $x(1)$& Second term of $AP$ & $14$ \\ \hline
        $x(2)$ &Third term of $AP$ & $18$ \\ \hline
         $x(0)$ & First term of $AP$ & $2x(1)-x(2)=10$ \\ \hline
         $d$ & Common difference of $AP$ $(x(2)-x(1))$ & $4$ \\ \hline
          $x(n)$& $n^{th}$ term of sequence& $(4n+10)u(n)$\\ \hline 
          
    \end{tabular}

    \caption{input parameters}
    \label{}
\end{table}
    
       For an $AP$,
\begin{align}
    X\brak z &= \frac{ x\brak 0 }{1-z^{-1}} + \frac{dz^{-1}}{{(1-z^{-1})}^{2}}    \\
    \implies X\brak z &= \frac{10}{(1-z^{-1})} + \frac{4z^{-1}}{{(1-z^{-1})}^{2}}, \abs{z}>1    \\
    y\brak{n}&=x\brak{n}\ast u\brak{n}\\
     Y\brak{z}&=X\brak{z}U\brak{z}   \\
     Y\brak z &= \frac{10}{(1-z^{-1})^{2}} + \frac{4z^{-1}}{{(1-z^{-1})}^{3}} \\  
     \implies Y\brak z &= \frac{(-6z^{-1}+10)}{(1-z^{-1})^{3}} , \abs{z}>1
\end{align}
Using Contour Integration to find the inverse $Z$-transform,
\begin{align}
    y(50)&=\frac{1}{2\pi j}\oint_{C}Y(z) \;z^{49} \;dz  \\
    &=\frac{1}{2\pi j}\oint_{C}\frac{(-6z^{-1}+10)z^{49}}{({1-z^{-1})}^{3}} \;dz 
\end{align}
We can observe that the pole is repeated $3$ times and thus $m=3$,
\begin{align}
    R&=\frac{1}{\brak {m-1}!}\lim\limits_{z\to a}\frac{d^{m-1}}{dz^{m-1}}\brak {{(z-a)}^{m}f\brak z}  \\
    \implies R &=\frac{1}{\brak {2}!}\lim\limits_{z\to 1}\frac{d^{2}}{dz^{2}}\brak {{(z-1)}^{3}\frac{(-6z^{-1}+10)z^{52}}{{(z-1)}^3}}   \\
    \implies R &=\frac{1}{2}\lim\limits_{z\to 1}\frac{d^2}{dz^2}(10z^{52}-6z^{51})   \\
    \implies R &=5610 \\
    \therefore y(50)&=5610
\end{align}
\begin{figure}[ht]
        \centering
        \includegraphics[width=\columnwidth]{ncert-maths/11/5/3/8/figs/plot.png}
        \caption{Analysis vs Simulation}
    \end{figure}


\pagebreak

\item Fill in the blanks in the following table given that $a$ is the first term, $d$ is the common difference, and $a_n$ is the $n$th term of the AP.\\
\begin{table}[h!]
  \centering
  \begin{tabular}{|c|c@{\hspace{1cm}}|c@{\hspace{1cm}}|c@{\hspace{1cm}}|c|}
    \hline
    $i$ & $x_i(0)$ & $d_i$ & $n_i$ & $a_i(n)$  \\
    \hline
    1 & 7 & 3 & 7 (8$^{th}$ term) & ... \\
    \hline
    2 & -18 & ... & 9 (10$^{th}$ term) & 0 \\
    \hline
    3 & ... & -3 & 17 (18$^{th}$ term) & -5 \\
    \hline
    4 & -18.9 & 2.5 & ... & 3.6 \\
    \hline
    5 & 3.5 & 0 & 104 (105$^{th}$ term) & ... \\
    \hline
\end{tabular}

   \label{tab:ESTable1}
\end{table}
\solution
\documentclass[journal,12pt,twocolumn]{IEEEtran}
\usepackage{cite}
\usepackage{amsmath,amssymb,amsfonts,amsthm}
\usepackage{algorithmic}
\usepackage{graphicx}
\usepackage{textcomp}
\usepackage{xcolor}
\usepackage{txfonts}
\usepackage{listings}
\usepackage{enumitem}
\usepackage{mathtools}
\usepackage{float}
\usepackage{gensymb}
\usepackage{comment}
\usepackage[breaklinks=true]{hyperref}
\usepackage{tkz-euclide} 
\usepackage{listings}
\usepackage{gvv}                                        
\def\inputGnumericTable{}                                 
\usepackage[latin1]{inputenc}                                
\usepackage{color}                                            
\usepackage{array}                                            
\usepackage{longtable}                                       
\usepackage{calc}            
\usepackage{multirow}                                         
\usepackage{hhline}                                           
\usepackage{ifthen}                                           
\usepackage{lscape}
\usepackage{amsmath}
\newtheorem{theorem}{Theorem}[section]
\newtheorem{problem}{Problem}
\newtheorem{proposition}{Proposition}[section]
\newtheorem{lemma}{Lemma}[section]
\newtheorem{corollary}[theorem]{Corollary}
\newtheorem{example}{Example}[section]
\newtheorem{definition}[problem]{Definition}
\newcommand{\BEQA}{\begin{eqnarray}}
\newcommand{\EEQA}{\end{eqnarray}}
\newcommand{\define}{\stackrel{\triangle}{=}}
\theoremstyle{remark}
\newtheorem{rem}{Remark}

\begin{document}

\bibliographystyle{IEEEtran}
\vspace{3cm}

\title{NCERT Discrete - 10.5.2.1}
\author{EE23BTECH11037 - M Esha$^{*}$}

\maketitle
\newpage
\bigskip

\renewcommand{\thefigure}{\theenumi}
\renewcommand{\thetable}{\theenumi}

\vspace{3cm}
\textbf{Question 10.5.2.1:} 

 Fill in the blanks in the following table given that $a$ is the first term, $d$ is the common difference, and $a_n$ is the $n$th term of the AP.

\begin{table}[h!]
  \centering
  \begin{tabular}{|c|c@{\hspace{1cm}}|c@{\hspace{1cm}}|c@{\hspace{1cm}}|c|}
    \hline
    $i$ & $x_i(0)$ & $d_i$ & $n_i$ & $a_i(n)$  \\
    \hline
    1 & 7 & 3 & 7 (8$^{th}$ term) & ... \\
    \hline
    2 & -18 & ... & 9 (10$^{th}$ term) & 0 \\
    \hline
    3 & ... & -3 & 17 (18$^{th}$ term) & -5 \\
    \hline
    4 & -18.9 & 2.5 & ... & 3.6 \\
    \hline
    5 & 3.5 & 0 & 104 (105$^{th}$ term) & ... \\
    \hline
\end{tabular}

   \label{tab:Table1}
\end{table}
\solution
for A.P,
\begin{align}
    x_i(n) = \sbrak{x_i(0) + nd_i} u(n)
    \label{eq:1}
\end{align}
\begin{enumerate}
\item From \eqref{eq:1} \tabref{tab:Table1} :
\begin{align}
x_1(n) &= \sbrak{7+3n}u(n)\\
x_1(7) &= 28\\
X_1(z) &= \frac{7 - 4z^{-1}}{(1 - z^{-1})^2} \quad |z| \neq 1
\end{align}
\item From \eqref{eq:1} and \tabref{tab:Table1} :
\begin{align}
x_2(n) &= \sbrak{-18 + d_2n}u(n)\\
x_2(9) &= 0\\
d_2 &= 2\\
X_2(z) &= \frac{-18 + 20z^{-1}}{(1 - z^{-1})^2}
 \quad |z| \neq 1
\end{align}
\item From \eqref{eq:1} \tabref{tab:Table1} :
\begin{align}
x_3(n) &= \sbrak{x_3(0)-3n}u(n)\\
x_3(17) &= -5\\
x_3(0) &= 49\\
X_3(z) &= \frac{49 - 52z^{-1}}{(1 - z^{-1})^2}
 \quad |z| \neq 1
\end{align}
\item From \eqref{eq:1} \tabref{tab:Table1} :
\begin{align}
x_4(n) &= \sbrak{18.9+2.5n}u(n)\\
x_4(n) &= 3.6\\
n_4    &= 9\\
X_4(z) &= \frac{18.9 - 16.4z^{-1}}{(1 - z^{-1})^2}
 \quad |z| \neq 1
\end{align}
\item From \eqref{eq:1} \tabref{tab:Table1} :
\begin{align}
x_5(n) &= \sbrak{3.5}u(n)\\
x_5(104) &= 3.5\\
X_5(z) &= \frac{3.5}{1-z^{-1}} \quad |z| \neq 1
\end{align}
\begin{figure}[h!]
    \centering
    \includegraphics[width=\columnwidth]{10/5/2/1/figs/10.png}
    \caption{stem plots }
    \label{fig:1}
\end{figure}
\end{enumerate}

\end{document}


\pagebreak
\item In the following APs, find the missing terms in the boxes:\\
(i) 2,\textunderscore, 26 \\
(ii) \textunderscore, 13,\textunderscore , 3\\
(iii)  5,  \textunderscore, \textunderscore,9\(\frac{1}{2}\) \\
(iv) - 4, \textunderscore,  \textunderscore, \textunderscore, \textunderscore, 6\\
(v)  \textunderscore,38, \textunderscore, \textunderscore, \textunderscore, - 22\\
\solution
\iffalse
\let\negmedspace\undefined
\let\negthickspace\undefined
\documentclass[journal,12pt,twocolumn]{IEEEtran}
\usepackage{cite}
\usepackage{amsmath,amssymb,amsfonts,amsthm}
\usepackage{algorithmic}
\usepackage{graphicx}
\usepackage{textcomp}
\usepackage{xcolor}
\usepackage{txfonts}
\usepackage{listings}
\usepackage{enumitem}
\usepackage{mathtools}
\usepackage{gensymb}
\usepackage{comment}
\usepackage[breaklinks=true]{hyperref}
\usepackage{tkz-euclide} 
\usepackage{listings}
\usepackage{gvv}                                        
\def\inputGnumericTable{}                                 
\usepackage[latin1]{inputenc}                                
\usepackage{color}                                            
\usepackage{array}                                            
\usepackage{longtable}                                       
\usepackage{calc}                                             
\usepackage{multirow}                                         
\usepackage{hhline}                                           
\usepackage{ifthen}                                           
\usepackage{lscape}

\newtheorem{theorem}{Theorem}[section]
\newtheorem{problem}{Problem}
\newtheorem{proposition}{Proposition}[section]
\newtheorem{lemma}{Lemma}[section]
\newtheorem{corollary}[theorem]{Corollary}
\newtheorem{example}{Example}[section]
\newtheorem{definition}[problem]{Definition}
\newcommand{\BEQA}{\begin{eqnarray}}
\newcommand{\EEQA}{\end{eqnarray}}
\newcommand{\define}{\stackrel{\triangle}{=}}
\theoremstyle{remark}
\newtheorem{rem}{Remark}

\begin{document}

\bibliographystyle{IEEEtran}
\vspace{3cm}
\title{\textbf{10.05.2.3}}
\author{EE23BTECH11053-R.Rahul$^{*}$% <-this % stops a space
}

\maketitle
\textbf{QUESTION:}
\\
1. In the following APs, find the missing terms in the boxes:\\
(i) 2,\textunderscore, 26 \\
(ii) \textunderscore, 13,\textunderscore , 3\\
(iii)  5,  \textunderscore, \textunderscore,9\(\frac{1}{2}\) \\
(iv) - 4, \textunderscore,  \textunderscore, \textunderscore, \textunderscore, 6\\
(v)  \textunderscore,38, \textunderscore, \textunderscore, \textunderscore, '- 22'\\

\solution\fi


\begin{table}[h]
  \centering
  \renewcommand{\arraystretch}{1.5}
\begin{tabular}{|c|c|c|}
\hline
Parameter & Description \\\hline
\( n \) & No. of terms in the A.P  \\\hline
\(x(0) \) & first term in the A.P \\\hline
\( d \) & common difference in the A.P  \\\hline
\(x(n)=x(0)+nd\) & $(n+1)^{th}$ term in A.P\\ \hline
\end{tabular}
\caption{variables}
  \label{tab:xn}
\end{table}


\begin{center}
\begin{enumerate}
    \item  
     \begin{align}
          26&=2+2d\\
        24&=2d \\
        \therefore d&=12\\
         x(1)&=14
     \end{align}
     The $Z$-transform of $x(n) = (2 + 12n)u(n)$ is given by:
     \begin{align}
    X(z)&=\frac{2+{10z^{-1}}}{(1-{z^{-1}})^2} \qquad|z|>1  \notag\\
     \end{align}     
     \item       
      \begin{align}
         3-13&=2d\\
           -10&=2d\\
           \therefore d&=-5\\
            x(1)&=18\\
            x(2)&=8
      \end{align}
     The $Z$-transform of $x(n) = (18 - 5n)u(n)$ is given by:
\begin{align}
    &X(z)=\frac{18-{23z^{-1}}}{(1-{z^{-1}})^2} \qquad  |z|>1  \notag\\
\end{align}
       \item    
     \begin{align}
           9\ \frac{1}{2}\ &=5+3d \\
           3d&=\frac{9}{2}\\
           \therefore d&=\frac{3}{2}\ \\ 
          x(1)&=6\ \frac{1}{2}\\
          x(2)&=8
     \end{align}
     $Z$-transform of $x(n) = (5 + \frac{3}{2}n)u(n)$ is given by:
\begin{align}
    &X(z)=\frac{5-\frac{7}{2}{z^{-1}}}{(1-{z^{-1}})^2} \qquad |z|>1 \notag\\
\end{align}
      \item       
    \begin{align}
     6&=-4+5d\\
     10&=5d\\
     \therefore d&=2\\
          x(1)&=-2\\
          x(2)&=0\\
          x(3)&=2 \\
          x(4)&=4 \notag\\
    \end{align}
    $Z-transform of x(n) = (-4 + 2n)u(n) $is given by:
    \begin{align}
       &X(z)=\frac{-4+6{z^{-1}}}{(1-{z^{-1}})^2} \qquad  |z|>1  \notag\\
    \end{align}
       \item    
     \begin{align}
         -22-38&=4d\\
          -60&=4d\\
           \therefore d&=-15\\
            x(0)&=53\\
            x(2)&=23\\
            x(3)&=8\\
            x(4)&=-7 \notag\\
     \end{align}
     $Z$-transform of $x(n) = (53 - 15n)u(n)$ is given by:
\begin{align}
       &X(z)=\frac{53-68{z^{-1}}}{(1-{z^{-1}})^2}\qquad|z|>1 \notag \\
\end{align}
\end{enumerate}             
          
\end{center}

\begin{figure}[h]
       \centering
        \includegraphics[width=1\linewidth]{ncert-maths/10/5/2/3/figs/plot1.png} % Adjust the width as needed
        \caption{}
\end{figure}

\begin{figure}[h]
      \centering
       \includegraphics[width=1\linewidth]{ncert-maths/10/5/2/3/figs/plot2.png} % Adjust the width as needed
        \caption{}
    \end{figure}
    
\begin{figure}[h]
      \centering
       \includegraphics[width=1\linewidth]{ncert-maths/10/5/2/3/figs/plot3.png} % Adjust the width as needed
        \caption{}
    \end{figure}
    
\begin{figure}[h] 
      \centering
       \includegraphics[width=1\linewidth]{ncert-maths/10/5/2/3/figs/plot4.png} % Adjust the width as needed
        \caption{}
    \end{figure}
    
\begin{figure}[h]
      \centering
       \includegraphics[width=1\linewidth]{ncert-maths/10/5/2/3/figs/plot5.png} % Adjust the width as needed
        \caption{}
\end{figure}


%\end{document}

\pagebreak
\end{enumerate}
