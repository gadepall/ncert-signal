\begin{enumerate}[label=\thesection.\arabic*,ref=\thesection.\theenumi]
\item Find the number of terms in each of the following APs. 
\begin{enumerate}
    \item 7, 13, 19, ... 205

    \item 18, 15$\frac{1}{2}$, 13, ... -47
\end{enumerate}
\solution
\iffalse
\let\negmedspace\undefined
\let\negthickspace\undefined
\documentclass[journal,12pt,twocolumn]{IEEEtran}
\usepackage{cite}
\usepackage{amsmath,amssymb,amsfonts,amsthm}
\usepackage{algorithmic}
\usepackage{graphicx}
\usepackage{textcomp}
\usepackage{xcolor}
\usepackage{txfonts}
\usepackage{listings}
\usepackage{enumitem}
\usepackage{mathtools}
\usepackage{gensymb}
\usepackage{comment}
\usepackage[breaklinks=true]{hyperref}
\usepackage{tkz-euclide} 
\usepackage{listings}
\usepackage{gvv}                                        
\def\inputGnumericTable{}                                 
\usepackage[latin1]{inputenc}                                
\usepackage{color}                                            
\usepackage{array}                                            
\usepackage{longtable}                                       
\usepackage{calc}                                             
\usepackage{multirow}                                         
\usepackage{hhline}                                           
\usepackage{ifthen}                                           
\usepackage{lscape}
\usepackage{placeins}
\usepackage{xparse}


\newtheorem{theorem}{Theorem}[section]
\newtheorem{problem}{Problem}
\newtheorem{proposition}{Proposition}[section]
\newtheorem{lemma}{Lemma}[section]
\newtheorem{corollary}[theorem]{Corollary}
\newtheorem{example}{Example}[section]
\newtheorem{definition}[problem]{Definition}
\newcommand{\BEQA}{\begin{eqnarray}}
\newcommand{\EEQA}{\end{eqnarray}}
\newcommand{\define}{\stackrel{\triangle}{=}}
\theoremstyle{remark}
\newtheorem{rem}{Remark}



\begin{document}

\bibliographystyle{IEEEtran}
\vspace{3cm}

\Large\title{NCERT Question 10.5.2.5}
\large\author{EE23BTECH11032 - Kaustubh Parag Khachane $^{*}$% <-this % stops a space
}
\maketitle
\newpage
\bigskip

\renewcommand{\thefigure}{\theenumi}
\renewcommand{\thetable}{\theenumi}
\large\textbf{Question 10.5.2.5} : \normalsize Find the number of terms in each of the following APs. Then express each term as x\brak{n} and find the z transform, ROC and plot the graph for x\brak{n}: 
\begin{enumerate}
    \item 7, 13, 19, ... 205

    \item 18, 15$\frac{1}{2}$, 13, ... -47
\end{enumerate}


\solution
\fi
\begin{table}[ht] 
\centering
\setlength{\extrarowheight}{8pt}
\begin{tabular}{|c|l|l|} 
 \hline
  \textbf{Parameter} & \textbf{Used to denote } & \textbf{Values} \\ 
 \hline
 $x_{i}$\brak{n} & $n^{th}$ term of $i^{th}$ series $\brak{i =\brak{1,2}}$  & $\brak{x_{i}\brak{0} + nd_{i}}u\brak{n}$ \\
 \hline
$x_{i}$\brak{0} & First term of $i^{th} $ AP &\multicolumn{1}{|p{1.5cm}|}{\centering $x_{1}\brak{0} = 7$ \\ $x_{2}\brak{0} = 18$ }\\
 \hline
  $d_{i}$ & Commmon difference of $i^{th}$ AP&\multicolumn{1}{|p{1.5cm}|}{\centering $d_{1} = 6 $ \\ $d_{2} = -2.5$}\\
 \hline

\end{tabular}
 \vspace{4mm}
 \caption{Parameter Table}
 \label{tab:table0}
\end{table}

The number of terms in the AP x\brak{n} is given by: 
\begin{align}  \label{eq:eq12}
    \frac{x\brak{n} - x\brak{0}}{d} + 1
\end{align}
\begin{align}
    &X_i(z) = \frac{x_i\brak{0}}{1 - z^{-1}} + d_i\frac{z^{-1}}{\brak{1-z^{-1}}^2} \text{ , for i=1,2} \label{eq:eq3}\\
    &\text{ROC : $\abs{z} > 1$ as it is an AP}   
\end{align}
\begin{enumerate}
    \item 
\begin{align}
x_{1}\brak{n} &= \brak{7 + \brak{n}6}u\brak{n}
\end{align}
Using the values in \tabref{tab:table0} and equation \eqref{eq:eq12},
\begin{align}
    k_1 = \frac{205 - 7}{6} + 1 = 34
\end{align}

Using the values in \tabref{tab:table0} and equation \eqref{eq:eq3} :
\begin{align}
 X_1\brak{z} = \frac{7 - z^{-1}}{\brak{1-z^{-1}}^2}
\end{align}

ROC is $\abs{z} > 1$
 
   \item
   
\begin{align}
    x_{2}\brak{n} &= \brak{18 + n\brak{-2.5}u\brak{n}}
\end{align}

Using the values in \tabref{tab:table0} and equation \eqref{eq:eq12},
\begin{align}
    k_2 = \frac{-47 - 18}{-2.5} + 1 = 27
\end{align}

Using the values in \tabref{tab:table0} and equation \eqref{eq:eq3} :
\begin{align} 
 X_2\brak{z} = \frac{18 - \brak{20.5}z^{-1}}{\brak{1 - z^{-1}}^2}
\end{align}

ROC is $\abs{z} > 1$.

\begin{figure}[!ht]
\centering
\begin{center}
\includegraphics[width=\columnwidth]{ncert-maths/10/5/2/5/figs/Figure_1}
\caption{Plot of $x_1\brak{n}$}
\end{center}
\end{figure}

\begin{figure}[!ht]
\centering
\begin{center}
\includegraphics[width=\columnwidth]{ncert-maths/10/5/2/5/figs/Figure_2}
\caption{Plot of $x_2\brak{n}$}
\end{center}
\end{figure}

\end{enumerate}
%\end{document}



\item For what value of $ n$, are the $ nth$ terms of two A.Ps: 63, 65, 67,\dots and 3, 10, 17,\dots equal?
\solution
\iffalse
\let\negmedspace\undefined
\let\negthickspace\undefined
\documentclass[journal,12pt,twocolumn]{IEEEtran}
\usepackage{cite}
\usepackage{amsmath,amssymb,amsfonts,amsthm}
\usepackage{algorithmic}
\usepackage{graphicx}
\usepackage{textcomp}
\usepackage{xcolor}
\usepackage{txfonts}
\usepackage{listings}
\usepackage{enumitem}
\usepackage{mathtools}
\usepackage{gensymb}
\usepackage{comment}
\usepackage[breaklinks=true]{hyperref}
\usepackage{tkz-euclide}
\usepackage{listings}
\usepackage{gvv}
\def\inputGnumericTable{}
\usepackage[latin1]{inputenc}
\usepackage{color}
\usepackage{array}
\usepackage{longtable}
\usepackage{calc}
\usepackage{multirow}
\usepackage{hhline}
\usepackage{ifthen}
\usepackage{lscape}

\newtheorem{theorem}{Theorem}[section]
\newtheorem{problem}{Problem}
\newtheorem{proposition}{Proposition}[section]
\newtheorem{lemma}{Lemma}[section]
\newtheorem{corollary}[theorem]{Corollary}
\newtheorem{example}{Example}[section]
\newtheorem{definition}[problem]{Definition}
\newcommand{\BEQA}{\begin{eqnarray}}
\newcommand{\EEQA}{\end{eqnarray}}
\newcommand{\define}{\stackrel{\triangle}{=}}
\theoremstyle{remark}
\newtheorem{rem}{Remark}
\begin{document}

\bibliographystyle{IEEEtran}
\vspace{3cm}

\title{NCERT Discrete 10.5.2 -15}
\author{EE23BTECH11057 - Shakunaveti Sai Sri Ram Varun$^{}$% &lt;-this % stops a space
}
\maketitle
\newpage
\bigskip

\vspace{2cm}
\textbf{Question: }
For what value of $ n$, are the $ nth$ terms of two A.Ps: 63, 65, 67,\dots and 3, 10, 17,\dots equal?\\
\vspace{0.5cm}
\textbf{Solution}:
\fi

\begin{table}[htbp] 
\centering
\begin{tabular}{|c|c|c|c|}
    \hline
    \textbf{Parameter} & \textbf{Sub-question} & \textbf{Description} & \textbf{Value} \\
    \hline
    \multirow{2}{*}{$x_i\brak{0}$} & $x_1\brak{0}$ & $1^{st}$ term of $1^{st}$ A.P. & 63 \\
    \cline{2-4}
    & $x_2\brak{0}$ & $1^{st}$ term of $2^{nd}$ A.P. & \phantom{0}3 \\
    \hline
    \multirow{2}{*}{$d_i$} & $d_1$ & Common difference of $1^{st}$ A.P. & \phantom{0}2 \\
    \cline{2-4}
    & $d_2$ & Common difference of $2^{nd}$ A.P. & \phantom{0}7 \\
    \hline
\end{tabular}

\caption{input values}
\label{tab: table10.5.2.15}
\end{table}
\begin{align}
x_i\brak{n} &= x\brak{0}u\brak{n} + dnu\brak{n}\\
X\brak{z} &= \frac{x\brak{0}}{1-z^{-1}} + \frac{dz^{-1}}{\brak{1-z^{-1}}^{2}} \quad |z|>1
\end{align}
\begin{enumerate}
\item
\begin{align}
x_1\brak{n} &= 63u\brak{n} + 2nu\brak{n} \\
%To find $ X_1\brak{z}$:
X_1\brak{z} &= \frac{63}{1-z^{-1}} + \frac{2z^{-1}}{\brak{1-z^{-1}}^{2}}  \quad |z|>1
\end{align}
\item
\begin{align}
x_2\brak{n} &= 3u\brak{n} + 7nu\brak{n}\\ 
%To find $ X_2\brak{z}$ :\\
X_2\brak{z} &= \frac{3}{1-z^{-1}} + \frac{7z^{-1}}{\brak{1-z^{-1}}^{2}} \quad |z|>1
\end{align}
\item

given,
\begin{align}
 x_1\brak{n} &= x_2\brak{n}\\
\therefore 63 + 2n &= 7n+3\\
\implies n &=12
\end{align}
\begin{figure}[h!]
    \includegraphics[width = \columnwidth]{ncert-maths/10/5/2/15/figs/Figure_1.png}
    \caption{Graphs of $ x_1\brak{n}$ and $ x_2\brak{n}$ and both are equal at $ n=12$}
    \label{fig: fig10.5.2.15}
\end{figure}
\end{enumerate}



\item Two APs have the same common difference.The difference between their $100${th} terms is 100,what is the difference between their $1000${th} terms?

\solution
\iffalse
\let\negmedspace\undefined
\let\negthickspace\undefined
\documentclass[journal,12pt,onecolumn]{IEEEtran}
\usepackage{cite}
\usepackage{amsmath,amssymb,amsfonts,amsthm}
\usepackage{algorithmic}
\usepackage{graphicx}
\usepackage{textcomp}
\usepackage{xcolor}
\usepackage{txfonts}
\usepackage{listings}
\usepackage{enumitem}
\usepackage{mathtools}
\usepackage{gensymb}
\usepackage{comment}
\usepackage[breaklinks=true]{hyperref}
\usepackage{tkz-euclide} 
\usepackage{listings}
\usepackage{gvv}                                        
\def\inputGnumericTable{}                                 
\usepackage[latin1]{inputenc}                                
\usepackage{color}                                            
\usepackage{array}                                            
\usepackage{longtable}                                       
\usepackage{calc}                                             
\usepackage{multirow}                                         
\usepackage{hhline}                                           
\usepackage{ifthen}                                           
\usepackage{lscape}
\newtheorem{theorem}{Theorem}[section]
\newtheorem{problem}{Problem}
\newtheorem{proposition}{Proposition}[section]
\newtheorem{lemma}{Lemma}[section]
\newtheorem{corollary}[theorem]{Corollary}
\newtheorem{example}{Example}[section]
\newtheorem{definition}[problem]{Definition}
\newcommand{\BEQA}{\begin{eqnarray}}
\newcommand{\EEQA}{\end{eqnarray}}
\newcommand{\define}{\stackrel{\triangle}{=}}
\theoremstyle{remark}
\newtheorem{rem}{Remark}
\begin{document}
\bibliographystyle{IEEEtran}
\vspace{3cm}
\title{NCERT 11.9.2 16Q}
\author{EE23BTECH11021 - GANNE GOPI CHANDU$^{*}$% <-this % stops a space
}
\maketitle
\bigskip
\renewcommand{\thefigure}{\theenumi}
\renewcommand{\thetable}{\theenumi}
\bibliographystyle{IEEEtran}
\textbf{Question}\\
Between 1 and 31, m numbers have been inserted in such a way that the resulting sequence is an A.P. and 
the ratio of 7 th and (m - 1) th numbers is 5:9. Find the value of m.\\
\textbf{Solution}\\
\fi
\begin{table}[!h]
\begin{center}
\renewcommand\thetable{1}
\begin{tabular}{ |c|c|c| } 
  \hline
    Symbol & Value & description \\ 
  \hline
  $x(0)$ & $1$ & First term of A.P  \\ 
  \hline
  $x(n)$ & $31$ & $\brak{n+1}\text{th}$ term \\
  \hline
  $\frac{x\brak{7}}{x\brak{m-1}}$ & $\frac{5}{9}$ & ratio of $7$ th  and $(m-1)$ th numbers\\ 
  \hline
  $n$ & $m+2$ & number of terms \\
  \hline
\end{tabular}
\end{center}
\caption{}
\end{table}\\
The last term is
\begin{align}
x(n)&=x(0)+\brak{n}d\\
\implies31 &= 1 + \brak{m + 1}d \\
\implies30 &= \brak{m + 1}d \\
\implies\frac{30}{m + 1} &= d \label{eq11.9.2.4}
\end{align}
Now $7$th and $\brak{m-1}$th terms
\begin{align}
x\brak{7} &= x(0) + 7d\label{eq11.9.2.5}\\
x\brak{m-1} &= x(0) + \brak{m-1}d\label{eq11.9.2.6}
\end{align}
From  equations \eqref{eq11.9.2.5} and \eqref{eq11.9.2.6}\\
\begin{align}
   \frac{x(0) + 7d}{x(0) + \brak{m-1}d} &= \frac{5}{9} \label{eq11.9.2.7}
\end{align}
Substituting  \eqref{eq11.9.2.4} in \eqref{eq11.9.2.7}\\
\begin{align}
\implies \frac{1+7\brak{{\frac{30}{m+1}}}}{1+\brak{{m-1}}\brak{\frac{30}{m+1}}} &= \frac{5}{9} \\
\implies \frac{m+1+210}{m+1+30m-30} &= \frac{5}{9}\\
\implies \frac{m+181}{31m-29} &= \frac{5}{9}\\
\implies 9m+1899 &=155m-145\\
\implies 155m-9m &=1899+145\\
\implies 146m &=2044\\
\implies m &=14
\end{align}
Therefore, $m = 14$ .\\
 \text{General term of AP is} \\
\begin{align}
    x\brak{n}&=\brak{2n+1}u(n)\\
    x\brak{n}&=\brak{2n}u\brak{n}+u\brak{n}
\end{align}
\begin{figure}
    \centering
    \includegraphics[width=1.0\linewidth]{ncert-maths/11/9/2/16/figs/test.png}
    \caption{Plot of x(n) vs n}
    \label{fig:11.9.2.1}
\end{figure}\\
The Z-Transform is\\
\begin{align}
    X\brak{z}&=2\brak{\dfrac{z}{\brak{z-1}^{2}}}+U\brak{z}\\
    &=\dfrac{2z}{\brak{z-1}^{2}}+\dfrac{1}{1-z^{-1}}\\
    X\brak{z}&=\dfrac{z^2+z}{\brak{z-1}^{2}} \quad{|z|>1}
\end{align}


\item Check whether -150 is a term of the AP: 11,8,5,2,....

 \solution
 \let\negmedspace\undefined
\let\negthickspace\undefined
\documentclass[journal,12pt,onecolumn]{IEEEtran}
\usepackage{cite}
\usepackage{amsmath,amssymb,amsfonts,amsthm}
\usepackage{algorithmic}
\usepackage{graphicx}
\usepackage{textcomp}
\usepackage{xcolor}
\usepackage{txfonts}
\usepackage{listings}
\usepackage{enumitem}
\usepackage{mathtools}
\usepackage{gensymb}
\usepackage{comment}
\usepackage[breaklinks=true]{hyperref}
\usepackage{tkz-euclide} % loads  TikZ and tkz-base
\usepackage{listings}
\usepackage[latin1]{inputenc}                                
\usepackage{color}                                            
\usepackage{array}                                            
\usepackage{longtable}                                       
\usepackage{calc}                                             
\usepackage{multirow}                                         
\usepackage{hhline}                                           
\usepackage{ifthen}                                           
\usepackage{lscape}
\usepackage{caption}


\newtheorem{theorem}{Theorem}[section]
\newtheorem{problem}{Problem}
\newtheorem{proposition}{Proposition}[section]
\newtheorem{lemma}{Lemma}[section]
\newtheorem{corollary}[theorem]{Corollary}
\newtheorem{example}{Example}[section]
\newtheorem{definition}[problem]{Definition}
%\newtheorem{thm}{Theorem}[section] 
%\newtheorem{defn}[thm]{Definition}
%\newtheorem{algorithm}{Algorithm}[section]
%\newtheorem{cor}{Corollary}
\newcommand{\BEQA}{\begin{eqnarray}}
\newcommand{\EEQA}{\end{eqnarray}}
\newcommand{\define}{\stackrel{\triangle}{=}}
\theoremstyle{remark}
\newtheorem{rem}{Remark}
%\bibliographystyle{ieeetr}

\begin{document}

%
\providecommand{\pr}[1]{\ensuremath{\Pr\left(#1\right)}}
\providecommand{\prt}[2]{\ensuremath{p_{#1}^{\left(#2\right)} }}        % own macro for this question
\providecommand{\qfunc}[1]{\ensuremath{Q\left(#1\right)}}
\providecommand{\sbrak}[1]{\ensuremath{{}\left[#1\right]}}
\providecommand{\lsbrak}[1]{\ensuremath{{}\left[#1\right.}}
\providecommand{\rsbrak}[1]{\ensuremath{{}\left.#1\right]}}
\providecommand{\brak}[1]{\ensuremath{\left(#1\right)}}
\providecommand{\lbrak}[1]{\ensuremath{\left(#1\right.}}
\providecommand{\rbrak}[1]{\ensuremath{\left.#1\right)}}
\providecommand{\cbrak}[1]{\ensuremath{\left\{#1\right\}}}
\providecommand{\lcbrak}[1]{\ensuremath{\left\{#1\right.}}
\providecommand{\rcbrak}[1]{\ensuremath{\left.#1\right\}}}
\newcommand{\sgn}{\mathop{\mathrm{sgn}}}
\providecommand{\abs}[1]{\left\vert#1\right\vert}
\providecommand{\res}[1]{\Res\displaylimits_{#1}} 
\providecommand{\norm}[1]{\left\lVert#1\right\rVert}
%\providecommand{\norm}[1]{\lVert#1\rVert}
\providecommand{\mtx}[1]{\mathbf{#1}}
\providecommand{\mean}[1]{E\left[ #1 \right]}
\providecommand{\cond}[2]{#1\middle|#2}
\providecommand{\fourier}{\overset{\mathcal{F}}{ \rightleftharpoons}}
\newenvironment{amatrix}[1]{%
  \left(\begin{array}{@{}*{#1}{c}|c@{}}
}{%
  \end{array}\right)
}
%\providecommand{\hilbert}{\overset{\mathcal{H}}{ \rightleftharpoons}}
%\providecommand{\system}{\overset{\mathcal{H}}{ \longleftrightarrow}}
        %\newcommand{\solution}[2]{\textbf{Solution:}{#1}}
\newcommand{\solution}{\noindent \textbf{Solution: }}
\newcommand{\cosec}{\,\text{cosec}\,}
\providecommand{\dec}[2]{\ensuremath{\overset{#1}{\underset{#2}{\gtrless}}}}
\newcommand{\myvec}[1]{\ensuremath{\begin{pmatrix}#1\end{pmatrix}}}
\newcommand{\mydet}[1]{\ensuremath{\begin{vmatrix}#1\end{vmatrix}}}
\newcommand{\myaugvec}[2]{\ensuremath{\begin{amatrix}{#1}#2\end{amatrix}}}
\providecommand{\rank}{\text{rank}}
\providecommand{\pr}[1]{\ensuremath{\Pr\left(#1\right)}}
\providecommand{\qfunc}[1]{\ensuremath{Q\left(#1\right)}}
        \newcommand*{\permcomb}[4][0mu]{{{}^{#3}\mkern#1#2_{#4}}}
\newcommand*{\perm}[1][-3mu]{\permcomb[#1]{P}}
\newcommand*{\comb}[1][-1mu]{\permcomb[#1]{C}}
\providecommand{\qfunc}[1]{\ensuremath{Q\left(#1\right)}}
\providecommand{\gauss}[2]{\mathcal{N}\ensuremath{\left(#1,#2\right)}}
\providecommand{\diff}[2]{\ensuremath{\frac{d{#1}}{d{#2}}}}
\providecommand{\myceil}[1]{\left \lceil #1 \right \rceil }
\newcommand\figref{Fig.~\ref}
\newcommand\tabref{Table~\ref}
\newcommand{\sinc}{\,\text{sinc}\,}
\newcommand{\rect}{\,\text{rect}\,}
%%
%       %\newcommand{\solution}[2]{\textbf{Solution:}{#1}}
%\newcommand{\solution}{\noindent \textbf{Solution: }}
%\newcommand{\cosec}{\,\text{cosec}\,}
%\numberwithin{equation}{section}
%\numberwithin{equation}{subsection}
%\numberwithin{problem}{section}
%\numberwithin{definition}{section}
%\makeatletter
%\@addtoreset{figure}{problem}
%\makeatother

%\let\StandardTheFigure\thefigure
\let\vec\mathbf

\bibliographystyle{IEEEtran}

\vspace{3cm}
\title{Assignment}
\author{EE23BTECH11001 - Aashna Sahu}
\maketitle
\bigskip

\renewcommand{\thefigure}{\theenumi}
\renewcommand{\thetable}{\theenumi}
%\renewcommand{\theequation}{\theenumi}
Q:Check whether -150 is a term of the AP: 11,8,5,2,....

 \solution

\begin{align}
x(n)&=x(0)+nd\\
n&=\frac{x(n)-x(0)}{d}
\end{align}
\begin{align}
x(n)-x(0) &\equiv 0 \pmod{d}
\end{align}
On substitutings values\\
\begin{align}
-161 &\equiv 2 \pmod{-3}
\end{align}
Thus -150 is not a term of the given AP.
\begin{align}
 \boxed{x(n)=(11-3n)\times u(n)}   
\end{align}

\begin{align}
   X(z)&=\frac{11}{1-z^{-1}}-\frac{3z^{-1}}{(1-z^{-1})^2}\quad
    |z|>1
\end{align}

    \begin{table}[h]
    \centering
    
        \begin{tabular}{|c|c|c|}
\hline 
   \textbf{Parameter}  &\textbf{Description} &\textbf{Value} \\
\hline
&&\\
$I_r$&Net Intensity of light at $\Delta x =\dfrac{\lambda}{3}$ &$\dfrac{K}{4}$ \\&&\\
\hline
\end{tabular}

        
    \caption{Input parameters}
    \label{tab:Table1}
\end{table}
\newpage
\begin{figure}[h]
  \centering
  \includegraphics[width=1.2\columnwidth]{figs/Figure_1.png}
  \captionsetup {justification=centering}
  \caption{Representation of x(n)}
  \label{fig:fig1}
\end{figure}
\end{document}

 

 \item Write the first five terms of the sequence \(a_n = \frac{n(n^2+5)}{4}\).

\solution
\input{ncert-maths/11/9/1/6/file1.tex}


\item
\begin{enumerate}
\item 30th term of the AP: 10, 7, 4, $\ldots$ is 
\item 11th term of the AP: $-3, -\frac{1}{2}, 2, \ldots$ is
\end{enumerate}
\solution
\let\negmedspace\undefined
\let\negthickspace\undefined
\documentclass[journal,12pt,twocolumn]{IEEEtran}
\usepackage{cite}
\usepackage{amsmath,amssymb,amsfonts,amsthm}
\usepackage{algorithmic}
\usepackage{graphicx}
\usepackage{textcomp}
\usepackage{xcolor}
\usepackage{txfonts}
\usepackage{listings}
\usepackage{enumitem}
\usepackage{mathtools}
\usepackage{gensymb}
\usepackage{comment}
\usepackage[breaklinks=true]{hyperref}
\usepackage{tkz-euclide}
\usepackage{listings}
\usepackage{gvv}
\def\inputGnumericTable{}
\usepackage[latin1]{inputenc}
\usepackage{color}
\usepackage{array}
\usepackage{longtable}
\usepackage{calc}
\usepackage{multirow}
\usepackage{hhline}
\usepackage{ifthen}
\usepackage{lscape}

\newtheorem{theorem}{Theorem}[section]
\newtheorem{problem}{Problem}
\newtheorem{proposition}{Proposition}[section]
\newtheorem{lemma}{Lemma}[section]
\newtheorem{corollary}[theorem]{Corollary}
\newtheorem{example}{Example}[section]
\newtheorem{definition}[problem]{Definition}
\newcommand{\BEQA}{\begin{eqnarray}}
\newcommand{\EEQA}{\end{eqnarray}}
\newcommand{\define}{\stackrel{\triangle}{=}}
\theoremstyle{remark}
\newtheorem{rem}{Remark}
\begin{document}

\bibliographystyle{IEEEtran}
\vspace{3cm}

\title{NCERT Discrete - 10.5.2.2}
\author{EE23BTECH11058 - Sindam Ananya$^{*}$% <-this % stops a space
}
\maketitle
\newpage
\bigskip

\renewcommand{\thefigure}{\theenumi}
\renewcommand{\thetable}{\theenumi}

\vspace{3cm}
\textbf{Question 10.5.2.2:} 
\begin{enumerate}
\item 30th term of the AP: 10, 7, 4, $\ldots$ is 
\item 11th term of the AP: $-3, -\frac{1}{2}, 2, \ldots$ is
\end{enumerate}
\solution
\begin{table}[h!]
    \centering
    \begin{tabular}{|c|c|c|}
\hline 
   \textbf{Parameter}  &\textbf{Description} &\textbf{Value} \\
\hline
&&\\
$I_r$&Net Intensity of light at $\Delta x =\dfrac{\lambda}{3}$ &$\dfrac{K}{4}$ \\&&\\
\hline
\end{tabular}

    \caption{Input Parameters}
    \label{tab:table1}
    \end{table}
\begin{equation}
    x_i(n) = \sbrak{x_i(0) + nd_i} u(n)
    \label{eq:eq1}
\end{equation}
\begin{enumerate}
\item From \eqref{eq:eq1} \tabref{tab:table1} :
\begin{align}
x_1(n) &= \sbrak{10 -3n}u(n)\\
x_1(29) &= -77\\
X_1(z) &= \frac{10 - 13z^{-1}}{(1-z^{-1})^2} \quad \abs{z} > 1
\end{align}
\item From \eqref{eq:eq1} and \tabref{tab:table1} :
\begin{align}
x_2(n) &= \sbrak{-3 + \frac{5}{2}n}u(n)\\
x_2(10) &= 22\\
X_2(z) &= \frac{0.5z^{-1}-3}{(1-z^{-1})^2} \quad \abs{z}> 1
\end{align}
\end{enumerate}
\begin{figure}[h!]
    \centering
    \includegraphics[width=\columnwidth]{figs/plot.png}
    \caption{stem plots of $x_1(n)$ and $x_2(n)$}
    \label{fig:1}
\end{figure}
\end{document}



\item Write the first five terms of the sequence whose nth term is $\frac{2n-3}{6}$ and obtain the Z transform of the series
\solution
\let\negmedspace\undefined
\let\negthickspace\undefined
\documentclass[journal,12pt,twocolumn]{IEEEtran}
\usepackage{cite}
\usepackage{amsmath,amssymb,amsfonts,amsthm}
\usepackage{algorithmic}
\usepackage{graphicx}
\usepackage{textcomp}
\usepackage{xcolor}
\usepackage{txfonts}
\usepackage{listings}
\usepackage{enumitem}
\usepackage{mathtools}
\usepackage{gensymb}
\usepackage{comment}
\usepackage[breaklinks=true]{hyperref}
\usepackage{tkz-euclide} 
\usepackage{listings}
\usepackage{gvv}                                        
\def\inputGnumericTable{}                                 
\usepackage[latin1]{inputenc}                                
\usepackage{color}                                            
\usepackage{array}                                            
\usepackage{longtable}                                       
\usepackage{calc}                                             
\usepackage{multirow}                                         
\usepackage{hhline}                                           
\usepackage{ifthen}                                           
\usepackage{lscape}

\newtheorem{theorem}{Theorem}[section]
\newtheorem{problem}{Problem}
\newtheorem{proposition}{Proposition}[section]
\newtheorem{lemma}{Lemma}[section]
\newtheorem{corollary}[theorem]{Corollary}
\newtheorem{example}{Example}[section]
\newtheorem{definition}[problem]{Definition}
\newcommand{\BEQA}{\begin{eqnarray}}
\newcommand{\EEQA}{\end{eqnarray}}
\newcommand{\define}{\stackrel{\triangle}{=}}
\theoremstyle{remark}
\newtheorem{rem}{Remark}

\begin{document}
\bibliographystyle{IEEEtran}

\vspace{3cm}

\title{}
\author{EE23BTECH11047 - Deepakreddy P
}
\maketitle
\newpage
\bigskip

\section*{Exercise 9.1}

\noindent \textbf{4} \quad Write the first five terms of the sequence whose nth term is $\frac{2n-3}{6}$ and obtain the Z transform of the series\\
\solution
\begin{align}
x \brak{n} &= \frac{2n-1}{6} \brak{u\brak{n}}
\label{x(n)}
\end{align}

\begin{figure}[h]
   \centering
   \includegraphics[width=1\columnwidth]{figs/plot.png}
   \caption{Plot of x(n) vs n}
   \label{fig: 9.1.4.1}
\end{figure}

\begin{align}
X(z) &= {\frac{3z^{-1}-1}{6(1-z^{-1})^{2}}\quad|z|>1}
\end{align}


\end{document}


 \item For what values of x, the numbers $-\frac{2}{7}\,,x,-\frac{7}{2}\,$ are in G.P ?

\solution
\iffalse
\let\negmedspace\undefined
\let\negthickspace\undefined
\documentclass[journal,12pt,twocolumn]{IEEEtran}
\usepackage{cite}
\usepackage{amsmath,amssymb,amsfonts,amsthm}
\usepackage{algorithmic}
\usepackage{graphicx}
\usepackage{textcomp}
\usepackage{xcolor}
\usepackage{txfonts}
\usepackage{listings}
\usepackage{enumitem}
\usepackage{mathtools}
\usepackage{gensymb}
\usepackage{comment}
\usepackage[breaklinks=true]{hyperref}
\usepackage{tkz-euclide} 
\usepackage{listings}
\usepackage{gvv}                                        
\def\inputGnumericTable{}                                 
\usepackage[latin1]{inputenc}                                
\usepackage{color}                                            
\usepackage{array}                                            
\usepackage{longtable}                                       
\usepackage{calc}                                             
\usepackage{multirow}                                         
\usepackage{hhline}                                           
\usepackage{ifthen}                                           
\usepackage{lscape}

\newtheorem{theorem}{Theorem}[section]
\newtheorem{problem}{Problem}
\newtheorem{proposition}{Proposition}[section]
\newtheorem{lemma}{Lemma}[section]
\newtheorem{corollary}[theorem]{Corollary}
\newtheorem{example}{Example}[section]
\newtheorem{definition}[problem]{Definition}
\newcommand{\BEQA}{\begin{eqnarray}}
\newcommand{\EEQA}{\end{eqnarray}}
\newcommand{\define}{\stackrel{\triangle}{=}}
\theoremstyle{remark}
\newtheorem{rem}{Remark}
\begin{document}

\bibliographystyle{IEEEtran}
\vspace{3cm}

\title{11.9.3.6}
\author{EE23BTECH11022 - G DILIP REDDY}
\maketitle
\newpage

\bigskip

\renewcommand{\thefigure}{\theenumi}
\renewcommand{\thetable}{\theenumi}
\textbf{Question}:\\
For what values of x, the numbers $-\frac{2}{7}\,,x,-\frac{7}{2}\,$ are in G.P ?
\\\\
\textbf{Solution: }\\
\fi
\begin{table}[h]
    \centering
    \begin{tabular}[12.1pt]{ |c| c| c|}
    \hline
    \textbf{Variable} & \textbf{Description} &\textbf{Value}\\ 
    \hline
    $x(0)$ & First term of the GP &$-\brak{\frac{2}{7}}$ \\
    \hline 
    $x(1)$ & Second term of the GP &$x$ \\
    \hline 
    $x(2)$ & Third term of the GP &$-\brak{\frac{7}{2}}$ \\
    \hline 
    $r$ & Common ratio of the GP & \\
    \hline
    $x(n)$ & General term & $x(0)\,r^n\,u(n)$\\
    \hline    
\end{tabular}

    \caption{Variables Used}
    \label{tab:table_11.9.3.6}
\end{table}
Let $r$ be the common ratio\\
From \tabref{tab:table_11.9.3.6}:
\begin{align}
\implies \frac{x}{\brak{-\frac{2}{7}\,}}\,&= \frac{\brak{-\frac{7}{2}\,}}{x}\,=r \\
x^2&=\brak{-\frac{2}{7}\,}\cdot\brak{-\frac{7}{2}\,}\\
x&=\pm 1\\
\implies r&=\pm \frac{7}{2}\,\\\notag
\end{align}
The signal corresponding to this is 
\begin{align}
x(n)=\brak{-\frac{2}{7}}\brak{\pm \frac{7}{2}}^n\,u(n)
\end{align}
Applying z-Transform :
\begin{align}
\implies X_1(z)&=\brak{\frac{1}{7}}\brak{\frac{4}{7z^{-1}+2}\,}
\quad \abs{z}>\frac{7}{2}\\
\implies X_2(z)&=\brak{\frac{1}{7}}\brak{\frac{4}{7z^{-1}-2}\,}
\quad \abs{z}>\frac{7}{2}
\end{align}
\begin{figure}[h]
    \centering
    \includegraphics[width=1.1\linewidth]{ncert-maths/11/9/3/6/figs/graph1.png}
    \caption{Stem Plot of $x_1$(n)}
    \label{stemplot1}
\end{figure}
\begin{figure}[h]
    \centering
    \includegraphics[width=1.1\linewidth]{ncert-maths/11/9/3/6/figs/graph2.png}
    \caption{Stem Plot of $x_2(n)$}
    \label{stemplot2}
\end{figure}
%\end{document}



\item Find the $20^{th}$ and $n^{th}$ terms of the G.P $\frac{5}{2}$, $\frac{5}{4}$, $\frac{5}{8}$,.....

\solution
% \iffalse
\let\negmedspace\undefined
\let\negthickspace\undefined
\documentclass[journal,12pt,twocolumn]{IEEEtran}
\usepackage{cite}
\usepackage{amsmath,amssymb,amsfonts,amsthm}
\usepackage{algorithmic}
\usepackage{graphicx}
\usepackage{textcomp}
\usepackage{xcolor}
\usepackage{txfonts}
\usepackage{listings}
\usepackage{enumitem}
\usepackage{mathtools}
\usepackage{gensymb}
\usepackage{comment}
\usepackage[breaklinks=true]{hyperref}
\usepackage{tkz-euclide} 
\usepackage{listings}
\usepackage{gvv}                                        
\def\inputGnumericTable{}                                 
\usepackage[latin1]{inputenc}                                
\usepackage{color}                                            
\usepackage{array}                                            
\usepackage{longtable}                                       
\usepackage{calc}                                             
\usepackage{multirow}                                         
\usepackage{hhline}                                           
\usepackage{ifthen}                                           
\usepackage{lscape}
\usepackage{caption}
\newtheorem{theorem}{Theorem}[section]
\newtheorem{problem}{Problem}
\newtheorem{proposition}{Proposition}[section]
\newtheorem{lemma}{Lemma}[section]
\newtheorem{corollary}[theorem]{Corollary}
\newtheorem{example}{Example}[section]
\newtheorem{definition}[problem]{Definition}
\newcommand{\BEQA}{\begin{eqnarray}}
\newcommand{\EEQA}{\end{eqnarray}}
\newcommand{\define}{\stackrel{\triangle}{=}}
\theoremstyle{remark}
\newtheorem{rem}{Remark}
\begin{document}
\parindent 0px
\bibliographystyle{IEEEtran}
\vspace{3cm}

\title{NCERT 11.9.3 1Q}
\author{EE23BTECH11013 - Avyaaz$^{*}$% <-this % stops a space
}
\maketitle
\newpage
\bigskip

\renewcommand{\thefigure}{\arabic{figure}}
\renewcommand{\thetable}{\arabic{table}}
\large\textbf{\textsl{Question:}}
Find the $20^{th}$ and $n^{th}$ terms of the G.P $\frac{5}{2}$, $\frac{5}{4}$, $\frac{5}{8}$,.....

\solution
 \begin{table}[htbp]
     \centering
     \setlength{\extrarowheight}{8pt}
    \begin{tabular}{|c|c|c|}
\hline 
   \textbf{Parameter}  &\textbf{Description} &\textbf{Value} \\
\hline
&&\\
$I_r$&Net Intensity of light at $\Delta x =\dfrac{\lambda}{3}$ &$\dfrac{K}{4}$ \\&&\\
\hline
\end{tabular}

     \caption{Parameters}
     \label{tab:table1.11.9.3.1}
 \end{table} 

% \begin{align}
%    x(n) = \dfrac{5}{2}\left(\dfrac{1}{2}\right)^n 
% \end{align}

% \begin{align}
% 	x \brak{n} & \system{Z} X \brak{z} \\
%    % x(n) &=\dfrac{5}{2}\left(\dfrac{1}{2}\right)^n u(n) \\
%     \therefore X(z) &= \sum_{n=-\infty}^{\infty}x(n)z^{-n}\label{eq:z-transform}  
% \end{align}
% Here, 
%          $    u(n) = \begin{cases}
%                 0 &\text{for } n < 0 \\
%                 1 & \text{for } n \geq 0
%             \end{cases}$       
 
%  \vspace{1cm}
From \tabref{tab:table1.11.9.3.1}:
\(Z\)-Transform of \(x(n)\):
\begin{align}
% \implies X(z) &= \sum_{n=-\infty}^{\infty}\left(\dfrac{5}{2}\left(\dfrac{1}{2}\right)^n u(n)\right) z^{-n} \\
 % \implies X(z) &= \dfrac{5}{2}\sum_{n=0}^{\infty}\left(\dfrac{z
 % ^{-1}}{2}\right)^n \\
\implies X(z) &=\dfrac{5}{2}\left(\dfrac{1}{1-\frac{z^{-1}}{2}}\right) ;\cbrak{z\in\mathbb{C} : |z|>\dfrac{1}{2}}
\end{align}

\begin{figure}[ht]
    \centering
    \includegraphics[width = \columnwidth]{figs/stem_plot.png}
    \caption{}
	\label{fig:graph1.11.9.3.1}
\end{figure} 

\bibliographystyle{IEEEtran}
\end{document}



\item 
Which term of the following sequences:\\
(a) 2,$2\sqrt{2}$,4\dots is 128
\quad(b) $\sqrt{3}$,3,$3\sqrt{3}$\dots is 729\\
(c) $\frac{1}{3}$,$\frac{1}{9}$,$\frac{1}{27}$\dots is $\frac{1}{19683}$ \\
\solution
\iffalse
\let\negmedspace\undefined
\let\negthickspace\undefined
\documentclass[journal,12pt,twocolumn]{IEEEtran}
\usepackage{cite}
\usepackage{amsmath,amssymb,amsfonts,amsthm}
\usepackage{algorithmic}
\usepackage{graphicx}
\usepackage{textcomp}
\usepackage{xcolor}
\usepackage{txfonts}
\usepackage{listings}
\usepackage{enumitem}
\usepackage{mathtools}
\usepackage{gensymb}
\usepackage{comment}
\usepackage[breaklinks=true]{hyperref}
\usepackage{tkz-euclide} 
\usepackage{listings}
\usepackage{gvv}                                        
\def\inputGnumericTable{}                                 
\usepackage[latin1]{inputenc}                                
\usepackage{color}                                            
\usepackage{array}                                            
\usepackage{longtable}                                       
\usepackage{calc}                                             
\usepackage{multirow}                                         
\usepackage{hhline}                                           
\usepackage{ifthen}                                           
\usepackage{lscape}
\usepackage[center]{caption} % center the captions to figure

\newtheorem{theorem}{Theorem}[section]
\newtheorem{problem}{Problem}
\newtheorem{proposition}{Proposition}[section]
\newtheorem{lemma}{Lemma}[section]
\newtheorem{corollary}[theorem]{Corollary}
\newtheorem{example}{Example}[section]
\newtheorem{definition}[problem]{Definition}
\newcommand{\BEQA}{\begin{eqnarray}}
\newcommand{\EEQA}{\end{eqnarray}}
\newcommand{\define}{\stackrel{\triangle}{=}}
\theoremstyle{remark}
\newtheorem{rem}{Remark}
\begin{document}

\newcolumntype{M}[1]{>{\centering\arraybackslash}m{#1}}
\newcolumntype{N}{@{}m{0pt}@{}}

\bibliographystyle{IEEEtran}
\vspace{3cm}

\title{NCERT 11.9.3 5Q} 
\author{ee23btech11223 - Soham Prabhakar More% <-this % stops a space
}
\maketitle
\newpage
\bigskip

\renewcommand{\thefigure}{\theenumi}
\renewcommand{\thetable}{\theenumi}

\bibliographystyle{IEEEtran}

\textbf{Question:}\\
Which term of the following sequences:\\
(a) 2,$2\sqrt{2}$,4\dots is 128
\quad(b) $\sqrt{3}$,3,$3\sqrt{3}$\dots is 729\\
(c) $\frac{1}{3}$,$\frac{1}{9}$,$\frac{1}{27}$\dots is $\frac{1}{19683}$
\fi 
For a general GP series and $k > 0$,
\begin{align}
    x\brak{k} &= x\brak{0}r^k \\
    \therefore k &= \log_r{\frac{x\brak{k}}{x\brak{0}}} \label{eq:gsoln}
\end{align}
And the Z-transform $X\brak{z}$:
\begin{align}
    X\brak{z} &= \frac{x\brak{0}}{1 - rz^{-1}} \quad {\abs{z} > \abs{r}} \label{eq:zresult}
\end{align}

\begin{enumerate}[label=(\alph*)]
\item By \tabref{Table:1}, \eqref{eq:gsoln} and \tabref{Table:1}: % prob:a
\begin{align}
    x_1\brak{n} &= x_1\brak{0} r_1^nu\brak{n} \\
    k_1 &= \log_{r_1}{\frac{128}{x_1\brak{0}}} \\
    \therefore k_1 &= 12 \\
	X_1\brak{z} &= \frac{2}{1 - \sqrt{2}z^{-1}} \quad \abs{z} > \sqrt{2}
\end{align}

\begin{figure}[h!]
    \renewcommand\thefigure{1}
    \centering
    \includegraphics[width=\columnwidth]{ncert-maths/11/9/3/5/figs/a.png}
    \caption[short]{Plot of $x_1$\brak{n} vs n. See \tabref{Table:1}}
    \label{fig:img1}
\end{figure}



\item By \eqref{eq:gsoln}, \eqref{eq:zresult} and \tabref{Table:1}: % prob:b
\begin{align}
    x_2\brak{n} &= x_2\brak{0} r_2^nu\brak{n} \\
    k_2 &= \log_{r_2}{\frac{729}{x_2\brak{0}}} \\
    \therefore k_2 &= 11 \\
    X_2\brak{z} &= \frac{\sqrt{3}}{1 - \sqrt{3}z^{-1}} \quad \abs{z} > \sqrt{3} 
\end{align}

\begin{figure}[h!]
    \renewcommand\thefigure{2}
    \centering
    \includegraphics[width=\columnwidth]{ncert-maths/11/9/3/5/figs/b.png}
    \caption[short]{Plot of $x_2$\brak{n} vs n. See \tabref{Table:1}}
    \label{fig:img2}
\end{figure}

\item By \eqref{eq:gsoln}, \eqref{eq:zresult} and \tabref{Table:1}: % prob:c
\begin{align}
    x_3\brak{n} &= x_3\brak{0} r_3^nu\brak{n} \\
    k_3 &= \log_{r_3}{\frac{1}{19683 x_3\brak{0}}} \\
    \therefore k_3 &= 8 \\
    X_3\brak{z} &= \frac{1}{3 - z^{-1}} \quad \abs{z} > \frac{1}{3}
\end{align}

\begin{figure}[h!]
    \renewcommand\thefigure{3}
    \centering
    \includegraphics[width=0.9\columnwidth]{ncert-maths/11/9/3/5/figs/c.png}
    \caption[short]{Plot of $x_3$\brak{n} vs n. See \tabref{Table:1}}
    \label{fig:img3}
\end{figure}

\begin{table}[ht]
\begin{tabular}{|c|c|c|}
    \hline 
    \textbf{Parameter}&\textbf{Description} &\textbf{Value}\\
    \hline 
    $r_i$ & Common ratio of G.P (a),(b),(c) & $\sqrt{2}, \sqrt{3}, \frac{1}{3}$ \\
    \hline
    $x_i(0)$ & Initial Values & $2, \sqrt{3}, \frac{1}{3}$ \\
    \hline
    $x_i(k_i)$ & Given Values & $128, 729, \frac{1}{19683}$ \\
    \hline 
    $k_i$ & Desired index & $12, 11, 8$ \\
    \hline 
    $x_i\brak{n}$ & Series & $x_i\brak{0}r_i^nu\brak{n}$ \\
    \hline
	$X_i\brak{z}$ & Z-Transform of $x_i\brak{n}$ & $\frac{x\brak{0}}{1-rz^{-1}}$ \\
    \hline
\end{tabular}

\caption{Table of parameters}
\label{Table:1}


\end{table}

\end{enumerate}

Find the $20^{th}$ and $n^{th}$ terms of the G.P $\frac{5}{2}$, $\frac{5}{4}$, $\frac{5}{8}$,.....

% \item 
% Which term of the following sequences:\\
% (a) 2,$2\sqrt{2}$,4\dots is 128
% \quad(b) $\sqrt{3}$,3,$3\sqrt{3}$\dots is 729\\
% (c) $\frac{1}{3}$,$\frac{1}{9}$,$\frac{1}{27}$\dots is $\frac{1}{19683}$ \\
% \solution
% \iffalse
\let\negmedspace\undefined
\let\negthickspace\undefined
\documentclass[journal,12pt,twocolumn]{IEEEtran}
\usepackage{cite}
\usepackage{amsmath,amssymb,amsfonts,amsthm}
\usepackage{algorithmic}
\usepackage{graphicx}
\usepackage{textcomp}
\usepackage{xcolor}
\usepackage{txfonts}
\usepackage{listings}
\usepackage{enumitem}
\usepackage{mathtools}
\usepackage{gensymb}
\usepackage{comment}
\usepackage[breaklinks=true]{hyperref}
\usepackage{tkz-euclide} 
\usepackage{listings}
\usepackage{gvv}                                        
\def\inputGnumericTable{}                                 
\usepackage[latin1]{inputenc}                                
\usepackage{color}                                            
\usepackage{array}                                            
\usepackage{longtable}                                       
\usepackage{calc}                                             
\usepackage{multirow}                                         
\usepackage{hhline}                                           
\usepackage{ifthen}                                           
\usepackage{lscape}
\usepackage[center]{caption} % center the captions to figure

\newtheorem{theorem}{Theorem}[section]
\newtheorem{problem}{Problem}
\newtheorem{proposition}{Proposition}[section]
\newtheorem{lemma}{Lemma}[section]
\newtheorem{corollary}[theorem]{Corollary}
\newtheorem{example}{Example}[section]
\newtheorem{definition}[problem]{Definition}
\newcommand{\BEQA}{\begin{eqnarray}}
\newcommand{\EEQA}{\end{eqnarray}}
\newcommand{\define}{\stackrel{\triangle}{=}}
\theoremstyle{remark}
\newtheorem{rem}{Remark}
\begin{document}

\newcolumntype{M}[1]{>{\centering\arraybackslash}m{#1}}
\newcolumntype{N}{@{}m{0pt}@{}}

\bibliographystyle{IEEEtran}
\vspace{3cm}

\title{NCERT 11.9.3 5Q} 
\author{ee23btech11223 - Soham Prabhakar More% <-this % stops a space
}
\maketitle
\newpage
\bigskip

\renewcommand{\thefigure}{\theenumi}
\renewcommand{\thetable}{\theenumi}

\bibliographystyle{IEEEtran}

\textbf{Question:}\\
Which term of the following sequences:\\
(a) 2,$2\sqrt{2}$,4\dots is 128
\quad(b) $\sqrt{3}$,3,$3\sqrt{3}$\dots is 729\\
(c) $\frac{1}{3}$,$\frac{1}{9}$,$\frac{1}{27}$\dots is $\frac{1}{19683}$
\fi 
For a general GP series and $k > 0$,
\begin{align}
    x\brak{k} &= x\brak{0}r^k \\
    \therefore k &= \log_r{\frac{x\brak{k}}{x\brak{0}}} \label{eq:gsoln}
\end{align}
And the Z-transform $X\brak{z}$:
\begin{align}
    X\brak{z} &= \frac{x\brak{0}}{1 - rz^{-1}} \quad {\abs{z} > \abs{r}} \label{eq:zresult}
\end{align}

\begin{enumerate}[label=(\alph*)]
\item By \tabref{Table:1}, \eqref{eq:gsoln} and \tabref{Table:1}: % prob:a
\begin{align}
    x_1\brak{n} &= x_1\brak{0} r_1^nu\brak{n} \\
    k_1 &= \log_{r_1}{\frac{128}{x_1\brak{0}}} \\
    \therefore k_1 &= 12 \\
	X_1\brak{z} &= \frac{2}{1 - \sqrt{2}z^{-1}} \quad \abs{z} > \sqrt{2}
\end{align}

\begin{figure}[h!]
    \renewcommand\thefigure{1}
    \centering
    \includegraphics[width=\columnwidth]{ncert-maths/11/9/3/5/figs/a.png}
    \caption[short]{Plot of $x_1$\brak{n} vs n. See \tabref{Table:1}}
    \label{fig:img1}
\end{figure}



\item By \eqref{eq:gsoln}, \eqref{eq:zresult} and \tabref{Table:1}: % prob:b
\begin{align}
    x_2\brak{n} &= x_2\brak{0} r_2^nu\brak{n} \\
    k_2 &= \log_{r_2}{\frac{729}{x_2\brak{0}}} \\
    \therefore k_2 &= 11 \\
    X_2\brak{z} &= \frac{\sqrt{3}}{1 - \sqrt{3}z^{-1}} \quad \abs{z} > \sqrt{3} 
\end{align}

\begin{figure}[h!]
    \renewcommand\thefigure{2}
    \centering
    \includegraphics[width=\columnwidth]{ncert-maths/11/9/3/5/figs/b.png}
    \caption[short]{Plot of $x_2$\brak{n} vs n. See \tabref{Table:1}}
    \label{fig:img2}
\end{figure}

\item By \eqref{eq:gsoln}, \eqref{eq:zresult} and \tabref{Table:1}: % prob:c
\begin{align}
    x_3\brak{n} &= x_3\brak{0} r_3^nu\brak{n} \\
    k_3 &= \log_{r_3}{\frac{1}{19683 x_3\brak{0}}} \\
    \therefore k_3 &= 8 \\
    X_3\brak{z} &= \frac{1}{3 - z^{-1}} \quad \abs{z} > \frac{1}{3}
\end{align}

\begin{figure}[h!]
    \renewcommand\thefigure{3}
    \centering
    \includegraphics[width=0.9\columnwidth]{ncert-maths/11/9/3/5/figs/c.png}
    \caption[short]{Plot of $x_3$\brak{n} vs n. See \tabref{Table:1}}
    \label{fig:img3}
\end{figure}

\begin{table}[ht]
\begin{tabular}{|c|c|c|}
    \hline 
    \textbf{Parameter}&\textbf{Description} &\textbf{Value}\\
    \hline 
    $r_i$ & Common ratio of G.P (a),(b),(c) & $\sqrt{2}, \sqrt{3}, \frac{1}{3}$ \\
    \hline
    $x_i(0)$ & Initial Values & $2, \sqrt{3}, \frac{1}{3}$ \\
    \hline
    $x_i(k_i)$ & Given Values & $128, 729, \frac{1}{19683}$ \\
    \hline 
    $k_i$ & Desired index & $12, 11, 8$ \\
    \hline 
    $x_i\brak{n}$ & Series & $x_i\brak{0}r_i^nu\brak{n}$ \\
    \hline
	$X_i\brak{z}$ & Z-Transform of $x_i\brak{n}$ & $\frac{x\brak{0}}{1-rz^{-1}}$ \\
    \hline
\end{tabular}

\caption{Table of parameters}
\label{Table:1}


\end{table}

\end{enumerate}

Find the $20^{th}$ and $n^{th}$ terms of the G.P $\frac{5}{2}$, $\frac{5}{4}$, $\frac{5}{8}$,.....

% \item 
% Which term of the following sequences:\\
% (a) 2,$2\sqrt{2}$,4\dots is 128
% \quad(b) $\sqrt{3}$,3,$3\sqrt{3}$\dots is 729\\
% (c) $\frac{1}{3}$,$\frac{1}{9}$,$\frac{1}{27}$\dots is $\frac{1}{19683}$ \\
% \solution
% \iffalse
\let\negmedspace\undefined
\let\negthickspace\undefined
\documentclass[journal,12pt,twocolumn]{IEEEtran}
\usepackage{cite}
\usepackage{amsmath,amssymb,amsfonts,amsthm}
\usepackage{algorithmic}
\usepackage{graphicx}
\usepackage{textcomp}
\usepackage{xcolor}
\usepackage{txfonts}
\usepackage{listings}
\usepackage{enumitem}
\usepackage{mathtools}
\usepackage{gensymb}
\usepackage{comment}
\usepackage[breaklinks=true]{hyperref}
\usepackage{tkz-euclide} 
\usepackage{listings}
\usepackage{gvv}                                        
\def\inputGnumericTable{}                                 
\usepackage[latin1]{inputenc}                                
\usepackage{color}                                            
\usepackage{array}                                            
\usepackage{longtable}                                       
\usepackage{calc}                                             
\usepackage{multirow}                                         
\usepackage{hhline}                                           
\usepackage{ifthen}                                           
\usepackage{lscape}
\usepackage[center]{caption} % center the captions to figure

\newtheorem{theorem}{Theorem}[section]
\newtheorem{problem}{Problem}
\newtheorem{proposition}{Proposition}[section]
\newtheorem{lemma}{Lemma}[section]
\newtheorem{corollary}[theorem]{Corollary}
\newtheorem{example}{Example}[section]
\newtheorem{definition}[problem]{Definition}
\newcommand{\BEQA}{\begin{eqnarray}}
\newcommand{\EEQA}{\end{eqnarray}}
\newcommand{\define}{\stackrel{\triangle}{=}}
\theoremstyle{remark}
\newtheorem{rem}{Remark}
\begin{document}

\newcolumntype{M}[1]{>{\centering\arraybackslash}m{#1}}
\newcolumntype{N}{@{}m{0pt}@{}}

\bibliographystyle{IEEEtran}
\vspace{3cm}

\title{NCERT 11.9.3 5Q} 
\author{ee23btech11223 - Soham Prabhakar More% <-this % stops a space
}
\maketitle
\newpage
\bigskip

\renewcommand{\thefigure}{\theenumi}
\renewcommand{\thetable}{\theenumi}

\bibliographystyle{IEEEtran}

\textbf{Question:}\\
Which term of the following sequences:\\
(a) 2,$2\sqrt{2}$,4\dots is 128
\quad(b) $\sqrt{3}$,3,$3\sqrt{3}$\dots is 729\\
(c) $\frac{1}{3}$,$\frac{1}{9}$,$\frac{1}{27}$\dots is $\frac{1}{19683}$
\fi 
For a general GP series and $k > 0$,
\begin{align}
    x\brak{k} &= x\brak{0}r^k \\
    \therefore k &= \log_r{\frac{x\brak{k}}{x\brak{0}}} \label{eq:gsoln}
\end{align}
And the Z-transform $X\brak{z}$:
\begin{align}
    X\brak{z} &= \frac{x\brak{0}}{1 - rz^{-1}} \quad {\abs{z} > \abs{r}} \label{eq:zresult}
\end{align}

\begin{enumerate}[label=(\alph*)]
\item By \tabref{Table:1}, \eqref{eq:gsoln} and \tabref{Table:1}: % prob:a
\begin{align}
    x_1\brak{n} &= x_1\brak{0} r_1^nu\brak{n} \\
    k_1 &= \log_{r_1}{\frac{128}{x_1\brak{0}}} \\
    \therefore k_1 &= 12 \\
	X_1\brak{z} &= \frac{2}{1 - \sqrt{2}z^{-1}} \quad \abs{z} > \sqrt{2}
\end{align}

\begin{figure}[h!]
    \renewcommand\thefigure{1}
    \centering
    \includegraphics[width=\columnwidth]{ncert-maths/11/9/3/5/figs/a.png}
    \caption[short]{Plot of $x_1$\brak{n} vs n. See \tabref{Table:1}}
    \label{fig:img1}
\end{figure}



\item By \eqref{eq:gsoln}, \eqref{eq:zresult} and \tabref{Table:1}: % prob:b
\begin{align}
    x_2\brak{n} &= x_2\brak{0} r_2^nu\brak{n} \\
    k_2 &= \log_{r_2}{\frac{729}{x_2\brak{0}}} \\
    \therefore k_2 &= 11 \\
    X_2\brak{z} &= \frac{\sqrt{3}}{1 - \sqrt{3}z^{-1}} \quad \abs{z} > \sqrt{3} 
\end{align}

\begin{figure}[h!]
    \renewcommand\thefigure{2}
    \centering
    \includegraphics[width=\columnwidth]{ncert-maths/11/9/3/5/figs/b.png}
    \caption[short]{Plot of $x_2$\brak{n} vs n. See \tabref{Table:1}}
    \label{fig:img2}
\end{figure}

\item By \eqref{eq:gsoln}, \eqref{eq:zresult} and \tabref{Table:1}: % prob:c
\begin{align}
    x_3\brak{n} &= x_3\brak{0} r_3^nu\brak{n} \\
    k_3 &= \log_{r_3}{\frac{1}{19683 x_3\brak{0}}} \\
    \therefore k_3 &= 8 \\
    X_3\brak{z} &= \frac{1}{3 - z^{-1}} \quad \abs{z} > \frac{1}{3}
\end{align}

\begin{figure}[h!]
    \renewcommand\thefigure{3}
    \centering
    \includegraphics[width=0.9\columnwidth]{ncert-maths/11/9/3/5/figs/c.png}
    \caption[short]{Plot of $x_3$\brak{n} vs n. See \tabref{Table:1}}
    \label{fig:img3}
\end{figure}

\begin{table}[ht]
\input{ncert-maths/11/9/3/5/tables/table.tex}
\end{table}

\end{enumerate}

Find the $20^{th}$ and $n^{th}$ terms of the G.P $\frac{5}{2}$, $\frac{5}{4}$, $\frac{5}{8}$,.....

% \item 
% Which term of the following sequences:\\
% (a) 2,$2\sqrt{2}$,4\dots is 128
% \quad(b) $\sqrt{3}$,3,$3\sqrt{3}$\dots is 729\\
% (c) $\frac{1}{3}$,$\frac{1}{9}$,$\frac{1}{27}$\dots is $\frac{1}{19683}$ \\
% \solution
% \input{ncert-maths/11/9/3/5/main.tex}
% \pagebreak

%\end{document}


% \pagebreak

%\end{document}


% \pagebreak

%\end{document}


\clearpage

\item The number of bacteria in a certain culture doubles every hour. If there were 30 bacteria present in the culture originally, how many bacteria will be present at the end of $2^{nd}$ hour, $4^{th}$ hour and $n^{th}$ hour?

\solution
\iffalse
\let\negmedspace\undefined
\let\negthickspace\undefined
\documentclass[journal,12pt,twocolumn]{IEEEtran}
\usepackage{cite}
\usepackage{amsmath,amssymb,amsfonts,amsthm}
\usepackage{algorithmic}
\usepackage{graphicx}
\usepackage{textcomp}
\usepackage{xcolor}
\usepackage{txfonts}
\usepackage{listings}
\usepackage{enumitem}
\usepackage{mathtools}
\usepackage{gensymb}
\usepackage{comment}
\usepackage[breaklinks=true]{hyperref}
\usepackage{tkz-euclide}
\usepackage{listings}
\usepackage{gvv}
\def\inputGnumericTable{}
\usepackage[latin1]{inputenc}
\usepackage{color}
\usepackage{array}
\usepackage{longtable}
\usepackage{calc}
\usepackage{multirow}
\usepackage{hhline}
\usepackage{ifthen}
\usepackage{lscape}

\newtheorem{theorem}{Theorem}[section]
\newtheorem{problem}{Problem}
\newtheorem{proposition}{Proposition}[section]
\newtheorem{lemma}{Lemma}[section]
\newtheorem{corollary}[theorem]{Corollary}
\newtheorem{example}{Example}[section]
\newtheorem{definition}[problem]{Definition}
\newcommand{\BEQA}{\begin{eqnarray}}
\newcommand{\EEQA}{\end{eqnarray}}
\newcommand{\define}{\stackrel{\triangle}{=}}
\theoremstyle{remark}
\newtheorem{rem}{Remark}
\begin{document}

\bibliographystyle{IEEEtran}
\vspace{3cm}

\title{NCERT Discrete - 11.9.3.30}
\author{EE23BTECH11007 - Aneesh Kadiyala$^{*}$% <-this % stops a space
}
\maketitle
\newpage
\bigskip

\renewcommand{\thefigure}{\theenumi}
\renewcommand{\thetable}{\theenumi}

\vspace{3cm}
\textbf{Question 11.9.3.30:} The number of bacteria in a certain culture doubles every hour. If there were 30 bacteria present in the culture originally, how many bacteria will be present at the end of $2^{nd}$ hour, $4^{th}$ hour and $n^{th}$ hour?
\\
\solution
\fi
\begin{table}[h!]
    \begin{tabular}{ | c | c | c | }
    \hline
    Parameter & Value & Description \\
    \hline
    $x(0)$ & 30 & Initial no. of bacteria\\
    \hline
    $r$ & 2 & Ratio of no. of bacteria at end of \\
    & & hour to start of hour (Common Ratio) \\
    \hline
    $x(n)$ & $r^nx(0)u(n)$ & $n^{th}$ term of the GP \\
    \hline
\end{tabular}
    \caption{Input Parameters}
    \label{tab:ncert_maths_11_9_3_30}
\end{table}
From \tabref{tab:ncert_maths_11_9_3_30}:
\begin{align}
x(2) &= 120 \\
x(4) &= 480 \\
x(n) &= 30(2^n)u(n)
\end{align}
\begin{figure}[h!]
    \centering
    \includegraphics[width=\columnwidth]{ncert-maths/11/9/3/30/figs/11_9_3_30.png}
    \caption{Plot of $x(n)$ vs $n$. See \tabref{tab:ncert_maths_11_9_3_30} for details.}
    \label{fig:ncert_maths_11_9_3_30}
\end{figure}
\begin{align}
X(z) = \frac{30z^{-1}}{1 - 2z^{-1}} \quad \abs{z} > 2
\end{align}
%\end{document}


\item Ramkali saved Rs 5 in the first week of a year and then increased her weekly savings by Rs 1.75. If in the $n$th week, her weekly savings become Rs 20.75, find $n$.

\solution
\let\negmedspace\undefined
\let\negthickspace\undefined
\documentclass[journal,12pt,twocolumn]{IEEEtran}
\usepackage{cite}
\usepackage{amsmath,amssymb,amsfonts,amsthm}
\usepackage{algorithmic}
\usepackage{graphicx}
\usepackage{textcomp}
\usepackage{xcolor}
\usepackage{txfonts}
\usepackage{listings}
\usepackage{enumitem}
\usepackage{mathtools}
\usepackage{gensymb}
\usepackage[breaklinks=true]{hyperref}
\usepackage{tkz-euclide} % loads  TikZ and tkz-base
\usepackage{listings}
\usepackage{gvv}


\newtheorem{theorem}{Theorem}[section]
\newtheorem{problem}{Problem}
\newtheorem{proposition}{Proposition}[section]
\newtheorem{lemma}{Lemma}[section]
\newtheorem{corollary}[theorem]{Corollary}
\newtheorem{example}{Example}[section]
\newtheorem{definition}[problem]{Definition}

\newcommand{\BEQA}{\begin{eqnarray}}
\newcommand{\EEQA}{\end{eqnarray}}
\newcommand{\define}{\stackrel{\triangle}{=}}
\theoremstyle{remark}
\newtheorem{rem}{Remark}

\graphicspath{./figs/}

%\bibliographystyle{ieeetr}
\begin{document}
%

\bibliographystyle{IEEEtran}


\vspace{3cm}

\title{
	%	\logo{
	Assignment-1 

	\large{EE:1205 Signals and Systems}

	Indian Institute of Technology, Hyderabad
	%	}
}
\author{Kunal Thorawade

EE23BTECH11035
}	

\maketitle


\newpage

%\tableofcontents

\bigskip
 
 \renewcommand{\thefigure}{\theenumi}
 \renewcommand{\thetable}{\theenumi}
 %\renewcommand{\theequation}{\theenumi}

 \section{\Large Question:}  Ramkali saved Rs 5 in the first week of a year and then increased her weekly savings by Rs 1.75. If in the $n$th week, her weekly savings become Rs 20.75, find $n$.

 \section{\Large Solution:} 
 \begin{tabular}{|c|c|c|}
\hline 
   \textbf{Parameter}  &\textbf{Description} &\textbf{Value} \\
\hline
&&\\
$I_r$&Net Intensity of light at $\Delta x =\dfrac{\lambda}{3}$ &$\dfrac{K}{4}$ \\&&\\
\hline
\end{tabular}


 \begin{align} 
	 x(n) &= x(0) + (n)(d)
	 \\ 20.75 &= 5 + (n)(1.75)  
	 \\ \implies 15.75 &= (n)(1.75)
	 \\ \implies n &= \frac{15.75}{1.75}
	 \\ \implies n &= 9
	 \\x(n) &= 5u(n) + 1.75nu(n)
 \end{align}
 The Z-transform of a sequence $x(n)$ is given by:
 \begin{align}
	  X(z) &= \frac{5z^{-1}}{1-z^{-1}}+\frac{1.75z^{-1}}{(1-z^{-1})^{2}} ; |z| > 1
 \end{align}

 \begin{figure}
	     \centering
	         \includegraphics[width = 8cm]{figs/fig1.png}
		     \caption{Plot of $x(n) = 5 + 1.75n$}
		         \label{fig:enter-label}
 \end{figure}
\end{document}



\item Show that the sum of $\brak {m+n}^{th}$ and $\brak {m-n}^{th}$ terms of an $A.P.,$ is equal to twice the $m^{th}$ terms.    \\
\solution
% \iffalse
\let\negmedspace\undefined
\let\negthickspace\undefined
\documentclass[journal,12pt,twocolumn]{IEEEtran}
\usepackage{cite}
\usepackage{amsmath,amssymb,amsfonts,amsthm}
\usepackage{algorithmic}
\usepackage{graphicx}
\usepackage{textcomp}
\usepackage{xcolor}
\usepackage{txfonts}
\usepackage{listings}
\usepackage{enumitem}
\usepackage{mathtools}
\usepackage{gensymb}
\usepackage{comment}
\usepackage[breaklinks=true]{hyperref}
\usepackage{tkz-euclide} 
\usepackage{listings}
\usepackage{gvv}                                        
\def\inputGnumericTable{}                                 
\usepackage[latin1]{inputenc}                                
\usepackage{color}                                            
\usepackage{array}                                            
\usepackage{longtable}                                       
\usepackage{calc}                                             
\usepackage{multirow}                                         
\usepackage{hhline}                                           
\usepackage{ifthen}                                           
\usepackage{lscape}

\newtheorem{theorem}{Theorem}[section]
\newtheorem{problem}{Problem}
\newtheorem{proposition}{Proposition}[section]
\newtheorem{lemma}{Lemma}[section]
\newtheorem{corollary}[theorem]{Corollary}
\newtheorem{example}{Example}[section]
\newtheorem{definition}[problem]{Definition}
\newcommand{\BEQA}{\begin{eqnarray}}
\newcommand{\EEQA}{\end{eqnarray}}
\newcommand{\define}{\stackrel{\triangle}{=}}
\theoremstyle{remark}
\newtheorem{rem}{Remark}
\begin{document}
\parindent 0px
\bibliographystyle{IEEEtran}
\title{Assignment 11.9.5\_1Q}
\author{EE22BTECH11219 - Rada Sai Sujan$^{}$% <-this % stops a space
}
\maketitle
\newpage
\bigskip
\section*{Question}
Show that the sum of $\brak {m+n}^{th}$ and $\brak {m-n}^{th}$ terms of an $A.P.,$ is equal to twice the $m^{th}$ terms.    \\
\solution

\begin{table}[ht]
    \centering
    \def\arraystretch{1.5}
    \begin{tabular}{|c|c|c|}
    \hline
    PARAMETER & VALUE & DESCRIPTION  \\ \hline
    $$x\brak0$$ & $$x\brak{0}$$ & First term \\ \hline
    $$d$$ & $$d$$ & common difference \\ \hline
    $$x(n)$$ & $$[x\brak{0}+nd]u\brak n$$ & General term of the series  \\ \hline
  \end{tabular}

    \caption{Parameter Table1}
    \label{tab:10.9.5.1.1}
\end{table}
For an $AP$,
\begin{align}
    x\brak{n}&=[x\brak{0}+nd]u\brak{n}   \\
    \implies x\brak{m+n}+x\brak{m-n}&=[x\brak{0}+\brak{m+n}d]+[x\brak{0}+\brak{m-n}d] \\
    &=2[x\brak{0}+md]   \\
    \therefore x\brak{m+n}+x\brak{m-n}&=2x\brak{m}
\end{align}
\begin{table}[ht]
    \centering
    \def\arraystretch{1.5}
    \begin{tabular}{|p{4.5cm}|p{4.5cm}|}
    \hline
      $$x\brak{0}$$ & $$3$$  \\ \hline
      $$d$$ & $$2$$  \\ \hline
      $$m$$ & $$6$$  \\ \hline
      $$n$$ & $$2$$  \\ \hline
      $$x\brak{m+n}$$ & $$19$$  \\ \hline
      $$x\brak{m-n}$$ & $$11$$  \\ \hline
      $$x\brak{m}$$ & $$15$$  \\ \hline
    \end{tabular}

    \caption{Verified Values}
    \label{tab:10.9.5.1.2}
\end{table}
\end{document}



\item The sum of the first three terms of a G.P is $39/10$ and their product is $1$. Find the common ratio and the terms.\\
\solution
\let\negmedspace\undefined
\let\negthickspace\undefined
\documentclass[journal,12pt,twocolumn]{IEEEtran}
\usepackage{cite}
\usepackage{amsmath,amssymb,amsfonts,amsthm}
\usepackage{algorithmic}
\usepackage{graphicx}
\usepackage{textcomp}
\usepackage{xcolor}
\usepackage{txfonts}
\usepackage{listings}
\usepackage{enumitem}
\usepackage{mathtools}
\usepackage{gensymb}
\usepackage{comment}
\usepackage[breaklinks=true]{hyperref}
\usepackage{tkz-euclide}
\usepackage{listings}
\usepackage{gvv}
\def\inputGnumericTable{}
\usepackage[latin1]{inputenc}
\usepackage{color}
\usepackage{array}
\usepackage{longtable}
\usepackage{calc}
\usepackage{multirow}
\usepackage{hhline}
\usepackage{ifthen}
\usepackage{lscape}

\newtheorem{theorem}{Theorem}[section]
\newtheorem{problem}{Problem}
\newtheorem{proposition}{Proposition}[section]
\newtheorem{lemma}{Lemma}[section]
\newtheorem{corollary}[theorem]{Corollary}
\newtheorem{example}{Example}[section]
\newtheorem{definition}[problem]{Definition}
\newcommand{\BEQA}{\begin{eqnarray}}
\newcommand{\EEQA}{\end{eqnarray}}
\newcommand{\define}{\stackrel{\triangle}{=}}
\theoremstyle{remark}
\newtheorem{rem}{Remark}
\begin{document}

\bibliographystyle{IEEEtran}
\vspace{3cm}

\title{NCERT Discrete - 11.9.3.12}
\author{EE23BTECH11058 - Sindam Ananya$^{*}$% <-this % stops a space
}
\maketitle
\newpage
\bigskip

\renewcommand{\thefigure}{\theenumi}
\renewcommand{\thetable}{\theenumi}

\vspace{3cm}
\textbf{Question : 11.9.3.12} 
The sum of the first three terms of a G.P is $39/10$ and their product is $1$. Find the common ratio and the terms.\\
\solution
\begin{table}[h!]
    \centering
    \begin{tabular}{|c|c|c|}
        \hline
        \textbf{Parameter} & \textbf{Value} & \textbf{Description} \\
        \hline
        $x(0)$ & & First term \\
        \hline
        $r$ & & Common ratio \\
        \hline
        $x(0)^3r^3$ & 1 & Product of terms \\
        \hline
        $x(0)$ + $x(0)r$ + $x(0)r^2$ & $\frac{39}{10}$ & Sum of terms \\
        \hline
    \end{tabular}

    \caption{Input Parameters}
    \label{tab:11.9.3.12table1}
\end{table}
\begin{equation}
y(n) = x(0)\brak{\frac{r^{n+1}-1}{r-1}}u(n)
\label{eq:11.9.3.12eq1}
\end{equation}
From \tabref{tab:11.9.3.12table1} and \eqref{eq:11.9.3.12eq1} :
\begin{align}
y(2) &= x(0)\brak{\frac{r^3-1}{r-1}}\\
\frac{39}{10} &= x(0)\brak{r^2+r+1}\\
\implies \frac{39r}{10} &= r^2+r+1 \quad \brak{\because x(0)r = 1}\\
\implies (2r-5)(5r-2) &=0\\
\implies r &= \frac{2}{5} \quad or \quad \frac{5}{2}
\end{align}
\begin{enumerate}
      \item If $r = \frac{2}{5}$, then terms are $\frac{5}{2}$, $1$, $\frac{2}{5}$.
      \item If $r = \frac{5}{2}$, then terms are $\frac{2}{5}$, $1$, $\frac{5}{2}$.
\end{enumerate}
\begin{figure}[h!]
    \centering
    \includegraphics[width=\columnwidth]{figs/graph1.png}
    \caption{stem plots of GP if $r=\frac{2}{5}$}
    \label{fig:11.9.3.12_1}
\end{figure}
\begin{figure}[h!]
    \centering
    \includegraphics[width=\columnwidth]{figs/graph2.png}
    \caption{stem plots of GP if $r=\frac{5}{2}$}
    \label{fig:11.9.3.12_2}
\end{figure}
\end{document}




\item The ratio of the A.M and G.M of two positive numbers $a$ and $b$ is $m:n$. Show that $a:b = \brak{ m + \sqrt{m^2 - n^2}} : \brak{ m - \sqrt{m^2 - n^2}}$.\\
\solution
\input{ncert-maths/11/9/5/19/file2.tex}

\item The sum of three numbers in an arithmetic progression (AP) is $24$ and the product of those three numbers is $440$, find the values of the three numbers.\\
\solution
\iffalse
\let\negmedspace\undefined
\let\negthickspace\undefined
\documentclass[journal,12pt,twocolumn]{IEEEtran}
\usepackage{cite}
\usepackage{amsmath,amssymb,amsfonts,amsthm}
\usepackage{algorithmic}
\usepackage{graphicx}
\usepackage{textcomp}
\usepackage{xcolor}
\usepackage{txfonts}
\usepackage{listings}
\usepackage{enumitem}
\usepackage{mathtools}
\usepackage{gensymb}
\usepackage{comment}
\usepackage[breaklinks=true]{hyperref}
\usepackage{tkz-euclide} 
\usepackage{listings}
\usepackage{gvv}

\def\inputGnumericTable{}                                
\usepackage[latin1]{inputenc}                 
\usepackage{color}                            
\usepackage{array}                            
\usepackage{longtable}                        
\usepackage{calc}                            
\usepackage{multirow}                      
\usepackage{hhline}                           
\usepackage{ifthen}                          
\usepackage{lscape}
\usepackage{amsmath}
\newtheorem{theorem}{Theorem}[section]
\newtheorem{problem}{Problem}
\newtheorem{proposition}{Proposition}[section]
\newtheorem{lemma}{Lemma}[section]
\newtheorem{corollary}[theorem]{Corollary}
\newtheorem{example}{Example}[section]
\newtheorem{definition}[problem]{Definition}
\newcommand{\BEQA}{\begin{eqnarray}}
\newcommand{\EEQA}{\end{eqnarray}}
\newcommand{\define}{\stackrel{\triangle}{=}}
\theoremstyle{remark}
\newtheorem{rem}{Remark}


\begin{document}
%

\bibliographystyle{IEEEtran}


\vspace{3cm}

\title{
%	\logo{
Discrete 11.9.2 

\large{EE:1205 Signals and System}

Indian Institute of Technology, Hyderabad
%	}
}
\author{Prashant Maurya

EE23BTECH11218
\maketitle

\newpage

%\tableofcontents

\bigskip

\renewcommand{\thefigure}{\arabic{figure}}
\renewcommand{\thetable}{\arabic{table}}
\flushleft{\textbf{Question-2 :} Find the sum of all natural numbers lying between 100 and 1000, which are
multiples of 5.}\\
\bigskip
\textbf{Solution:}
\fi
\begin{table}[!h]
	\centering
	\begin{tabular}{|c|c|c|}
    \hline
     Parameter & Description & Value \\
    \hline
     $x(0)$ & First Term & 105\\
     \hline
     $d$ & Common Difference & 5\\
    \hline
    $n$ & Total terms & 179 \\ 
    \hline
    $x(178)$ & Last Term & 995\\
    \hline
    $m$ & No of poles & 3\\
    \hline
\end{tabular}

	\vspace{6 pt}
	\caption{Given Parameters}
\end{table}
\begin{align}
	x\brak{n}= & \brak{105+5n}\brak{u(n)}
\end{align}
On taking Z transform
\begin{align}
X\brak{z}= &\frac {x\brak{0}} {\brak{1-z^{-1}}} + \frac {dz^{-1}} {\brak{1-z^{-1}}^2} \\
= &\dfrac{105}{1-z^{-1}} + \frac{5z^{-1}}{\brak{1-z^{-1}}^{2}}\\
\implies X\brak{z}=& \dfrac{105-100z^{-1}}{{\brak{1-z^{-1}}}^2} \quad |z|>1\\
y\brak{n}=& x\brak{n}* u\brak{n}\\
\implies Y\brak{z}=& X\brak{z}U\brak{z}\\
=& \dfrac{105-100z^{-1}}{{(1-z^{-1})}^2}\frac1 {\brak{1-z^{-1}}} \\
=& \dfrac{105-100z^{-1}}{\brak{1-z^{-1}}^{3}} \quad |z|>1
\end{align}
Using contour integration to find the inverse Z-transform:\\
\begin{align}
    \implies y\brak{178}=&\dfrac{1}{2\pi j}\oint_{C}Y\brak{z} \;z^{177} \;dz\\
    =&\dfrac{1}{2\pi j}\oint_{C}\dfrac{\brak{105-100z^{-1}}{z^{177}}}{\brak{{1-z^{-1}}}^{3}} \;dz 
\end{align}
We can observe that there is only a 3 times repeated pole at $z=1$,
\begin{align}
    \implies R&=\dfrac{1}{\brak {m-1}!}\lim\limits_{z\to a}\dfrac{d^{m-1}}{dz^{m-1}}\brak {{\brak{z-a}}^{m}f\brak z}  
\end{align}
\begin{align}
    &=\dfrac{1}{\brak {2}!}\lim\limits_{z\to 1}\dfrac{d^{2}}{dz^{2}}\brak {{\brak{z-1}}^{3}\dfrac{\brak{105-100z^{-1}}z^{180}}{{\brak{z-1}}^3}}\\
    &=\dfrac{1}{2}{\lim\limits_{z\to 1}\dfrac{d^2}{dz^2}\brak{105z^{180}-100z^{179}}}\\
     &=98450
\end{align}
\begin{align}
    \therefore y\brak{178}=98450
\end{align}
\begin{figure}[ht]
    \centering
    \includegraphics[width=\columnwidth]{ncert-maths/11/9/2/2/figures/fig1.png}
    \caption{Plot of $x(n)$ $vs$ $n$}
\end{figure}
%\end{document}

\pagebreak

\item The sum of some terms of G.P. is $315$ whose first term and the common ratio are $5$ and $2$ , respectively. Find the last term and the number of terms.\\
\solution
\let\negmedspace\undefined
\let\negthickspace\undefined
\documentclass[journal,12pt,twocolumn]{IEEEtran}
\usepackage{cite}
\usepackage{amsmath,amssymb,amsfonts,amsthm}
\usepackage{algorithmic}
\usepackage{graphicx}
\usepackage{textcomp}
\usepackage{xcolor}
\usepackage{txfonts}
\usepackage{listings}
\usepackage{enumitem}
\usepackage{mathtools}
\usepackage{gensymb}
\usepackage{comment}
\usepackage[breaklinks=true]{hyperref}
\usepackage{tkz-euclide}
\usepackage{listings}
\usepackage{gvv}
\def\inputGnumericTable{}
\usepackage[latin1]{inputenc}
\usepackage{color}
\usepackage{array}
\usepackage{longtable}
\usepackage{calc}
\usepackage{multirow}
\usepackage{hhline}
\usepackage{ifthen}
\usepackage{lscape}

\newtheorem{theorem}{Theorem}[section]
\newtheorem{problem}{Problem}
\newtheorem{proposition}{Proposition}[section]
\newtheorem{lemma}{Lemma}[section]
\newtheorem{corollary}[theorem]{Corollary}
\newtheorem{example}{Example}[section]
\newtheorem{definition}[problem]{Definition}
\newcommand{\BEQA}{\begin{eqnarray}}
\newcommand{\EEQA}{\end{eqnarray}}
\newcommand{\define}{\stackrel{\triangle}{=}}
\theoremstyle{remark}
\newtheorem{rem}{Remark}
\begin{document}

\bibliographystyle{IEEEtran}
\vspace{3cm}

\title{NCERT Discrete - 10.5.3.20}
\author{EE23BTECH1205 - Avani Chouhan$^{*}$% <-this % stops a space
}
\maketitle
\newpage
\bigskip

\renewcommand{\thefigure}{\theenumi}
\renewcommand{\thetable}{\theenumi}

\vspace{3cm}
\textbf{Question : 10.5.3.20} 
The sum of some terms of G.P. is 315 whose first term and the common ratio are $5$ and $2$ , respectively. Find the last term and the number of terms.\\
\solution

\begin{table}
  \centering
  
  \begin{tabular}{|c|c|c|c|}
\hline
\textbf{Parameter}&\textbf{Description} &\textbf{subquestion}& \textbf{Value}\\
\hline
     \multirow{4}{*}{$\Delta \theta$} & \multirow{4}{*}{$\theta_1 - \theta_2$} &\brak{a}& 6.4$\pi$ \, radians \\
     \cline{3-4}
     & & \brak{b}& 0.8$\pi$ \, radians \\
     \cline{3-4}
     & &\brak{c}& $\pi$ \, radians \\
     \cline{3-4}
     & & \brak{d} & $\dfrac{3\pi}{2\vphantom{\brak{0.1}}}$ \, radians \\
     \hline
\end{tabular}

  \caption{Input Parameters}
  \label{tab:10.5.3.20table1}
\end{table}
\begin{align}
x(n) = x(0)r^{n}u(n)
\label{eq:10.5.3.20eq}
\end{align}
From \eqref{eq:gpz}
\begin{align}
X(z) =\frac{5}{1-2z^{-1}} \quad \abs{z} > \abs{2}
\end{align}
By contour integration:
\begin{align}
y(n) &= x(0)\brak{\frac{r^{n+1}-1}{r-1}}u(n)\\
315 &= 5\brak{2^{n+1}- 1}  \\
\implies n &= 5
\end{align}
The number of terms is \(n + 1 = 6\)\\
From \eqref{eq:10.5.3.20eq}:
\begin{align}
x(5) &= 5\brak{2^{5}}\\
 &= 160 
\end{align}

\begin{figure}
    \centering
    \includegraphics[width=\columnwidth]{figs/plot1.png}
    \caption{Stem plot of x(n)}
    \label{fig:10.5.3.20fig1}
\end{figure}
\begin{figure}
    \centering
    \includegraphics[width=\columnwidth]{figs/plot2.png}
    \caption{Stem plot of y(n)}
    \label{fig:10.5.3.20fig2}
\end{figure}
\end{document}


\item  Find the sum of n terms of the A.P. whose kth term is \(5k + 1\).\\
\solution

\item How many 3 digit numbers are divisible by 7? \\
\solution

\item A person writes a letter to four of his friends. He asks each one of them to copy the letter and mail to four different persons with instruction that they move the chain similarly. Assuming that the chain is not broken and that it costs 50 paise to mail one letter. Find the amount spent on the postage when 8th set of letter is mailed.\\
\solution 
\pagebreak

\item If $a$, $b$, $c$ are in A.P.; $b$, $c$, $d$ are in G.P and $\frac{1}{c}$, $\frac{1}{d}$, $\frac{1}{e}$ are in A.P. prove that $a$, $c$, $e$ are in G.P.\\
\solution
\pagebreak
\item Find the 31st term of an AP whose $11$th term is 38 and the $16$th term is 73.\\ \hfill{NCERT 10.5.2.7}
\\
\solution
\pagebreak
\item If $a\left(\frac{1}{b} + \frac{1}{c}\right)$, $b\left(\frac{1}{c} + \frac{1}{a}\right)$, $c\left(\frac{1}{a} + \frac{1}{b}\right)$ are in arithmetic progression (AP), prove that $a$, $b$, $c$ are also in AP. \\
\solution
% \iffalse
\let\negmedspace\undefined
\let\negthickspace\undefined
\documentclass[journal,12pt,twocolumn]{IEEEtran}
\usepackage{cite}
\usepackage{amsmath,amssymb,amsfonts,amsthm}
\usepackage{algorithmic}
\usepackage{graphicx}
\usepackage{textcomp}
\usepackage{xcolor}
\usepackage{txfonts}
\usepackage{listings}
\usepackage{enumitem}
\usepackage{mathtools}
\usepackage{gensymb}
\usepackage{comment}
\usepackage[breaklinks=true]{hyperref}
\usepackage{tkz-euclide} 
\usepackage{listings}
\usepackage{gvv}                                        
\def\inputGnumericTable{}                                 
\usepackage[latin1]{inputenc}                                
\usepackage{color}                                            
\usepackage{array}                                            
\usepackage{longtable}                                       
\usepackage{calc}                                             
\usepackage{multirow}                                         
\usepackage{hhline}                                           
\usepackage{ifthen}                                           
\usepackage{lscape}
\usepackage{caption}
\newtheorem{theorem}{Theorem}[section]
\newtheorem{problem}{Problem}
\newtheorem{proposition}{Proposition}[section]
\newtheorem{lemma}{Lemma}[section]
\newtheorem{corollary}[theorem]{Corollary}
\newtheorem{example}{Example}[section]
\newtheorem{definition}[problem]{Definition}
\newcommand{\BEQA}{\begin{eqnarray}}
\newcommand{\EEQA}{\end{eqnarray}}
\newcommand{\define}{\stackrel{\triangle}{=}}
\theoremstyle{remark}
\newtheorem{rem}{Remark}
\begin{document}
\parindent 0px
\bibliographystyle{IEEEtran}
\vspace{3cm}

\title{NCERT 11.9.3 1Q}
\author{EE23BTECH11013 - Avyaaz$^{*}$% <-this % stops a space
}
\maketitle
\newpage
\bigskip

\renewcommand{\thefigure}{\arabic{figure}}
\renewcommand{\thetable}{\arabic{table}}
\large\textbf{\textsl{Question:}}
Find the $20^{th}$ and $n^{th}$ terms of the G.P $\frac{5}{2}$, $\frac{5}{4}$, $\frac{5}{8}$,.....

\solution
 \begin{table}[htbp]
     \centering
     \setlength{\extrarowheight}{8pt}
    \begin{tabular}{|c|c|c|}
\hline 
   \textbf{Parameter}  &\textbf{Description} &\textbf{Value} \\
\hline
&&\\
$I_r$&Net Intensity of light at $\Delta x =\dfrac{\lambda}{3}$ &$\dfrac{K}{4}$ \\&&\\
\hline
\end{tabular}

     \caption{Parameters}
     \label{tab:table1.11.9.3.1}
 \end{table} 

% \begin{align}
%    x(n) = \dfrac{5}{2}\left(\dfrac{1}{2}\right)^n 
% \end{align}

% \begin{align}
% 	x \brak{n} & \system{Z} X \brak{z} \\
%    % x(n) &=\dfrac{5}{2}\left(\dfrac{1}{2}\right)^n u(n) \\
%     \therefore X(z) &= \sum_{n=-\infty}^{\infty}x(n)z^{-n}\label{eq:z-transform}  
% \end{align}
% Here, 
%          $    u(n) = \begin{cases}
%                 0 &\text{for } n < 0 \\
%                 1 & \text{for } n \geq 0
%             \end{cases}$       
 
%  \vspace{1cm}
From \tabref{tab:table1.11.9.3.1}:
\(Z\)-Transform of \(x(n)\):
\begin{align}
% \implies X(z) &= \sum_{n=-\infty}^{\infty}\left(\dfrac{5}{2}\left(\dfrac{1}{2}\right)^n u(n)\right) z^{-n} \\
 % \implies X(z) &= \dfrac{5}{2}\sum_{n=0}^{\infty}\left(\dfrac{z
 % ^{-1}}{2}\right)^n \\
\implies X(z) &=\dfrac{5}{2}\left(\dfrac{1}{1-\frac{z^{-1}}{2}}\right) ;\cbrak{z\in\mathbb{C} : |z|>\dfrac{1}{2}}
\end{align}

\begin{figure}[ht]
    \centering
    \includegraphics[width = \columnwidth]{figs/stem_plot.png}
    \caption{}
	\label{fig:graph1.11.9.3.1}
\end{figure} 

\bibliographystyle{IEEEtran}
\end{document}

\pagebreak
\item If \(\frac{a^n +b^n}{a^{n-1} + b^{n-1}}\) is A.M between a and b, then find value of n.\\
\solution
\pagebreak

\item Write the first five terms of the sequence whose $n^{th}$ \text{term is} : $x(n) = (-1)^{n-1}5^{n+1}$.\\
\solution
\pagebreak

\item Shamshad Li buys a scooter for RS 22000. He pays RS 4000 cash and agrees to pay the balance in annual instalment of RS 1000 plus 10\% interest on the unpaid amount. How much will the scooter cost him?
\solution
\pagebreak

\end{enumerate}
