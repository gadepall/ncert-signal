\begin{enumerate}[label=\thesection.\arabic*,ref=\thesection.\theenumi]
\item An LC circuit contains a $50 \mu H$ inductor and a $50 \mu F$ capacitor with an initial charge of $10 mC$. The resistance of the circuit is negligible. Let the instant the circuit is closed by $t = 0$.

\textbf{a)} What is the total energy stored initially? Is it conserved during LC oscillations?

\textbf{b)} What is the natural frequency of the circuit?

\textbf{c)} At what time is the energy stored \textbf{(i)} completely electrical (i.e., stored in the capacitor)? \textbf{(ii)} completely magnetic (i.e., stored in the inductor)?

\textbf{d)} At what times is the total energy shared equally between the inductor and the capacitor?

\textbf{e)} If a resistor is inserted in the circuit, how much energy is eventually dissipated as heat? \\
\hfill(NCERT-Physics 12.7 12Q)\\
\solution 
\pagebreak 

\item Obtain the resonant frequency and Q-factor of a series LCR circuit with $L = 3.0\, H$, $C = 27\, \mu F$, and $R = 7.4\, \Omega$. It is desired to improve the sharpness of the resonance of the circuit by reducing its `full width at half maximum' by a factor of 2. Suggest a suitable way.\\
\solution
\input{ncert-physics/12/7/21/Phy_12_7_21.tex}

\pagebreak
\item A circuit containing a $80 mH$ inductor and a $60 \mu F$ capacitor in series is connected to a $230 V$, $50 Hz$ supply. A resistance of $15 \Omega $ is connected in series. Obtain the average power transferred to each element of the circuit, and the total power absorbed.\\
\solution
\pagebreak

\item A series LCR circuit with 
$L$ = $0.12 H$
$C$ = $480 nF$
$R$ = $23 \Omega$
is connected to a $230 V$ variable frequency supply.\\
(a) What is the source frequency for which current amplitude is maximum? Obtain this maximum value.\\
(b) What is the source frequency for which the average power absorbed by the circuit is maximum? Obtain the value of this maximum power.\\
(c) For which frequencies of the source is the power transferred to the circuit half the power at resonant frequency? What is the current amplitude at these frequencies?\\
(d) What is the Q-factor of the given circuit?\\
\solution
\pagebreak
\item A radio can tune over the frequency range of a portion of the MW broadcast band: (800 kHz to 1200 kHz). If its LC circuit has an effective inductance (\(L\)) and a variable capacitor with capacitance (\(C\)), what must be the range of \(C\)?\\
\solution
\pagebreak

\item A 100$\mu$F capacitor in series with a 40$\Omega$ resistance is connected to a $110 V$, $12 kHz$ supply.
\begin{enumerate}[label=(\alph*)]
\item What is the maximum current in the circuit?
\item What is the time lag between the current maximum and the voltage maximum?
\end{enumerate}
Hence, explain the statement that a capacitor is a conductor at very high frequencies. Compare this behaviour with that of a capacitor in a dc circuit after the steady state.
\end{enumerate}
