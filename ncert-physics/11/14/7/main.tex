\iffalse
\let\negmedspace\undefined
\let\negthickspace\undefined
\documentclass[journal,12pt,twocolumn]{IEEEtran}
\usepackage{cite}
\usepackage{amsmath,amssymb,amsfonts,amsthm}
\usepackage{algorithmic}
\usepackage{graphicx}
\usepackage{textcomp}
\usepackage{caption}
\usepackage{xcolor}
\usepackage{txfonts}
\usepackage{listings}
\usepackage{enumitem}
\usepackage{mathtools}
\usepackage{gensymb}
\usepackage[breaklinks=true]{hyperref}
\usepackage{tkz-euclide} % loads  TikZ and tkz-base
\usepackage{listings}
\usepackage{gvv}
\newtheorem{theorem}{Theorem}[section]
\newtheorem{problem}{Problem}
\newtheorem{proposition}{Proposition}[section]
\newtheorem{lemma}{Lemma}[section]
\newtheorem{corollary}[theorem]{Corollary}
\newtheorem{example}{Example}[section]
\newtheorem{definition}[problem]{Definition}
%\newtheorem{thm}{Theorem}[section] 
%\newtheorem{defn}[thm]{Definition}
%\newtheorem{algorithm}{Algorithm}[section]
%\newtheorem{cor}{Corollary}
\newcommand{\BEQA}{\begin{eqnarray}}
\newcommand{\EEQA}{\end{eqnarray}}
\newcommand{\define}{\stackrel{\triangle}{=}}
\theoremstyle{remark}
\newtheorem{rem}{Remark}

%\bibliographystyle{ieeetr}
\begin{document}
%

\bibliographystyle{IEEEtran}


\vspace{3cm}

\title{
%	\logo{
Assignment-11.14.7 

\large{EE:1205-Signals and Systems}

Indian Institute of Technology, Hyderabad
%	}
}
\author{Md Ayaan Ashraf

EE23BTECH11041
}	

\maketitle

\newpage

%\tableofcontents

\bigskip
\renewcommand{\thefigure}{\arabic{figure}}
\renewcommand{\thetable}{\arabic{table}}
%\renewcommand{\theequation}{\theenumi}
\section*{\textbf{\textit{Question}}}
The motion of a particle executing simple harmonic motion is described by the
displacement function, $x(t)$ = $A$ $cos$ ($\omega$$t$ +$\phi$).
If the initial $(t = 0)$ position of the particle is $1 cm$ and its initial velocity is $\omega\quad cm/s$, what are its amplitude and initial phase angle ? The angular frequency of the particle is $\pi\quad s^{-1}$. If instead of the cosine function, we choose the sine function to describe the SHM : $x$ = $B$ $sin$ ($\omega$$t$ +$\alpha$), what are the amplitude and initial phase of the
particle with the above initial conditions.
\section*{\textit{\textbf{Solution}}}
\fi

\begin{table}[h]
  \centering
  \begin{tabular}{|c|c|c|}
    \hline
Parameter & Description & Value \\ \hline
$x_1(0)$ & Initial position of particle in 1)& 1cm\\ \hline
$x_2(0)$ & Initial position of particle in 2)  & 1cm\\ \hline
$f$ & Frequency of particle & $\frac{\omega}{2\pi}$ \\ \hline
$x_1'(0)$ & Initial velocity of particle in 1) & $\omega$ \\ \hline
$x_2'(0)$ & Initial velocity of particle in 2)& $\omega$ \\ \hline
$\phi$ & Initial Phase Angle & ?\\
\hline
$\alpha$ & New Phase Angle & ?\\
\hline
$A$ & Initial Amplitude & ?\\
\hline
$B$ & New Amplitude & ?\\
\hline
  \end{tabular}
  \vspace{2mm}
  \caption{Parameter Table 11.14.7}
\end{table}



\begin{enumerate}
\item In \figref{Fig1_11.14.7}, the equation is
\begin{align}
   x_1(t) = A \cos(2\pi f t + \phi)  
\end{align}
Given:
\begin{align}
     x_1(0)&= A \cos(\phi) = 1 \text{ cm} \\
 x_1'(0)&= -A 2\pi f \sin(\phi) = 2\pi f \text{ cm/s}
 \end{align}
Solving for $\phi$ and $A$:
\begin{align}
    \tan(\phi) = -1\\
\implies
\phi = -\frac{\pi}{4} \\
\implies 
A= \sqrt{2}cm
\end{align}
\begin{figure}[h]
\renewcommand\thefigure{1}
    \centering
    \includegraphics[width=0.8\columnwidth]{ncert-physics/11/14/7/figs/fig1.png}
    \caption{$x_1(t) = \sqrt{2}\cos(\pi t - \frac{\pi}{4}$)}
    \label{Fig1_11.14.7}
\end{figure}
    \item In \figref{Fig2_11.14.7}, the equation is
\begin{align}
    x_2(t) = B \sin(2\pi f t + \alpha) 
    \end{align}
    Given:\\
    \begin{align}
     x_2(0)=& B \sin(\alpha) = 1 \text{ cm} \\
    x_2'(0)=& B 2\pi f \cos(\alpha) = 2\pi f \text{ cm/s}
\end{align}
Solving for $\alpha$ and $B$:
\begin{align}
    \tan(\alpha)& = 1\\
\implies
\alpha &= \frac{\pi}{4} \\
\implies 
B &=\sqrt{2}cm
\end{align}
\begin{figure}[h]
\renewcommand\thefigure{2}
    \centering
    \includegraphics[width=0.8\columnwidth]{ncert-physics/11/14/7/figs/fig2.png}
    \caption{$x_2(t) = \sqrt{2}\sin(\pi t + \frac{\pi}{4})$}
    \label{Fig2_11.14.7}
\end{figure}
\end{enumerate}
