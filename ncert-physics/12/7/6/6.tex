\let\negmedspace\undefined
\let\negthickspace\undefined
\documentclass[journal,12pt,twocolumn]{IEEEtran}

\usepackage{cite}
\usepackage{amsmath,amssymb,amsfonts,amsthm}
\usepackage{algorithmic}
\usepackage{graphicx}
\usepackage{textcomp}
\usepackage{xcolor}
\usepackage{txfonts}
\usepackage{listings}
\usepackage{enumitem}
\usepackage{mathtools}
\usepackage{gensymb}
\usepackage[breaklinks=true]{hyperref}
\usepackage{tkz-euclide} % loads  TikZ and tkz-base
\usepackage{listings}
\usepackage{circuitikz}
\usepackage{graphicx}

%\newcounter{MYtempeqncnt}
\DeclareMathOperator*{\Res}{Res}
%\renewcommand{\baselinestretch}{2}
\renewcommand\thesection{\arabic{section}}
\renewcommand\thesubsection{\thesection.\arabic{subsection}}
\renewcommand\thesubsubsection{\thesubsection.\arabic{subsubsection}}

\renewcommand\thesectiondis{\arabic{section}}
\renewcommand\thesubsectiondis{\thesectiondis.\arabic{subsection}}
\renewcommand\thesubsubsectiondis{\thesubsectiondis.\arabic{subsubsection}}

% correct bad hyphenation here
\hyphenation{op-tical net-works semi-conduc-tor}
\def\inputGnumericTable{}                                 %%

\lstset{
	frame=single,
	breaklines=true,
	columns=fullflexible
}



\newtheorem{theorem}{Theorem}[section]
\newtheorem{problem}{Problem}
\newtheorem{proposition}{Proposition}[section]
\newtheorem{lemma}{Lemma}[section]
\newtheorem{corollary}[theorem]{Corollary}
\newtheorem{example}{Example}[section]
\newtheorem{definition}[problem]{Definition}
\newcommand{\BEQA}{\begin{eqnarray}}
	\newcommand{\EEQA}{\end{eqnarray}}
\newcommand{\define}{\stackrel{\triangle}{=}}
\newcommand\figref{Fig.~\ref}
\newcommand\tabref{Table~\ref}
\bibliographystyle{IEEEtran}
%\bibliographystyle{ieeetr}


\providecommand{\mbf}{\mathbf}
\providecommand{\pr}[1]{\ensuremath{\Pr\left(#1\right)}}
\providecommand{\qfunc}[1]{\ensuremath{Q\left(#1\right)}}
\providecommand{\sbrak}[1]{\ensuremath{{}\left[#1\right]}}
\providecommand{\lsbrak}[1]{\ensuremath{{}\left[#1\right.}}
\providecommand{\rsbrak}[1]{\ensuremath{{}\left.#1\right]}}
\providecommand{\brak}[1]{\ensuremath{\left(#1\right)}}
\providecommand{\lbrak}[1]{\ensuremath{\left(#1\right.}}
\providecommand{\rbrak}[1]{\ensuremath{\left.#1\right)}}
\providecommand{\cbrak}[1]{\ensuremath{\left\{#1\right\}}}
\providecommand{\lcbrak}[1]{\ensuremath{\left\{#1\right.}}
\providecommand{\rcbrak}[1]{\ensuremath{\left.#1\right\}}}
\theoremstyle{remark}
\newtheorem{rem}{Remark}
\newcommand{\sgn}{\mathop{\mathrm{sgn}}}
\providecommand{\abs}[1]{\left\vert#1\right\vert}
\providecommand{\res}[1]{\Res\displaylimits_{#1}}
\providecommand{\norm}[1]{\left\lVert#1\right\rVert}
%\providecommand{\norm}[1]{\lVert#1\rVert}
\providecommand{\mtx}[1]{\mathbf{#1}}
\providecommand{\mean}[1]{E\left[ #1 \right]}
\providecommand{\fourier}{\overset{\mathcal{F}}{ \rightleftharpoons}}
%\providecommand{\hilbert}{\overset{\mathcal{H}}{ \rightleftharpoons}}
\providecommand{\system}{\overset{\mathcal{H}}{ \longleftrightarrow}}
%\newcommand{\solution}[2]{\textbf{Solution:}{#1}}
\newcommand{\solution}{\noindent \textbf{Solution: }}
\newcommand{\cosec}{\,\text{cosec}\,}
\providecommand{\dec}[2]{\ensuremath{\overset{#1}{\underset{#2}{\gtrless}}}}
\newcommand{\myvec}[1]{\ensuremath{\begin{pmatrix}#1\end{pmatrix}}}
\newcommand{\mydet}[1]{\ensuremath{\begin{vmatrix}#1\end{vmatrix}}}
\renewcommand{\abstractname}{Question}

\let\vec\mathbf

	
	\vspace{3cm}
	
	


\newcommand{\permcomb}[4][0mu]{{{}^{#3}\mkern#1#2_{#4}}}
\newcommand{\comb}[1][-1mu]{\permcomb[#1]{C}}

%\IEEEpeerreviewmaketitle

\newcommand \tab [1][1cm]{\hspace*{#1}}
%\newcommand{\Var}{$\sigma ^2$}
\usepackage{amssymb}
\usepackage{amsmath}
\title{
	
\title{NCERT Physics 12.7 Q6}
\author{EE23BTECH11061 - SWATHI DEEPIKA$^{*}$% <-this % stops a space
}


}
\begin{document}

\maketitle

\textbf{Question:} 
Obtain the resonance frequency of a series LCR circuit with $L = 2.0\, H$, $C = 32\, \mu F$, and $R = 10\, \Omega$. What is the Q-value of the circuit.\\

\begin{figure}[!h]
	\centering
	    %\begin{circuitikz}
		% Draw the components
	%	\draw (0,0) to[V, v=$230\,V$, f=$50\,Hz$] (0,3)
	%	to[R, l=$15\,\Omega$] (3,3)
	%	to[L, l=$80\,mH$] (6,3)
	%	to[C, l=$60\,\mu F$] (6,0)
	%	-- (0,0);
    % \end{circuitikz}
\begin{circuitikz}
	\draw(0, 0) -- (1, 0);
	\draw(1, 0) to [L, l = $80\,mH$](2, 0);
	\draw(2, 0) -- (3, 0);
	\draw(3, 0) to [C, l = $60\,\mu F$](4, 0);
	\draw(4, 0) -- (5, 0);
	\draw(5, 0) to [R, l = $15\,\Omega$](6, 0);
	\draw(0, 0) -- (0, -2);
	\draw[->] (0, -1) node[left] {$I(s)$} -- (0, -1);
	\draw(6, 0) -- (7, 0);
	\draw(7, 0) -- (7, -2);
	\draw(0, -2) -- (3, -2);
	\draw(7, -2) -- (7, -2);
	\draw(3, -2) to [sV, l = $230\,V$](4, -2);
	\draw(4, -2) -- (7, -2);
\end{circuitikz}


	\caption{LCR Circuit}
	\label{fig: cirk1}
\end{figure}
     
\textbf{Solution: }
 \begin{table}[h]
 	\centering
 	\resizebox{6 cm}{!}{
 		\begin{tabular}{|c|c|c|}
\hline 
   \textbf{Parameter}  &\textbf{Description} &\textbf{Value} \\
\hline
&&\\
$I_r$&Net Intensity of light at $\Delta x =\dfrac{\lambda}{3}$ &$\dfrac{K}{4}$ \\&&\\
\hline
\end{tabular}

 	}
 	\vspace{6 pt}
 	\caption{Parameters}
 	\label{tab: cirktabel} 
 \end{table}
 
\begin{figure}[!h]
 \centering
    
\begin{circuitikz}
\tikzstyle{every node}=[font=\Large]
\draw (6.75,13.5) to[L,l={ \LARGE $sL$} ] (6.75,10.75);
\draw [](6.75,13.5) to[short] (10.75,13.5);
\draw [](6.75,10.75) to[short] (10.75,10.75);
\draw (10.75,13.5) to[C,l={ \LARGE $1/sC$}] (10.75,12);
\draw (10.75,12) to[american voltage source,l={ \LARGE $V_0/s$}] (10.75,10.75);
\draw [->, >=Stealth] (8.5,13.5) -- (8.25,13.5);
\draw (8.5,13.5) -- (8.25,13.5) node[midway, above] {$I(s)$};
\end{circuitikz}


    \caption{LCR Circuit}
    \label{fig: cirk2}
\end{figure}

\begin{enumerate}
\item {Frequency Response of the Circuit}


From \figref{fig: cirk2},
\begin{align}
   V(j\omega) &= I(j\omega)\left(R + Lj\omega + \dfrac{1}{j\omega C}\right)\\
    \implies I(j\omega) &= \dfrac{V(s)}{\left(R + Lj\omega + \dfrac{1}{j\omega C}\right)}\label{eq: 4}
\end{align}
At resonance,
\begin{align}
    Lj\omega + \dfrac{1}{j\omega C} &= 0
\end{align}
\begin{equation}
    \omega = \dfrac{1}{\sqrt{LC}}
\end{equation}

At resonance, Resonant frequency($\omega_0$) = $\dfrac{1}{\sqrt{LC}}$
\item{Quality Factor}

\begin{enumerate}
\item voltage across inductor,
\begin{align}
    Q &= \left(\dfrac{V_L}{V_R}\right)_{\omega_0} = \dfrac{\lvert{j\omega_0 LI(j\omega)}\rvert}{\lvert RI(j\omega) \rvert}\\
    &= \dfrac{1}{\sqrt{LC}}\dfrac{L}{R}\\
    &= \dfrac{1}{R}\sqrt{\dfrac{L}{C}}
\end{align}
\item Using voltage across capacitor,
\begin{align}
	Q &= \left(\dfrac{V_C}{V_R}\right)_{\omega_0} = \dfrac{\abs{\frac{I(j\omega)}{j\omega_0 C}}}{\lvert RI(j\omega) \rvert}\\
    &= \dfrac{\sqrt{LC}}{RC}\\
    &= \dfrac{1}{R}\sqrt{\dfrac{L}{C}}
\end{align}
\end{enumerate}
\item{Plot of Impedance vs Angular Frequency}
\begin{equation}
    H(j\omega) = \dfrac{V(j\omega)}{I(j\omega)}
\end{equation}
Using \eqref{eq: 4},
\begin{align}
      H(j\omega) &= R + j\omega L + \dfrac{1}{j\omega C}\\
     \implies \lvert H(j\omega) \rvert &= \sqrt{R^2 + \left(\omega L - \dfrac{1}{\omega C}\right)^2}
\end{align}
\begin{figure}[!h]
    \centering
    \includegraphics[width = \columnwidth]{figs/q_plot.png}
    \caption{Impedance vs $\omega$ (using values in \tabref{tab: cirktabel})}
    \label{fig:h_plot}
\end{figure}
\end{enumerate}

Substituting values,
\begin{align}
\omega_0 &= \dfrac{1}{\sqrt{(2.0)(32 \times 10^{-6})}}
\end{align}
\begin{align}
\omega_0 &= 125 \text{ Hz}
\end{align}


\begin{align}
Q &= \frac{1}{10}\sqrt{\frac{2}{32 \times 10^{-6}}}
\end{align}
\begin{align}
Q &= 25
\end{align}


\end{document}



