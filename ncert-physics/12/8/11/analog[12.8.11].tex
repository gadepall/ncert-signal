 \iffalse
\let\negmedspace\undefined
\let\negthickspace\undefined
\documentclass[journal,12pt,twocolumn]{IEEEtran}
\usepackage{xparse}
\usepackage{cite}
\usepackage{amsmath,amssymb,amsfonts,amsthm}
\usepackage{algorithmic}
\usepackage{graphicx}
\usepackage{textcomp}
\usepackage{xcolor}
\usepackage{txfonts}
\usepackage{listings}
\usepackage{enumitem}
\usepackage{mathtools}
\usepackage{gensymb}
\usepackage{comment}
\usepackage[breaklinks=true]{hyperref}
\usepackage{tkz-euclide} 
\usepackage{listings}
\usepackage{gvv}                                        
\def\inputGnumericTable{}                                 
\usepackage[latin1]{inputenc}                                
\usepackage{color}                                            
\usepackage{array}                                            
\usepackage{longtable}                                       
\usepackage{calc}                                             
\usepackage{multirow}                                         
\usepackage{hhline}                                           
\usepackage{ifthen}                                           
\usepackage{lscape}

\newtheorem{theorem}{Theorem}[section]
\newtheorem{problem}{Problem}
\newtheorem{proposition}{Proposition}[section]
\newtheorem{lemma}{Lemma}[section]
\newtheorem{corollary}[theorem]{Corollary}
\newtheorem{example}{Example}[section]
\newtheorem{definition}[problem]{Definition}
\newcommand{\BEQA}{\begin{eqnarray}}
\newcommand{\EEQA}{\end{eqnarray}}
\newcommand{\define}{\stackrel{\triangle}{=}}
\theoremstyle{remark}
\newtheorem{rem}{Remark}
\begin{document}

\bibliographystyle{IEEEtran}
\vspace{3cm}

\title{ANALOG NCERT 12.8.11}
\author{EE23BTECH11046 - Poluri Hemanth$^{*}$}
\maketitle
\textbf{Question:}
Suppose that the electric field part of an electromagnetic wave
in vacuum given as\\ \textbf{E} =\{(3.1N/C)cos[(1.8 rad/m)y+(5.4$\times$10$^{6}$rad/s)t]\}$\vec{e_1}$ \\
(a) What is the direction of propagation ?\\
(b) What is the wavelength ? \\
(c) What is the frequency ?\\
(d) What is the amplitude of the magnetic field part of the wave?\\
(e) Write an expression for the magnetic field part of the wave.\\
\textbf{Solution:}
\fi
\begin{table}[h!]
    % Change address in github
    \begin{tabular}{|c|c|c|}
\hline 
   \textbf{Parameter}  &\textbf{Description} &\textbf{Value} \\
\hline
&&\\
$I_r$&Net Intensity of light at $\Delta x =\dfrac{\lambda}{3}$ &$\dfrac{K}{4}$ \\&&\\
\hline
\end{tabular}

    \caption{Input Parameters}
    \label{tab:12.8.11}
\end{table}\\
(a)\\
As the wave is in form of $cos(ky+wt)$
the wave is propagating along $-y$ axis, represented by $\vec{e_2}$\\
(b)
\begin{align}
	k&=\frac{2\pi}{\lambda} \\
	\Rightarrow\lambda&=\frac{2\pi}{1.8}\\
	&\approx3.5m
\end{align}
(c)
\begin{align} 
	\omega&=2\pi.f \\
	 5.4 x 10^{6} &=2.\pi.f \\
	\Rightarrow f &= 0.859 \times 10^{6} Hz    
\end{align}
(d)
\begin{align}
	B_o&=\frac{E_o}{c}
\end{align}
where $c$ is velocity of propagation of wave which is given by
\begin{align}
	c&=\frac{\omega}{k} \\
	&=\frac{5.4 \times 10^{6}}{1.8}\\
	&=3 \times 10^{6}m/s.\\
	B_o&= \frac{3.1}{3 \times 10^{6}}\\
	&= 1.03 \times 10^{-6}T\label{1[12.8.11]}
\end{align}
(e)
Direction of magnetic field is $\vec{e_3}$\\
where,
\begin{align}
	\vec{e_3}&=\vec{e_2} \times \vec{e_1}\\
	\textbf{B}&=B_ocos(ky+wt)\vec{e_3}\label{b[12.8.11]}
\end{align}
 From \eqref{1[12.8.11]},\eqref{b[12.8.11]}
\begin{align}
	\textbf{B}=1.03\times10^{-6}T\{cos[(1.8rad/m)y+(5.4x10^{6}rad/s)t]\}\vec{e_3}
\end{align}
\begin{figure}[h]
\centering
\includegraphics[width=1\columnwidth]{ncert-physics/12/8/11/figures/ewave.png}
\caption{Electric field part}
\label{ewave[12.8.11]}
\end{figure}
\begin{figure}[h]
\centering
\includegraphics[width=1\columnwidth]{ncert-physics/12/8/11/figures/mwave.png}
\caption{Magnetic field part}
\label{mwave[12.8.11]}
\end{figure}



%\end{document}

