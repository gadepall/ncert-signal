\begin{enumerate}[label=\thesection.\arabic*,ref=\thesection.\theenumi]
\item Suppose that the electric field amplitude of an electromagnetic wave is $E_0$ = 120N/C and that its frequency is $f$ = 50.0 MHz.
\begin{enumerate} [label=(\alph*)]
    \item Determine, $B_0, \omega, k$ and $\lambda$
    \item Find expressions for \textbf{E} and \textbf{B}
\end{enumerate}
\solution
\let\negmedspace\undefined
\let\negthickspace\undefined
\documentclass[journal,12pt,twocolumn]{IEEEtran}
\usepackage{cite}
\usepackage{amsmath,amssymb,amsfonts,amsthm}
\usepackage{algorithmic}
\usepackage{graphicx}

\usepackage{textcomp}
\usepackage{xcolor}
\usepackage{txfonts}
\usepackage{listings}
\usepackage{enumitem}
\usepackage{mathtools}
\usepackage{gensymb}
\usepackage{comment}
\usepackage[breaklinks=true]{hyperref}
\usepackage{tkz-euclide} 
\usepackage{listings}
\usepackage{gvv}                                                                      
\usepackage[latin1]{inputenc}                                
\usepackage{color}                                            
\usepackage{array}                                            
\usepackage{longtable}                                       
\usepackage{calc}                                             
\usepackage{multirow}                                         
\usepackage{hhline}                                           
\usepackage{ifthen}                                           
\usepackage{lscape}
\setlength{\arrayrulewidth}{0.5mm}
\setlength{\tabcolsep}{18pt}
\renewcommand{\arraystretch}{1.5}
\newtheorem{theorem}{Theorem}[section]
\newtheorem{problem}{Problem}
\newtheorem{proposition}{Proposition}[section]
\newtheorem{lemma}{Lemma}[section]
\newtheorem{corollary}[theorem]{Corollary}
\newtheorem{example}{Example}[section]
\newtheorem{definition}[problem]{Definition}
\newcommand{\BEQA}{\begin{eqnarray}}
\newcommand{\EEQA}{\end{eqnarray}}
\newcommand{\define}{\stackrel{\triangle}{=}}
\theoremstyle{remark}
\newtheorem{rem}{Remark}

\begin{document}

\bibliographystyle{IEEEtran}
\vspace{3cm}

\title{NCERT 12.8 8}
\author{EE23BTECH11054 - Sai Krishna Shanigarapu% <-this % stops a space
}
\maketitle
\newpage
\bigskip

\begin{flushleft}
\textbf{Question 8}\\
Suppose that the electric field amplitude of an electromagnetic wave is $E_0$ = 120N/C and that its frequency is $f$ = 50.0 MHz.\\
(a) Determine, $B_0, \omega, k$ and $\lambda$\\
(b) Find expressions for \textbf{E} and \textbf{B}\\
\end{flushleft}

\bigskip

\begin{flushleft}
Solution:
\end{flushleft}

\begin{center}
    \begin{table}[ht]
        \caption{Input Parameters}
        \begin{tabular}{|c|c|c|}
\hline 
   \textbf{Parameter}  &\textbf{Description} &\textbf{Value} \\
\hline
&&\\
$I_r$&Net Intensity of light at $\Delta x =\dfrac{\lambda}{3}$ &$\dfrac{K}{4}$ \\&&\\
\hline
\end{tabular}

        \label{tab:table1.12.8.8}
    \end{table}
\end{center}


\begin{flushleft}
    \begin{table}[ht]
       \caption{Formulae and Output}
       \begin{tabular}{|c|c|c|c|}
\hline
\textbf{Parameter}&\textbf{Description} &\textbf{subquestion}& \textbf{Value}\\
\hline
     \multirow{4}{*}{$\Delta \theta$} & \multirow{4}{*}{$\theta_1 - \theta_2$} &\brak{a}& 6.4$\pi$ \, radians \\
     \cline{3-4}
     & & \brak{b}& 0.8$\pi$ \, radians \\
     \cline{3-4}
     & &\brak{c}& $\pi$ \, radians \\
     \cline{3-4}
     & & \brak{d} & $\dfrac{3\pi}{2\vphantom{\brak{0.1}}}$ \, radians \\
     \hline
\end{tabular}

       \label{tab:table2.12.8.8}
    \end{table}
\bigskip
\end{flushleft}

\bigskip
%\bigskip
%\bigskip
%\begin{flushleft}
%    (a)
%\end{flushleft}
%
%\begin{align}
%B_0  &= 400nT\\
% \omega &= 3.14 x 10^8 rad/s\\
 %  k &= 1.05 rad/m\\
%   \lambda &=  6.0m 
%\end{align}
%
%\begin{flushleft}
%    (b)\\ 
%    \begin{align}
%    \vec{E} &= 120 \sin[1.05x - 3.1 x 10^8t]\vec{e_2}\\
%    \vec{B} &= (4 x 10^{-7})\sin[1.05x - 3.14 x 10^8t]\vec{e_3}
%    \end{align} 
%\end{flushleft}
%
%
%\bigskip

%\begin{align}
%    c &= \frac{2\pi f}{k}\\
%    c &= f\lambda\\
%    \lambda &= \frac{c}{f}
%\end{align}

%\begin{flushleft}
%    \begin{table}[ht]
 %       \caption{Output Parameters}
  %      \begin{tabular}{|c|c|c|c|}
\hline
\textbf{Parameter}&\textbf{Description} &\textbf{subquestion}& \textbf{Value}\\
\hline
     \multirow{4}{*}{$\Delta \theta$} & \multirow{4}{*}{$\theta_1 - \theta_2$} &\brak{a}& 6.4$\pi$ \, radians \\
     \cline{3-4}
     & & \brak{b}& 0.8$\pi$ \, radians \\
     \cline{3-4}
     & &\brak{c}& $\pi$ \, radians \\
     \cline{3-4}
     & & \brak{d} & $\dfrac{3\pi}{2\vphantom{\brak{0.1}}}$ \, radians \\
     \hline
\end{tabular}

   %     \label{tab:table3.12.8.8}
   % \end{table}
%\end{flushleft}

\newpage
\renewcommand{\thefigure}{\theenumi}
\renewcommand{\thetable}{\theenumi}

\begin{flushleft}

\begin{figure}[h]
\renewcommand\thefigure{1}
  \caption{Graphs of $\vec{E} \text{ and } \vec{B}$}
  \includegraphics[width=1.05\columnwidth]{figs/Figure_1.png}
  \label{fig:fig1.12.8.8}

\end{figure}

\end{flushleft}

\end{document}


\pagebreak
\item A charged particle oscillates about its mean equilibrium position with a frequency of $10^9Hz$. What is the frequency of the electromagnetic waves produced by the oscillator? \\
\solution
\iffalse
\let\negmedspace\undefined
\let\negthickspace\undefined
\documentclass[journal,12pt,twocolumn]{IEEEtran}
\usepackage{cite}
\usepackage{amsmath,amssymb,amsfonts,amsthm}
\usepackage{algorithmic}
\usepackage{graphicx}
\usepackage{textcomp}
\usepackage{xcolor}
\usepackage{txfonts}
\usepackage{listings}
\usepackage{enumitem}
\usepackage{mathtools}
\usepackage{gensymb}
\usepackage{comment}
\usepackage[breaklinks=true]{hyperref}
\usepackage{tkz-euclide} 
\usepackage{listings}
\usepackage{gvv}                                        
\def\inputGnumericTable{}                                 
\usepackage[latin1]{inputenc}                                
\usepackage{color}                                            
\usepackage{array}                                            
\usepackage{longtable}                                       
\usepackage{calc}                                             
\usepackage{multirow}                                         
\usepackage{hhline}                                           
\usepackage{ifthen}                                           
\usepackage{lscape}
\usepackage[export]{adjustbox}

\newtheorem{theorem}{Theorem}[section]
\newtheorem{problem}{Problem}
\newtheorem{proposition}{Proposition}[section]
\newtheorem{lemma}{Lemma}[section]
\newtheorem{corollary}[theorem]{Corollary}
\newtheorem{example}{Example}[section]
\newtheorem{definition}[problem]{Definition}
\newcommand{\BEQA}{\begin{eqnarray}}
\newcommand{\EEQA}{\end{eqnarray}}
\newcommand{\define}{\stackrel{\triangle}{=}}
\theoremstyle{remark}
\newtheorem{rem}{Remark}
\begin{document}
\parindent 0px
\bibliographystyle{IEEEtran}

\title{Assignment\\[1ex]12.8 - 6}
\author{EE23BTECH11034 - Prabhat Kukunuri$^{}$% <-this % stops a space
}
\maketitle
\newpage
\bigskip

\renewcommand{\thefigure}{\theenumi}
\renewcommand{\thetable}{\theenumi}
\section*{Question}
A charged particle oscillates about its mean equilibrium position with a frequency of $10^9Hz$. What is the frequency of the electromagnetic waves produced by the oscillator?

\section*{Solution}
\fi
\begin{table}[h]
    \centering
    \begin{tabular}{|c|c|c|}
    \hline
   Symbol&Value&Description\\ \hline
   $y(t)$&$\cos\brak{2{\pi}f_ct}$&Wave equation of electro-magnetic wave\\ \hline
   $f_c$&$10^9$&Frequency of electromagnetic wave\\ \hline
   $t$&seconds&Time\\ \hline

    \end{tabular}
    \caption{Variable description}
    \label{tab:12.8.6.1}
\end{table}

\begin{figure}[h]
    \centering
    \includegraphics[width=\columnwidth]{ncert-physics/12/8/6/figs/Figure_1.png}
    \caption{$y(t)=\cos\brak{2{\pi}\times 10^9t}$}
    \label{fig:12.8.6.2}
\end{figure}
%\end{document}

\pagebreak
\item Given below are some functions of x and t to 
represent the displacement (transverse
or longitudinal) of an elastic wave. State which of these represents \brak i travelling
wave, \brak {ii} a stationary wave or \brak {iii} none at all: \\
\begin{enumerate}
\item $y = 2\cos \brak{3x} \sin \brak{10t}$
\item $y=2\sqrt{x-vt}$
\item $y = 3\sin \brak{5x - 0.5t} + 4\cos \brak{5x - 0.5t}$
\item $y = \cos x \sin t + \cos 2x \sin 2t$
\end{enumerate}
\solution
% \iffalse
\let\negmedspace\undefined
\let\negthickspace\undefined
\documentclass[journal,12pt,twocolumn]{IEEEtran}
\usepackage{cite}
\usepackage{amsmath,amssymb,amsfonts,amsthm}
\usepackage{algorithmic}
\usepackage{graphicx}
\usepackage{textcomp}
\usepackage{xcolor}
\usepackage{txfonts}
\usepackage{listings}
\usepackage{enumitem}
\usepackage{mathtools}
\usepackage{gensymb}
\usepackage{comment}
\usepackage[breaklinks=true]{hyperref}
\usepackage{tkz-euclide} 
\usepackage{listings}
\usepackage{gvv}                                        
\def\inputGnumericTable{}                                 
\usepackage[latin1]{inputenc}                                
\usepackage{color}                                            
\usepackage{array}                                            
\usepackage{longtable}                                       
\usepackage{calc}                                             
\usepackage{multirow}                                         
\usepackage{hhline}                                           
\usepackage{ifthen}                                           
\usepackage{lscape}
\newtheorem{theorem}{Theorem}[section]
\newtheorem{problem}{Problem}
\newtheorem{proposition}{Proposition}[section]
\newtheorem{lemma}{Lemma}[section]
\newtheorem{corollary}[theorem]{Corollary}
\newtheorem{example}{Example}[section]
\newtheorem{definition}[problem]{Definition}
\newcommand{\BEQA}{\begin{eqnarray}}
\newcommand{\EEQA}{\end{eqnarray}}
\newcommand{\define}{\stackrel{\triangle}{=}}
\theoremstyle{remark}
\newtheorem{rem}{Remark}
\begin{document}
\parindent 0px
\bibliographystyle{IEEEtran}
\title{ASSIGNMENT11.15\_13Q}
\author{EE22BTECH11219 - Sai Sujan Rada$^{}$% <-this % stops a space
}
\maketitle
\newpage
\bigskip
\section*{QUESTION}
Given below are some functions of x and t to 
represent the displacement (transverse
or longitudinal) of an elastic wave. State which of these represents \brak i travelling
wave, \brak {ii} a stationary wave or \brak {iii} none at all: \\
\begin{enumerate}
\item $y = 2\cos \brak{3x} \sin \brak{10t}$
\item $y=2\sqrt{x-vt}$
\item $y = 3\sin \brak{5x - 0.5t} + 4\cos \brak{5x - 0.5t}$
\item $y = \cos x \sin t + \cos 2x \sin 2t$
\end{enumerate}
\solution 

\begin{tabular}{|c|c|c|}
    \hline
    PARAMETER & VALUE & DESCRIPTION  \\ \hline
    $$x\brak0$$ & $$x\brak{0}$$ & First term \\ \hline
    $$d$$ & $$d$$ & common difference \\ \hline
    $$x(n)$$ & $$[x\brak{0}+nd]u\brak n$$ & General term of the series  \\ \hline
  \end{tabular}

Let us assume an equation:
\begin{align}
y=A(x)\cos \brak{\omega t+\phi\brak {x}}
\end{align}
\begin{tabular}{|p{4.5cm}|p{4.5cm}|}
    \hline
      $$x\brak{0}$$ & $$3$$  \\ \hline
      $$d$$ & $$2$$  \\ \hline
      $$m$$ & $$6$$  \\ \hline
      $$n$$ & $$2$$  \\ \hline
      $$x\brak{m+n}$$ & $$19$$  \\ \hline
      $$x\brak{m-n}$$ & $$11$$  \\ \hline
      $$x\brak{m}$$ & $$15$$  \\ \hline
    \end{tabular}

\begin{figure}[ht]
                        \centering
                        \includegraphics[width=\columnwidth]{figs/a.png}
                        \caption{DIPLACEMENT $vs$ TIME-graph1}
                        \label{fig:1}
\end{figure}
\begin{figure}[ht]
                            \centering
                            \includegraphics[width=\columnwidth]{figs/b.png}
                            \caption{DIPLACEMENT $vs$ TIME-graph2}
                            \label{fig:2}
\end{figure}   
\begin{figure}[ht]
                             \centering
                             \includegraphics[width=\columnwidth]{figs/c.png}
                             \caption{DIPLACEMENT $vs$ TIME-graph3}
                             \label{fig:3}
\end{figure}
\begin{figure}[ht]
                            \centering
                            \includegraphics[width=\columnwidth]{figs/d.png}
                            \caption{DIPLACEMENT $vs$ TIME-graph4}
                            \label{fig:4}
\end{figure}
\figref{fig:1} and \figref{fig:3} are self explanatory for stationary and travelling waves.
\figref{fig:2} and \figref{fig:4} are neither stationary nor travelling waves. 
\end{document}


\pagebreak
\item For the travelling harmonic wave
$y\brak {x, t} = 2.0 \cos 2\pi \brak{10t - 0.0080 x + 0.35}$ where $x$ and $y$ are in $cm$ and $t$ in $s$. Calculate the phase difference between oscillatory
motion of two points separated by a distance of 

\begin{enumerate} [label=(\alph*)]
    \item $4 m$
    \item $0.5 m$
    \item $\lambda/2$
    \item $3\lambda/4$
\end{enumerate}
\solution
% \iffalse
\let\negmedspace\undefined
\let\negthickspace\undefined
\documentclass[journal,12pt,twocolumn]{IEEEtran}
\usepackage{cite}
\usepackage{amsmath,amssymb,amsfonts,amsthm}
\usepackage{algorithmic}
\usepackage{graphicx}
\usepackage{textcomp}
\usepackage{xcolor}
\usepackage{txfonts}
\usepackage{listings}
\usepackage{enumitem}
\usepackage{mathtools}
\usepackage{gensymb}
\usepackage{comment}
\usepackage[breaklinks=true]{hyperref}
\usepackage{tkz-euclide} 
\usepackage{listings}
\usepackage{gvv}                                        
\def\inputGnumericTable{}                                 
\usepackage[latin1]{inputenc}                               \usepackage{caption}
\usepackage{color}                                            
\usepackage{array}                                            
\usepackage{longtable}                                       
\usepackage{calc}                                             
\usepackage{multirow}                                         
\usepackage{hhline}                                           
\usepackage{ifthen}                                           
\usepackage{lscape}

\newtheorem{theorem}{Theorem}[section]
\newtheorem{problem}{Problem}
\newtheorem{proposition}{Proposition}[section]
\newtheorem{lemma}{Lemma}[section]
\newtheorem{corollary}[theorem]{Corollary}
\newtheorem{example}{Example}[section]
\newtheorem{definition}[problem]{Definition}
\newcommand{\BEQA}{\begin{eqnarray}}
\newcommand{\EEQA}{\end{eqnarray}}
\newcommand{\define}{\stackrel{\triangle}{=}}
\theoremstyle{remark}
\newtheorem{rem}{Remark}
\begin{document}

\bibliographystyle{IEEEtran}
\vspace{3cm}

\title{NCERT 11.15. Q10}
\author{EE23BTECH11010 - Venkatesh Bandawar$^{*}$% <-this % stops a space
}
\maketitle
\newpage
\bigskip

\renewcommand{\thefigure}{\arabic{figure}}
\renewcommand{\thetable}{\arabic{table}}

\bibliographystyle{IEEEtran}

\parindent 0px
\textbf{Question:} For the travelling harmonic wave
$y\brak {x, t} = 2.0 \cos 2\pi \brak{10t - 0.0080 x + 0.35}$ where $x$ and $y$ are in $cm$ and $t$ in $s$. Calculate the phase difference between oscillatory
motion of two points separated by a distance of 

\begin{enumerate} [label=(\alph*)]
    \item $4 m$
    \item $0.5 m$
    \item $\lambda/2$
    \item $3\lambda/4$
\end{enumerate}

\textbf{Solution:}  
\begin{table}[htbp] \small
\centering
\begin{tabular}{|c|c|c|}
\hline 
   \textbf{Parameter}  &\textbf{Description} &\textbf{Value} \\
\hline
&&\\
$I_r$&Net Intensity of light at $\Delta x =\dfrac{\lambda}{3}$ &$\dfrac{K}{4}$ \\&&\\
\hline
\end{tabular}

\caption{Given \, parameters list}
\label{tab:given parameters list}
\end{table}
\begin{align}
    \brak{\Delta \theta} &= \brak{ 2\pi f t - kx_1 + \phi}  - \brak{2\pi f t -kx_2 + \phi}\\
    &= k\brak{x_2 - x_1} 
\end{align}

\begin{table}[htbp] 
\centering
\begin{tabular}{|c|c|c|c|}
\hline
\textbf{Parameter}&\textbf{Description} &\textbf{subquestion}& \textbf{Value}\\
\hline
     \multirow{4}{*}{$\Delta \theta$} & \multirow{4}{*}{$\theta_1 - \theta_2$} &\brak{a}& 6.4$\pi$ \, radians \\
     \cline{3-4}
     & & \brak{b}& 0.8$\pi$ \, radians \\
     \cline{3-4}
     & &\brak{c}& $\pi$ \, radians \\
     \cline{3-4}
     & & \brak{d} & $\dfrac{3\pi}{2\vphantom{\brak{0.1}}}$ \, radians \\
     \hline
\end{tabular}

\caption{Phase \, differences}
\label{tab:phase differences}
\end{table}

\begin{figure}[!h] 
\centering
\includegraphics[width=\columnwidth]{figs/graph1.png}
\captionsetup{justification=centering}
\caption{}
\label{fig:Graph1}
\end{figure}

\begin{figure}[!h] 
\centering
\includegraphics[width=\columnwidth]{figs/graph2.png}
\captionsetup{justification=centering}
\caption{}
\label{fig:Graph2}
\end{figure}

\begin{figure}[!h] 
\centering
\includegraphics[width=\columnwidth]{figs/graph3.png}
\captionsetup{justification=centering}
\caption{}
\label{fig:Graph3}
\end{figure}

\begin{figure}[!h] 
\centering
\includegraphics[width=\columnwidth]{figs/graph4.png}
\captionsetup{justification=centering}
\caption{}
\label{fig:Graph4}
\end{figure}



\end{document}

\pagebreak
\item 
\begin{enumerate}
\item The peak voltage of an AC supply is 300 V. What is the rms voltage?
\item The rms value of current in an AC circuit is 10 A. What is the peak current?
\end{enumerate}
\solution
\let\negmedspace\undefined
\let\negthickspace\undefined
\documentclass[journal,12pt,onecolumn]{IEEEtran}
\usepackage{cite}
\usepackage{amsmath,amssymb,amsfonts,amsthm}
%\usepackage{algorithmic}
\usepackage{graphicx}
\usepackage{textcomp}
\usepackage{xcolor}
\usepackage{txfonts}
\usepackage{listings}
\usepackage{enumitem}
\usepackage{mathtools}
\usepackage{gensymb}
\usepackage[breaklinks=true]{hyperref}
\usepackage{tkz-euclide} % loads  TikZ and tkz-base
\usepackage{listings}
\usepackage{float}



\newtheorem{theorem}{Theorem}[section]
\newtheorem{problem}{Problem}
\newtheorem{proposition}{Proposition}[section]
\newtheorem{lemma}{Lemma}[section]
\newtheorem{corollary}[theorem]{Corollary}
\newtheorem{example}{Example}[section]
\newtheorem{definition}[problem]{Definition}
%\newtheorem{thm}{Theorem}[section] 
%\newtheorem{defn}[thm]{Definition}
%\newtheorem{algorithm}{Algorithm}[section]
%\newtheorem{cor}{Corollary}
\newcommand{\BEQA}{\begin{eqnarray}}
\newcommand{\EEQA}{\end{eqnarray}}
\newcommand{\define}{\stackrel{\triangle}{=}}
\theoremstyle{remark}
\newtheorem{rem}{Remark}
%\bibliographystyle{ieeetr}
\begin{document}
%
\providecommand{\pr}[1]{\ensuremath{\Pr\left(#1\right)}}
\providecommand{\prt}[2]{\ensuremath{p_{#1}^{\left(#2\right)} }}        % own macro for this question
\providecommand{\qfunc}[1]{\ensuremath{Q\left(#1\right)}}
\providecommand{\sbrak}[1]{\ensuremath{{}\left[#1\right]}}
\providecommand{\lsbrak}[1]{\ensuremath{{}\left[#1\right.}}
\providecommand{\rsbrak}[1]{\ensuremath{{}\left.#1\right]}}
\providecommand{\brak}[1]{\ensuremath{\left(#1\right)}}
\providecommand{\lbrak}[1]{\ensuremath{\left(#1\right.}}
\providecommand{\rbrak}[1]{\ensuremath{\left.#1\right)}}
\providecommand{\cbrak}[1]{\ensuremath{\left\{#1\right\}}}
\providecommand{\lcbrak}[1]{\ensuremath{\left\{#1\right.}}
\providecommand{\rcbrak}[1]{\ensuremath{\left.#1\right\}}}
\newcommand{\sgn}{\mathop{\mathrm{sgn}}}
\providecommand{\abs}[1]{\left\vert#1\right\vert}
\providecommand{\res}[1]{\Res\displaylimits_{#1}} 
\providecommand{\norm}[1]{\left\lVert#1\right\rVert}
%\providecommand{\norm}[1]{\lVert#1\rVert}
\providecommand{\mtx}[1]{\mathbf{#1}}
\providecommand{\mean}[1]{E\left[ #1 \right]}
\providecommand{\cond}[2]{#1\middle|#2}
\providecommand{\fourier}{\overset{\mathcal{F}}{ \rightleftharpoons}}
\newenvironment{amatrix}[1]{%
  \left(\begin{array}{@{}*{#1}{c}|c@{}}
}{%
  \end{array}\right)
}
%\providecommand{\hilbert}{\overset{\mathcal{H}}{ \rightleftharpoons}}
%\providecommand{\system}{\overset{\mathcal{H}}{ \longleftrightarrow}}
	%\newcommand{\solution}[2]{\textbf{Solution:}{#1}}
\newcommand{\solution}{\noindent \textbf{Solution: }}
\newcommand{\cosec}{\,\text{cosec}\,}
\providecommand{\dec}[2]{\ensuremath{\overset{#1}{\underset{#2}{\gtrless}}}}
\newcommand{\myvec}[1]{\ensuremath{\begin{pmatrix}#1\end{pmatrix}}}
\newcommand{\mydet}[1]{\ensuremath{\begin{vmatrix}#1\end{vmatrix}}}
\newcommand{\myaugvec}[2]{\ensuremath{\begin{amatrix}{#1}#2\end{amatrix}}}
\providecommand{\rank}{\text{rank}}
\providecommand{\pr}[1]{\ensuremath{\Pr\left(#1\right)}}
\providecommand{\qfunc}[1]{\ensuremath{Q\left(#1\right)}}
	\newcommand*{\permcomb}[4][0mu]{{{}^{#3}\mkern#1#2_{#4}}}
\newcommand*{\perm}[1][-3mu]{\permcomb[#1]{P}}
\newcommand*{\comb}[1][-1mu]{\permcomb[#1]{C}}
\providecommand{\qfunc}[1]{\ensuremath{Q\left(#1\right)}}
\providecommand{\gauss}[2]{\mathcal{N}\ensuremath{\left(#1,#2\right)}}
\providecommand{\diff}[2]{\ensuremath{\frac{d{#1}}{d{#2}}}}
\providecommand{\myceil}[1]{\left \lceil #1 \right \rceil }
\newcommand\figref{Fig.~\ref}
\newcommand\tabref{Table~\ref}
\newcommand{\sinc}{\,\text{sinc}\,}
\newcommand{\rect}{\,\text{rect}\,}
%%
%	%\newcommand{\solution}[2]{\textbf{Solution:}{#1}}
%\newcommand{\solution}{\noindent \textbf{Solution: }}
%\newcommand{\cosec}{\,\text{cosec}\,}
%\numberwithin{equation}{section}
%\numberwithin{equation}{subsection}
%\numberwithin{problem}{section}
%\numberwithin{definition}{section}
%\makeatletter
%\@addtoreset{figure}{problem}
%\makeatother

%\let\StandardTheFigure\thefigure
\let\vec\mathbf

\bibliographystyle{IEEEtran}





\bigskip

%\renewcommand{\thefigure}{\theenumi}
%\renewcommand{\thetable}{\theenumi}
%\renewcommand{\theequation}{\theenumi}

Q: \\
\begin{enumerate}
\item The peak voltage of an AC supply is 300 V. What is the rms voltage?
\item The rms value of current in an AC circuit is 10 A. What is the peak current?
\end{enumerate}

\solution

\begin{table}[!h]

  \centering
  \begin{tabular}{|c|c|c|}
    \hline
    parameter & value & description \\
    \hline
    $V(t)$ & $V_{\text{0}} \cdot \sin(2\pi ft + \phi)$ & voltage in terms of time \\
    \hline
    $I(t)$ & $I_{\text{0}} \cdot \sin(2\pi ft + \phi)$ & current in terms of time \\
    \hline
    $V_0$ & $300 \, \text{V}$ & peak voltage \\
    \hline
    $V_ \text{rms}$ & $\sqrt{\frac{1}{T} \int_{0}^{T} [V(t)]^2 \, dt}$ & rms value of Voltage \\
    \hline 
    $I_ \text{rms}$ & $10 \, \text{A}$ & rms value of current\\
    \hline
    $I_0$ & $\sqrt{2} \times I_{\text{rms}}$ & peak current \\
    \hline
    $f$ & $50 \, \text{Hz}$ & frequence of the sinosoidal wave. \\
    \hline
    $T$ & $0.02 \, \text{s}$ & time period of sinosoidal wave. \\
    \hline
  \end{tabular}


\caption{Input Parameter Table}
\label{tab:input_parameters}
\end{table}


\begin{enumerate}
\item
\begin{align}
V_{\text{rms}}^2 &= {\frac{1}{T} \int_{0}^{T} [V(t)]^2 \, dt} \\
&= {f \int_{0}^{\frac{1}{f}} V_{\text{0}}^2 \cdot \sin^2(2\pi ft + \phi) \, dt} \\
&= \frac{1}{2} V_{0}^2 \left(1 - \frac{1}{f}\int_{0}^{\frac{1}{f}} \cos(4\pi ft + 2\phi) \, dt \right) \\
&= \frac{1}{2} V_{0}^2 \left(1 - \frac{1}{f}\left[\frac{\sin(4\pi ft + 2\phi)}{4\pi f}\right]_{0}^{\frac{1}{f}}\right) \\
&= \frac{1}{2} V_{0}^2 \left(1 - \frac{1}{f} \cdot \frac{\sin\left(4\pi + 2\phi\right) - \sin(0 + 2\phi)}{4\pi f}\right) \\
V_{\text{rms}} &= \frac{V_{0}}{\sqrt{2}} \label{eq:12.7.2_voltage}
\end{align}

To find the RMS voltage (\(V_{\text{rms}})\) when the peak voltage (\(V_{\text{0}})\) is 300V, you can use equation  \eqref{eq:12.7.2_voltage}

\begin{align}
V_{\text{rms}} &= \frac{300V}{\sqrt{2}} \approx 212.13V
\end{align}

\item 
\begin{align}
I_{\text{rms}}^2 &= {\frac{1}{T} \int_{0}^{T} [I(t)]^2 \, dt} \\
&= {f \int_{0}^{\frac{1}{f}} I_{\text{0}}^2 \cdot \sin^2(2\pi ft + \phi) \, dt} \\
&= \frac{1}{2} I_{0}^2 \left(1 - \frac{1}{f}\left[\frac{\sin(4\pi ft + 2\phi)}{4\pi f}\right]_{0}^{\frac{1}{f}}\right) \\
&= \frac{1}{2} I_{0}^2 \left(1 - \frac{1}{f} \cdot \frac{\sin\left(4\pi + 2\phi\right) - \sin(0 + 2\phi)}{4\pi f}\right) \\
I_{\text{rms}} &= \frac{I_{0}}{\sqrt{2}} \label{eq:12.7.2_current}
\end{align}

To find the peak current (\(I_{\text{0}}\)) when the RMS current (\(I_{\text{rms}}\)) is given, you can use equation  \eqref{eq:12.7.2_current}

\begin{align}
I_{\text{0}} \approx 10 \, \text{A} \times 1.414 \approx 14.14 \, \text{A}  
\end{align}

\begin{figure}[H]
    \centering
    \includegraphics[width=\columnwidth]{./figs/merged_sine_wave_plots.png}
\end{figure}

\end{enumerate}
\end{document}


\pagebreak
\item In Young’s double-slit experiment using monochromatic light of wavelength $\lambda$, the intensity of light at a point on the screen where path difference is $\lambda$, is $K$ units. What is the intensity of light at a
point where path difference is $\lambda$/3?\\

\solution

% \iffalse
\let\negmedspace\undefined
\let\negthickspace\undefined
\documentclass[journal,12pt,twocolumn]{IEEEtran}
\usepackage{cite}
\usepackage{amsmath,amssymb,amsfonts,amsthm}
\usepackage{algorithmic}
\usepackage{graphicx}
\usepackage{textcomp}
\usepackage{xcolor}
\usepackage{txfonts}
\usepackage{listings}
\usepackage{enumitem}
\usepackage{mathtools}
\usepackage{gensymb}
\usepackage{comment}
\usepackage[breaklinks=true]{hyperref}
\usepackage{tkz-euclide} 
\usepackage{listings}
\usepackage{gvv} 
\usepackage{caption}
\def\inputGnumericTable{}                                 
%\usepackage[latin1]{inputenc}                                
\usepackage{color}                                            
\usepackage{array}                                            
\usepackage{longtable}                                       
\usepackage{calc}                                             
\usepackage{multirow}                                         
\usepackage{hhline}                                           
\usepackage{ifthen}                                           
\usepackage{lscape}

\newtheorem{theorem}{Theorem}[section]
\newtheorem{problem}{Problem}
\newtheorem{proposition}{Proposition}[section]
\newtheorem{lemma}{Lemma}[section]
\newtheorem{corollary}[theorem]{Corollary}
\newtheorem{example}{Example}[section]
\newtheorem{definition}[problem]{Definition}
\newcommand{\BEQA}{\begin{eqnarray}}
\newcommand{\EEQA}{\end{eqnarray}}
\newcommand{\define}{\stackrel{\triangle}{=}}
\theoremstyle{remark}
\newtheorem{rem}{Remark}

\begin{document}

\bibliographystyle{IEEEtran}
\vspace{3cm}

\title{NCERT 12.10 5Q}
\author{EE23BTECH11013 - Avyaaz$^{*}$% <-this % stops a space 
}
\maketitle
\newpage
\bigskip

\renewcommand{\thefigure}{\arabic{figure}}
\renewcommand{\thetable}{\arabic{table}}

\large\textbf{\textsl{Question:}}
In Young’s double-slit experiment using monochromatic light of wavelength $\lambda$, the intensity of light at a point on the screen where path difference is $\lambda$, is $K$ units. What is the intensity of light at a
point where path difference is $\lambda$/3?\\
\large\textbf{\textsl{Solution:}}
\begin{table}[htbp]
\setlength{\extrarowheight}{8pt}
\centering
\setlength{\arrayrulewidth}{0.2mm}
\setlength{\tabcolsep}{15pt}
\renewcommand{\arraystretch}{1.15}


\begin{table}[ht]
  \centering
  \begin{tabular}{|c|c|c|}
    \hline
    	Symbol & Parameters & value\\
    \hline
	  $u\brak{n}$ & unit step function & 1, if n$\geq$ 0; \\& &0 otherwise \\
    \hline
	  $x\brak{n}$ & general term of the series & $\brak{n+1}\brak{n+3}u\brak{n}$ \\
    \hline 
	 $X\brak{z}$ & Z-transform of $x\brak{n}$ & ? \\
    \hline
  \end{tabular}
  \vspace{0.3cm}
  \caption{Input Parameters}
  \label{tab:24.11.9.1.1}
\end{table}


\caption{Parameters}
\label{tab:parameters}
\end{table}

From \tabref{tab:parameters}:
\begin{align}
%%y\brak{t} &= y_1\brak{t} + y_2\brak{t}  \\
y\brak{t} &= A\sin({2\pi f t - kx_1})  + A\sin({ 2\pi f t - kx_2}) \\
y\brak{t} &=  2A\cos\left(\dfrac{k\Delta x}{2}\right)\sin\left(2\pi f t - \dfrac{k(x_1+x_2)}{2} \right) \label{eq:superposition}
\end{align}
From \tabref{tab:parameters} and equation \eqref{eq:superposition}: 
\begin{align}
\therefore I \propto 4A^2\cos^2\left(\dfrac{k\Delta x}{2}\right)  \label{eq:intensity}
\end{align}
From \tabref{tab:parameters} and equation \eqref{eq:intensity}: 
\begin{align}
 \dfrac{K}{I_r} = \dfrac{4A^2\cos^2\left(\dfrac{2\pi}{2}\right)}{4A^2\cos^2\left(\dfrac{\pi}{3}\right)}
 \implies I_r = \dfrac{K}{4}
 \end{align}
 Hence, the Intensity of light at a point where path difference is $\dfrac{\lambda}{3}$ is $\dfrac{K}{4}$ units.

\begin{table}[htbp]
\centering
\begin{tabular}{|c|c|c|}
\hline 
   \textbf{Parameter}  &\textbf{Description} &\textbf{Value} \\
\hline
&&\\
$I_r$&Net Intensity of light at $\Delta x =\dfrac{\lambda}{3}$ &$\dfrac{K}{4}$ \\&&\\
\hline
\end{tabular}

\caption{}
\label{tab:intensity}
\end{table}
Assuming $\Delta x= r\lambda$, 

From equation \eqref{eq:intensity}:
\begin{figure}[htbp]
    \centering
    \includegraphics[width = \columnwidth]{figs/intensity_plot.png}
  \caption{}
    \label{fig:graph1}
\end{figure}
\bibliographystyle{IEEEtran}
\end{document}

\pagebreak

\item In a plane electromagnetic wave, the electric field oscillates sinusoidally at a frequency of $2.0 \text{ x } 10^{10}$ Hz and amplitude 48 $Vm^{-1}$.
\begin{enumerate}[label=(\alph*)]
    \item What is the wavelength of the wave?
    \item What is the amplitude of the oscillating magnetic field?
    \item Show that the average energy density of the $\vec{E}$ field equals the
average energy density of the $\vec{B}$ field. $[c = 3 \text{ x } 10^{8}ms^{-1} ]$
\end{enumerate}

\item \begin{enumerate}
\item For the wave on the string $y(x, t) = 0.06 \sin(\frac{2\pi x}{3}) \cos(120\pi t)$ , do all the points on the string     oscillate with the same (a)frequency , (b)phase , (c)amplitude ? Explain your answers. \\

 \item What is the amplitude of a point 0.375m away from one end? \\
 \end{enumerate}
 \solution
 \pagebreak
 
 \item 
 A transverse harmonic wave on a string is described by
\begin{align}
    y\brak{x,t}=3.0 \sin\brak{36t+0.018x+\frac{\pi}{4}}
\end{align}
where $x$ and $y$ are in cm and $t$ in s. The positive direction of $x$ is from left to right.
\begin{enumerate}[label=(\alph*)]
    \item Is this a travelling wave or a stationary wave? If it is travelling, what are the speed and direction of its propogation?
    \item What are its amplitude and frequency?
    \item What is the initial phase at the origin?
    \item  What is the least distance between two succesive crests in the wave?
\end{enumerate}

\solution
\pagebreak

\item In deriving the single slit diffraction pattern, it was stated that the intensity is zero at angles of $\frac{n\lambda}{a}$. Justify this by suitably dividing the slit to bring out the cancellation.\\
\solution
\pagebreak

\item A 60 $\mu$ F capacitor is connected to a 110 V, 60 Hz ac supply. Determine the rms value of the current in the circuit.\\
\solution
\pagebreak

\item A charged  $30\mu F$ capacitor is connected to a $27 mH$ inductor. What is the angular frequency of free oscillations of the circuit?\\
\solution
\pagebreak
\item Obtain the resonance frequency of a series LCR circuit with $L = 2.0\, H$, $C = 32\, \mu F$, and $R = 10\, \Omega$. What is the Q-value of the circuit.\\
\solution
\pagebreak
\item A charged 30 $\mu$F capacitor is connected to a 27 mH inductor. Suppose the initial charge on the capacitor is 6mC.What is the total energy stored in the circuit initially? What is the
total energy at later time? \\
\solution
\pagebreak

\item A wire stretched between two rigid supports vibrates in its fundamental mode with a frequency of $45 \, \text{Hz}$. The mass of the wire is $3.5 \times 10^{-2} \, \text{kg}$, and its linear mass density is $4.0 \times 10^{-2} \, \text{kg/m}$. The length of the wire is $0.875 \, \text{m}$. Determine the speed of a transverse wave on the string and the tension in the string.\\
\solution
\pagebreak

\item The given figure shows a series LCR circuit connected to a variable
frequency 230 V source. \\
L = 5.0 H, C = 80 $\mu$F, R = 40 $\Omega$.

\begin{figure}[h!]
\begin{center}
\begin{circuitikz}[american voltages]
      \draw (0,0)
      to[sV, l=$\varepsilon$] (0,2) 
      to[R, l=$R$, v=$V_R$] (4,2) 
      to[C, l=$C$, v=$V_C$] (4,0)
      to[L, l=$L$, v=$V_L$] (0,0);
\end{circuitikz}
\end{center}
\end{figure}

\begin{enumerate}
    \item Determine the source frequency which drives the circuit in resonance.
    \item Obtain the impedance of the circuit and the amplitude of current
at the resonating frequency.
    \item Determine the rms potential drops across the three elements of
the circuit. Show that the potential drop across the LC
combination is zero at the resonating frequency.\\
\end{enumerate}
\solution
\pagebreak

\item Q23) A narrow sound pulse (for example, a short pip by a whistle) is sent across a
	medium.\\ \brak{\text{a}} Does the pulse have a definite \brak{\text{i}} frequency, \brak{\text{ii}} wavelength, \brak{\text{iii}} speed
	of propagation?\\[1ex]\brak{\text{b}} If the pulse rate is 1 after every 20 s, (that is the whistle is
	blown for a split of second after every 20 s), Is the frequency of note produced
	by whistle equal to 1/20 or 0.05 Hz ?\\
\solution
\pagebreak
\item Suppose that the electric field part of an electromagnetic wave in vacuum given as\\ \textbf{E} =\{(3.1N/C)cos[(1.8 rad/m)y+(5.4$\times$10$^{6}$rad/s)t]\}\^i \\
(a) What is the direction of propagation ?\\
(b) What is the wavelength ? \\
(c) What is the frequency ?\\
(d) What is the amplitude of the magnetic field part of the wave?\\
(e) Write an expression for the magnetic field part of the wave.\\
\solution
\pagebreak

\item A 44 mH inductor is connected to 220 V, 50 Hz ac supply. Determine
the rms value of the current in the circuit.\\
\solution
\pagebreak

\item The 6563 \AA\, H$\alpha$ line emitted by hydrogen in a star is found to be redshifted by 15 \AA. Estimate the speed with which the star is receding from the Earth.
\solution
\pagebreak

\item A steel rod $100$cm long is clamped at its middle. The fundamental frequency of the longitudinal vibrations of the rod are given to be $2.53$kHz. What is the speed of sound in steel? \\
\solution
\pagebreak

\end{enumerate}
