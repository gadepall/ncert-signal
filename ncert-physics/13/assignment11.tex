% \iffalse
\let\negmedspace\undefined
\let\negthickspace\undefined
\documentclass[journal,12pt,twocolumn]{IEEEtran}
\usepackage{cite}
\usepackage{amsmath,amssymb,amsfonts,amsthm}
\usepackage{algorithmic}
\usepackage{graphicx}
\usepackage{textcomp}
\usepackage{xcolor}
\usepackage{txfonts}
\usepackage{listings}
\usepackage{enumitem}
\usepackage{mathtools}
\usepackage{gensymb}
\usepackage{comment}
\usepackage[breaklinks=true]{hyperref}
\usepackage{tkz-euclide} 
\usepackage{listings}
\usepackage{gvv}                                        
\def\inputGnumericTable{}                                 
\usepackage[latin1]{inputenc}                                
\usepackage{color}                                            
\usepackage{array}                                            
\usepackage{longtable}                                       
\usepackage{calc}                                             
\usepackage{multirow}                                         
\usepackage{hhline}                                           
\usepackage{ifthen}                                           
\usepackage{lscape}
\newtheorem{theorem}{Theorem}[section]
\newtheorem{problem}{Problem}
\newtheorem{proposition}{Proposition}[section]
\newtheorem{lemma}{Lemma}[section]
\newtheorem{corollary}[theorem]{Corollary}
\newtheorem{example}{Example}[section]
\newtheorem{definition}[problem]{Definition}
\newcommand{\BEQA}{\begin{eqnarray}}
\newcommand{\EEQA}{\end{eqnarray}}
\newcommand{\define}{\stackrel{\triangle}{=}}
\theoremstyle{remark}
\newtheorem{rem}{Remark}
\begin{document}
\parindent 0px
\bibliographystyle{IEEEtran}
\title{ASSIGNMENT11.15\_13Q}
\author{EE22BTECH11219 - Sai Sujan Rada$^{}$% <-this % stops a space
}
\maketitle
\newpage
\bigskip
\section*{QUESTION}
Given below are some functions of x and t to 
represent the displacement (transverse
or longitudinal) of an elastic wave. State which of these represents \brak i travelling
wave, \brak {ii} a stationary wave or \brak {iii} none at all: \\
\begin{enumerate}
\item $y = 2\cos \brak{3x} \sin \brak{10t}$
\item $y=2\sqrt{x-vt}$
\item $y = 3\sin \brak{5x - 0.5t} + 4\cos \brak{5x - 0.5t}$
\item $y = \cos x \sin t + \cos 2x \sin 2t$
\end{enumerate}
\solution 

\begin{tabular}{|c|c|c|}
    \hline
    PARAMETER & VALUE & DESCRIPTION  \\ \hline
    $$x\brak0$$ & $$x\brak{0}$$ & First term \\ \hline
    $$d$$ & $$d$$ & common difference \\ \hline
    $$x(n)$$ & $$[x\brak{0}+nd]u\brak n$$ & General term of the series  \\ \hline
  \end{tabular}

Let us assume an equation:
\begin{align}
y=A(x)\cos \brak{\omega t+\phi\brak {x}}
\end{align}
\begin{tabular}{|p{4.5cm}|p{4.5cm}|}
    \hline
      $$x\brak{0}$$ & $$3$$  \\ \hline
      $$d$$ & $$2$$  \\ \hline
      $$m$$ & $$6$$  \\ \hline
      $$n$$ & $$2$$  \\ \hline
      $$x\brak{m+n}$$ & $$19$$  \\ \hline
      $$x\brak{m-n}$$ & $$11$$  \\ \hline
      $$x\brak{m}$$ & $$15$$  \\ \hline
    \end{tabular}

\begin{figure}[ht]
                        \centering
                        \includegraphics[width=\columnwidth]{figs/a.png}
                        \caption{DIPLACEMENT $vs$ TIME-graph1}
                        \label{fig:1}
\end{figure}
\begin{figure}[ht]
                            \centering
                            \includegraphics[width=\columnwidth]{figs/b.png}
                            \caption{DIPLACEMENT $vs$ TIME-graph2}
                            \label{fig:2}
\end{figure}   
\begin{figure}[ht]
                             \centering
                             \includegraphics[width=\columnwidth]{figs/c.png}
                             \caption{DIPLACEMENT $vs$ TIME-graph3}
                             \label{fig:3}
\end{figure}
\begin{figure}[ht]
                            \centering
                            \includegraphics[width=\columnwidth]{figs/d.png}
                            \caption{DIPLACEMENT $vs$ TIME-graph4}
                            \label{fig:4}
\end{figure}
\figref{fig:1} and \figref{fig:3} are self explanatory for stationary and travelling waves.
\figref{fig:2} and \figref{fig:4} are neither stationary nor travelling waves. 
\end{document}
